\only<beamer>{\titleframe}




\begin{frame}{Overview}
    \tableofcontents
\end{frame}




\sectionframe{Here Comes Section 1}

\section{Section 1}

\begin{frame}{First Frame}
    This is the first frame.
    It's not very exciting, but still it's the first frame.

    Slides should contain a descriptive text.
    But not full sentences like these.
\end{frame}

\begin{frame}{Second Frame}
    The second frame\dots it doesn't get any better.

    Here is an \texttt{itemize} list:
    \begin{itemize}
        \item
        First item

        \item
        Second item

        \item
        Last item
    \end{itemize}
    Note the round dots before the items.
\end{frame}



\sectionframe{This Is Section 2}

\section{Section 2}

% text along an arrow
% (from LaTeX Stack Exchange: https://tex.stackexchange.com/a/154769,
% by Gonzalo Medina: https://tex.stackexchange.com/users/3954/gonzalo-medina,
% licensed under CC BY-SA: https://creativecommons.org/licenses/by-sa/3.0/)
\usetikzlibrary{decorations.text}
\tikzset{
    textalongpath/.style 2 args={
        decoration={
            text align={left indent=#1},
            text along path,
            text={#2}
        },
        decorate
    }
}

\begin{frame}{Ti\textit{k}Z Example}
  \centering
  \begin{tikzpicture}
      \visible<4->{\node[white,fill=mittelblau,circle,minimum size=25mm,align=center]
          at (0,0) (parameters) {\textbf{Parameters}\\$\vec{x} = (a, b, c)$};}
      \visible<3->{\node[white,fill=mittelblau,circle,minimum size=25mm,align=center]
          at (3.5,0) (model) {\textbf{Model}};}
      \node[white,fill=mittelblau,circle,minimum size=25mm,align=center] at (7,0) (simulation)
      {\textbf{Simulation}};
      \node[white,fill=hellblau,circle,minimum size=25mm,align=center] at (7,-3.5) (reality)
      {\textbf{Reality}};
      \visible<4->{\draw[->,line width=1mm,anthrazit,>=stealth] (parameters) -- (model);}
      \visible<3->{\draw[->,line width=1mm,anthrazit,>=stealth] (model) -- (simulation);}
      \visible<2->{\draw[<->,dots,line width=1mm,anthrazit,>=stealth] (simulation)
          to[bend left=45] (reality);}
      \visible<6->{
          \draw[->,dots,line width=1mm,anthrazit,>=stealth] (reality)
              to[bend left=45] (parameters);
          \path[postaction={textalongpath={20mm}{%
              |\usebeamercolor[fg]{structure}\bfseries|Inverse problem},
              /pgf/decoration/raise=-5mm}] (parameters)
              to[bend right=45] (reality);
      }
      \visible<5->{\node at (3.5,1.5) {\term{Forward problem}};}
      \visible<2->{\node at (7.2,-1.75) {\term{Matches?}};}
  \end{tikzpicture}
\end{frame}

\begin{frame}{Another Frame}
    At the top, you can see the progress throughout the presentation.
    Each circle (round shapes -- yay!) stands for one frame.
    
    Also, there is a progress bar beneath.
\end{frame}



\sectionframe{Now the Last Section}

\section{Section 3}

\begin{frame}{Second to Last Frame}
    Now we're in Section 3.

    Here is an \texttt{enumerate} list:
    \begin{enumerate}
        \item
        First item

        \item
        Second item

        \item
        Last item
    \end{enumerate}
    Again note the round shapes\dots
\end{frame}

\begin{frame}{Last Frame}
    This is the final frame.
\end{frame}



\thanksframe
