\documentclass{beamer}
%Information to be included in the title page:
\title{Piecewise Linear Function}
\subtitle{with discontinuity at $x = 0$}
\author{Fabian Weik}
\institute{Overleaf}
\date{\today}

\usepackage{listings}

\graphicspath{{../../Simulation/Models/00_Examples/05_Homework_0124}}

\usepackage{forloop}
\newcounter{n}
\newcounter{mu}
\makeatletter
\newcommand{\twodigits}{\two@digits}
\makeatother

\begin{document}

\frame{\titlepage}

\begin{frame}[fragile]{Code}
    \begin{lstlisting}
real_t x = currentState[0];
real_t y = 0;

if (x < 0) {
    y = aL * x + mu;
} else {
    y = aR * x + mu;
}

RHS[0] = y;
return true;
    \end{lstlisting}
    
    Executed using AnT
    
\end{frame}

\begin{frame}{$a_L = -1, a_R = 2$}
    \begin{figure}
        \centering
        \includegraphics[width=0.6\textwidth]{1D_i/result.png}
        \caption{Varying $\mu$}
    \end{figure}
\end{frame}

\begin{frame}{$a_L = -1, a_R = 2$}
    \begin{figure}
        \centering
            \forloop{n}{0}{\value{n} < 5}{
                \includegraphics<\arabic{n}>[width=0.6\textwidth]{
                    Cob_i/Autogen/Frame_00\twodigits{\value{n}}/result.png
                }
            }
        \caption{Cobwebs $\mu \in \{0.01, 0, -0.01, -0.02, -0.03\}$}
    \end{figure}
\end{frame}

\end{document}
