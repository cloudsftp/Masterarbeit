\sectionframe{Original Model}

\section{Original Model}

\begin{frame}{Original Model Problem Domain}
    Power Converter DC $\to$ AC.
    Types:
    \begin{itemize}
        \item fixed frequency (stroboscopic mapping, continuous)
        \item variable frequency (first return mapping, discontinuous)
              \hspace{2em}
              \raisebox{-.7em}{\includegraphics[height=2em]{gfx/you_are_here.png}}
    \end{itemize}

    \pause
    \vspace{2em}
    Model maps
    \begin{itemize}
        \item Input: Point at which the converter switched the last time $\theta \in [0, 2 \pi)$
        \item Ouput: Point at which the converter will switch the next time $F(\theta) \in [0, 2 \pi)$
    \end{itemize}
\end{frame}

\begin{frame}{Original Model Definition (1/3)}
    \vspace{-3.0em}
    \begin{align}
        \theta \mapsto F(\theta) & = \begin{cases}
                                         F_1(\theta) & \text{if } q \cdot \cos(\theta) > 0 \\
                                         F_2(\theta) & \text{if } q \cdot \cos(\theta) < 0
                                     \end{cases}
        \\
        F_1(\theta)              & = \begin{cases}
                                         \theta + z_{L_+} + z_1 & \text{if } z_{L_+} < z_{L_0} \\
                                         \theta + z_{L_0} + z_2 & \text{if } z_{L_+} > z_{L_0}
                                     \end{cases}
        \\
        F_1(\theta)              & = \begin{cases}
                                         \theta + z_{R_+} + z_3 & \text{if } z_{R_+} < z_{R_0} \\
                                         \theta + z_{R_0} + z_4 & \text{if } z_{R_+} > z_{R_0}
                                     \end{cases}
    \end{align}

    \pause
    \vspace{2em}
    This looks ok, but how are these values defined?
    \begin{align*}
        q, z_1, z_2, z_3, z_4, z_{L_+}, z_{L_-}, z_{R_+}, \text{ and } z_{R_0}
    \end{align*}
\end{frame}

\begin{frame}{Original Model Definition (2/3)}
    The following implicit equations define these values!
    \begin{subequations}
        \begin{align}
            (q \cdot \cos(\theta) + \mu \cdot \chi) \cdot e^{\lambda \cdot z_{L_+}}
             & = q \cdot \cos(\theta + z_{L_+} + z_1) + \mu \cdot \chi \\
            (q \cdot \cos(\theta) + \mu \cdot \chi) \cdot e^{\lambda \cdot z_{L_0}}
             & = q \cdot \cos(\theta + z_{L_0} + z_1) - \mu \cdot \chi \\
            (q \cdot \cos(\theta) + \mu \cdot \chi) \cdot e^{\lambda \cdot z_{R_+}}
             & = q \cdot \cos(\theta + z_{R_+} + z_1) + \mu \cdot \chi \\
            (q \cdot \cos(\theta) + \mu \cdot \chi) \cdot e^{\lambda \cdot z_{R_0}}
             & = q \cdot \cos(\theta + z_{R_0} + z_1) - \mu \cdot \chi
        \end{align}
    \end{subequations}
    \vspace{-2em}
    \begin{subequations}
        \begin{align}
            (q \cdot \cos(\theta + z_{L_+}) + \chi + 1) \cdot e^{\lambda \cdot z_1} - 1
             & = q \cdot  \cos(\theta + z_{L_+} + z_1) + \mu \cdot \chi \\
            (q \cdot \cos(\theta + z_{L_0}) + \chi + 1) \cdot e^{\lambda \cdot z_2} + 1
             & = q \cdot  \cos(\theta + z_{L_0} + z_2) - \mu \cdot \chi \\
            (q \cdot \cos(\theta + z_{R_+}) + \chi + 1) \cdot e^{\lambda \cdot z_3} - 1
             & = q \cdot  \cos(\theta + z_{L_+} + z_3) + \mu \cdot \chi \\
            (q \cdot \cos(\theta + z_{R_0}) + \chi + 1) \cdot e^{\lambda \cdot z_4} + 1
             & = q \cdot  \cos(\theta + z_{R_0} + z_4) - \mu \cdot \chi
        \end{align}
    \end{subequations}
\end{frame}

\begin{frame}{Original Model Definition (3/3)}
    \vspace{-3.0em}
    \begin{align}
        \chi    & = \dfrac{R \cdot \chi_0}{\beta \cdot E_0} \\
        \lambda & = \dfrac{-R}{L \cdot 2 \cdot \pi \cdot f} \\
        q       & = \dfrac{R \cdot V_m}{\beta \cdot E_0}
    \end{align}

    Parameters varied in the 2D scan above:
    \begin{align*}
        E_0, \chi_0
    \end{align*}

    Symmetry in this model:
    \begin{align}
        F(\theta + \pi) = F(\theta) + \pi \mod 2 \pi
    \end{align}
\end{frame}

\begin{frame}{Original Model}
    \begin{figure}
        \centering
        \includegraphics[height=0.7 \textheight]{99_Yunus/2D_Period_Zoomed/result.png}
        \caption{2D scan showing periods of cycles of the original model}
    \end{figure}
\end{frame}

\begin{frame}{Original Model}
    \begin{columns}[t]
        \begin{column}{.9 \textwidth}
            \vspace{-2em}
            \begin{center}
                \begin{figure}
                    \centering
                    \subfloat[$A$: Period 16, $\Cycle{\A^3\B^3\C^3\D^3}$]{
                        \includegraphics[width=0.5 \textheight]{99_Yunus/Period12/Cobweb_A_12/result.png}
                    }
                    \subfloat[$B$: Period 16, $\Cycle{\A^3\B^3\C^2\D^4}$ and $\Cycle{\A^2\B^4\C^3\D^3}$]{
                        \includegraphics[width=0.5 \textheight]{99_Yunus/Period12/Cobweb_B_12/result.png}
                    }
                    \subfloat[$C$: Period 16, $\Cycle{\A^2\B^4\C^2\D^4}$]{
                        \includegraphics[width=0.5 \textheight]{99_Yunus/Period12/Cobweb_C_12/result.png}
                    }
                    \caption{Cobwebs at selected parameter values of the original model}
                \end{figure}
            \end{center}
        \end{column}
        \begin{column}{.2 \textwidth}
            \vspace{-4em}
            \begin{center}
                \hspace{-2em}
                \includegraphics[height=0.3 \textheight]{99_Yunus/2D_Period_Zoomed/result.png}
            \end{center}
        \end{column}
    \end{columns}
\end{frame}

\begin{frame}{Parameter Regions in Original Model}
    \begin{itemize}
        \item ``Type A'': At points $A$ and $B$ in 2D scan
              \begin{itemize}
                  \item One cycle
                  \item Symmetrical
                  \item Example at point $A$: $\Cycle{\A^3\B^3\C^3\D^3}$
                  \item Example at point $C$: $\Cycle{\A^2\B^4\C^2\D^4}$ \vspace*{1em}
              \end{itemize}
        \item ``Type B'': At point $B$ in 2D scan
              \begin{itemize}
                  \item Two coexisting cycles
                  \item Asymmetrical
                  \item Example at point $B$: $\Cycle{\A^3\B^3\C^2\D^4}$ and $\Cycle{\A^2\B^4\C^3\D^3}$
              \end{itemize}
    \end{itemize}
\end{frame}


%\subfloat[Halved model]{
%    \includegraphics[height=0.6 \textheight]{98_Yunus_modpi/2D_Period_Zoomed/result.png}

%}

\begin{frame}{Overlaping Parameter Regions in Original Model}
    \begin{figure}
        \centering
        \includegraphics[height=0.7 \textheight]{98_Yunus_modpi/2D_Regions_Zoomed2/result.png}
        \caption{2D Scan of Period Regions of Original Model}
    \end{figure}
\end{frame}

\begin{frame}{Research Questions}
    \begin{itemize}
        \item Can this bifurcation scenario be described by a simpler, more general model?
        \item Was there something missing in the original analysis of this model?
    \end{itemize}
\end{frame}
