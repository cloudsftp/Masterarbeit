\sectionframe{What is Period-adding?}
\section{What is Period-adding?}


\begin{frame}{Period-adding}
	\todo{example with graphs: bifurcation, period \\ from L08 p15}
	\todo{chain example from simpson thesis}
\end{frame}

\begin{frame}{Farey Sequences}
	\begin{definition}
		$\F^m$
	\end{definition}
	\todo{Definition of Farey Sequences $\F^m$}
	\begin{theorem}
		$\dfrac{a_1}{b_1}$
	\end{theorem}
	\todo{Theorem of Farey Addition}
	\todo{overlay: tree}
\end{frame}

\begin{frame}{Farey-trees with Symbolic Sequences}
	\begin{itemize}
		\item Keep the structure of the tree
		\item Replace starting nodes with symbolic sequences
		\item Use concatenation instead of Farey-addition $\oplus$
	\end{itemize}
	\todo{overlay: tree with L and R symbolic sequences}
\end{frame}

\begin{frame}{Rotation Numbers}
	\begin{definition}
		$\dfrac{|\sigma|_L}{|\sigma|}$
	\end{definition}
	\todo{define rotation numbers for systems w 2 symbols}
	\begin{theorem}
		Concatenation of two cycles is Farey-addition of their rotation numbers
	\end{theorem}
	\todo{complete theorem}
	So we can replace the symbolic sequences in the last tree with their rotation numbers and will get a valid Farey-tree.
	\todo{overlay: tree}
\end{frame}

\begin{frame}
	This concatenation of the symbolic sequences associated with the parent nodes is precisely what happens in period-adding structures
	\todo{example from slides}

	Hence the periods add
\end{frame}