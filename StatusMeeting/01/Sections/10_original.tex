\sectionframe{Original Model}

\section{Original Model}

\begin{frame}{Original Model Problem Domain}
	Power Converter DC $\to$ AC.
	Types:
	\begin{itemize}
		\item fixed frequency (stroboscopic mapping, continuous)
		\item variable frequency (first return mapping, discontinuous)
		      \hspace{2em}
		      \raisebox{-.7em}{\includegraphics[height=2em]{gfx/you_are_here.png}}
	\end{itemize}

	\pause
	\vspace{2em}
	Model maps
	\begin{itemize}
		\item Input: Point at which the converter switched the last time $\theta \in [0, 2 \pi)$
		\item Ouput: Point at which the converter will switch the next time $F(\theta) \in [0, 2 \pi)$
	\end{itemize}
\end{frame}

\begin{frame}{Original Model (1/3)}
	\vspace{-2.0em}
	\begin{align}
		\theta      & \mapsto  F(\theta) \mod 2 \pi
		\\
		F(\theta)   & = \begin{cases}
			                F_1(\theta) & \text{if } q \cdot \cos(\theta) > 0 \\
			                F_2(\theta) & \text{if } q \cdot \cos(\theta) < 0
		                \end{cases}
		\\
		F_1(\theta) & = \begin{cases}
			                \theta + z_{L_+} + z_1 & \text{if } z_{L_+} < z_{L_0} \\
			                \theta + z_{L_0} + z_2 & \text{if } z_{L_+} > z_{L_0}
		                \end{cases}
		\\
		F_2(\theta) & = \begin{cases}
			                \theta + z_{R_+} + z_3 & \text{if } z_{R_+} < z_{R_0} \\
			                \theta + z_{R_0} + z_4 & \text{if } z_{R_+} > z_{R_0}
		                \end{cases}
	\end{align}

	\pause
	\vspace{2em}
	This looks ok, but how are these values defined?
	\begin{align*}
		z_1, z_2, z_3, z_4, z_{L_+}, z_{L_-}, z_{R_+}, \text{ and } z_{R_0}
	\end{align*}
\end{frame}

\begin{frame}{Original Model (2/3)}
	\vspace{-1em}
	The smallest non-negative solutions to the following implicit equations
	\begin{subequations}
		\begin{align}
			(q \cdot \cos(\theta) + \mu \cdot \chi) \cdot e^{\lambda \cdot z_{L_+}}
			 & = q \cdot \cos(\theta + z_{L_+}) + \chi \\
			(q \cdot \cos(\theta) + \mu \cdot \chi) \cdot e^{\lambda \cdot z_{L_0}}
			 & = q \cdot \cos(\theta + z_{L_0}) - \chi \\
			(q \cdot \cos(\theta) - \mu \cdot \chi) \cdot e^{\lambda \cdot z_{R_+}}
			 & = q \cdot \cos(\theta + z_{R_+}) - \chi \\
			(q \cdot \cos(\theta) - \mu \cdot \chi) \cdot e^{\lambda \cdot z_{R_0}}
			 & = q \cdot \cos(\theta + z_{R_0}) + \chi
		\end{align}
	\end{subequations}
	\vspace{-2em}
	\begin{subequations}
		\begin{align}
			(q \cdot \cos(\theta + z_{L_+}) + \chi + 1) \cdot e^{\lambda \cdot z_1} - 1
			 & = q \cdot  \cos(\theta + z_{L_+} + z_1) + \mu \cdot \chi \\
			(q \cdot \cos(\theta + z_{L_0} + z_2) - \chi - 1) \cdot e^{\lambda \cdot z_2} + 1
			 & = q \cdot  \cos(\theta + z_{L_0} + z_2) - \mu \cdot \chi \\
			(q \cdot \cos(\theta + z_{R_+}) + \chi + 1) \cdot e^{\lambda \cdot z_3} - 1
			 & = q \cdot  \cos(\theta + z_{L_+} + z_1) + \mu \cdot \chi \\
			(q \cdot \cos(\theta + z_{R_0} + z_4) - \chi - 1) \cdot e^{\lambda \cdot z_4} + 1
			 & = q \cdot  \cos(\theta + z_{R_0} + z_2) - \mu \cdot \chi
		\end{align}
	\end{subequations}
\end{frame}

\begin{frame}{Original Model (3/3)}
	\vspace{-3.0em}
	\begin{align}
		\chi    & = \dfrac{R \cdot \chi_0}{\beta \cdot E_0} \\
		\lambda & = \dfrac{-R}{L \cdot 2 \cdot \pi \cdot f} \\
		q       & = \dfrac{R \cdot V_m}{\beta \cdot E_0}
	\end{align}

	Normalized and varied Parameters:
	\begin{align*}
		E_0, \chi_0
	\end{align*}

	Symmetry in this model:
	\begin{align}
		F(\theta + \pi) = F(\theta) + \pi \mod 2 \pi
	\end{align}

	\begin{flushright}
		Definition and symmetry from \cite{akyuz2022}
	\end{flushright}
\end{frame}

\begin{frame}{Original Model}
	\begin{figure}
		\centering
		\includegraphics[height=0.7 \textheight]{99_Yunus/2D_Period_Zoomed/result.png}
		\caption*{2D scan showing periods of cycles of the original model}
	\end{figure}
\end{frame}

\begin{frame}{Original Model}
	\begin{columns}
		\begin{column}{.9 \textwidth}
			\vspace{-2em}
			\begin{center}
				\begin{figure}
					\centering
					\subfloat[$A$: Period 12, $\Cycle{\A^3\B^3\C^3\D^3}$]{
						\includegraphics[width=0.5 \textheight]{99_Yunus/Period12/Cobweb_A_12/result.png}
					}
					\subfloat[$B$: Period 12, $\Cycle{\A^3\B^3\C^2\D^4}$ and $\Cycle{\A^2\B^4\C^3\D^3}$]{
						\includegraphics[width=0.5 \textheight]{99_Yunus/Period12/Cobweb_B_12/result.png}
					}
					\subfloat[$C$: Period 12, $\Cycle{\A^2\B^4\C^2\D^4}$]{
						\includegraphics[width=0.5 \textheight]{99_Yunus/Period12/Cobweb_C_12/result.png}
					}
					\caption*{Cobwebs at selected parameter values of the original model}
				\end{figure}
			\end{center}
		\end{column}
		\begin{column}{.2 \textwidth}
			\vspace{-4em}
			\begin{center}
				\hspace{-2em}
				\includegraphics[height=0.3 \textheight]{99_Yunus/2D_Period_Zoomed/result.png}
			\end{center}
		\end{column}
	\end{columns}
\end{frame}

\begin{frame}{Parameter Regions in Original Model}
	\vspace{-1em}
	\begin{columns}
		\begin{column}{0.7 \textwidth}
			Chains of parameter regions with the same period but different symbolic sequences.
			Alternating between two types:

			\begin{itemize}
				\item ``Type A'': At points $A$ and $B$ in 2D scan
				      \begin{itemize}
					      \item One cycle
					      \item Symmetrical
					      \item Example at point $A$: $\Cycle{\A^3\B^3\C^3\D^3}$
					      \item Example at point $C$: $\Cycle{\A^2\B^4\C^2\D^4}$ \vspace*{1em}
				      \end{itemize}
				\item ``Type B'': At point $B$ in 2D scan
				      \begin{itemize}
					      \item Two coexisting cycles
					      \item Asymmetrical
					      \item Example at point $B$: $\Cycle{\A^3\B^3\C^2\D^4}$ and $\Cycle{\A^2\B^4\C^3\D^3}$
				      \end{itemize}
			\end{itemize}

			\begin{flushright}
				Observations taken from \cite{akyuz2022}
			\end{flushright}
		\end{column}
		\begin{column}{0.3 \textwidth}
			\only<1>{
				\begin{figure}
					\centering
					\includegraphics[height=0.5 \textheight]{99_Yunus/2D_Period_Zoomed/Manual_Chain16/result.png}
					\caption*{2D scan of chain with period 12 in the original model}
				\end{figure}
			}
			\only<2>{
				\begin{tikzpicture}
					\node (A5) at (0, 0) {$\A^5\B^1\C^5\D^1$};

					\node (B54) at (-1, -1) {$\A^5\B^1\C^4\D^2$};
					\node (B45) at (1, -1) {$\A^4\B^2\C^5\D^1$};

					\node (A4) at (0, -2) {$\A^4\B^2\C^4\D^2$};

					\node (B43) at (-1, -3) {$\A^4\B^2\C^3\D^3$};
					\node (B34) at (1, -3) {$\A^3\B^3\C^4\D^2$};

					\node (A3) at (0, -4) {$\A^3\B^3\C^3\D^3$};

					\node (B32) at (-1, -5) {\dots};
					\node (B23) at (1, -5) {\dots};

					\graph {
					(A5)
					-> {(B54), (B45)}
					-> (A4)
					-> {(B43), (B34)}
					-> (A3)
					-> {(B32), (B23)}
					};
				\end{tikzpicture}
			}
		\end{column}
	\end{columns}
\end{frame}


%\subfloat[Halved model]{
%    \includegraphics[height=0.6 \textheight]{98_Yunus_modpi/2D_Period_Zoomed/result.png}

%}

\begin{frame}{Overlaping Parameter Regions in Original Model}
	\begin{figure}
		\centering
		\includegraphics[height=0.7 \textheight]{98_Yunus_modpi/2D_Regions_Zoomed2/result.png}
		\caption*{2D Scan of period regions of original model}
	\end{figure}
\end{frame}

\begin{frame}{Research Questions}
	\begin{itemize}
		\item Can this bifurcation scenario be reproduced by a simpler, more general model?
		\item Was there something overlooked in the original analysis of this model?
		\item What else can happen in models that are similar to the original model?
	\end{itemize}
\end{frame}
