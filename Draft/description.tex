\documentclass[12pt]{article}

\newif\ifFigs       \Figsfalse
\Figstrue


\usepackage{amssymb,amsmath}
\usepackage{fbox}
\renewcommand{\P}{{\mathcal P}}
\renewcommand{\L}{{\mathcal L}}
\newcommand{\A}{{\mathcal A}}
\newcommand{\B}{{\mathcal B}}
\newcommand{\C}{{\mathcal C}}
\newcommand{\D}{{\mathcal D}}
\newcommand{\K}{{\mathcal K}}
\newcommand{\R}{{\mathcal R}}
\newcommand{\N}{{\mathcal N}}
\newcommand{\M}{{\mathcal M}}
\newcommand{\I}{{\mathcal I}}
\newcommand{\J}{{\mathcal J}}
\newcommand{\Q}{{\mathcal Q}}
\newcommand{\E}{{\mathcal E}}
\newcommand{\U}{{\mathcal U}}
\renewcommand{\l}{\text{\tiny $\mathcal L$}}
\renewcommand{\r}{\text{\tiny $\mathcal R$}}
\newcommand{\tR}{\tau_\r}
\newcommand{\tL}{\tau_\l}
\newcommand{\dR}{\delta_\r}
\newcommand{\dL}{\delta_\l}
\newcommand{\LC}{{\mathcal O}}

\usepackage{graphicx}
\usepackage{xcolor}
\usepackage{paralist}
\usepackage{mathtools}
\renewcommand{\leq}{\leqslant}
\renewcommand{\geq}{\geqslant}

\setlength{\parindent}{0mm}
\setlength{\textwidth}{160mm}
\setlength{\textheight}{240mm}
\setlength{\topmargin}{-19mm}
\setlength{\oddsidemargin}{0mm}

\newcommand{\Includegraphics}[2]%
           {\centering
             \ifFigs\includegraphics[width=#1]{#2}%
             \else  \includegraphics[draft,width=#1]{#2}\fi}

\newcommand{\Includesubgraphics}[3]%
{\begin{minipage}[b]{#1}
    \Includegraphics{\textwidth}{#2}\\
    \centerline{{\footnotesize (#3)}}
  \end{minipage}
}


\begin{document}

\section{Bifurcation structure in the archetypal model}

\begin{enumerate}
  \item It follows from the definition of the function $f(x)$ that if a
        point $x$ belongs to an $m$-cycle, then the point $x+\frac{1}{2}$
        belongs to an $m$-cycle as well. Therefore, there are only two
        possible cases:
        \begin{itemize}
          \item[(A)] The points $x$ and $x+\frac{1}{2}$ belong to the same
            cycle.  Then, this cycle has necessarily an even period with the
            same number and the same location of the points in the intervals
            $I_\L = I_\A \cup I_\B$ and $I_\R = I_\C \cup I_\D$.  In the
            following, such cycles are referred to as symmetric cycles or
            cycles of type A.
          \item[(B)] The points $x$ and $x+\frac{1}{2}$ belong to different cycles.
            Then, the map has necessarily two coexisting cycles
            of the same period, referred to in the following as asymmetric cycles
            or cycles of type B.
        \end{itemize}
  \item
        The considered bifurcation structure of map~\eqref{} consists of regions
        associated with stable cycles of even periods only.  For the case A,
        it follows from the symmetry. For the case B, it does not.
  \item
        The cycles of period $2m$ forming the considered bifurcation
        structure are associated with symbolic sequences
        $\A^{m_\A}\:\B^{m_\B}\:\C^{m_\C}\:\D^{m_\D}$ with
        $m_{\A,\B,\C,D}>0$, (i.e., they have points in all partitions)
        $m_\A+m_\B+m_\C+m_\D=2m$ and $|m_\A-m_\C|\leq 1$, $|m_\B-m_\D|\leq
          1$.

  \item
        For each period $2m$, the bifurcation structure of map~\eqref{}
        contains a chain of pairwise overlapping periodicity regions of
        cycles with the same period $2m$ but differing in the corresponding
        symbolic sequences. Each chain start with the region associated with
        the symmetric cycle such that among $m$ of its points located in
        each of the intervals $I_\L$ and $I_\R$, all but one are located in
        $I_\A$ and $I_\D$, respectively. Then, the chain is built up by
        consecutive movement of these points, one by one, from these
        intervals to the adjacent intervals $I_\B$ and $I_\D$. Graphically,
        the beginning of each chain can be illustrated by the following
        scheme:
        \begin{equation}
          \fbox{
            \begin{minipage}[c][3.5em][c]{7em}\small\centering
              $\A^{m-1}\:\B\:\C^{m-1}\:\D$
            \end{minipage}
          }
          \hspace*{-1ex}
          \fbox{
            \begin{minipage}[c][2.5em][c]{7em}\small\centering
              $\A^{m-2}\:\B^2\:\C^{m-1}\:\D$\\
              $\A^{m-1}\:\B  \:\C^{m-2}\:\D^2$
            \end{minipage}
          }
          \hspace*{-1ex}
          \fbox{
            \begin{minipage}[c][3.5em][c]{7em}\small\centering
              $\A^{m-2}\:\B^2\:\C^{m-2}\:\D^2$
            \end{minipage}
          }
          \hspace*{-1ex}
          \fbox{
            \begin{minipage}[c][2.5em][c]{7em}\small\centering
              $\A^{m-3}\:\B^3\:\C^{m-2}\:\D^2$\\
              $\A^{m-2}\:\B^2\:\C^{m-3}\:\D^3$
            \end{minipage}
          }
          \hspace*{-1ex}
          \fbox[LTB]{
            \begin{minipage}[c][3.5em][c]{1ex}\small
              \dots
            \end{minipage}
          }
        \end{equation}
        and a generic step in this scheme can be specified as follows
        \begin{equation}
          \fbox[RTB]{
            \begin{minipage}[c][2.5em][c]{1ex}\small
              \hspace*{-.7em}\dots
            \end{minipage}
          }
          \hspace*{-1ex}
          \fbox{
            \begin{minipage}[c][3.5em][c]{7em}\small\centering
              $\A^{m-k}\:\B^k\:\C^{m-k}\:\D^k$
            \end{minipage}
          }
          \hspace*{-1ex}
          \fbox{
            \begin{minipage}[c][2.5em][c]{10em}\small\centering
              $\A^{m-(k+1)}\:\B^{k+1}\:\C^{m-k}    \:\D^{k}$\\
              $\A^{m-k}   \:\B^{k}  \:\C^{m-(k+1)}\:\D^{k+1}$
            \end{minipage}
          }
          \hspace*{-1ex}
          \fbox[LTB]{
            \begin{minipage}[c][3.5em][c]{1ex}\small
              \dots
            \end{minipage}
          }
        \end{equation}
        with $0<k<m$ (provided, the chain is not truncated).
  \item
        Each region in the chain is confined by four border collision
        bifurcation curves. Due to the symmetry, each border collision bifurcation
        in map~\eqref{} is a double border collision: either a
        border is collided by two points of the same (symmetric) cycle,
        or it is collided by two points of two cycles belonging to a pair
        of coexisting asymmetric cycles. More specifically,
        \begin{itemize}
          \item[(A)]
            Taking into account, that the points of a type A cycle
            $\LC_{\A^{m-k}\B^k\C^{m-k}\D^k}$ are
            located with respect to the border points as follows
            \begin{equation}
              \left.
              \underbrace{x_0 \dots x_{m-k-1}}_{I_\A}
              \right|_{d_1}\hspace*{-1em}
              \left.
              \underbrace{x_{m-k} \dots x_{m-1}}_{I_\B}
              \right|_{d_2}\hspace*{-1em}
              \left.
              \underbrace{x_{m} \dots x_{2m-k-1}}_{I_\C}
              \right|_{d_3}\hspace*{-1em}
              \left.
              \underbrace{x_{2m-k} \dots x_{2m-1}}_{I_\D}
              \right|_{d_0}
            \end{equation}
            we conclude that the  border collision bifurcations
            this cycle can undergo
            are given by
            \begin{align}
              \xi^{\A^{m-k}\B^k\C^{m-k}\D^k}_{d_0, d_2} & =
              \{ p \mid
              x_0^{\A^{m-k}\B^k\C^{m-k}\D^k} = d_0
              \:\:\:\text{and}\:\:\:
              x_{m}^{\A^{m-k}\B^k\C^{m-k}\D^k} = d_2
              \}                                            \\
              %
              \xi^{\A^{m-k}\B^k\C^{m-k}\D^k}_{d_1, d_0} & =
              \{ p \mid
              x_{m-k}^{\A^{m-k-1}\B^k\C^{m-k}\D^k} = d_1
              \:\:\:\text{and}\:\:\:
              x_{2m-k-1}^{\A^{m-k}\B^k\C^{m-k}\D^k} = d_0
              \}                                            \\
              %
              \xi^{\A^{m-k}\B^k\C^{m-k}\D^k}_{d_1, d_3} & =
              \{ p \mid
              x_{m-k}^{\A^{m-k}\B^k\C^{m-k}\D^k} = d_1
              \:\:\:\text{and}\:\:\:
              x_{2m-k}^{\A^{m-k}\B^k\C^{m-k}\D^k} = d_3
              \}                                            \\
              %
              \xi^{\A^{m-k}\B^k\C^{m-k}\D^k}_{d_2, d_0} & =
              \{ p \mid
              x_{m-1}^{\A^{m-k}\B^k\C^{m-k}\D^k} = d_2
              \:\:\:\text{and}\:\:\:
              x_{2m-1}^{\A^{m-k}\B^k\C^{m-k}\D^k} = d_0
              \}
            \end{align}
            Note that in the notation for the border collision bifurcations
            the upper index refers to the cycle undergoing the bifurcation
            and the lower one to the discontinuities the cycle collides with.
          \item[(B)]
            Similarly, since the points of the coexisting cycles
            $\LC_{\A^{m-k-1}\B^{k+1}\C^{m-k}\D^k}$
            and
            $\LC_{\A^{m-k}\B^k\C^{m-k-1}\D^{k+1}}$
            of type B are located with respect to the border points as
            follows
            \begin{align}
               &
              \left.
              \underbrace{x_0 \dots x_{m-k-2}}_{I_\A}
              \right|_{d_1}\hspace*{-1em}
              \left.
              \underbrace{x_{m-k-1} \dots x_{m-1}}_{I_\B}
              \right|_{d_2}\hspace*{-1em}
              \left.
              \underbrace{x_{m} \dots x_{2m-k-1}}_{I_\C}
              \right|_{d_3}\hspace*{-1em}
              \left.
              \underbrace{x_{2m-k\phantom{-1}} \dots x_{2m-1}}_{I_\D}
              \right|_{d_0} \\
              %
               &
              \left.
              \underbrace{x_0 \dots x_{m-k-1}}_{I_\A}
              \right|_{d_1}\hspace*{-1em}
              \left.
              \underbrace{x_{m-k\phantom{-1}} \dots x_{m-1}}_{I_\B}
              \right|_{d_2}\hspace*{-1em}
              \left.
              \underbrace{x_{m} \dots x_{2m-k-2}}_{I_\C}
              \right|_{d_3}\hspace*{-1em}
              \left.
              \underbrace{x_{2m-k-1} \dots x_{2m-1}}_{I_\D}
              \right|_{d_0}
            \end{align}
            the  border collision bifurcations
            confining the periodicity regions of these cycles are
            \begin{align}
              \xi^{\A^{m-k-1}\B^{k+1}\C^{m-k}\D^k}_{d_0} & =
              \{ p \mid
              x_0^{\A^{m-k-1}\B^{k+1}\C^{m-k}\D^k} = d_0
              \}                                             \\
              \xi^{\A^{m-k}\B^k\C^{m-k-1}\D^{k+1}}_{d_2} & =
              \{ p \mid
              x_{m}^{\A^{m-k}\B^k\C^{m-k-1}\D^{k+1}} = d_2
              \}
              \\
              \xi^{\A^{m-k-1}\B^{k+1}\C^{m-k}\D^k}_{d_2} & =
              \{ p \mid
              x_{m-1}^{\A^{m-k-1}\B^{k+1}\C^{m-k}\D^k} = d_2
              \}                                             \\
              \xi^{\A^{m-k}\B^k\C^{m-k-1}\D^{k+1}}_{d_0} & =
              \{ p \mid
              x_{2m-1}^{\A^{m-k}\B^k\C^{m-k-1}\D^{k+1}} = d_0
              \}
              \\
              \xi^{\A^{m-k-1}\B^{k+1}\C^{m-k}\D^k}_{d_3} & =
              \{ p \mid
              x_{2m-k-1}^{\A^{m-k-1}\B^{k+1}\C^{m-k}\D^k} = d_3
              \}                                             \\
              \xi^{\A^{m-k}\B^k\C^{m-k-1}\D^{k+1}}_{d_1} & =
              \{ p \mid
              x_{m-k-1}^{\A^{m-k}\B^k\C^{m-k-1}\D^{k+1}} = d_1
              \}
              \\
              \xi^{\A^{m-k-1}\B^{k+1}\C^{m-k}\D^k}_{d_3} & =
              \{ p \mid
              x_{2m-k}^{\A^{m-k-1}\B^{k+1}\C^{m-k}\D^k} = d_3
              \}                                             \\
              \xi^{\A^{m-k}\B^k\C^{m-k-1}\D^{k+1}}_{d_1} & =
              \{ p \mid
              x_{m-k}^{\A^{m-k}\B^k\C^{m-k-1}\D^{k+1}} = d_1
              \}
            \end{align} \\
            As already mentioned, it follows from the symmetry that
            \begin{align}
              \xi^{\A^{m-k-1}\B^{k+1}\C^{m-k}\D^k}_{d_0} & \equiv
              \xi^{\A^{m-k}\B^k\C^{m-k-1}\D^{k+1}}_{d_2}          \\
              \xi^{\A^{m-k-1}\B^{k+1}\C^{m-k}\D^k}_{d_2} & \equiv
              \xi^{\A^{m-k}\B^k\C^{m-k-1}\D^{k+1}}_{d_0}          \\
              \xi^{\A^{m-k-1}\B^{k+1}\C^{m-k}\D^k}_{d_3} & \equiv
              \xi^{\A^{m-k}\B^k\C^{m-k-1}\D^{k+1}}_{d_1}          \\
              \xi^{\A^{m-k-1}\B^{k+1}\C^{m-k}\D^k}_{d_3} & \equiv
              \xi^{\A^{m-k}\B^k\C^{m-k-1}\D^{k+1}}_{d_1}
            \end{align}
            \\
            Kommentar: Ich habe hier nur die Bifurkationen aufgeschrieben, die vorkommen.
            Die Glechungen sind geordnet: 10, 11: $\rightarrow$, 12, 13: $\leftarrow$, 14, 15: $\uparrow$, 16, 17: $\downarrow$.
        \end{itemize}
        \clearpage
  \item
        Each two adjacent chains (associated with cycles of periods $m$ and
        $m+2$) overlap as well, as illustrated by the following diagram
        showing the changes of regions associated with cycles of period $2m$,
        surrounded from the top and from the bottom with the chains
        associated with cycles of periods $2(m-1)$, and $2(m+1)$,
        respectively:
        \begin{equation}
          \begin{split}
            &
            \fbox[RB]{
              \begin{minipage}[c][4.5em][c]{3ex}\small
                \hspace*{-.7em}\dots
              \end{minipage}
            }\hspace*{-1ex}
            \fbox[LBR]{
              \begin{minipage}[c][3.5em][c]{11em}\small\centering
                $\A^{m-k-2}\:\B^{k+1}\:\C^{m-1-k}    \:\D^{k}$\\
                $\A^{m-1-k}   \:\B^{k}  \:\C^{m-k-2}\:\D^{k+1}$
              \end{minipage}
            }\hspace*{-1ex}
            \fbox[LBR]{
              \begin{minipage}[c][4.5em][c]{12em}\small\centering
                $\A^{m-k-1}\:\B^k\:\C^{m-k-1}\:\D^k$
              \end{minipage}
            }
            \hspace*{-1ex}
            \fbox[LB]{
              \begin{minipage}[c][3.5em][c]{1ex}\small
                \dots
              \end{minipage}
            }\\[-1.5em]
            %%
            &
            \fbox[RTB]{
              \begin{minipage}[c][3.5em][c]{1ex}\small
                \hspace*{-.7em}\dots
              \end{minipage}
            }
            \hspace*{-1ex}
            \fbox{
              \begin{minipage}[c][4.5em][c]{13em}\small\centering
                $\A^{m-k}\:\B^k\:\C^{m-k}\:\D^k$
              \end{minipage}
            }
            \hspace*{-1ex}
            \fbox{
              \begin{minipage}[c][3.5em][c]{10em}\small\centering
                $\A^{m-(k+1)}\:\B^{k+1}\:\C^{m-k}    \:\D^{k}$\\
                $\A^{m-k}   \:\B^{k}  \:\C^{m-(k+1)}\:\D^{k+1}$
              \end{minipage}
            }
            \hspace*{-1ex}
            \fbox[LTB]{
              \begin{minipage}[c][4.5em][c]{4ex}\small
                \dots
              \end{minipage}
            }\\[-1.5em]
            %%
            &
            \fbox[RT]{
              \begin{minipage}[c][4.5em][c]{3ex}\small
                \hspace*{-.7em}\dots
              \end{minipage}
            }\hspace*{-1ex}
            \fbox[RTL]{
              \begin{minipage}[c][3.5em][c]{11em}\small\centering
                $\A^{m-k}\:\B^{k+1}\:\C^{m-k+1}    \:\D^{k}$\\
                $\A^{m-k+1}   \:\B^{k}  \:\C^{m-k}\:\D^{k+1}$
              \end{minipage}
            }\hspace*{-1ex}
            \fbox[RTL]{
              \begin{minipage}[c][4.5em][c]{12em}\small\centering
                $\A^{m-k}\:\B^{k+1}\:\C^{m-k}\:\D^{k+1}$
              \end{minipage}
            }
            \hspace*{-1ex}
            \fbox[LT]{
              \begin{minipage}[c][3.5em][c]{1ex}\small
                \dots
              \end{minipage}
            }
          \end{split}
        \end{equation}
        Accordingly, depending on actual parameter values, we may observe
        either a single cycle or two, three, or even four coexisting cycles.
        Specifically, let us denote the periodicity region of the cycle
        $\LC_{\A^{m-k}\:\B^k\:\C^{m-k}\:\D^k}$ by $P^m_k$ and the
        periodicity region of the pair of cycles
        $\LC_{\A^{m-(k+1)}\:\B^{k+1}\:\C^{m-k} \:\D^{k}}$ $\LC_{\A^{m-k}
            \:\B^{k} \:\C^{m-(k+1)}\:\D^{k+1}}$ by $Q^m_k$. In these terms,
        the above diagram becomes
        \begin{equation}
          \begin{split}
            &
            \fbox[RB]{
              \begin{minipage}[c][4em][c]{3em}\small
                \hspace*{-.7em}
                $P^{m-2}_k$
              \end{minipage}
            }\hspace*{-1ex}
            \fbox[LBR]{
              \begin{minipage}[c][3em][c]{3em}\small\centering
                $Q^{m-2}_k$
              \end{minipage}
            }\hspace*{-1ex}
            \fbox[LBR]{
              \begin{minipage}[c][4em][c]{5em}\small\centering
                $P^{m-2}_{k+1}$
              \end{minipage}
            }
            \hspace*{-1ex}
            \fbox[LBR]{
              \begin{minipage}[c][3em][c]{3em}\small
                $Q^{m-2}_{k+1}$
              \end{minipage}
            }
            \hspace*{-1ex}
            \fbox[LB]{
              \begin{minipage}[c][4em][c]{2em}\small
                \dots
              \end{minipage}
            }\\[-1.5em]
            %%
            &
            \fbox[RTB]{
              \begin{minipage}[c][3em][c]{2em}\small
                \hspace*{-.7em}
                $Q^{m}_{k-1}$
              \end{minipage}
            }
            \hspace*{-1ex}
            \fbox{
              \begin{minipage}[c][4em][c]{5em}\small\centering
                $P^{m}_k$
              \end{minipage}
            }
            \hspace*{-1ex}
            \fbox{
              \begin{minipage}[c][3em][c]{3em}\small\centering
                $Q^{m}_{k}$
              \end{minipage}
            }
            \hspace*{-1ex}
            \fbox{
              \begin{minipage}[c][4em][c]{5em}\small\centering
                $P^{m}_{k+1}$
              \end{minipage}
            }
            \hspace*{-1ex}
            \fbox[LBT]{
              \begin{minipage}[c][3em][c]{1em}\small
                \dots
              \end{minipage}
            }\\[-1.5em]
            %%
            &
            \fbox[RT]{
              \begin{minipage}[c][4em][c]{3em}\small
                \hspace*{-.7em}
                $P^{m+2}_{k}$
              \end{minipage}
            }\hspace*{-1ex}
            \fbox[RTL]{
              \begin{minipage}[c][3em][c]{3em}\small\centering
                $Q^{m+2}_{k}$
              \end{minipage}
            }\hspace*{-1ex}
            \fbox[RTL]{
              \begin{minipage}[c][4em][c]{5em}\small\centering
                $P^{m+2}_{k+1}$
              \end{minipage}
            }\hspace*{-1ex}
            \fbox[RTL]{
              \begin{minipage}[c][3em][c]{3em}\small\centering
                $Q^{m+2}_{k+1}$
              \end{minipage}
            }
            \hspace*{-1ex}
            \fbox[LT]{
              \begin{minipage}[c][4em][c]{2em}\small
                \dots
              \end{minipage}
            }
          \end{split}
        \end{equation}
        Then, the regions of the considered bifurcations structure where
        map~\eqref{} has a single (glaobally attracting) stable cycle
        are giveb by
        \begin{equation}
          P^{m}_k \setminus \left(
          Q^{m}_{k} \cup
          Q^{m}_{k-1} \cup
          Q^{m-2}_{k} \cup
          Q^{m+2}_{k} \cup
          P^{m-2}_{k} \cup
          P^{m-2}_{k+1} \cup
          P^{m+2}_{k} \cup
          P^{m+2}_{k+1}
          \right)
        \end{equation}
\end{enumerate}

\end{document}
