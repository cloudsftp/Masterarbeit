\documentclass[10pt]{article}

\newif\ifFigs       \Figsfalse
\Figstrue


\usepackage{amssymb,amsmath}
\usepackage{fbox}
\usepackage{paralist}
\usepackage{graphpap}
\renewcommand{\P}{{\mathcal P}}
\renewcommand{\L}{{\mathcal L}}
\newcommand{\A}{{\mathcal A}}
\newcommand{\B}{{\mathcal B}}
\newcommand{\C}{{\mathcal C}}
\newcommand{\D}{{\mathcal D}}
\newcommand{\K}{{\mathcal K}}
\newcommand{\R}{{\mathcal R}}
\newcommand{\N}{{\mathcal N}}
\newcommand{\M}{{\mathcal M}}
\newcommand{\I}{{\mathcal I}}
\newcommand{\J}{{\mathcal J}}
\newcommand{\Q}{{\mathcal Q}}
\newcommand{\E}{{\mathcal E}}
\newcommand{\U}{{\mathcal U}}
\renewcommand{\l}{\text{\tiny $\mathcal L$}}
\renewcommand{\r}{\text{\tiny $\mathcal R$}}
\newcommand{\tR}{\tau_\r}
\newcommand{\tL}{\tau_\l}
\newcommand{\dR}{\delta_\r}
\newcommand{\dL}{\delta_\l}
\newcommand{\LC}{{\mathcal O}}


\newcommand{\fA}{f_\A}
\newcommand{\fB}{f_\B}
\newcommand{\fC}{f_\C}
\newcommand{\fD}{f_\D}

\newcommand{\FA}{F_\A}
\newcommand{\FB}{F_\B}
\newcommand{\FC}{F_\C}
\newcommand{\FD}{F_\D}

\newcommand{\gR}{g_\r}
\newcommand{\gL}{g_\l}

\newcommand{\aR}{a_\r}
\newcommand{\aL}{a_\l}

\newcommand{\bR}{b_\r}
\newcommand{\bL}{b_\l}

\newcommand{\cR}{c_\r}
\newcommand{\cL}{c_\l}


\usepackage{graphicx}
\usepackage{xcolor}
\usepackage{paralist}
\usepackage{mathtools}
\renewcommand{\leq}{\leqslant}
\renewcommand{\geq}{\geqslant}

\setlength{\parindent}{0mm}
\setlength{\textwidth}{160mm}
\setlength{\textheight}{240mm}
\setlength{\topmargin}{-19mm}
\setlength{\oddsidemargin}{0mm}

\newcommand{\Includegraphics}[2]%
           {\centering
             \ifFigs\includegraphics[width=#1]{#2}%
             \else  \includegraphics[draft,width=#1]{#2}\fi}

\newcommand{\Includesubgraphics}[3]%
{\begin{minipage}[b]{#1}
    \Includegraphics{\textwidth}{#2}\\
    \centerline{{\footnotesize (#3)}}
  \end{minipage}
}


\begin{document}

\tableofcontents

\clearpage

\section{Introduction}
\section{Investigated system}
\subsection{Circuit}
\subsection{Model in continuous time}
\subsection{Model in discrete time}
\begin{equation}
	\label{eq:map:orig}
	\theta_{n+1} = F (\theta_n) \bmod 2\pi
\end{equation}
\subsection{Bifurcation structure}

\clearpage
\section{Archetypal model}
\subsection{Idea of the approach}


The original model is suitable for numerical investigation of the
systems dynamics. However, it has a major drawback common for
many similar models defined by implicit equations. In fact,
this model can be seen as a kind black box, providing no information
on how the system dynamics may change under parameter variation.

Since exactly this question is pivotal for the bifurcation
analysis, we propose the following approach for investigation of the
bifurcation structure exhibited by the original model:
\begin{itemize}
	\item
	      First, we identify the characteristic properties of the
	      original model at the parameter values leading to the
	      considered bifurcation structure.
	\item
	      Then, construct a simple model possessing these properties
	\item
	      Next, we investigate the constructed models and provide an
	      explanation of the considered bifurcation structure.
	\item
	      Eventually, we validate to which extent the predictions made
	      based on the simplified model apply to the original one.
\end{itemize}

\subsection{Properties of the original map}

Proceeding according to the approach described above, let us first
identify the properties the map~\eqref{eq:map:orig}
which we assume to be relevant for the formation of the bifurcation
structure the map
exhibits in the
considered parameter domain. As illustrated in Fig.~\ref{}:
\begin{itemize}
	\item The function $F$ governing the dynamics of
	      map~\eqref{eq:map:orig} is discontinuous and consists, in the phase
	      interval $[2, 2\pi)$, of four continuous branches denoted in the
	      following by $\FA$, $\FB$, $\FC$ and $\FD$.
	      The domains in the state space associated with these branches
	      are referred to as $I_\A$, $I_\B$, $I_\C$ and $I_\D$, respectively,
	      with  $I_\A \cup I_\B \cup I_\C \cup I_\D = [2, 2\pi)$.
	      The appearance of
	      discontinuities of $F$ is explained by the effect described
	      in~\cite{} and related to the tendencies of the flow of the underlying
	      model in continuous time to a boundary of the hysteresis zone.  In
	      the following, the discontinuities separating the
	      $I_\A$, $I_\B$, $I_\C$ and $I_\D$ from each other
	      a denoted by $d_0$, $d_1$, $d_2$, $d_3$, where $d_0$ is the boundary
	      between $I_\D$ and $I_\A$,  $d_1$ is the boundary
	      between $I_\A$ and $I_\B$, and so on.
	\item
	      The function $F$ fulfills the symmetry
	      \begin{equation}
		      \label{eq:symmetry}
		      F(x+\pi)=F(x)+\pi
	      \end{equation}
	      Therefore, the shapes of the branches $\FA$, $\FC$,
	      as well as  $\FB$, $\FD$ are identical, and the distance
	      between the discontinuities $d_0$, $d_2$, as well as between
	      $d_1$, $d_3$ is equal $\pi$.
	\item
	      Branches of $F$ are non-monotonous.  Note that the appearance of
	      local minima and maxima of $F$ is described in~\cite{} and --
	      similar to the points of discontinuities -- is related to
	      tangencies of the flow in the underlying continuous time model and
	      a boundary of the hysteresis zone.  The shape of the branches
	      $\FA$ and $\FC$ is similar to a quadratic polynomial, while
	      shape of the branches $\FB$ and $\FD$ is similar to a cubic one.
\end{itemize}

In order to reproduce the bifurcation structure observable in
map~\eqref{eq:map:orig}, the archetypal model has to replicate not
only the shape of the function $F$ but also how this shape depends on
parameters.  In map~\eqref{eq:map:orig}, the considered bifurcation
structure appears under variation of parameters $E_0$ and $\chi_0$.
Numerical experiments show that that
\begin{itemize}
	\item the parameter $\chi_0$ influences mainly the branches $\FA$ and
	      $\FC$. For increasing values of this parameter,
	      the vertical offset of these branches increases but the overall
	      shape of the branches remain approximately unchanged;
	\item the parameter $E_0$ influences mainly the branches $\FB$ and
	      $\FD$. The kind of influence is different than in the previous
	      case: for increasing $E_0$, the values $\FB(d_1)$ and $\FD(d_3)$
	      increase, while the values $\FB(d_2)$ and $\FD(d_0)$ remain
	      approximately unchanged;
	\item additionally, both parameters $E_0$ and $\chi_0$ influence the location
	      of the discontinuity points of $F$.
\end{itemize}

\dotfill

It follows from the definition of the function $f(x)$ (specifically,
from Eq.~\eqref{eq:f:def:pre}) that if a point $x$ belongs to an
$m$-cycle of map~\eqref{eq:map}, then the point $x+\frac{1}{2}$
belongs to a cycle as well. Therefore, only the following two
cases are possible:
\begin{itemize}
	\item[(A)] The points $x$ and $x+\frac{1}{2}$ belong to the same
		cycle.  Then, this cycle has necessarily an even period with the
		same number and the same location of the points in the intervals
		$I_\A \cup I_\B$ and $I_\C \cup I_\D$.  In the
		following, such cycles are referred to as symmetric cycles or
		cycles of type A.
	\item[(B)] The points $x$ and $x+\frac{1}{2}$ belong to different cycles.
		Then, the map has necessarily at least two coexisting cycles
		of the same period, referred to in the following as asymmetric cycles
		or cycles of type B.
\end{itemize}

\subsection{\label{sec:map}Construction of the archetypal model}
Let us now define a map with the properties similar to the
ones described above.
First, we scale the phase space of the model to the interval $[0,1)$, i.e,
\begin{subequations}
	\label{eq:map}
	\begin{align}
		x_{n+1}                                                       & = f(x_n) \bmod 1.                                                     \\
		\intertext{Then, we fix the location of the discontinuity points to be}
		\label{eq:disc:def}
		d_0                                                           & = 0, \quad
		d_1 = \tfrac{1}{4}, \quad
		d_2 = \tfrac{1}{2}, \quad
		d_3 = \tfrac{3}{4},
		%
		\intertext{Indeed, numerical experiments evidence that the
			dependency of the discontinuity points on system parameters is
			not significant for the resulting bifurcation structure.  To
			achieve the symmetry similar to the one defined by
			Eq.~\eqref{eq:symmetry}, we require}
		%
		\label{eq:f:def:pre}
		f(x)                                                          & = \left\{\!\!
		\begin{array}{ll}
			g(x)          & \text{if} \:\: x \in I_\l = \left[d_0, d_2\right), \\[.5ex]
			g(x-d_2) +d_2 & \text{if} \:\: x \in I_\r = \left[d_2, 1\right).
		\end{array} \right.
		\intertext{On the partition $I_\l$, the function $g$ is defined by
			two nonlinear branches:}
		\label{eq:g:def}
		g(x)                                                          & = \left\{\!\!
		\begin{array}{ll}
			\gL(x) & \text{if} \:\: x \in I_\A = \left[d_0, d_1\right), \\[.5ex]
			\gR(x) & \text{if} \:\: x \in I_\B = \left[d_1, d_2\right).
		\end{array} \right.
		%
		\intertext{In this way, the function $f$ is defined by four branches $\fA$, $\fB$, $\fC$, and $\fD$
			corresponding to the branches $\FA$, $\FB$, $\FC$, and $\FD$ of the original map~\eqref{eq:map:orig}:}
		\label{eq:f:def}
		f(x)                                                          & = \left\{\!\!
		\begin{array}{lll}
			\fA(x) \!\!\!\! & = \gL(x)          & \text{if} \:\: x \in I_\A = \left[d_0, d_1\right), \\[.5ex]
			\fB(x) \!\!\!\! & = \gR(x)          & \text{if} \:\: x \in I_\B = \left[d_1, d_2\right), \\[.5ex]
			\fC(x) \!\!\!\! & = \gL(x-d_2) +d_2 & \text{if} \:\: x \in I_\C = \left[d_2, d_3\right), \\[.5ex]
			\fD(x) \!\!\!\! & = \gR(x-d_2) +d_2 & \text{if} \:\: x \in I_\D = \left[d_3, 1\right).
		\end{array} \right.
		\intertext{A suitable definition of the functions $\gL$, $\gR$ is
			a key issue in the development of the archetypal model. Here, an
			obvious choice is to define them by polynomial functions.  As
			for the function $\gL$, it turns out that its non-monotonicity
			is essential for the appearance of the considered bifurcation
		structure, so we define it by a quadratic polynomial:} \gL(x) & =
		\aL x^2 + \bL x + \cL
		%
		\intertext{Here, a vertical displacement of the branches $f_\A$,
			$f_\C$ mimicking the shift of the branches $f_\A$, $f_\C$ in the
			original map~\eqref{eq:map:orig} under variation of the
			parameter $\chi_0$ is achieved by variation of the offset
			$\cL$.\newline As for the definition of the function $\gR$,
			numerical experiments show that it is not necessary to define it
			by a cubic polynomial. It turns out that the essential features
			of the bifurcation structure exhibited by the original map can
			be reproduced by using a linear function instead. The property
			of map~\eqref{eq:map:orig} we have to maintain here is that the
			variation of parameters influences the value of the function at
			the left boundary of its domain and does not change the value on
			the right one. Accordingly, for the values $d_1$, $d_2$ given by
			Eq.~\eqref{eq:disc:def}, we define the function $\gR$ as
			follows}
		%
		\gR(x)                                                        & = \bR x + \cR\qquad\text{with}\:\: \bR = 4 (\gR(d_2) - \gR(d_1)),\:\:
		\cR = 2\gR(d_1) - \gR(d_2)
		%
	\end{align}
\end{subequations}
Thus, the
influence of the parameter $E_0$ on the branches $\FB$ and $\FD$ in
the original map is mimicked by fixing the value $\gR(d_2)$ while the
value $\gR(d_1)$ is varied, so that both the slope $\bR$ and the offset
$\cR$ are changed simultaneously.


In order to maintain the shape of the function $f$ similar to the one
of the function $F$ in the relevant parameter region, the parameters
$\aL$, $\bL$, $\cL$, $\bR$, and $\cR$ have to fulfill some fulfill
some conditions. To this end we require the function $\gL$ to be
located above the diagonal (i.e., $\gL(x)>x$ for all $x\in [0,
		\frac{1}{4}]$), to have a minimum in its application domain (i.e.,
$\aL>0$ and $x=-\frac{\bL}{2\aL} \in (0, \frac{1}{4})$), and to map
the orbits into the domain of the function $\gR$ (i.e.,
$\gL(0)<\frac{1}{2}$, $\gL(\frac{1}{4})<\frac{1}{2}$).  As for the
function $\gR$, it has to be contracting and increasing (i.e., $0 <
	\bR < 1$), and to have no fixed points (i.e.,
$\gR(\frac{1}{2})>\frac{1}{2}$).  Under these conditions, the shapes of
the functions $f$ are sufficiently close to each other and we can
expect the map~\eqref{eq:map} to exhibit dynamics similar to
map~\eqref{eq:map:orig}.


Varied parameters: $\alpha=-\gR(\frac{1}{2})$, $\beta=\cR$

\clearpage
\section{Bifurcation structure in the archetypal model}

\begin{figure}[t]
	%\Includesubgraphics{.48\textwidth}{figs/scanMode2/overall/periods2/bif.png}{a}
	%\Includesubgraphics{.48\textwidth}{figs/scanMode2/overall/symbolic/bif.png}{b}
	\caption{\label{fig:2D:overall}Overall bifurcation structure in
		map~\eqref{eq:map}.  In (a), periods of cycles in the
		corresponding regions are indicated, some chains of regions
		associated with cycles of the same period are shown in different
		colors.  In (b), the symbolic descriptions of the regions are
		specified, the chain of regions associated with $16$-cycles is
		highlighted. Rectangels marked in (a) with A, B are shown in
		Figs.~\ref{fig:2D:blowup:symbolic}, \ref{fig:2D:blowup:bifs}, and
		\ref{fig:1D:P16:bifs}(a),(b), respectively.
	}
\end{figure}




An example of the 2D bifurcation structure in map~\eqref{eq:map} is
shown in Fig.~\ref{fig:2D:overall}(a).  As one can see, the structure is
formed by partially overlapping regions associated with various
cycles.  It is worth noting that all cycles belonging to this
structure have even periods. Recall that symmetric cycles of
map~\eqref{eq:map} are necessarily of even periods, while the pairs of
coexisting asymmetric cycles may have odd periods as well.  However,
such cycles do not appear within the considered bifurcation structure.

The cycles of period $2m$, $m\geq 2$, forming the considered
bifurcation structure have points in all four partitions, and, more
specifically, are associated with symbolic sequences
$\A^{m_\A}\:\B^{m_\B}\:\C^{m_\C}\:\D^{m_\D}$ with $m_{\A,\B,\C,D}>0$,
$m_\A+m_\B+m_\C+m_\D=2m$.  This follows from the restrictions for the
parameter of function $f$ mentioned in Sec.~\ref{sec:map} which
guarantee that from each partition the orbits are mapped into the next
one.  Moreover, the number of points in the corresponding partitions
on the left and the right halves of the state space (i.e., in $I_\A$
and $I_\C$, as well as in $I_\B$ and $I_\D$) differs at most by one,
i.e., $|m_\A-m_\C|\leq 1$, $|m_\B-m_\D|\leq 1$ (see
Fig.~\ref{fig:2D:overall}(b))

As one can clearly see in Fig.~\ref{fig:2D:overall}(b), for each
period $2m$, the bifurcation structure of map~\eqref{eq:map} contains
a chain of pairwise overlapping periodicity regions of cycles with the
same period $2m$ but differing in the corresponding symbolic
sequences. Each chain start with the region associated with the
symmetric cycle $\LC_{\A^{m-1}\:\B\:\C^{m-1}\:\D}$ such that among $m$
of its points located in each of the partitions $I_\L$ and $I_\R$, all
but one are located in $I_\A$ and $I_\C$, respectively. Then, the
chain is built up by consecutive movement of these points, one by one,
from these intervals to the adjacent intervals $I_\B$ and
$I_\D$. Graphically, the beginning of each chain can be illustrated by
the following scheme:
\begin{equation}
	\fbox{
		\begin{minipage}[c][3.5em][c]{7em}\small\centering
			$\A^{m-1}\:\B\:\C^{m-1}\:\D$
		\end{minipage}
	}
	\hspace*{-1ex}
	\fbox{
		\begin{minipage}[c][2.5em][c]{7em}\small\centering
			$\A^{m-2}\:\B^2\:\C^{m-1}\:\D$\\
			$\A^{m-1}\:\B  \:\C^{m-2}\:\D^2$
		\end{minipage}
	}
	\hspace*{-1ex}
	\fbox{
		\begin{minipage}[c][3.5em][c]{7em}\small\centering
			$\A^{m-2}\:\B^2\:\C^{m-2}\:\D^2$
		\end{minipage}
	}
	\hspace*{-1ex}
	\fbox{
		\begin{minipage}[c][2.5em][c]{7em}\small\centering
			$\A^{m-3}\:\B^3\:\C^{m-2}\:\D^2$\\
			$\A^{m-2}\:\B^2\:\C^{m-3}\:\D^3$
		\end{minipage}
	}
	\hspace*{-1ex}
	\fbox[LTB]{
		\begin{minipage}[c][3.5em][c]{1ex}\small
			\dots
		\end{minipage}
	}
\end{equation}
and a generic step in this scheme can be specified as follows
\begin{equation}
	\fbox[RTB]{
		\begin{minipage}[c][2.5em][c]{1ex}\small
			\hspace*{-.7em}\dots
		\end{minipage}
	}
	\hspace*{-1ex}
	\fbox{
		\begin{minipage}[c][3.5em][c]{7em}\small\centering
			$\A^{m-k}\:\B^k\:\C^{m-k}\:\D^k$
		\end{minipage}
	}
	\hspace*{-1ex}
	\fbox{
		\begin{minipage}[c][2.5em][c]{10em}\small\centering
			$\A^{m-(k+1)}\:\B^{k+1}\:\C^{m-k}    \:\D^{k}$\\
			$\A^{m-k}   \:\B^{k}  \:\C^{m-(k+1)}\:\D^{k+1}$
		\end{minipage}
	}
	\hspace*{-1ex}
	\fbox[LTB]{
		\begin{minipage}[c][3.5em][c]{1ex}\small
			\dots
		\end{minipage}
	}
\end{equation}
with $0<k<m$ (provided, the chain is not truncated).


\begin{figure}[t]
	%\Includesubgraphics{.48\textwidth}{figs/scanMode2/blowup2/periods/bif.png}{a}
	%\Includesubgraphics{.48\textwidth}{figs/scanMode2/blowup1/periods/bif.png}{b}
	\caption{\label{fig:2D:blowup:symbolic}Regions associated (a) with
		the symmetric $16$-cycle $\LC_{\A^4\B^4\C^4\D^4}$ and (b) with the
		pair of asymmetric $16$-cycle $\LC_{\A^5\B^3\C^4\D^4}$
		$\LC_{\A^4\B^4\C^5\D^3}$. The chain or regions corresponding to
		period 16 is highlighted. Rectangle marked in (a) is shown
		magnified in (b).
	}
\end{figure}

As an example, Fig.~\ref{fig:2D:blowup:symbolic} shows some of the
regions forming the chain corresponding to period 16, specifically,
the regions associated with a symmetric $16$-cycle
$\LC_{\A^4\B^4\C^4\D^4}$ (see Fig.~\ref{fig:2D:blowup:symbolic}(a))
and with a pair of asymmetric $16$-cycle $\LC_{\A^5\B^3\C^4\D^4}$
$\LC_{\A^4\B^4\C^5\D^3}$ (Fig.~\ref{fig:2D:blowup:symbolic}(b)).  It
is clearly visible that the regions overlap pairwise and that the
number of points located in the partitions $I_\A$, $I_\B$ decreases
along a chain, while the number of points in the partitions $I_\B$,
$I_\D$ increases.

%\Figstrue

\begin{figure}[t]
	%\Includesubgraphics{.48\textwidth}{figs/scanMode2/blowup2/symbolic/bif.png}{a}
	%\Includesubgraphics{.48\textwidth}{figs/scanMode2/blowup1/symbolic/bif.png}{b}
	\caption{\label{fig:2D:blowup:bifs}Border collision bifurcations
		confining the regions (a) $\P_{\A^4\B^4\C^4\D^4}$ and (b)
		$\P_{\A^5\B^3\C^4\D^4}\equiv\P_{\A^4\B^4\C^5\D^3}$. The chain of
		regions corresponding to period 16 is highlighted. Rectangle
		marked in (a) is shown magnified in (b). Parameter paths marked
		with A and B in (b) correspond to bifurcation diagrams shown in
		Fig.~\ref{fig:1D:P16:bifs}(a),(b), respectively.}
\end{figure}


Each region in a chain has four boundaries corresponding to four
border collision bifurcation curves. It is worth noting that due to
the symmetry of the map, each border collision bifurcation in
map~\eqref{eq:map} is a double border collision: either a border is
collided by two points of the same symmetric cycle, or it is collided
by two points of two coexisting asymmetric cycles.

Taking into account, that the points of a symmetric $2m$-cycle
$\LC_{\A^{m-k}\B^k\C^{m-k}\D^k}$ are located with respect to the
border points as follows
\begin{equation}
	\label{eq:cycle:points:symmetric}
	\raisebox{-1.2ex}{$\bigg|_{d_0}$\hspace*{-1em}}
	\underbrace{x_0 \dots x_{m-k-1}}_{I_\A}
	\raisebox{-1.2ex}{$\bigg|_{d_1}$\hspace*{-1em}}
	\underbrace{x_{m-k} \dots x_{m-1}}_{I_\B}
	\raisebox{-1.2ex}{$\bigg|_{d_2}$\hspace*{-1em}}
	\underbrace{x_{m} \dots x_{2m-k-1}}_{I_\C}
	\raisebox{-1.2ex}{$\bigg|_{d_3}$\hspace*{-1em}}
	\underbrace{x_{2m-k} \dots x_{2m-1}}_{I_\D}
	\raisebox{-1.2ex}{$\bigg|_{d_0}$}
\end{equation}
we conclude that the border collision bifurcations this cycle can
undergo are given by
\begin{align}
	\label{eq:BCB:symmetric}
	\xi^{\underline{\A}^{m-k}\B^k\underline{\C}^{m-k}\D^k}_{d_0, d_2} & =
	\{ (\alpha,\beta) \mid
	x_0^{\A^{m-k}\B^k\C^{m-k}\D^k} = d_0
	\:\:\:\text{and}\:\:\:
	x_{m}^{\A^{m-k}\B^k\C^{m-k}\D^k} = d_2
	\}                                                                    \\
	%
	\xi^{\underline{\A}^{m-k}\B^k\underline{\C}^{m-k}\D^k}_{d_1, d_3} & =
	\{ (\alpha,\beta) \mid
	x_{m-k-1}^{\A^{m-k-1}\B^k\C^{m-k}\D^k} = d_1
	\:\:\:\text{and}\:\:\:
	x_{2m-k-1}^{\A^{m-k}\B^k\C^{m-k}\D^k} = d_3
	\}                                                                    \\
	%
	\xi^{\A^{m-k}\underline{\B}^k\C^{m-k}\underline{\D}^k}_{d_1, d_3} & =
	\{ (\alpha,\beta) \mid
	x_{m-k}^{\A^{m-k}\B^k\C^{m-k}\D^k} = d_1
	\:\:\:\text{and}\:\:\:
	x_{2m-k}^{\A^{m-k}\B^k\C^{m-k}\D^k} = d_3
	\}                                                                    \\
	%
	\xi^{\A^{m-k}\underline{\B}^k\C^{m-k}\underline{\D}^k}_{d_2, d_0} & =
	\{ (\alpha,\beta) \mid
	x_{m-1}^{\A^{m-k}\B^k\C^{m-k}\D^k} = d_2
	\:\:\:\text{and}\:\:\:
	x_{2m-1}^{\A^{m-k}\B^k\C^{m-k}\D^k} = d_0
	\}
\end{align}
Note that in the notation for the border collision bifurcations the
lower index refers to the discontinuities the cycle collides with and
the upper one to the cycle undergoing the bifurcation.  The
underliened letters indicate which points of the cycle are colliding
with the discontinuities: for example, the underliened letter $\A$
implies by Eq.~\eqref{eq:cycle:points:symmetric} that either the point
$x^{\A^{m-k}\B^k\C^{m-k}\D^k}_0$ collides with the discontinuity $d_0$
or the point $x^{\A^{m-k}\B^k\C^{m-k}\D^k}_{m-k-1}$ with $d_1$.


For the $16$-cycle $\LC_{\A^5\B^3\C^5\D^3}$, the four border collision
bifurcations given by Eq.~\eqref{eq:BCB:symmetric} are shown
in Fig.~\ref{fig:2D:blowup:bifs}(a). Note that
at the corner points of the region  $\P_{\A^5\B^3\C^5\D^3}$
the cycle undergoes codimension-2 border collision bifurcations
at which it collides with all four discontinuities simultaneously.



\medskip
Similar results can be obtained for a pair of coexisting asymmetric
$2m$-cycles $\LC_{\A^{m-k-1}\B^{k+1}\C^{m-k}\D^k}$ and
$\LC_{\A^{m-k}\B^k\C^{m-k-1}\D^{k+1}}$. One can easily see that their
points are located with respect to the border points as follows
\begin{align}
	 &
	\raisebox{-1.2ex}{$\bigg|_{d_0}$\hspace*{-1em}}
	\underbrace{x_0 \dots x_{m-k-2}}_{I_\A}
	\raisebox{-1.2ex}{$\bigg|_{d_1}$\hspace*{-1em}}
	\underbrace{x_{m-k-1} \dots x_{m-1}}_{I_\B}
	\raisebox{-1.2ex}{$\bigg|_{d_2}$\hspace*{-1em}}
	\underbrace{x_{m} \dots x_{2m-k-1}}_{I_\C}
	\raisebox{-1.2ex}{$\bigg|_{d_3}$\hspace*{-1em}}
	\underbrace{x_{2m-k\phantom{-1}} \dots x_{2m-1}}_{I_\D}
	\raisebox{-1.2ex}{$\bigg|_{d_0}$} \\
	%
	 &
	\raisebox{-1.2ex}{$\bigg|_{d_0}$\hspace*{-1em}}
	\underbrace{x_0 \dots x_{m-k-1}}_{I_\A}
	\raisebox{-1.2ex}{$\bigg|_{d_1}$\hspace*{-1em}}
	\underbrace{x_{m-k\phantom{-1}} \dots x_{m-1}}_{I_\B}
	\raisebox{-1.2ex}{$\bigg|_{d_2}$\hspace*{-1em}}
	\underbrace{x_{m} \dots x_{2m-k-2}}_{I_\C}
	\raisebox{-1.2ex}{$\bigg|_{d_3}$\hspace*{-1em}}
	\underbrace{x_{2m-k-1} \dots x_{2m-1}}_{I_\D}
	\raisebox{-1.2ex}{$\bigg|_{d_0}$}
\end{align}
Therefore, the  border collision bifurcations
confining the periodicity regions of these cycles are
\begin{align}
	\xi^{\underline{\A}^{m-k-1}\B^{k+1}\C^{m-k}\D^k}_{d_0} & =
	\{ (\alpha,\beta) \mid
	x_0^{\A^{m-k-1}\B^{k+1}\C^{m-k}\D^k} = d_0
	\}                                                         \\
	\xi^{\A^{m-k-1}\underline{\B}^{k+1}\C^{m-k}\D^k}_{d_2} & =
	\{ (\alpha,\beta) \mid
	x_{m-1}^{\A^{m-k-1}\B^{k+1}\C^{m-k}\D^k} = d_2
	\}                                                         \\
	\xi^{\A^{m-k-1}\B^{k+1}\underline{\C}^{m-k}\D^k}_{d_3} & =
	\{ (\alpha,\beta) \mid
	x_{2m-k-1}^{\A^{m-k-1}\B^{k+1}\C^{m-k}\D^k} = d_3
	\}                                                         \\
	\xi^{\A^{m-k-1}\B^{k+1}\C^{m-k}\underline{\D}^k}_{d_3} & =
	\{ (\alpha,\beta) \mid
	x_{2m-k}^{\A^{m-k-1}\B^{k+1}\C^{m-k}\D^k} = d_3
	\}                                                         \\
	%
	\xi^{\underline{\A}^{m-k}\B^k\C^{m-k-1}\D^{k+1}}_{d_1} & =
	\{ (\alpha,\beta) \mid
	x_{m-k-1}^{\A^{m-k}\B^k\C^{m-k-1}\D^{k+1}} = d_1
	\}                                                         \\
	\xi^{\A^{m-k}\underline{\B}^k\C^{m-k-1}\D^{k+1}}_{d_1} & =
	\{ (\alpha,\beta) \mid
	x_{m-k}^{\A^{m-k}\B^k\C^{m-k-1}\D^{k+1}} = d_1
	\}                                                         \\
	\xi^{\A^{m-k}\B^k\underline{\C}^{m-k-1}\D^{k+1}}_{d_2} & =
	\{ (\alpha,\beta) \mid
	x_{m}^{\A^{m-k}\B^k\C^{m-k-1}\D^{k+1}} = d_2
	\}                                                         \\
	\xi^{\A^{m-k}\B^k\C^{m-k-1}\underline{\D}^{k+1}}_{d_0} & =
	\{ (\alpha,\beta) \mid
	x_{2m-1}^{\A^{m-k}\B^k\C^{m-k-1}\D^{k+1}} = d_0
	\}
\end{align}
As already mentioned,
the existence regions $\P_{\A^{m-k}\B^k\C^{m-k-1}\D^{k+1}}$
and $\P_{\A^{m-k-1}\B^{k+1}\C^{m-k}\D^k}$
are identical and their borders coincide as follows
\begin{align}
	\xi^{\A^{m-k-1}\B^{k+1}\C^{m-k}\underline{\D}^k}_{d_0} & \equiv
	\xi^{\A^{m-k}\underline{\B}^k\C^{m-k-1}\D^{k+1}}_{d_2}          \\
	\xi^{\underline{\A}^{m-k-1}\B^{k+1}\C^{m-k}\D^k}_{d_0} & \equiv
	\xi^{\A^{m-k}\B^k\underline{\C}^{m-k-1}\D^{k+1}}_{d_2}          \\
	\xi^{\underline{\A}^{m-k-1}\B^{k+1}\C^{m-k}\D^k}_{d_1} & \equiv
	\xi^{\A^{m-k}\B^k\underline{\C}^{m-k-1}\D^{k+1}}_{d_3}          \\
	\xi^{\A^{m-k-1}\underline{\B}^{k+1}\C^{m-k}\D^k}_{d_1} & \equiv
	\xi^{\A^{m-k}\B^k\C^{m-k-1}\underline{\D}^{k+1}}_{d_3}
\end{align}
as illustrated in Fig.~\ref{fig:2D:blowup:bifs}(b) for the
region $\P_{\A^{5}\B^3\C^{4}\D^{4}} \equiv \P_{\A^{4}\B^{4}\C^{5}\D^3}$.  Note that at the corner points of this
region, two codimension-2 border collision bifurcations occur.
At these points, similar to the previous example, all four border
points are collided, but here two of them are collided by the points
of the cycle $\LC_{\A^{5}\B^3\C^{4}\D^{4}}$ and the other two by
points of $\LC_{\A^{4}\B^{4}\C^{5}\D^3}$.

\medskip


\begin{figure}[t]
	%\Includesubgraphics{.48\textwidth}{figs/scanMode1//1D_Bif_LFU16/Manual/result.png}{a}
	%\Includesubgraphics{.48\textwidth}{figs/scanMode1//1D_Bif_LFR16/Manual/result.png}{b}
	\caption{\label{fig:1D:P16:bifs}Bifurcation diagrams along the
		parameter paths intersecting the boundaries of the region
		$\P_{\A^{5}\B^3\C^{4}\D^{4}} \equiv \P_{\A^{4}\B^{4}\C^{5}\D^3}$
		as indicated in Fig.~\ref{fig:2D:blowup:bifs}. Parameters: (a)
		$\alpha=-0.375$; (b) $\beta=0.1675$.}
\end{figure}


Examples of bifurcation diagrams across the boundaries of the region
$\P_{\A^{5}\B^3\C^{4}\D^{4}} \equiv \P_{\A^{4}\B^{4}\C^{5}\D^3}$ are
shown in Fig.~\ref{fig:1D:P16:bifs}. As illustrated in
Fig.~\ref{fig:1D:P16:bifs}(a), at the border collision bifurcation
$\xi^{\A^{5}\B^{3}\underline{\C}^{4}\D^4}_{d_3} \equiv
	\xi^{\underline{\A}^{4}\B^4\C^{5}\D^{3}}_{d_1}$ the asymmetric
$16$-cycles $\LC_{\A^{4}\B^4\C^{5}\D^{3}}$ and
$\LC_{\A^{5}\B^{3}\C^{4}\D^4}$ collide with the border points $d_1$
and $d_3$, respectively. Additionally, in the same bifurcation diagram
one can see the border collision bifurcation
$\xi^{\A^{4}\underline{\B}^{3}\C^{4}\underline{\D}^3}_{d_1,d_3}$ of
the symmetric $16$-cycle $\LC_{\A^{4}\B^{4}\C^{4}\D^4}$ which collides
with the same discontinuities $d_1$,$d_3$ from the opposite side.
Similarly, Fig.~\ref{fig:1D:P16:bifs}(a) shows the border collision
bifurcation $\xi^{\underline{\A}^{4}\B^4\C^{5}\D^{3}}_{d_0} \equiv
	\xi^{\A^{5}\B^{3}\underline{\C}^{4}\D^4}_{d_2}$ at which the same pair
of asymmetric $16$-cycles $\LC_{\A^{5}\B^{3}\C^{4}\D^4}$ and
$\LC_{\A^{4}\B^4\C^{5}\D^{3}}$ collide with the border points $d_0$
and $d_2$, respectively. In addition, the border collision bifurcation
$\xi^{\A^{5}\underline{\B}^{4}\C^{5}\underline{\D}^4}_{d_0,d_2}$ of
the symmetric $18$-cycle $\LC_{\A^{5}\B^{4}\C^{5}\D^4}$ collides with
the same discontinuities from the opposite side. It is worth
mentioning that in the considered bifurcation structure the border
collision bifurcation curves do not intersect. However, it is possible
that under variation of some parameters, such an intersection may
occur. It is known~\cite{} that in discontinuous 1D maps with a single
border point and increasing branches, a codimension-2 border collision
bifurcation at which two distinct cycles collide with the
discontinuity from opposite sides acts as an organizing center of a
period adding structure issuing from this point.  This lead us to the
question whether a similar structure can be expected to appear in
map~\eqref{eq:map}. This question is discussed in detail in our
forthcoming work.


\medskip

\begin{figure}[t]
	%\Includesubgraphics{.48\textwidth}{figs/scanMode2/blowup2/numbers/bif.png}{a}
	%\Includesubgraphics{.48\textwidth}{figs/scanMode2/blowup1/numbers/bif.png}{b}
	\caption{\label{fig:2D:blowup:numbers}Number of coexisting
		attracting cycles (indicated by encircled numbers) close to the
		boundaries of the regions (a) $\P_{\A^4\B^4\C^4\D^4}$ and (b)
		$\P_{\A^5\B^3\C^4\D^4}\equiv\P_{\A^4\B^4\C^5\D^3}$. The chain or
		regions corresponding to period 16 is highlighted. Rectangle
		marked in (a) is shown magnified in (b).  Coexisting attractors at
		the parameter values marked with A,B,C and D are shown in
		Fig.~\ref{fig:multistability}(a),(b), respectively.
	}
\end{figure}


It can clearly be seen in Fig.~\ref{fig:2D:overall} that not only
regions forming a chain of regions of the same period overlap, but
also each two adjacent chains overlap as well.  This can be
illustrated by the following diagram showing the changes of regions
associated with cycles of period $2m$, surrounded by the chains
associated with cycles of periods $2(m-1)$, and $2(m+1)$:
\begin{equation*}
	\begin{split}
		2(m-1)\hspace*{3em}
		&
		\fbox[RB]{
			\begin{minipage}[c][4.5em][c]{3ex}\small
				\hspace*{-.7em}\dots
			\end{minipage}
		}\hspace*{-1ex}
		\fbox[LBR]{
			\begin{minipage}[c][3.5em][c]{11em}\small\centering
				$\A^{m-k-2}\:\B^{k+1}\:\C^{m-1-k}    \:\D^{k}$\\
				$\A^{m-1-k}   \:\B^{k}  \:\C^{m-k-2}\:\D^{k+1}$
			\end{minipage}
		}\hspace*{-1ex}
		\fbox[LBR]{
			\begin{minipage}[c][4.5em][c]{12em}\small\centering
				$\A^{m-k-1}\:\B^k\:\C^{m-k-1}\:\D^k$
			\end{minipage}
		}
		\hspace*{-1ex}
		\fbox[LB]{
			\begin{minipage}[c][3.5em][c]{1ex}\small
				\dots
			\end{minipage}
		}\\[-1.5em]
		%%
		2m\hspace*{4.2em}
		&
		\fbox[RTB]{
			\begin{minipage}[c][3.5em][c]{1ex}\small
				\hspace*{-.7em}\dots
			\end{minipage}
		}
		\hspace*{-1ex}
		\fbox{
			\begin{minipage}[c][4.5em][c]{13em}\small\centering
				$\A^{m-k}\:\B^k\:\C^{m-k}\:\D^k$
			\end{minipage}
		}
		\hspace*{-1ex}
		\fbox{
			\begin{minipage}[c][3.5em][c]{10em}\small\centering
				$\A^{m-(k+1)}\:\B^{k+1}\:\C^{m-k}    \:\D^{k}$\\
				$\A^{m-k}   \:\B^{k}  \:\C^{m-(k+1)}\:\D^{k+1}$
			\end{minipage}
		}
		\hspace*{-1ex}
		\fbox[LTB]{
			\begin{minipage}[c][4.5em][c]{4ex}\small
				\dots
			\end{minipage}
		}\\[-1.5em]
		%%
		2(m+1)\hspace*{3em}
		&
		\fbox[RT]{
			\begin{minipage}[c][4.5em][c]{3ex}\small
				\hspace*{-.7em}\dots
			\end{minipage}
		}\hspace*{-1ex}
		\fbox[RTL]{
			\begin{minipage}[c][3.5em][c]{11em}\small\centering
				$\A^{m-k}\:\B^{k+1}\:\C^{m-k+1}    \:\D^{k}$\\
				$\A^{m-k+1}   \:\B^{k}  \:\C^{m-k}\:\D^{k+1}$
			\end{minipage}
		}\hspace*{-1ex}
		\fbox[RTL]{
			\begin{minipage}[c][4.5em][c]{12em}\small\centering
				$\A^{m-k}\:\B^{k+1}\:\C^{m-k}\:\D^{k+1}$
			\end{minipage}
		}
		\hspace*{-1ex}
		\fbox[LT]{
			\begin{minipage}[c][3.5em][c]{1ex}\small
				\dots
			\end{minipage}
		}
	\end{split}
\end{equation*}
As one can see in this diagram, depending on actual parameter values,
map~\eqref{eq:map} may exhibit either a single cycle or two, three, or
four coexisting cycles.  Indeed, each region associated with a pair of
asymmetric $2m$-cycles has a non-empty overlap with a neighboring
region associated with a symmetric cycle.  Specifically, the region
$\P_{\A^{m-(k+1)}\B^{k+1}\C^{m-k}\D^{k}} \equiv P_{\A^{m-k} \B^{k}
			\C^{m-(k+1)}\D^{k+1}}$ overlaps with
\begin{compactenum}[C1:]
	\item the regions $\P_{\A^{m-k}\B^{k}\C^{m-k}\D^{k}}$ or
	$\P_{\A^{m-(k+1)}\B^{k+1}\C^{m-(k-1)}\D^{k+1}}$ belonging to
	the same chain. Accordingly, at the parameter values in these overlaps,
	map~\eqref{eq:map} exhibits three coexisting $2m$ cycles.
	\item
	the regions $\P_{\A^{m-k-1}\B^{k}\C^{m-k-1}\D^{k}}$ or $\P_{\A^{m-k}\B^{k+1}\C^{m-k}\D^{k+1}}$ belonging to
	the neighboring chains. In these overlaps,
	map~\eqref{eq:map} exhibits two coexisting $2m$-cycles
	and one $2(m-1)$- or $2(m+1)$-cycle.
\end{compactenum}
Additionally, close it its corner points,
the region $\P_{\A^{m-(k+1)}\B^{k+1}\C^{m-k}\D^{k}} \equiv P_{\A^{m-k}   \B^{k}  \C^{m-(k+1)}\D^{k+1}}$
overlaps with
\begin{compactenum}[C1:]
	\item[C3:] one of the regions $\P_{\A^{m-k}\B^{k}\C^{m-k}\D^{k}}$ or
	$\P_{\A^{m-(k+1)}\B^{k+1}\C^{m-(k-1)}\D^{k+1}}$ belonging to the
	same chain, and one of the regions
	$\P_{\A^{m-k-1}\B^{k}\C^{m-k-1}\D^{k}}$ or
	$\P_{\A^{m-k}\B^{k+1}\C^{m-k}\D^{k+1}}$ belonging to the neighboring
	chains.  Hence, at the parameter values belonging to these overlaps,
	map~\eqref{eq:map} exhibits four coexisting cycles, namely three
	coexisting $2m$-cycles and one $2(m-1)$- or $2(m+1)$-cycle.
\end{compactenum}
Moreover, as one can see in the above diagram, a
region  $\P_{\A^{m-k}\B^{k}\C^{m-k}\D^{k}}$ associated with
a symmetric $2m$-cycle overlaps close to its corner point
with
\begin{compactenum}[C1:]
	\item[C4:]
	another region associated with a symmetric cycle belonging to a
	neighboring chain, namely {\color{red} \dotfill check it
			\dotfill}\\ In these cases, map~\eqref{eq:map} exhibits
	coexistence of two symmetric cycles, one of period $2m$ and the other one
	of period $2(m+1)$ or $2(m-1)$.
\end{compactenum}
It follows immediately that in all cases listed above the periods of
coexisting cycles are either identical or differ by two.


\begin{figure}[t]
	%\Includesubgraphics{.48\textwidth}{figs/scanMode0/Cobweb_M/Manual/result.png}{a}
	%\Includesubgraphics{.48\textwidth}{figs/scanMode0/Cobweb_Q/Manual/result.png}{b}\\
	%\Includesubgraphics{.48\textwidth}{figs/scanMode0/Cobweb_U/Manual/result.png}{c}
	%\Includesubgraphics{.48\textwidth}{figs/scanMode0/Cobweb_X2/Manual/result.png}{d}  
	\caption{\label{fig:multistability}Coexisting attracting cycles in map~\eqref{eq:map}:
		(a) a symmetric $16$-cycle $\LC_{\A^5\B^3\C^5\D^3}$ and
		a symmetric $14$-cycle $\LC_{\A^4\B^3\C^4\D^3}$
		at $\alpha=-0.4$, $\beta=0.168$;
		(b) a pair of asymmetric $16$-cycles $\LC_{\A^5\B^3\C^4\D^4}$, $\LC_{\A^4\B^4\C^5\D^3}$
		at $\alpha=-0.375$, $\beta=0.1678$;
		(c) a symmetric $14$-cycle $\LC_{\A^4\B^3\C^4\D^3}$ and a pair of
		asymmetric $16$-cycles $\LC_{\A^5\B^3\C^4\D^4}$, $\LC_{\A^4\B^4\C^5\D^3}$
		at $\alpha=-0.3797$, $\beta=0.168$;
		(d) a symmetric $16$-cycle $\LC_{\A^5\B^3\C^5\D^3}$,
		a symmetric $14$-cycle $\LC_{\A^4\B^3\C^4\D^3}$,
		and a pair of asymmetric $\LC_{\A^5\B^3\C^4\D^4}$, $\LC_{\A^4\B^4\C^5\D^3}$
		$16$-cycles at $\alpha=-0.3805$, $\beta=0.1672$.
		The basins of attraction are shown different colors.
		Corrsponding parameter values are indicated in
		Fig.~\ref{fig:2D:blowup:numbers}.}
\end{figure}

To sum up, each region where map~\eqref{eq:map} has exactly two
asymmetric cycles is surrounded by four regions where is has exactly
three cycles (cases~C1, C2), and by four regions where exactly four
cycles coexist (case~C3).  Similarly, each region associated with a
single (globally attracting) symmetric cycle is surrounded by four
regions where the map has exactly two cycles (case~C4), by four
regions four regions where the map exhibits exactly three cycles
(cases~C1, C2), as well as eight regions where four cycles coexist
(case~C3).  This is illustrated in Fig.~\ref{fig:2D:blowup:numbers},
showing the number of attractors in the regions associated with a
symmetric $16$-cycle $\LC_{\A^5\B^3\C^5\D^3}$
(Fig.~\ref{fig:2D:blowup:numbers}(a)) and with a pair of asymmetric
cycles $\LC_{\A^{5}\B^3\C^{4}\D^{4}}$, $\LC_{\A^{4}\B^{4}\C^{5}\D^3}$
(Fig.~\ref{fig:2D:blowup:numbers}(b)).  Examples of two, three and
four coexisting cycles in the presented parameter region are shown in
Fig.~\ref{fig:multistability}. Note that the basins of coexisting
cycles are separated by the points of discontinuities and their
preimages. Indeed, as at the considered parameter values the function
$f$ is everywhere contractive and therefore cannot have repelling
cycles. Therefore, discontinuities and their preimages are the only
points which can separate the basins, and the number of coexisting
cycles cannot exceed twice the number of discontinuities (which
corresponds to the case that each one-side-neighborhood of each
discontinuity belongs to the basin of a different cycle). However, due
to the symmetry~\ref{eq:symmetry} of the map, the neighborhoods of the
discontinuities $d_0$, $d_2$, as well as of $d_1$, $d_2$ must belong
to the same basins (e.g., if the left neighborhood of $d_1$ belongs to
the basin of some cycle, then the left neighborhood of $d_3$ belongs
to the basin of the same cycle). Accordingly, four is indeed the
maximal number of coexisting cycles in the consideded map
(see the example shown in Fig.~\ref{fig:multistability}(d)).

\clearpage
\subsection{Changing the shape of $f$}
Question: which properties are indeed important? We have already shown
that the non-monotonicity of $\FB$, $\FD$ does not matter, thus the
question arises whether the non-monotonicity of $\FA$, $\FC$ matters?
Yes, it does, as illustrated in Fig.~\ref{fig:vaying}.

\begin{figure}[t]
	%\Includesubgraphics{.48\textwidth}{figs/scanMode0/varying/result.png}{a}
	%\Includesubgraphics{.48\textwidth}{figs/scanMode2/varying/DATA/07/bif.png}{b}\\
	%\Includesubgraphics{.48\textwidth}{figs/scanMode2/varying/DATA/09/bif.png}{c}
	%%\Includesubgraphics{.48\textwidth}{figs/scanMode2/varying/DATA/11/bif.png}{d}  
	\caption{\label{fig:vaying}
		(b) $\bL=0.7$;
		(c) $\bL=0.9$;
		(d) $\bL=1.1$;
	}
	\vspace*{-1em}
\end{figure}
\end{document}



\begin{equation}
	\P_{\A^{m-(k+1)}\:\B^{k+1}\:\C^{m-k}\:\D^{k}} \cap
	\P_{\A^{m-k}\:\B^{k}\:\C^{m-k}\:\D^{k}} \cap
	\P_{\A^{m-(k+1)}\:\B^{k}\:\C^{m-(k-1)}\:\D^{k}}
\end{equation}

\clearpage
Specifically, let us denote the periodicity region of the $2m$-cycle
$\LC_{\A^{m-k}\:\B^k\:\C^{m-k}\:\D^k}$ by $P^{2m}_k$ and the
periodicity region of the pair of $2m$-cycles
$\LC_{\A^{m-(k+1)}\:\B^{k+1}\:\C^{m-k} \:\D^{k}}$ $\LC_{\A^{m-k}
		\:\B^{k} \:\C^{m-(k+1)}\:\D^{k+1}}$ by $Q^{2m}_k$. In these terms,
the above diagram becomes
\begin{equation}
	\begin{split}
		&
		\fbox[RB]{
			\begin{minipage}[c][4em][c]{3em}\small
				\hspace*{-.7em}
				$P^{2m-2}_k$
			\end{minipage}
		}\hspace*{-1ex}
		\fbox[LBR]{
			\begin{minipage}[c][3em][c]{3em}\small\centering
				$Q^{2m-2}_k$
			\end{minipage}
		}\hspace*{-1ex}
		\fbox[LBR]{
			\begin{minipage}[c][4em][c]{5em}\small\centering
				$P^{2m-2}_{k+1}$
			\end{minipage}
		}
		\hspace*{-1ex}
		\fbox[LBR]{
			\begin{minipage}[c][3em][c]{3em}\small
				$Q^{2m-2}_{k+1}$
			\end{minipage}
		}
		\hspace*{-1ex}
		\fbox[LB]{
			\begin{minipage}[c][4em][c]{2em}\small
				\dots
			\end{minipage}
		}\\[-1.5em]
		%%
		&
		\fbox[RTB]{
			\begin{minipage}[c][3em][c]{2em}\small
				\hspace*{-.7em}
				$Q^{2m}_{k-1}$
			\end{minipage}
		}
		\hspace*{-1ex}
		\fbox{
			\begin{minipage}[c][4em][c]{5em}\small\centering
				$P^{2m}_k$
			\end{minipage}
		}
		\hspace*{-1ex}
		\fbox{
			\begin{minipage}[c][3em][c]{3em}\small\centering
				$Q^{2m}_{k}$
			\end{minipage}
		}
		\hspace*{-1ex}
		\fbox{
			\begin{minipage}[c][4em][c]{5em}\small\centering
				$P^{2m}_{k+1}$
			\end{minipage}
		}
		\hspace*{-1ex}
		\fbox[LBT]{
			\begin{minipage}[c][3em][c]{1em}\small
				\dots
			\end{minipage}
		}\\[-1.5em]
		%%
		&
		\fbox[RT]{
			\begin{minipage}[c][4em][c]{3em}\small
				\hspace*{-.7em}
				$P^{2m+2}_{k}$
			\end{minipage}
		}\hspace*{-1ex}
		\fbox[RTL]{
			\begin{minipage}[c][3em][c]{3em}\small\centering
				$Q^{2m+2}_{k}$
			\end{minipage}
		}\hspace*{-1ex}
		\fbox[RTL]{
			\begin{minipage}[c][4em][c]{5em}\small\centering
				$P^{2m+2}_{k+1}$
			\end{minipage}
		}\hspace*{-1ex}
		\fbox[RTL]{
			\begin{minipage}[c][3em][c]{3em}\small\centering
				$Q^{2m+2}_{k+1}$
			\end{minipage}
		}
		\hspace*{-1ex}
		\fbox[LT]{
			\begin{minipage}[c][4em][c]{2em}\small
				\dots
			\end{minipage}
		}
	\end{split}
\end{equation}


\begin{equation}
	Q^{2m}_k \cap P^{2(m-1)}_{k+1} \cap P^{2m}_{k}
\end{equation}


\clearpage
Then, the regions of the considered bifurcations structure where
map~\eqref{} has a single (globally attracting) stable cycle
are given by
\begin{equation}
	P^{2m}_k \setminus \left(
	Q^{2m}_{k} \cup
	Q^{2m}_{k-1} \cup
	Q^{2m-2}_{k} \cup
	Q^{2m+2}_{k} \cup
	P^{2m-2}_{k} \cup
	P^{2m-2}_{k+1} \cup
	P^{2m+2}_{k} \cup
	P^{2m+2}_{k+1}
	\right)
\end{equation}
Two cycles

\begin{tabular}{|l|l|}
	\hline
	region                                                           & periods of coexisting cycles \\
	\hline
	$Q^{2m}_k$                                                       & $2m$, $2m$                   \\
	$(P^{2m}_k \cap P^{2m-2}) \setminus (Q^{2m}_k \cup Q^{2m-2}_k) $ &
	$2m$, $2m-2$                                                                                    \\
	\hline
\end{tabular}
\end{document}

\clearpage
\begin{figure}
	\setlength{\unitlength}{.005\textwidth}
	\begin{picture}(100,50)(0,0)
		\put(-15,-13.2){\Includegraphics{140\unitlength}{figs/Chains.drawio.png}}
		%
		\put(4,2){\scriptsize $1$}
		\put(4,44){\scriptsize $1$}
		\put(53,2){\scriptsize $1$}
		\put(53,44){\scriptsize $1$}
		\put(26.5,22){\scriptsize $1$}
		\put(80,22){\scriptsize $1$}

		\put(4,22){\scriptsize $2$}
		\put(53,22){\scriptsize $2$}
		\put(26.5,2){\scriptsize $2$}
		\put(26.5,44){\scriptsize $2$}
		\put(78.5,2){\scriptsize $2$}
		\put(78.5,44){\scriptsize $2$}

		\put(13.5,32){\scriptsize $2$}
		\put(13.5,12){\scriptsize $2$}
		\put(39.8,32){\scriptsize $2$}
		\put(39.8,12){\scriptsize $2$}
		\put(66.3,32){\scriptsize $2$}
		\put(66.3,12){\scriptsize $2$}
		\put(92,32){\scriptsize $2$}
		\put(92,12){\scriptsize $2$}


		\put(10.5,22){\scriptsize $3$}
		\put(42.5,22){\scriptsize $3$}
		\put(63,22){\scriptsize $3$}
		\put(26.5,9){\scriptsize $3$}
		\put(26.5,35.3){\scriptsize $3$}
		\put(78.5,9){\scriptsize $3$}
		\put(78.5,35.3){\scriptsize $3$}
		\put(16.5,2){\scriptsize $3$}
		\put(16.5,44){\scriptsize $3$}
		\put(36.7,2){\scriptsize $3$}
		\put(36.7,44){\scriptsize $3$}
		\put(68.7,2){\scriptsize $3$}
		\put(68.7,44){\scriptsize $3$}
		\put(89.2,2){\scriptsize $3$}
		\put(89.2,44){\scriptsize $3$}
		\put(53,29.5){\scriptsize $3$}
		\put(53,14.7){\scriptsize $3$}


		\put(16.5,35.3){\scriptsize $4$}
		\put(16.5,9){\scriptsize $4$}
		\put(36.7,35.3){\scriptsize $4$}
		\put(36.7,9){\scriptsize $4$}
		\put(68.7,35.3){\scriptsize $4$}
		\put(68.7,9){\scriptsize $4$}
		\put(89.2,35.3){\scriptsize $4$}
		\put(89.2,9){\scriptsize $4$}

		\put(10.5,29.5){\scriptsize $4$}
		\put(10.5,14.7){\scriptsize $4$}
		\put(42.8,29.5){\scriptsize $4$}
		\put(42.8,14.7){\scriptsize $4$}
		\put(62.7,29.5){\scriptsize $4$}
		\put(62.7,14.7){\scriptsize $4$}
		\put(95,29.5){\scriptsize $4$}
		\put(95,14.7){\scriptsize $4$}


		%\graphpaper[5](0,0)(100,50)
	\end{picture}
	\caption{Number of coexisting attractors}
\end{figure}



\end{document}
