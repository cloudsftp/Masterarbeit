% !TeX spellcheck = de_DE

\documentclass[10pt]{article}
%\usepackage[german]{babel}
%\usepackage{fancyhdr}
%\usepackage{amssymb}
%\usepackage{amsmath}
%\usepackage{graphicx}
%\usepackage[utf8]{inputenc}
\setlength{\parindent}{0pt}
\usepackage{fullpage}

%\addtolength{\topmargin}{-3em}
%\addtolength{\textheight}{4em}


\usepackage{paralist}

\pagestyle{empty}

\begin{document}

% OG title: Investigation of multistability-affected period adding structure
% using an archetypal model of power converters with hysteresis control

\title{Investigation of Multistability-affected Period Adding Structure}
\author{Viktor Avrutin}
\date{}
\maketitle

Identification and investigation of generic bifurcation structures
appearing in several applied systems and explanation of the specific
mechanisms leading to their formation is one of the central tasks in
nonlinear dynamics.  As an example of such structures, one can mention
period-doubling cascades which are well-known to exist in all kinds of
dynamic systems.  Another prominent class of such structures is
period-adding cascades typically occurring in discrete-time models
(maps) whose dynamics is governed by a discontinuous function
(so-called discontinuous maps) as well as in continuous-time models
that can be reduced to such maps.

Recently, our investigations of a model of a power converter with
hysteresis control revealed an interesting and previously unknown
bifurcation structure combining some features of period adding and
period incrementing scenarios.  A structural unit of this structure is
given by a chain of periodicity regions associated with cycles of the
same period (such chains have previously been observed in
period-adding structures).  However, neighboring chains in the
observed structure are partially overlapping (which is typical for
period incrementing structures).  Characteristic features of this
bifurcation structure include also unusual kinds of multistability and
not yet clear regularities in the appearance of symbolic sequences
associated with stable cycles.

Unfortunately, the specific model exhibiting this highly interesting
bifurcation structure is quite inconvenient for the investigation of
the mechanism leading to the appearance of the observed bifurcation
structure. Indeed, the model is given by a set of implicit equations,
so that it can not be immediately concluded how specific parameters
influence the function governing the dynamics.

The primary goal of the master thesis is to develop a minimal model
serving as an archetypal model for the class of power converters with
hysteresis control and showing the bifurcation scenario observed in
these systems.  Then, using this model, the mechanism leading to the
appearance of this scenario has to be explained.  The specific steps
of the proposed approach are as follows:
\begin{compactenum}[1.]
\item to identify characteristic properties of the original functions.
  (such as periodicity, symmetries, monotonicity, etc.)
\item to identify how the implicitly defined function of the original
  model changes under the variation of parameters along one chain of regions
  forming a structural unit of the bifurcation structure under consideration.
\item to define a minimal model exhibiting the same dynamics leading to the
  interesting bifurcation structures.
\item to track the bifurcations leading to the appearance and disappearance
  of periodic orbits in this model, to identify the underlying regularities
  and in this way to
  explain the mechanisms that lead to the considered
  bifurcation structures..
\end{compactenum}

The proposed mode of operation follows a novel approach recently
developed for investigation of bifurcation structures in discontinuous
maps (so-called archetypal map approach).  In a very general sense,
this approach can be seen as a kind of model reduction technique which
might be applicable in the cases where other techniques
(being developed specifically for continuous maps) fail.
Until now, the applicability of the  archetypal map approach
has been demonstrated by a few examples only.
Therefore, the second goal of the master thesis is to  
examine the applicability of this approach and to contribute to
the understanding of its strong and
weak sides.

\end{document}