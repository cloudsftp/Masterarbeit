% !TeX spellcheck = de_DE

\documentclass{article}
\usepackage[german]{babel}
\usepackage{fancyhdr}
\usepackage{amssymb}
\usepackage{amsmath}
\usepackage{graphicx}
\usepackage[utf8]{inputenc}
\setlength{\parindent}{0pt}
%\pagestyle{fancy}
%\fancyhead[R]{Frank Bastian}
\renewcommand{\headrulewidth}{0.1pt}
\renewcommand{\footrulewidth}{0pt}
\renewcommand{\headrulewidth}{0.1pt}
\renewcommand{\footrulewidth}{0pt}

\usepackage{paralist}

\begin{document}
\title{Task description}
\author{Viktor Avrutin}
\date{\today}
\maketitle

Investigation of generic bifurcation structures appearing in several
applied systems and explanation of the specific mechanisms leading to
their formation is one of the central tasks in nonlinear dynamics.  As
an example of such structures one can mention period doubling cascades
which are well known to exist in all kinds of dynamical systems.
Another prominent class of such structures are period adding cascades
typically occurring in discrete-time models (maps) whose dynamics is
governed by a discontinuous function as well as in continuous-time
models that can be reduced to such maps.

Recently, our investigations of a model of a power converter with
hysteresis control revealed an interesting and previousely unknown
bifurcation structure combining some features of period adding and
period incrementing scenarios.  A structural unit of this structure is
given by a chain of periodicity regions of cycles of the same period
(such chains have previousle been observed in period adding
structures). However, neighboring chains in the observed structure are
partially overlapping (which is typical for period incrementing
structures).  Caracteristic features of this bifurcation structure
include also unusual kinds of multistability and not yet clear
regularities in the apprarance of symbolic sequences associated with
stable cycles.

Unfortulately, the specific model exhibiting this highly interesting
bifurcation structure is quite inconvenient for investigation of the
mechanism leading to the apprarance of the observed bifurcation
structure. Indeed, the model is given by a set of implicite equations,
so that it can not be immediately concluded how specific parameters
influence the function governing the dynamics.

The goal of the master thesis is to develop a minimal model showing
the bifurcation scenario observed so far and -- using this model -- to
explain the mechanism leading to its appearance.  The specific steps
of the proposed approach are as follows:
\begin{compactenum}
\item to identify characteristic properties of the original function
  (such as periodicity, symmetries, etc.)
\item to identify how the implicitly defined function of the original
  model changes under the variation of parameters along one chain of regions
  formig a structural unit of the bifurcation structure under consideation.
\item to define a 
\end{compactenum}

\end{document}

