\chapter{Future Work}
\label{chap:outlook}

\todo{Ok as seperate chapter? should I put it as a subsection in the concusion instead?}

In this chapter, we will touch upon topics that were not covered in this thesis and would be interesting to investigate in future work.

First, the theory of the halved archetypal model was fully developed in the later part of this thesis.
We suspect that it is possible to explain the regularities of the bifurcations at the boundaries of ``type B'' parameter regions as they are described in \Cref{sec:arch.bif.sum}.
The regularities we mean is that at the upper boundary of a ``type A'' parameter region, the \gls{bcb} is $\BCB_{d1, d_2}^{\underline{\A}^a\B^b\underline{\C}^a\D^b}$ while at the upper boundary of a ``type B'' parameter region, there are two \glspl{bcb} $\BCB_{d_1}^{\underline{\A}^a\B^b\C^c\D^d}$ and $\BCB_{d_3}^{\A^c\B^d\underline{\C}^a\D^d}$.
Note that in the ``type A'' parameter boundary \gls{bcb}, two points of the branches $f_\A$ and $f_\C$ collide with the borders $d_1$ and $d_3$ respectively.
While in the ``type B'' \glspl{bcb} this distributes onto both \glspl{bcb}.
Here, the point of the branch $f_\A$ collides with the border $d_1$ for the cycle $\Cycle{\A^a\B^b\C^c\D^d}$ and the point of the branch $f_\C$ collides with $d_3$ for the cycle $\Cycle{\A^c\B^d\C^a\D^b}$.

Even more interesting is that this distribution inverts for the lower boundary.
Now the point of the branch $f_\D$ collides with the border $d_3$ for the cycle $\Cycle{\A^a\B^b\C^c\D^d}$ and the point of the branch $f_\B$ collides with the border $d_1$ for the cycle $\Cycle{\A^c\B^d\C^a\D^b}$.
And for the right and left boundaries of the parameter regions this distribution of border collisions onto two bifurcations in the case of ``type B'' parameter regions is similar.

Secondly, in \Cref{sec:arch.end} we describe how the chains of the same period start falling apart for larger values of $\beta = c_L$.
This agrees with the behavior of the original model where the chains also fall apart for larger values of $\chi_0$.
It was described by \Citeauthor{akyuz2022} in his thesis~\cite{akyuz2022}.
Our shared consensus is that we might be able to observe ``full'' chains in another parameter plane.
With ``full'' chains we mean chains that start with the parameter region $P^{n}_{1}$ and stop with the parameter region $P^{n}_{n-1}$ while following all rules laid out for these chains.
This was first formulated by \Citeauthor{akyuz2022} in his thesis~\cite{akyuz2022}.

We propose an alternative archetypal model, where it might be possible to observe ``full'' chains.
First, we need four quadratic branches.
The piecewise quadratic model defined in \Cref{sec:setup.quad} would suffice.
Then we choose new compound parameters $\alpha'$ and $\beta'$ in such a way that $\alpha'$ has the effect of $\alpha$ in the archetypal model proposed in this thesis while also having the opposite effect of $\beta$ in the archetypal model.
So increasing alpha should decrease all values of the branches $f_\A$ and $f_\C$, while the values at the left borders of the branches should decrease less or even increase.
The effect of $\beta'$ should then have the opposite effect on the branches $f_\B$ and $f_\D$.
This will make a 2D scan in this parameter plane symmetric, provided $\alpha'$ and $\beta'$ are varied in the same ranges.
And it also preserves the effects, we identified as causes for the bifurcation structure with the chains in the original model.

\todo{Adding / chaos upper right its in 64 adding large aL?}
