\section{Future Work}
\label{sec:concl.future}

This section touches on some phenomena that were discovered but not explained in this thesis.
This section is divided into \hl{three} logical parts.

First, the theory of the halved archetypal model was fully developed in the later part of this thesis.
It \hl{may} be possible to explain the regularities of the bifurcations at the boundaries of ``type B'' parameter regions as they are described in \Cref{sec:arch.bif.sum}.
The regularities being that at the upper boundary of a ``type A'' parameter region, the border collision bifurcation is $\BCB_{d1, d_2}^{\underline{\A}^a\B^b\underline{\C}^a\D^b}$ while at the upper boundary of a ``type B'' parameter region, there are two border collision bifurcations $\BCB_{d_1}^{\underline{\A}^a\B^b\C^c\D^d}$ and $\BCB_{d_3}^{\A^c\B^d\underline{\C}^a\D^d}$.
Note that in the ``type A'' parameter boundary border collision bifurcation, two points of the branches $f_\A$ and $f_\C$ collide with the borders $d_1$ and $d_3$ respectively.
While in the ``type B'' border collision bifurcations this distributes onto both border collision bifurcations.
Here, the point of the branch $f_\A$ collides with the border $d_1$ for the cycle $\Cycle{\A^a\B^b\C^c\D^d}$ and the point of the branch $f_\C$ collides with $d_3$ for the cycle $\Cycle{\A^c\B^d\C^a\D^b}$.
The interesting part is that this distribution inverts for the lower boundary.
At the lower boundary, the point of the branch $f_\D$ collides with the border $d_3$ for the cycle $\Cycle{\A^a\B^b\C^c\D^d}$ and the point of the branch $f_\B$ collides with the border $d_1$ for the cycle $\Cycle{\A^c\B^d\C^a\D^b}$.
And for the right and left boundaries of the parameter regions this distribution of border collisions onto two bifurcations in the case of ``type B'' parameter regions is similar.

Second, \Cref{sec:arch.end} describes how the chains of parameter regions associated with the same period start falling apart for larger values of $\beta = c_L$.
This agrees with the behavior of the original model where the chains also start falling apart for larger values of $\chi_0$.
Nonetheless, this thesis proposes ways to obtain full chains with the archetypal model or a similar model.
These could be investigated in future work.

\hl{
	Third, there is another kind of bifurcation structure that the archetypal model can exhibit.
	It is possible if the parameter $a_L$ is increased.
}
\Cref{fig:concl.fut.addincr} \hl{shows this bifurcation structure}.
\hl{
	In this figure we can see \gls{pa} and \gls{pi} incrementing behavior close to each other.
	On the left side there is \gls{pa} and on the right side there is \gls{pi}.
	A similar bifurcation structure was observed...
}
\todo{citation where this bifurcation structure also turned up}

\begin{figure}
	\centering
	\includegraphics[width=.7 \textwidth]{../Figures/8/8.1/result.png}
	\caption[2D scan of the periods associated with parameter regions in the archetypal model]{
		2D scan of the periods associated with parameter regions in the archetypal model.
		The parameters $a_L = 8, b_L = -\frac{1}{2},$ and $g_R\left(\frac{1}{2}\right) = \frac{1}{2} + \frac{1}{4}$ are fixed.
		The parameters $\alpha = g_R\left(\frac{1}{4}\right)$ and $\beta = c_L$ are varied in the ranges $[-0.48, -0.44]$ and $[0.091, 0.094]$.
	}
	\label{fig:concl.fut.addincr}
\end{figure}
