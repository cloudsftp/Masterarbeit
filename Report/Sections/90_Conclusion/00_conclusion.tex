\chapter{Conclusion}

This chapter will summarize all findings of this thesis.
It is divided into two bigger parts.

First, we analyzed the original model function and the parameter effects on the model function in the original model.
We then identified the characteristics of the model function and parameter effects that are needed for the \gls{pi} structure of interest by constructing the archetypal model.
This model has 4 branches $f_\A, f_\B, f_\C,$ and $f_\D$ like the original model.
And it is symmetric like the original model.
The branches $f_\A$ and $f_\C$ are quadratic and both governed by a function $g_L(x) = a_L \cdot x^2 + b_L \cdot x + c_L$, while the branches $f_\B$ and $f_\D$ are linear and governed by a function $g_R(x) = b_L \cdot x + c_L$.
One parameter of the archetypal model is chosen to directly manipulate a specific characteristic of the model function, namely the value of $g_R$ at the left border of the branches $f_\B$ and $f_\D$.
It is named $\alpha = g_R\left(\frac{1}{4}\right)$ and its value influences the parameters $b_R$ and $c_R$.
Hence, it is called a compound parameter.
The other parameter is just the offset of the branches $f_\A$ and $f_\D$, $\beta = c_L$.

\todo{Too detailed? Previously I felt it was too shallow, so I added the archetypal model definition}

We confirmed that this archetypal model in fact exhibits the same behavior as the original model.
While doing so we even found the coexistence of four cycles which was not discovered previously in the original model and confirmed its existence in the original model.
Using the characteristics of that we found to be essential for the \gls{pi} structure of interest, we were able to explain the structure.

Secondly, we demonstrated that the archetypal model proposed in this thesis can also exhibit \gls{pa} behavior.
We found that these structures defied our expectations for \gls{pa}, since the periods did not add up as they normally would and there were no obvious regularities in the symbolic sequences associated with the parameter regions in these structures.
Hence, we called them \gls{pal} structures.
Then we introduced the halved archetypal model where we took advantage of the symmetry in the archetypal model.
The halved archetypal model exhibits the expected behavior of \gls{pa} structures.
With this model we were able to formulate rules for the \gls{pal} structures.
This includes rules for the periods, symbolic sequences, coexistence, and rotation tuples associated with the parameter regions of these \gls{pal} structures.

\todo{Rules too complicated to describe in a few sentences. Should I write something about the periods? Sometimes double sometimes not}

\todo{Coexistence similar to ``type B'' parameter regions}
