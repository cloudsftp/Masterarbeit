\chapter{Approach}
\label{chap:approach}

The model definition of the original model is very complicated.
We have algorithms to simulate it, so it can be worked with.
But since the model involves a lot of implicit equations, it is almost impossible to see, which effects the parameters have on the model function.
And therefore, what characteristics of the model function lead to the observed bifurcation structure.
Meaning the chains of alternating ``type A'' and ``type B'' parameter regions with the same period, laying next to each other with increasing periods.
We want a simplified model showing the same bifurcation structure, to explain them better.

In non-linear dynamics, there is the concept of normal forms.
They are the simplest model that exhibits some bifurcation.
As we mentioned in \Cref{sec:state.pws}, it is not possible to find a normal form for this bifurcation structure since the model is discontinuous.
Therefore, we will not construct a normal form but rather an archetypal model.
%That is a simplified model, but not necessarily the simplest, showing the same bifurcation structure as the original model.

We will construct this model by imitating the model function of the original model.
For this, we first need to analyze the original function and note the characteristics, such as number and shape of branches.
Arguably more important characteristics are the effects, the parameters have on the original function.

Then we will construct many different models that exhibit some of the characteristics of the original model function and analyze their behavior.
This is a process of trial and error, but also analysis.
When the constructed models exhibit behavior that is similar or hints at similar behavior as the original model function, we will use this model as the basis for further modifications.
We will start with the most obvious characteristics, such as the strongest effects of parameters.
This way, we can figure out, which of the characteristics are essential for the observed bifurcation structure in the original model.
And since we only imitate the characteristics that are important for the desired behavior, the resulting model will be simple.
