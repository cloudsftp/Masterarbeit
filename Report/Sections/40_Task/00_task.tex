\chapter{Task Definition}
\label{chap:task}

The first task of this thesis is to find an archetypal model for the observed bifurcation structure in the original model.
This task consists of two subtasks.
First, the characteristics of the original model have to be identified.
Then different models with similar characteristics have to be constructed and analyzed.
The details of both tasks is outlined in \Cref{chap:approach}.
The result will be the archetypal model for the \gls{pi} structure observed in the original model and also the characteristics that are identified to be important for this structure.

The next task is to investigate the archetypal model.
This is done for two reasons.
First, it has to be certain that the bifurcation structure in the archetypal model is the same structure as in the original model.
For this, numerical evidence has to be provided.
Secondly, the bifurcation structure has to be explained.
The numerical evidence and the identified characteristics that are important for the bifurcation structure are used for this.

The last task is to explore the possibilities of the archetypal model and to find parameter values for which it exhibits different bifurcation structures.
