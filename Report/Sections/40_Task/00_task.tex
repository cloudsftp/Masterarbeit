\chapter{Task}
\label{chap:task}

The task of this thesis is to find an archetypal model for the observed bifurcation structures in the original model.
For this we employ the approach outlined in \Cref{chap:approach}.
The result will be not only the archetypal model, but also the characteristics that are important for such bifurcation structures.
%Then we also need to confirm that the archetypal model does not show any behavior that the original model does not show.

The next task is to analyze the archetypal model for two reasons.
First, we have to explore whether there are any behaviors that were missed in the analysis of the original model by \Citeauthor{akyuz2022}.
While doing this we also confirm that our archetypal model does not show any behavior that was not intended.
Secondly, we have to explain the bifurcation structure.

The last task is to explore the archetypal model further.
Period-adding structures are common in discontinuous models, especially in models of power converters.
\todo{source}
Does our archetypal model also show period-adding when modified slightly?
