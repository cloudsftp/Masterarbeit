\chapter{Task Definition}
\label{chap:task}

The first task of this thesis is to \hl{develop} an archetypal model for the observed bifurcation structure in the original model.
This task consists of two subtasks.
First, the characteristics of the original model have to be identified.
Then different models with similar characteristics have to be constructed and analyzed.
The details of both tasks are outlined in \Cref{chap:approach} and its ex.
\Cref{chap:approach} \hl{outlines the details of both these tasks, and} \Cref{chap:setup} \hl{documents the execution}.
The result will be the archetypal model for the \gls{pi} structure observed in the original model and also the characteristics that are identified to be important for this structure.

The next task is to investigate the \hl{bifurcation structure in the} archetypal model.
This is done for two reasons.
\hl{
	First, the \gls{pi} structure has to be explained.
	Secondly, the capacity of the archetypal model to emulate the behavior of the original model has to be validated.
	Both is achieved with numerical evidence that results from the investigation of the bifurcation structure in the archetypal model and the characteristics that lead to the \gls{pi} structure identified by the first task.
}
\Cref{chap:arch} \hl{covers this task}.

\hl{
	The last task is to find out which other bifurcation structures may occur in the archetypal model, and which kinds of changes to the model function are required for this.
}
\Cref{chap:add} \hl{covers this task}.
