The theory of non-linear dynamical systems is very effective at explaining physical systems and phenomena.
Unfortunately, the theory is developed mainly for smooth systems.
And the models of some systems, such as power converters, are inherently piecewise-smooth.
This thesis is concerned with such a model that is piecewise-smooth, discontinuous, and symmetric.
The definition of this model is very complex, and it exhibits a unique period-incrementing structure that is affected by multistability.
This thesis identifies the characteristics of the model that lead to this unique bifurcation structure by constructing a simplified model that exhibits the same bifurcation structure, the archetypal model.
It follows a description of the dynamics of the archetypal model and an explanation of the unique bifurcation structure using the description of the dynamics.
Also, this thesis demonstrates that the proposed archetypal model can exhibit structures related to \glsentrylong{pa}.
These structures behave unexpectedly.
And this behavior is explained by taking advantage of the symmetry in the archetypal model.
In the same way, rules for these period-adding structures are constructed.
