\chapter{Introduction}

In this thesis we will construct an archetypal model for a bifurcation scenario that was observed in a complex model.
This archetypal model should be a much simpler model than the original model.

The original model is taken from the thesis of \Citeauthor{akyuz2022}~\cite{akyuz2022}.
This model is itself a simplification of another model, which describes the behavior of a DC to AC power converter in continuous time.
It defines the time-derivative of the present output current.
The analyzed power converter switches when the output current hits defined limits, which causes the derivative to change sign.
Therefore, the output current stays in a predefined range which has the form of a sine wave.
We can see why this model cannot be smooth.
In this case it has multiple discontinuities.

The simplified model in discrete time ignores the actual value of the output current.
Instead, it maps the phase, when the converter switched the last time, to the phase, when it will switch the next time.
This model is also discontinuous with multiple discontinuities.
It is a 1D map with input and output being phases, at which the power converter switches.

\todo{Completely rewrite the following}
The original, complex model is from a class of models that are non-continuous and piecewise smooth.
It has multiple discontinuities.
\todo{For such models the theory is not mature yet}
\todo{this class of maps not not theoretically explored fully}

The topic of this thesis is rather part of the field non-linear dynamics than computer science.
The following chapter will therefore be a summary of fundamentals of non-linear dynamics used in this thesis.
