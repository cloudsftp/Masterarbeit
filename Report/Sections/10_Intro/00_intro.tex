\chapter{Introduction}
\label{chap:intro}

Non-linear \hl{dynamical} systems are everywhere in nature and their study has gained popularity in recent decades.
It is very effective at explaining a wide range of physical phenomena such as fluid dynamics and weather \hl{patterns}~\cite{bernardo2008piecewise}.
Scientists in this field mostly focus on qualitative changes in the behavior of the systems under parameter \hl{variations}, so-called bifurcations~\cite{simpson2010}.

However, most the developed theory of non-linear \hl{dynamical} systems relies heavily on the smoothness of the model functions.
\hl{Several} real-world applications of dynamical systems are not only non-linear, but \hl{only \gls{pws} or even \gls{pws} discontinuous as well}.
Although the study of non-linear \hl{dynamical} systems is itself not very old, the study of \hl{\gls{pws} dynamical} systems is even younger.
It is gaining popularity, since all electrical systems with switching behavior are inherently \hl{\gls{pws}}, many even \hl{\gls{pws}} discontinuous~\cite{simpson2010}.
Non-smoothness introduces many different phenomena that don't happen in smooth systems.
One of the bifurcation classes that are new is the class of \glspl{bcb}.
This is the only class of bifurcations, this thesis touches on.

This thesis concentrates on a dynamical system which models a DC to AC power converter.
The power converter switches and therefore its model is \hl{\gls{pws}}.
\hl{The model happens to be discontinuous as well.}
Its model is not only \hl{\gls{pws}} discontinuous but very complicated too.
The main goal of this thesis is to find a simpler model that exhibits the same behavior as the original one.

Since the topic of this thesis focuses on non-linear dynamics more than computer science, the next chapter will introduce fundamentals needed for the rest of the thesis.
