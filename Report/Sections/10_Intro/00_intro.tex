\chapter{Introduction}
\label{chap:intro}

Non-linear dynamical systems are everywhere in nature.
And their study has gained popularity in recent decades.
It is very effective at explaining a wide range of physical phenomena such as fluid dynamics and weather patterns~\cite{bernardo2008piecewise}.
Scientists in this field mostly focus on qualitative changes in the behavior of the systems under parameter variations, so-called bifurcations~\cite{simpson2010}.

However, most of the developed theory of non-linear dynamical systems relies heavily on the smoothness of the model functions.
Several real-world applications of dynamical systems are not only non-linear but also \gls{pws} or even \gls{pws} and discontinuous as well.
Although the study of non-linear dynamical systems is not very old, the study of \gls{pws} dynamical systems is even younger.
It is gaining popularity since all electrical systems with switching behavior are inherently \gls{pws}, many even \gls{pws} discontinuous~\cite{simpson2010}.
Non-smoothness introduces many different phenomena that don't happen in smooth systems.
The class of border collision bifurcations is one of the new bifurcation classes.
It is the only class of bifurcations encountered in this thesis.

This thesis concentrates on a dynamical system that models a DC-to-AC power converter.
The power converter switches and therefore its model is \gls{pws}.
The model happens to be discontinuous as well.
Its model is not only \gls{pws} discontinuous but very complicated as well.
The main goal of this thesis is to find a simpler model that exhibits the same behavior as the original one.

Since the topic of this thesis focuses on non-linear dynamics more than computer science, the next chapter will introduce the fundamentals needed for the rest of the thesis.
