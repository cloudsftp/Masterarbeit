\chapter{Introduction}
\label{chap:intro}

Non-linear dynamical systems are ubiquitous in nature.
And their study has gained popularity in recent decades.
The study of non-linear systems is very effective at explaining a wide range of physical phenomena such as fluid dynamics and weather patterns~\cite{bernardo2008piecewise}.
Scientists in this field focus mostly on qualitative changes in the behavior of the systems under parameter variations, so-called bifurcations~\cite{simpson2010}.

However, most of the developed theory of non-linear dynamical systems relies heavily on the smoothness of the model functions.
Several real-world applications of dynamical systems are not only non-linear but also \gls{pws} or even \gls{pws} and discontinuous as well.
Although the study of non-linear dynamical systems is still quite young, the study of \gls{pws} dynamical systems is even younger.
The area has seen a recent surge in interest due to the fact that all electrical systems with switching behavior are inherently \gls{pws}, many even \gls{pws} discontinuous~\cite{simpson2010}.
Non-smoothness introduces phenomena that do not occur in smooth systems.
The class of border collision bifurcations is one of the new bifurcation classes.
It is the only class of bifurcations encountered in this thesis.

This thesis focuses on the simplification of an existing model of a DC/AC power converter.
The existing model is \gls{pws} and discontinuous due to the switching behavior of the power converter.
It is highly complex due to a high number of implicit equations and exhibits behavior that leads to an unusual bifurcation structure.
The goal of this thesis is to explain the unusual bifurcation structure in the existing model using the simplified model.

Since the topic of this thesis focuses on non-linear dynamics more than computer science, the next chapter will introduce the fundamental concepts and ideas required for the rest of the thesis.
