\chapter{Introduction}
\label{chap:intro}

Non-linear dynamical systems are \hl{ubiquitous} in nature.
And their study has gained popularity in recent decades.
\hl{The study of non-linear systems} is very effective at explaining a wide range of physical phenomena such as fluid dynamics and weather patterns~\cite{bernardo2008piecewise}.
Scientists in this field mostly focus on qualitative changes in the behavior of the systems under parameter variations, so-called bifurcations~\cite{simpson2010}.

However, most of the developed theory of non-linear dynamical systems relies heavily on the smoothness of the model functions.
Several real-world applications of dynamical systems are not only non-linear but also \gls{pws} or even \gls{pws} and discontinuous as well.
Although the study of non-linear dynamical systems is \hl{still quite young}, the study of \gls{pws} dynamical systems is even younger.
The area has seen a recent surge in interest due to the fact that all electrical systems with switching behavior are inherently \gls{pws}, many even \gls{pws} discontinuous~\cite{simpson2010}.
Non-smoothness introduces many phenomena that \hl{do not occur} in smooth systems.
The class of border collision bifurcations is one of the new bifurcation classes.
It is the only class of bifurcations encountered in this thesis.

This thesis concentrates on a \hl{model of} a DC-to-AC power converter.
\hl{
	The power converter switches and therefore the existing model is \gls{pws}, additionally the existing model is discontinuous.
	The definition of the existing model includes many implicit equations and is therefore highly complex.
}
% ask matthew again about this
The main goal of this thesis is to find a simpler model that exhibits the same behavior as the original one.

Since the topic of this thesis focuses on non-linear dynamics more than computer science, the next chapter will introduce the \hl{fundamental concepts and ideas required} for the rest of the thesis.
