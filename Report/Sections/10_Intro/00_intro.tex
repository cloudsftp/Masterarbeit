\chapter{Introduction}
\label{chap:intro}

Non-linear systems are everywhere in nature and their study has gained popularity in recent decades.
It is very effective at explaining a wide range of physical phenomena such as fluid dynamics and weather predictions~\cite{bernardo2008piecewise}.
Scientists in this field mostly focus on qualitative changes in the behavior of the systems under parameter changes, so-called bifurcations~\cite{simpson2010}.

However, most the developed theory of non-linear systems relies heavily on the smoothness of the model functions.
Many real-world applications of dynamical systems are not only non-linear, but non-smooth or even discontinuous as well.
These systems are often also called piecewise-smooth.
Although the study of non-linear systems is itself not very old, the study of non-smooth systems is even younger.
It is gaining popularity, since all electrical systems with switching behavior are inherently non-smooth, many even discontinuous~\cite{simpson2010}.
Non-smoothness introduces many different phenomena that don't happen in smooth systems.
One of the bifurcation classes that are new is the class of border collision bifurcations.
This is the only class of bifurcations, this thesis touches on.

This thesis concentrates on a dynamical system which models a DC to AC power converter.
The power converter switches and therefore its model is discontinuous.
Its model is not only discontinuous but very complicated too.
The main goal of this thesis is to find a simpler model that exhibits the same behavior as the original one.

Since the topic of this thesis focuses on non-linear dynamics more than computer science, the next chapter will introduce fundamentals needed for the rest of the thesis.
