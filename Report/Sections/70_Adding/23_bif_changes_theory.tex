\subsection{Summary of the Changes to the Bifurcation Structure}

\begin{enumerate}
	\item \begin{itemize}
		      \item ``type A'' parameter regions stop overlapping with the ``type A'' period regions above them (lower period, same number of points on branches $f_\B$ and $f_\D$)
		      \item Point where boundaries cross. On the left: no overlap, period-adding. On the right: overlap
		      \item This point moves right until no overlap exists anymore
	      \end{itemize}
	\item \begin{itemize}
		      \item ``type A'' parameter regions start overlapping with the ``type A'' period regions above right them (same period)
		      \item Point where boundaries cross. On the left: overlap. On the right: no overlap, ``type B'' parameter region
		      \item This point moves right, until no ``type B'' parameter region exists anymore
	      \end{itemize}
	\item \begin{itemize}
		      \item ``type A'' parameter regions stop overlapping with the ``type A'' parameter regions right to them (higher period, one more point on branches $f_\B$ and $f_\D$)
		      \item This seems to happen in an instant
	      \end{itemize}
\end{enumerate}

Timeline:
On the parameter line outlined above, the processes happen as follows.
First, the process (i) starts.
While it is happening, the process (ii) starts and finishes, before (i) finishes.
Lastly the process (iii) starts and finishes directly.
After all that, process (i) is the last to complete.

\todo{what to do with timeline?}

\todo{Sketches of period region shapes}

In ``conclusion'', the local minima on branches $f_\A$ and $f_\C$ seem to be important for the ``type B'' parameter regions.
That means, parameter regions with coexisting asymmetrical cycles with the \textbf{same} period.
At the same time, these minima seem to prevent period-adding structures.

\begin{theorem}[No ``Type B'' Parameter Regions with only Increasing Branches]
	The ``type B'' parameter regions are not possible in the increasing archetypal model.
	The minima on the branches $f_\A$ and $f_\C$ are essential for the bifurcation structure.
\end{theorem}

\begin{proof} \phantom{x} \\
	Let's assume that all branches of the archetypal model $f_\A, f_\B, f_\C,$ and $f_\D$ are increasing.
	And let $\sigma_1$ and $\sigma_2$ be ``type B'' twin cycles.
	The following conditions are true for such cycles.
	\begin{subequations}
		\begin{align}
			|\sigma_1|_\A - 1 & = |\sigma_2|_\A \label{equ:add.change.conseq.sigmaA} \\
			|\sigma_1|_\B + 1 & = |\sigma_2|_\B \label{equ:add.change.conseq.sigmaB} \\
			|\sigma_1|_\C + 1 & = |\sigma_2|_\C \label{equ:add.change.conseq.sigmaC} \\
			|\sigma_1|_\D - 1 & = |\sigma_2|_\D \label{equ:add.change.conseq.sigmaD}
		\end{align}
	\end{subequations}

	For \Cref{equ:add.change.conseq.sigmaA} to hold, the first point of $\sigma_1$ on the branch $f_\A$ needs to be left of first point of $\sigma_2$ on this branch, because the branch is increasing.
	At the same time must its last point on this branch be right of the last point of $\sigma_2$ on this branch.
	This way, $\sigma_1$ has exactly one more point on the branch $f_\A$ as required by \Cref{equ:add.change.conseq.sigmaA}.

	The order of the first points on the next branch, $f_\B$, is the same as for the last points on the branch $f_\A$.
	So the first point of $\sigma_2$ on this branch is left of the first point of $\sigma_1$ on this branch.
	For \Cref{equ:add.change.conseq.sigmaB} to hold, its last point on this branch must be right of the last point of $\sigma_1$ on this branch per the same logic as before.

	The order of the first points on the next branch, $f_\C$, is the same as for the last points on the branch $f_\B$.
	So the first point of $\sigma_1$ on this branch is left of the first point of $\sigma_2$ on this branch.
	This is a contradiction, since the first point of $\sigma_1$ on branch $f_\A$ is also left of the first point of $\sigma_2$ on that branch.
	This violates the symmetry.
	Also, \Cref{equ:add.change.conseq.sigmaC} can't be fulfilled if the first point of $\sigma_1$ is left of the first point of $\sigma_2$ on the branch $f_\C$.
	\hfill $\blacksquare$
\end{proof}

\todo{Odd number of splits => no neg slope needed for asymmetry. odd number => needed (reorders cycles)}
