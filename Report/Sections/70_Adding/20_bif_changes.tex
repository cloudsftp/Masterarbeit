\section{Changes to the Bifurcation Structure}
\label{sec:add.change}

\begin{enumerate}
	\item \begin{itemize}
		      \item ``type A'' parameter regions stop overlapping with the ``type A'' period regions above them (lower period, same number of points on branches $f_\B$ and $f_\D$)
		      \item Point where boundaries cross. On the left: no overlap, period-adding. On the right: overlap
		      \item This point moves right until no overlap exists anymore
	      \end{itemize}
	\item \begin{itemize}
		      \item ``type A'' parameter regions start overlapping with the ``type A'' period regions above right them (same period)
		      \item Point where boundaries cross. On the left: overlap. On the right: no overlap, ``type B'' parameter region
		      \item This point moves right, until no ``type B'' parameter region exists anymore
	      \end{itemize}
	\item \begin{itemize}
		      \item ``type A'' parameter regions stop overlapping with the ``type A'' parameter regions right to them (higher period, one more point on branches $f_\B$ and $f_\D$)
		      \item This seems to happen in an instant
	      \end{itemize}
\end{enumerate}

Timeline:
On the parameter line outlined above, the processes happen as follows.
First, the process (i) starts.
While it is happening, the process (ii) starts and finishes, before (i) finishes.
Lastly the process (iii) starts and finishes directly.
After all that, process (i) is the last to complete.

\todo{Remove previous}

\Cref{fig:add.change.regions} shows diagrams depicting the borders of parameter regions with different symbolic sequences.
We will use these diagrams as basis for our exploration of the changes to the bifurcation structure.
The lower left corner of the diagrams is always in the period region with the stable cycle $\Cycle{\A^7\B^3\C^7\D^3}$.
The cycles in the upper left ($\Cycle{\A^6\B^3\C^6\D^3}$), upper right ($\Cycle{\A^6\B^4\C^6\D^4}$), lower right ($\Cycle{\A^7\B^4\C^7\D^4}$), and middle ($\Cycle{\A^7\B^3\C^6\D^4}$ and $\Cycle{\A^6\B^4\C^7\D^3}$) follow.
We introduce a short notation for the symbolic sequences here.
``Type A'' parameter regions where the cycle is associated with the symbolic sequence $\A^a\B^b\C^a\D^b$ are written as $P^{2 \cdot \left(a + b\right)}_b$
``Type B'' parameter regions where the coexisting twin cycles are associated with the symbolic sequences $\A^a\B^b\C^c\D^d$ and $\A^c\B^d\C^a\D^b$ where $c = a - 1$ and $d = b + 1$ are written as $P^{a + b + c + d}_b$.
So for example, the ``type A'' parameter region in the lower left corner of every diagram in \Cref{fig:add.change.regions} is written as $P^{20}_3$.
\todo{
	Special case $\left[P^{2 \cdot \left(a + b\right)}_b \mid P^{2 \cdot \left(c + d\right)}_d\right]$ where $c \neq a - 1$ or $d \neq b + 1$ (it's not a ``type B'' parameter regions as we described them earlier).
	Where $c < a$ or $d > b$.
	Operation defined later in \Cref{sec:add.add.halved}.
}

The four different diagrams in \Cref{fig:add.change.regions} show the evolution at different points during the transition from \Cref{fig:add.arch} to \Cref{fig:add.arch.new}.
For this, the parameter values for $a_L$ and $b_L$ are chosen along a line given by \Cref{equ:add.change.paramline}
\begin{align}
	a_L = \frac{5}{2} - 3 \cdot b_L
	\label{equ:add.change.paramline}
\end{align}

\todo{Labels and remember to check parameters in captions / move to large caption}
\begin{figure}
	\centering
	\subfloat[$a_L = 3.5, b_L = -0.\overline{3}$]{
		\includegraphics[width=.5 \textwidth]{62_MinimalRepr_Adding/2D_Regions_3.5/Manual/result.png}
		\label{fig:add.change.regions.1}
	}
	\subfloat[$a_L = 2.8, b_L = -0.1$]{
		\includegraphics[width=.5 \textwidth]{62_MinimalRepr_Adding/2D_Regions_2.8/Manual/result.png}
		\label{fig:add.change.regions.2}
	} \\
	\subfloat[$a_L = 2.725, b_L = -0.075$]{
		\includegraphics[width=.5 \textwidth]{62_MinimalRepr_Adding/2D_Regions_2.725/Manual/result.png}
		\label{fig:add.change.regions.3}
	}
	\subfloat[$a_L = 2.5, b_L = 0$]{
		\includegraphics[width=.5 \textwidth]{62_MinimalRepr_Adding/2D_Regions_2.5/Manual/result.png}
		\label{fig:add.change.regions.4}
	}
	\caption[Changes to the bifurcation structures of the archetypal model during its transition to the increasing archetypal model]{
		2D boundary scans of the archetypal model showing the borders of parameter regions that are associated with different symbolic sequences.
		The parameter $g_R\left(\frac{1}{2}\right) = \frac{1}{2} + \frac{1}{40}$ is fixed in every diagram.
		The parameters $a_L$ and $b_L$ are fixed, but different in every diagram.
		Their values are on the line given by \Cref{equ:add.change.paramline}.
		The parameters $\alpha = g_R\left(\frac{1}{4}\right)$ and $\beta = c_L$ are varied.
		The diagrams illustrate the changes to the bifurcation structure in the archetypal model when transitioning to the increasing archetypal model.
	}
	\label{fig:add.change.regions}
\end{figure}

First, we will examine the disappearance of the ``type B'' parameter regions.
After that, we will examine the appearance of period-adding structures between the chains of the same period.

\subsection{Disappearance of ``Type B'' Parameter Regions}
\label{sec:add.change.disb}

For \Cref{fig:add.change.regions.1,fig:add.change.regions.2}, the ``type B'' parameter region $Q^{20}_3$ is complete.
In \Cref{fig:add.change.regions.4}, it is gone completely, instead the two ``type A'' parameter regions $P^{20}_3$ and $P^{20}_4$ now overlap.

\begin{figure}
	\centering
	\subfloat[Regions]{
		\includegraphics[width=.3 \textwidth]{../Figures/7/7.4a/result.png}
		\label{fig:add.change.disb.regions}
	}
	\subfloat[Cobweb at point $A$]{
		\includegraphics[width=.3 \textwidth]{../Figures/7/7.4b/result.png}
		\label{fig:add.change.disb.cob.A}
	}
	\subfloat[Cobweb at point $B$]{
		\includegraphics[width=.3 \textwidth]{../Figures/7/7.4c/result.png}
		\label{fig:add.change.disb.cob.B}
	}
	\caption[Short]{
		Disappearance of the ``type B'' parameter region
		\todo{Update caption}
	}
\end{figure}

In between those two stages, we can see how the ``type B'' parameter region $Q^{20}_3$ disappears.
\hl{Somewhere along the parameter line given by} \Cref{equ:add.change.paramline} between \Cref{fig:add.change.regions.2} and \Cref{fig:add.change.regions.3}, \hl{the lower left corner of the parameter region $P^{20}_4$ crosses the upper boundary of the parameter region $P^{20}_3$}.
\hl{This causes the ``type A'' parameter regions $P^{20}_3$ and $P^{20}_4$ to overlap in} \Cref{fig:add.change.regions.3}.
\hl{
	The point, where both boundaries cross is not a codimension-2 point, since the bifurcation at the lower boundary of the overlapping parameter region
}
\hl{We know from} \Cref{sec:arch.bif.sum} \hl{that the border collision bifurcation at the upper boundary of $P^{20}_3$ is $\BCB_{d_1, d_3}^{\A^6\underline{\B}^4\C^6\underline{\D}^4}$ and the border collision bifurcation at the lower boundary of $P^{20}_4$ is $\BCB_{d_1, d_3}^{\underline{\A}^7\B^3\underline{\C}^7\D^3}$}.
\hl{
	Therefore, this point is \textbf{not} a codimension-2 point, since the bifurcations happen to different cycle.
	Nonetheless, this is the right corner of the overlapping parameter region of $P^{20}_3 \cap P^{20}_4$.
}

\hl{
	At similar parameter values, where the lower left corner of $P^{20}_4$ crosses the upper boundary of $P^{20}_3$, the upper left corner and the lower left corner of the ``type B'' parameter region $Q^{20}_3$ collide causing the lower and upper boundaries of this parameter region to cross also.
}
\hl{We know from} \Cref{sec:arch.bif.sum} \hl{that the border collision bifurcations at the upper boundary of $Q^{20}_3$ are $\BCB_{d_1}^{\underline{\A}^7\B^3\C^6\D^4}$ and $\BCB_{d_3}^{\A^6\B^4\underline{\C}^7\D^3}$ and the border collision bifurcations at the lower boundary of $Q^{20}_3$ are $\BCB_{d_3}^{\A^7\B^3\C^6\underline{\D}^4}$ and $\BCB_{d_1}^{\A^6\underline{\B}^4\C^7\D^3}$}.
\hl{
	This is a codimension-2 point, since each of the coexisting cycles undergoes two different bifurcations at this point.
}
\hl{Also, this codimension-2 point is different from the codimension-2 points listed in} \Cref{sec:arch.bif.sum}, \hl{since those points always involve all four borders}.
\hl{
	Here, only the borders associated with vertical boundaries, namely $d_1$ and $d_3$, are involved in all four bifurcations at this point.
}

\hl{Along the parameter line given by} \Cref{equ:add.change.paramline} \hl{for increasing values of $b_L$, the lower boundary of $P^{20}_4$ and the upper boundary of $Q^{20}_3$ move down while the upper boundary of $P^{20}_3$ and the lower boundary of $Q^{20}_3$ moves up}.
\hl{
	This leads to both the right corner of the overlapping region $P^{20}_3 \cap P^{20}_4$ and the codimension-2 point which is the left corner of the ``type B'' parameter region $Q^{20}_3$ to move right.
}
\hl{We can observe this movement in} \Cref{fig:add.change.disb.regions}.
\hl{
	Here, those two corner points are near the right boundaries of $Q^{20}_3$ and $P^{20}_3$.
	As soon as the codimension-2 point of the boundaries ``type B'' parameter region crosses the right boundary of the ``type B'' parameter region, the ``type B'' parameter region vanishes.
	And as soon as the right corner point of the overlapping parameter region $P^{20}_3 \cap P^{20}_4$ collides with the upper right corner of $P^{20}_3$, the upper boundary of $P^{20}_3$ stops crossing the lower boundary $P^{20}_4$ and the overlapping parameter region $P^{20}_3 \cap P^{20}_4$ has four boundaries instead of three.
}

\Cref{fig:add.change.disb.cob.A} \hl{shows a cobweb diagram of the coexisting ``type A'' cycles in the emerging overlapping parameter region $P^{20}_3 \cap P^{20}_4$}.
\hl{
	We can see that the cycles are very close to colliding $d_1$ and $d_3$.
}
\hl{The same is true for the coexisting ``type B'' twin cycles in the cobweb diagram in} \Cref{fig:add.change.disb.cob.B}.

\subsection{Appearance of Period-adding structures}
\label{sec:add.change.appa}

In this section we will explore the appearance of the period-adding structures in between the chains of the same period.
This happens at the horizontal boundaries between ``type A'' parameter regions of different chains, as well as at the vertical boundaries.
We will first take care of the horizontal period-adding structures and then move on to the vertical period-adding structures.

\subsubsection{Horizontal Period-adding Structures}
\label{sec:add.change.appa.hor}

In \Cref{fig:add.change.regions.1}, the ``type A'' parameter regions $P^{20}_3$ and $P^{18}_3$, as well as $P^{22}_4$ and $P^{20}_4$ overlap.
This changes in \Cref{fig:add.change.regions.2}.
Here only the ``type A'' parameter regions $P^{20}_3$ and $P^{18}_3$ overlap, the parameter regions $P^{22}_4$ and $P^{20}_4$ stopped overlapping.
Instead, in the space between the two ``type A'' parameter region there are now two asymmetric coexisting twin cycles $\Cycle{\A^8\B^3\C^8\D^2}$ and $\Cycle{\A^8\B^2\C^8\D^3}$.
Those cycles are \textbf{not} ``type B'' cycles, because they only differ in the number of points on the branches $f_\B$ and $f_\D$.
Instead, we will call them hybrid cycles and ``type B'' cycles are a special case of hybrid cycles.
The notation $\left[P^{22}_4 \mid P^{20}_4\right]$ used in the diagrams was introduced in \Cref{sec:add.change} and is formally defined later in \Cref{sec:add.add.halved}.
Later in \Cref{fig:add.change.regions.4}, the ``type A'' parameter regions $P^{20}_3$ and $P^{18}_3$ also stop overlapping.
In between, there are also hybrid cycles, $\Cycle{\A^7\B^3\C^6\D^3}$ and $\Cycle{\A^6\B^3\C^7\D^3}$.
This parameter region is therefore labeled $\left[P^{20}_3 \mid P^{18}_3\right]$.

In between \Cref{fig:add.change.regions.2,fig:add.change.regions.3}, the ``type A'' parameter regions $P^{22}_4$ and $P^{20}_4$ don't stop overlapping completely.
Instead, they only stop overlapping on the left side of their shared boundaries and the rest is not pictured in these diagrams.
\Cref{fig:add.change.appa.hor.regions} shows better what happens to this overlapping region between \Cref{fig:add.change.regions.1,fig:add.change.regions.4}.
We also have a codimension-2 point that moves right as was the case in \Cref{sec:add.change.disb}.

\todo{Regions: labels wrong}
\todo{Cobwebs: enhance borders, replace (c), wrong pic}
\begin{figure}
	\centering
	\subfloat[Regions]{
		\includegraphics[width=.3 \textwidth]{62_MinimalRepr_Adding/2D_Regions_2.8_add_hor/Manual/result.png}
		\label{fig:add.change.appa.hor.regions}
	}
	\subfloat[At point $A$]{
		\includegraphics[width=.3 \textwidth]{62_MinimalRepr_Adding/Cob_2.8_add_hor_A/Manual/result.png}
		\label{fig:add.change.appa.hor.cob.A}
	}
	\subfloat[At point $B$]{
		\includegraphics[width=.3 \textwidth]{62_MinimalRepr_Adding/Cob_2.8_add_hor_A/Manual/result.png}
		\label{fig:add.change.appa.hor.cob.B}
	}
	\caption{Appearance of the horizontal period-adding cascade}
\end{figure}

We know from \Cref{sec:arch.bif.sum} that the \gls{bcb} at the upper boundary of the ``type A'' parameter region $P^{22}_4$ is $\BCB_{d_1, d_3}^{\underline{\A}^7\B^4\underline{\C}^7\D^4}$.
And the \gls{bcb} at the lower boundary of the ``type A'' parameter region $P^{20}_4$ is $\BCB_{d_1, d_3}^{\A^6\underline{\B}^4\C^6\underline{\D}^4}$.
Both these \glspl{bcb} are at the upper and lower boundaries of the overlapping region $P^{22}_4 \cup P^{20}_4$.
At the codimension-2 point, both these \glspl{bcb} happen at the same time and both cycles vanish.
We can see in \Cref{fig:add.change.appa.hor.cob.A} that the ``type A'' cycles are very close to the borders $d_1$ and $d_3$, respectively.

This codimension-2 point moves right with higher values for $b_L$ along our line.
As soon as the codimension-2 point crosses the right boundary of either the ``type A'' parameter region $P^{22}_4$ or $P^{20}_4$, the overlapping parameter region $P^{22}_4 \cup P^{20}_4$ ceases to exist.
Instead, there is space between the two ``type A'' parameter regions where there are now two hybrid cycles and period-adding between the hybrid cycles and either ``type A'' parameter region.

\todo{Labels for bifurcations missing underline}
\begin{figure}
	\centering
	\includegraphics[width=.7 \textwidth]{62_MinimalRepr_Adding/1D_Bif_2.8_add_hor_AU/Manual/result.png}
	\caption{Bifurcation diagram at the upper boundary of $\left[P^{22}_4 \mid P^{20}_4\right]$}
	\label{fig:add.change.appa.hor.bif}
\end{figure}

We assume that the \glspl{bcb} bounding the parameter regions with hybrid cycles follow the same rules as the \glspl{bcb} bounding the ``type B'' parameter regions.
\Cref{fig:add.change.appa.hor.bif} confirms this for the upper boundary.
So it is bounded at the top by the \glspl{bcb} $\BCB_{d_1}^{\underline{\A}^7\B^4\C^6\D^4}$ and $\BCB_{d_3}^{\A^6\B^4\underline{\C}^7\D^4}$.
And bounded at the bottom by the \glspl{bcb} $\BCB_{d_3}^{\A^7\B^4\C^6\underline{\D}^4}$ and $\BCB_{d_1}^{\underline{\A}^6\B^4\C^7\D^4}$.
At the codimension-2 point, both \glspl{bcb} $\BCB_{d_1}^{\underline{\A}^7\B^4\C^6\D^4}$ and $\BCB_{d_3}^{\A^7\B^4\C^6\underline{\D}^4}$ happen to the cycle $\Cycle{\A^7\B^4\C^6\D^4}$ at the same time and it vanishes.
Because of the symmetry, the \glspl{bcb} $\BCB_{d_3}^{\A^6\B^4\underline{\C}^7\D^4}$ and $\BCB_{d_1}^{\A^6\underline{\B}^4\C^7\D^4}$ happen to the cycle $\Cycle{\A^6\B^4\C^7\D^4}$ at the same time and it vanishes also.

\todo{also confirmed by cobweb with enhanced cycles at borders}

How this overlapping parameter regions disappears is similar to how the overlapping parameter region appears in \Cref{sec:add.change.disb}.
There, the codimension-2 point removes the ``type B'' parameter region and opens the overlapping parameter region.
Here, the codimension-2 point removes the overlapping parameter region and opens the parameter region with hybrid cycles, which behave very similar to ``type B'' cycles.

\subsubsection{Vertical Period-adding Structures}
\label{sec:add.change.appa.vert}

For the vertical period-adding structures, we could not find a codimension-2 point like for \Cref{sec:add.change.disb,sec:add.change.appa.hor}.
In \Crefrange{fig:add.change.regions.1}{fig:add.change.regions.2}, the parameter regions $P^{20}_3$ and $P^{22}_4$, as well as $P^{18}_3$ and $P^{20}_4$ overlap.
The appearance of the parameter region $\left[P^{20}_3 \mid P^{22}_4\right]$ in between $P^{20}_3$ and $P^{22}_4$ seems to happen at the same time as the appearance of the hybrid parameter region $\left[P^{18}_3 \mid P^{20}_4\right]$ in between $P^{18}_3$ and $P^{20}_4$.
At some parameter values on the line between \Cref{fig:add.change.regions.3,fig:add.change.regions.4}.
And with these hybrid parameter regions the vertical period-adding-like structures between them and the neighboring ``type A'' parameter regions.

\Cref{fig:add.change.appb.regions.A,fig:add.change.appb.regions.B} show this transition again for the parameter regions $P^{20}_3$ and $P^{22}_4$.
The transition might be similar to the previous \Cref{sec:add.change.disb,sec:add.change.appa.hor} with a codimension-2 point that moves up or down, or it could happen at once with the left boundary of $P^{22}_4$ and the right boundary of $P^{20}_3$ being aligned perfectly for some parameter values of $a_L$ and $b_L$ on the line given by \Cref{equ:add.change.paramline}.

\todo{Regions: labels wrong}
\todo{Cobwebs: replace (c) and enhance cycles at borders}
\begin{figure}
	\centering
	\subfloat[Regions scan before period-adding\\at $a_L = 2.8, b_L = -0.1$]{
		\includegraphics[width=.4 \textwidth]{62_MinimalRepr_Adding/2D_Regions_2.8_add_vert/Manual/result.png}
		\label{fig:add.change.appa.vert.regions.A}
	}
	\subfloat[Regions with period-adding\\at $a_L = 2.65, b_L = -0.05$]{
		\includegraphics[width=.4 \textwidth]{62_MinimalRepr_Adding/2D_Regions_2.65_add_vert/Manual/result.png}
		\label{fig:add.change.appa.vert.regions.B}
	} \\
	\subfloat[At point $A$]{
		\includegraphics[width=.4 \textwidth]{62_MinimalRepr_Adding/Cob_2.8_add_vert_A/Manual/result.png}
		\label{fig:add.change.appa.vert.cobweb.A}
	}
	\subfloat[At point $B$]{
		\includegraphics[width=.4 \textwidth]{62_MinimalRepr_Adding/Cob_2.65_add_vert_B/Manual/result.png}
		\label{fig:add.change.appa.vert.cobweb.B}
	}
	\caption{Appearance of the vertical period-adding cascade}
\end{figure}

\Cref{fig:add.change.appa.vert.cobweb.A} shows the coexistence of the two coexisting ``type A'' cycles while the ``type A'' parameter regions still overlap.
\Cref{fig:add.change.appa.vert.cobweb.B} shows the coexistence of the two coexisting hybrid cycles in the newly created parameter region $\left[P^{20}_3 \mid P^{22}_4\right]$.

\begin{figure}
	\centering
	\includegraphics[width=.7 \textwidth]{62_MinimalRepr_Adding/1D_Bif_2.65_add_vert_BR/Manual/result.png}
	\caption{Bifurcation diagram of the right boundary of $P_{10}^3 \oplus P_{11}^4$}
	\label{fig:add.appa.vert.bif}
\end{figure}

\todo{old:}

\todo{odd number of splits => no neg slope needed for asymmetry. odd number => needed (reorders cycles)}

