\section{Adding in Between Chains of the Same Period}

\Cref{fig:minrep.adding1.overview} shows both the periods of the model initially, as described in the previous chapter (\Cref{chap:minrep}), as well as the model with period-adding structures.
For \Cref{fig:minrep.adding1.overview.adding}, only the values of the fixed parameters $a_L = 1, b_L = 0.5$ are changed.
Initially, the values for these parameters were $a_L = 4, b_L = -0.5$.
The only other fixed parameter $B$ stays the same and the parameter ranges of $p_x$ and $p_y$ were adjusted slightly.
That means, that the shape of the function only changed on the branches $f_{\A}$ and $f_{\C}$.

When comparing both figures, we can see that there are no ``type B'' period regions in \Cref{fig:minrep.adding1.overview.adding}, the period-adding situation.
Instead, it looks like the ``type A'' period regions of the same period overlap now.
Also, the regions of higher periods in between the chains are new, these are the period-adding regions.
At the points, where these regions make a turn, there are period regions of even higher periods.
The meaning of these is explored in a later section, the next section will focus on the disappearance of the ``type B'' parameter regions.

\begin{figure}
    \centering
    \subfloat[Initial situation]{
        \includegraphics[width=.5 \textwidth]{62_MinimalRepr_Adding/2D_Period_4/result.png}
    }
    \subfloat[Period-adding]{
        \includegraphics[width=.5 \textwidth]{62_MinimalRepr_Adding/2D_Period_1/result.png}
        \label{fig:minrep.adding1.overview.adding}
    }
    \caption{Overview of period-adding structures in between chains of the same period}
    \label{fig:minrep.adding1.overview}
\end{figure}

\input{Sections/70_Adding/11_disapp_b.tex}
\subsection{Appearance of Period Adding}

