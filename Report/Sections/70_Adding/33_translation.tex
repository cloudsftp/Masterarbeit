\subsection{Translating Symbolic Sequences}
\label{sec:add.halved.tanslating}

First, we define naive algorithms for translating symbolic sequences between the archetypal and the halved archetypal model based on the idea of the lifted model.
\hl{
	Then, we derive regularities for translated symbolic sequences in the archetypal model.
}
Finally, we refine the algorithms \hl{using the regularities of translated symbolic sequences}.

In the following, cycles of the archetypal model get the symbols $\phi, \psi,$ and $\pi$.
Cycles of the halved archetypal model get the symbols $\sigma, \rho,$ and $\tau$.
If there are two coexisting twin cycles, they are written as $\phi^a$ and $\phi^b$.
Also, the symbols for the symbolic sequences in the halved archetypal model are $\L$ and $\R$ instead of $\A, \B, \C,$ and $\D$ to avoid confusion.

\subsubsection{Naive Algorithms}

We start \hl{by} formulating a naive algorithm for translating symbolic sequences from the archetypal model to the halved archetypal model.
This is the easier direction.
From this algorithm we don't learn much about the nature of the \gls{pal} structures in the archetypal model.
The algorithm for translating symbolic sequences in the other direction, from the halved archetypal model to the archetypal model, will be more important for that.

To translate a symbolic sequence of the archetypal model we start by writing it down.
For example, \hl{let} $\phi = \A^4\B^3\C^4\D^3$.
Then we replace the symbols $\A$ and $\C$ by $\L$ and the symbols $\B$ and $\D$ by $\R$.
Now we have $\L^4\R^3\L^4\R^3$.
Finally, we have to check for redundancy in the resulting cycle.
In our example, the cycle $\L^4\R^3$ repeats twice in $\L^4\R^3\L^4\R^3$, so we just keep $\L^4\R^3$.

The opposite direction is trickier.
We start by writing down the symbolic sequence in the halved archetypal model.
For example $\sigma = \L^4\R^3\L4\R^3\L^3\R^3$.
Now we need to build pairs of rotations since each blue block fits exactly two red blocks.
If there is one rotation left over at the end, we wrap around or equivalently write down the original sequence again.
We repeat this until all rotations we have written down are paired up.

\begin{lemma}[How Often to Write Down the Symbolic Sequence]
	\label{lemma:writing.down}
	For cycles in the halved archetypal model $\sigma$ with an even number of rotations $n$, we only need to write the original cycle down once.
	For cycles in the halved archetypal model $\sigma$ with an odd number of rotations $n$, we need to write the original cycle down exactly twice.
\end{lemma}

\clearpage

\begin{proof} \phantom{x}
	\begin{enumerate}
		\item Let $n = 2i$. Then, we can build $i$ pairs of rotations and fit all $2i$ rotations of the original model.
		\item Let $n = 2i + 1$. We start by building $i$ pairs of rotations, fitting $2i$ rotations.
		      This will leave the last rotation unpaired, so we write down the sequence of $2i + 1$ rotations again.
		      Now we can pair up the last rotation of the first sequence we wrote down with the first rotation of the sequence we just wrote down.
		      $2i$ rotations remain, which we can fit into $i$ pairs.
	\end{enumerate}
	\hfill $\blacksquare$
\end{proof}

Notice that our example symbolic sequence has 3 rotations.
This means we have to write down the original sequence twice $\sigma^2 = \L^4\R^3\L4\R^3\L^3\R^3\L^4\R^3\L4\R^3\L^3\R^3$.

Then we pair up the rotations, this corresponds to drawing blue boxes around the red boxes in the infinite model.
In our example, we get the pairs $\left(\L^4\R^3\L^4\R^3\right)\left(\L^3\R^3\L^4\R^3\right)\left(\L^4\R^3\L^3\R^3\right)$.
The pairs then have to be translated using the function $t$ defined below in \Cref{def:t}.
The resulting symbolic sequence is $T(h) = \A^4\B^3\C^4\D^3\A^3\B^3\C^4\D^3\A^4\B^3\C^3\D^3$.
The formal definition of $T$ is below in \Cref{def:T}.

\begin{definition}[Syllables]
	A syllable is a maximal sequence of the same symbol of a symbolic sequence.
	A 2-syllable is a pair of syllables that are next to each other.
	And a 4-syllable is a pair of 2-syllables that are next to each other.
\end{definition}

So for example, the symbolic sequence $\L^4\R^3\L^3\R^3$ has 4 syllables.
Its syllables are $\L^4$, $\L^3$, and two times $\R^3$.
And a 2-syllable of the cycle is $\L^4\R^3$.
In the halved archetypal model, a 2-syllable corresponds to one rotation.
The 2-syllables of a cycle $\sigma$ in the halved archetypal model are written as $\sigma_i$.
In the archetypal model, a 4-syllable corresponds to one rotation.
The 4-syllables of a cycle $\phi$ in the archetypal model are written as $\phi_i$.
These terms are used interchangeably in the rest of this chapter.

\begin{definition}[Translating 4-syllables from Halved Archetypal to Archetypal]
	\label{def:t}
	The function $t$ maps two rotations, or a 4-syllable that starts with the symbol $\L$ of a symbolic sequence in the halved archetypal model to a single rotation in the archetypal model.
	It is defined in the following way.
	\begin{align}
		t: & \L^a\R^b\L^c\R^d \mapsto \A^a\B^b\C^c\D^d
	\end{align}
\end{definition}

\begin{definition}[Translating Symbolic Sequences from Halved Archetypal to Archetypal]
	\label{def:T}
	The function $T$ translates a symbolic sequence $\sigma = \sigma_1\sigma_2 \dots \sigma_n$ in the halved archetypal model to the archetypal model.
	Where $\sigma_i$ are the 2-syllables of $\sigma$.
	From \Cref{lemma:writing.down} we know how often we need to write down $\sigma$, and therefore also which 4-syllables to translate with $t$.
	\begin{align}
		T(\sigma) & = \begin{cases}
			              t(\sigma_1\sigma_2) \dots t(\sigma_{n-1}\sigma_n)                           & \text{ if } n | 2 \\
			              t(\sigma_1\sigma_2) \dots t(\sigma_n\sigma_1) \dots t(\sigma_{n-1}\sigma_n) & \text{ else }
		              \end{cases}
	\end{align}
\end{definition}

\begin{definition}[Shifting Symbolic Sequences]
	The function $s_2$ shifts a symbolic sequence $\sigma$ in the halved archetypal model by a single rotation, or equivalently by a 2-syllable.
	Let $\sigma = \sigma_1\sigma_2 \dots \sigma_n$.
	Then $s_2$ is defined in the following way.
	\begin{align}
		s_2: & \sigma_1\sigma_2 \dots \sigma_n \mapsto \sigma_2 \dots \sigma_n\sigma_1
	\end{align}
	In the archetypal model, there is a similar function, $s_4$ that shifts a symbolic sequence $\phi$ in the archetypal model by a single rotation.
	Let $\phi = \phi_1\phi_2 \dots \phi_n$.
	Then $s_4$ is defined in the following way.
	\begin{align}
		s_4: & \phi_1\phi_2 \dots \phi_n \mapsto \phi_2 \dots \phi_n\phi_1
	\end{align}
\end{definition}

\begin{definition}[Shift-equivalence]
	The two symbolic sequences $\phi$ and $\psi$ in the archetypal model are shift-equivalent $\phi \equiv \psi$,
	if they both have the same number of rotations $n$
	and there is a number $0 \leq i < n$, such that $\phi = s_4^i(\psi)$.
	Where $s_4^i(\psi)$ is defined as $s_4^{i-1}(s_4(\psi))$ and $s_4^1(\psi) = s_4(\psi)$.
\end{definition}

We need to repeat the whole process for each shift $s_2^i$ of the original symbolic sequence for $0 < i < n$ where $n$ is the number of 2-syllables of the original symbolic sequence.
And we only keep the results that are not shift-equivalent to any previous result, since shift-equivalent cycles are identical in the archetypal model.
In our example, we would repeat the process for $s_2(\sigma) = \L^4\R^3\L3\R^3\L^4\R^3$ and get the result $T(s_2(\sigma)) = \A^4\B^3\C^3\D^3\A^4\B^3\C^4\D^3\A^3\B^3\C^4\D^3$.
This result is shift-equivalent to the first result by shifting it 2 times ($s_4^2$).
Last we need to repeat it for $s_2^2(\sigma) = \L^3\R^3\L4\R^3\L^4\R^3$ and get the result $T(s_2^2(\sigma)) = \A^3\B^3\C^4\D^3\A^4\B^3\C^3\D^3\A^4\B^3\C^4\D^3$.
This result is shift-equivalent to the first result by shifting it once ($s_4$).

Therefore, the cycle $\sigma$ in the halved archetypal model manifests as a single cycle $T(\sigma)$ in the archetypal model.
We write it as $F(\sigma) = \{T(\sigma)\} = \{\A^4\B^3\C^4\D^3\A^3\B^3\C^4\D^3\A^4\B^3\C^3\D^3\}$.
The result of $F$ is a set because the cycle $\sigma$ in the halved archetypal model may manifest as multiple coexisting cycles in the archetypal model.

\subsubsection{Properties of Translated Symbolic Sequences in the Full Model}

With this naive algorithm, we can start to investigate rules for the \gls{pal} structures in the archetypal model.

\begin{lemma}[Shif-equivalence of Translated Symbolic Sequences in the Archetypal Model]
	\label{lemma:equivalence.translations}
	The translations of the two cycles $\sigma$ and $\rho = s_2^{2i}(\sigma)$ in the halved archetypal model are shift-equivalent $T(\sigma) \equiv T(\rho)$ in the archetypal model for all integers $i$.
\end{lemma}

\begin{proof} \phantom{x} \\
	Let $\sigma = \sigma_1\sigma_2 \dots \sigma_n$, therefore $\rho = \sigma_{2i+1} \dots \sigma_n\sigma_1 \dots \sigma_{2i}$.
	The translations are $T(\sigma) = t(\sigma_1\sigma_2)t(\sigma_3\sigma_4) \dots t(\sigma_{n-1}\sigma_n)$
	and $T(\rho) = t(\sigma_{2i+1}\sigma_{2i+2}) \dots t(\sigma_{n-1}\sigma_n)t(\sigma_1\sigma_2) \dots t(\sigma_{2i-i}\sigma_{2i})$.
	We can see that $T(\rho) = s_4^i(T(\sigma))$ and therefore $T(\sigma) \equiv T(\rho)$.

	\hfill $\blacksquare$
\end{proof}

\clearpage

\begin{theorem}[Coexistence of Translated Symbolic Sequences in the Archetypal Model]
	\label{theorem:coexistence.even}
	The manifestations of a cycle in the halved archetypal model $\sigma$ is either $F(\sigma) = \{T(\sigma), T(s_2(\sigma))\}$ or $F(\sigma) = \{T(\sigma)\}$.
	And \begin{enumerate}
		\item $F(\sigma) = \{T(\sigma), T(s_2(\sigma))\}$ if the number of rotations of the sequence $\sigma$ is even.
		\item $F(\sigma) = \{T(\sigma)\}$ if the number of rotations of the sequence $\sigma$ is odd.
	\end{enumerate}
\end{theorem}

\begin{proof} \phantom{x} \\
	Let $\sigma = \sigma_1\sigma_2 \dots \sigma_n$ a symbolic sequence in the halved archetypal model with $n$ rotations.
	We know from \Cref{lemma:equivalence.translations} that the only possible candidates for $F(\sigma)$ are $T(\sigma)$ and $T(s_2(\sigma))$.
	These are the first two possibilities we check in the algorithm and all other shifts $T(s_2^i(\sigma))$ with $2 \leq i < n$ are shift-equivalent to either $T(\sigma)$ or $T(s_2(\sigma))$.
	This follows directly from \Cref{lemma:equivalence.translations}.
	So, in the following, we will only check for the shift-equivalence of these two candidates.
	\begin{enumerate}
		\item Let $n = 2i$.
		      \begin{align*}
			      T(h)                 & = t(\sigma_1\sigma_2) t(\sigma_3\sigma_4) \dots t(\sigma_{n-1}\sigma_n) \\
			      \nequiv \: T(s_2(h)) & = t(\sigma_2\sigma_3) t(\sigma_4\sigma_5) \dots t(\sigma_n\sigma_1)
		      \end{align*}
		      The two candidates are not shift-equivalent because the pair $t(\sigma_1\sigma_2)$ in $T(\sigma)$ is not included in the other candidate $T(s_2(\sigma))$.
		      The same is true for any other pair, and therefore $F(\sigma) = \{T(\sigma), T(s_2(\sigma))\}$.
		\item Let $n = 2i + 1$.
		      \begin{align*}
			      T(h)                & = t(\sigma_1\sigma_2) t(\sigma_3\sigma_4) \dots t(\sigma_n\sigma_1) t(\sigma_2\sigma_3) \dots t(\sigma_{n-1}\sigma_n) \\
			      \equiv \: T(s_2(h)) & = t(\sigma_2\sigma_3) \dots t(\sigma_{n-1}\sigma_n) t(\sigma_1\sigma_2) t(\sigma_3\sigma_4) \dots t(\sigma_n\sigma_1)
		      \end{align*}
		      The two candidates are shift-equivalent.
		      By shifting the second candidate $T(s_2(\sigma))$ $2i$ times, we get the first candidate $T(\sigma)$.
		      Therefore, the second candidate is discarded and $F(\sigma) = \{T(\sigma)\}$.
	\end{enumerate}
	\hfill $\blacksquare$
\end{proof}

Note that a result of $F(\sigma) = \{T(\sigma), T(s_2(\sigma))\}$ means that the cycle $\sigma$ in the halved archetypal model manifests as two coexisting cycles in the archetypal model.

\subsubsection{Revised Algorithms}

With all these properties and functions we now can formulate a more compact algorithm, \Cref{alg:halved.to.full}, for translating symbolic sequences from the halved archetypal model into the archetypal model.
This revised algorithm will be used in the following to construct the rules for the \gls{pal} structures in the archetypal model.

\begin{algorithm}
	\caption{Algorithm for the Translation of Symbolic Sequences from the Halved Archetypal Model to the Archetypal Model}\label{alg:halved.to.full}
	\begin{algorithmic}
		\Require $\sigma = \sigma_1\sigma_2 \dots \sigma_n$ with $n > 0$
		\If{$n$ is even}
		\State \Return $\{t(\sigma_1\sigma_2) t(\sigma_3\sigma_4) \dots t(\sigma_{n-1}\sigma_n), t(\sigma_2\sigma_3) t(\sigma_4\sigma_5) \dots t(\sigma_n\sigma_1)\}$
		\ElsIf{$n$ is odd}
		\State \Return $\{t(\sigma_1\sigma_2) \dots t(\sigma_{n}\sigma_1) \dots t(\sigma_{n-1}\sigma_n)\}$
		\EndIf
	\end{algorithmic}
\end{algorithm}

\Cref{alg:full.to.halved} shows the inverse algorithm for translating symbolic sequences from the archetypal model to the halved archetypal model for completeness.
It uses the inverse $t^{-1}$ of the function $t$.

\begin{definition}[Translating 4-syllables from Archetypal to Halved Archetypal]
	The function $t^{-1}$ maps a 4-syllable of a symbolic sequence in the archetypal model to the halved archetypal model.
	It is defined in the following way.
	\begin{align}
		t^{-1}: \quad \A^a\B^b\C^c\D^d \mapsto \L^a\R^b\L^c\R^d
	\end{align}
\end{definition}

\begin{algorithm}
	\caption{Algorithm for the Translation of Symbolic Sequences from the Halved Model to the Halved Archetypal Model}\label{alg:full.to.halved}
	\begin{algorithmic}
		\Require $\phi = \phi_1\phi_2 \dots \phi_n$ with $n > 0$
		\State $d \gets t^{-1}(\phi_1)t^{-1}(\phi_2) \dots t^{-1}(\phi_n) = \sigma_1\sigma_2 \dots \sigma_m$
		\Comment $m = 2n$ is even
		\State $\tau \gets \sigma_1\sigma_2 \dots \sigma_{\frac{m}{2}}$
		\If{$\sigma = \tau^2$}
		\State \Return $\tau$
		\ElsIf{$\sigma \neq \tau^2$}
		\State \Return $\sigma$
		\EndIf
	\end{algorithmic}
\end{algorithm}
