\subsection{Appearance of Period-adding-like structures}
\label{sec:add.change.appa}

In this section we will explore the appearance of the period-adding-like structures in between the chains of the same period.
This happens at the horizontal boundaries between ``type A'' parameter regions of different chains, as well as at the vertical boundaries.
We will first take care of the horizontal period-adding-like structures and then move on to the vertical period-adding-like structures.

\subsubsection{Horizontal Period-adding-like Structures}
\label{sec:add.change.appa.hor}

In \Cref{fig:add.change.regions.1}, the ``type A'' parameter regions $P^{20}_3$ and $P^{18}_3$, as well as $P^{22}_4$ and $P^{20}_4$ overlap.
This changes in \Cref{fig:add.change.regions.2}.
Here only the ``type A'' parameter regions $P^{20}_3$ and $P^{18}_3$ overlap, the parameter regions $P^{22}_4$ and $P^{20}_4$ stopped overlapping.
Instead, in the space between the two ``type A'' parameter region there are now two asymmetric coexisting twin cycles $\Cycle{\A^8\B^3\C^8\D^2}$ and $\Cycle{\A^8\B^2\C^8\D^3}$.
Those cycles are \textbf{not} ``type B'' cycles, because they only differ in the number of points on the branches $f_\B$ and $f_\D$.
Instead, we will call them hybrid cycles and ``type B'' cycles are a special case of hybrid cycles.
The notation $\left[P^{22}_4 \mid P^{20}_4\right]$ used in the diagrams was introduced in \Cref{sec:add.change.num} and is formally defined later in \Cref{sec:add.add.halved}.
Later in \Cref{fig:add.change.regions.4}, the ``type A'' parameter regions $P^{20}_3$ and $P^{18}_3$ also stop overlapping.
In between, there are also hybrid cycles, $\Cycle{\A^7\B^3\C^6\D^3}$ and $\Cycle{\A^6\B^3\C^7\D^3}$.
This parameter region is therefore labeled $\left[P^{20}_3 \mid P^{18}_3\right]$.

In between \Cref{fig:add.change.regions.2,fig:add.change.regions.3}, the ``type A'' parameter regions $P^{22}_4$ and $P^{20}_4$ don't stop overlapping completely.
Instead, they only stop overlapping on the left side of their shared boundaries and the rest is not pictured in these diagrams.
\Cref{fig:add.change.appa.hor.regions} shows better what happens to this overlapping region between \Cref{fig:add.change.regions.1,fig:add.change.regions.4}.
We also have a codimension-2 point that moves right as was the case in \Cref{sec:add.change.disb}.

\begin{figure}
	\centering
	\subfloat[Regions]{
		\includegraphics[width=.3 \textwidth]{62_MinimalRepr_Adding/2D_Regions_2.8_add_hor/Manual/result.png}
		\label{fig:add.change.appa.hor.regions}
	}
	\subfloat[At point $A$]{
		\includegraphics[width=.3 \textwidth]{62_MinimalRepr_Adding/Cob_2.8_add_hor_A/Manual/result.png}
		\label{fig:add.change.appa.hor.cob.A}
	}
	\subfloat[At point $B$]{
		\includegraphics[width=.3 \textwidth]{62_MinimalRepr_Adding/Cob_2.8_add_hor_B/Manual/result.png}
		\label{fig:add.change.appa.hor.cob.B}
	}
	\caption[2D boundary scan and cobweb diagrams showing the appearance of horizontal period-adding-like structures in the archetypal model]{
		2D boundary scan and cobweb diagrams showing the appearance of horizontal period-adding-like structures in the archetypal model.
		The parameters $a_L = 2.8, b_L = -0.1,$ and $g_R\left(\frac{1}{2}\right) = \frac{1}{2} + \frac{1}{40}$ are fixed.
		(a) shows a boundary scan of parameter regions associated with different symbolic sequences.
		The parameters $\alpha = g_R\left(\frac{1}{4}\right)$ and $\beta = c_L$ are varied.
		(b) and (c) show the cobweb diagrams at the parameter values marked with the points $A$ and $B$ in (a).
	}
\end{figure}

We know from \Cref{sec:arch.bif.sum} that the \gls{bcb} at the upper boundary of the ``type A'' parameter region $P^{22}_4$ is $\BCB_{d_1, d_3}^{\underline{\A}^7\B^4\underline{\C}^7\D^4}$.
And the \gls{bcb} at the lower boundary of the ``type A'' parameter region $P^{20}_4$ is $\BCB_{d_1, d_3}^{\A^6\underline{\B}^4\C^6\underline{\D}^4}$.
Both these \glspl{bcb} are at the upper and lower boundaries of the overlapping region $P^{22}_4 \cup P^{20}_4$.
At the codimension-2 point, both these \glspl{bcb} happen at the same time and both cycles vanish.
We can see in \Cref{fig:add.change.appa.hor.cob.A} that the ``type A'' cycles are very close to the borders $d_1$ and $d_3$, respectively.

This codimension-2 point moves right with higher values for $b_L$ along the line given by \Cref{equ:add.change.paramline}.
As soon as the codimension-2 point crosses the right boundary of either the ``type A'' parameter region $P^{22}_4$ or $P^{20}_4$, the overlapping parameter region $P^{22}_4 \cup P^{20}_4$ ceases to exist.
Instead, there is space between the two ``type A'' parameter regions where there are now two hybrid cycles and period-adding-like between the hybrid cycles and their neighboring ``type A'' parameter regions.

\begin{figure}
	\centering
	\subfloat[Lower boundary]{
		\includegraphics[width=.45 \textwidth]{62_MinimalRepr_Adding/1D_Bif_2.8_add_hor_AD/Manual/result.png}
		\label{fig:add.change.appa.hor.bif.lower}
	}
	\subfloat[Upper boundary]{
		\includegraphics[width=.45 \textwidth]{62_MinimalRepr_Adding/1D_Bif_2.8_add_hor_AU/Manual/result.png}
		\label{fig:add.change.appa.hor.bif.upper}
	}
	\caption[Bifurcation diagrams for the horizontal hybrid parameter regions in the increasing archetypal model]{
		Bifurcation diagrams at the horizontal boundaries of the horizontal hybrid parameter region $\left[P^{22}_4 \mid P^{20}_4\right]$.
		The parameters $a_L = 2.8, b_L = -0.1, g_R\left(\frac{1}{2}\right) = \frac{1}{2} + \frac{1}{40},$ and $\alpha = g_R\left(\frac{1}{4}\right) = 0.366362$ are fixed.
		The parameter $\beta = c_L$ is varied.
		(a) shows the bifurcation diagram at the lower boundary, while (b) shows the bifurcation diagram at the upper boundary.
	}
	\label{fig:add.change.appa.hor.bif}
\end{figure}

The \glspl{bcb} at the upper and lower boundaries of the parameter region with hybrid cycles $\left[P^{22}_4 \mid P^{20}_4\right]$ are shown in \Cref{fig:add.change.appa.hor.bif}.
Therefore, the \glspl{bcb} at the upper boundary are $\BCB_{d_1}^{\underline{\A}^7\B^4\C^6\D^4}$ and $\BCB_{d_3}^{\A^6\B^4\underline{\C}^7\D^4}$.
At the lower boundary, the \glspl{bcb} are $\BCB_{d_3}^{\A^7\B^4\C^6\underline{\D}^4}$ and $\BCB_{d_1}^{\A^6\underline{\B}^4\C^7\D^4}$.
We can see that the \glspl{bcb} of the hybrid parameter regions follow similar rules as the \glspl{bcb} of ``type B'' parameter regions given in \Cref{sec:arch.bif.sum}.
At the codimension-2 point, both \glspl{bcb} $\BCB_{d_1}^{\underline{\A}^7\B^4\C^6\D^4}$ and $\BCB_{d_3}^{\A^7\B^4\C^6\underline{\D}^4}$ happen to the cycle $\Cycle{\A^7\B^4\C^6\D^4}$ at the same time and it vanishes.
Because of the symmetry, the \glspl{bcb} $\BCB_{d_3}^{\A^6\B^4\underline{\C}^7\D^4}$ and $\BCB_{d_1}^{\A^6\underline{\B}^4\C^7\D^4}$ happen to the cycle $\Cycle{\A^6\B^4\C^7\D^4}$ at the same time and it vanishes also.

How this overlapping parameter region disappears is similar to how the overlapping parameter region appears in \Cref{sec:add.change.disb}.
There, the codimension-2 point removes the ``type B'' parameter region and opens the overlapping parameter region.
Here, the codimension-2 point removes the overlapping parameter region and opens the parameter region with hybrid cycles, which behave very similar to ``type B'' cycles.

\subsubsection{Vertical Period-adding-like Structures}
\label{sec:add.change.appa.vert}

For the vertical period-adding-like structures, we could not find a codimension-2 point like for \Cref{sec:add.change.disb,sec:add.change.appa.hor}.
In \Crefrange{fig:add.change.regions.1}{fig:add.change.regions.2}, the parameter regions $P^{20}_3$ and $P^{22}_4$, as well as $P^{18}_3$ and $P^{20}_4$ overlap.
The appearance of the parameter region $\left[P^{20}_3 \mid P^{22}_4\right]$ in between $P^{20}_3$ and $P^{22}_4$ seems to happen at the same time as the appearance of the hybrid parameter region $\left[P^{18}_3 \mid P^{20}_4\right]$ in between $P^{18}_3$ and $P^{20}_4$, at some parameter values on the line given by \Cref{equ:add.change.paramline} between the parameter values of \Cref{fig:add.change.regions.3,fig:add.change.regions.4}.
And with these hybrid parameter regions the vertical period-adding-like structures between them and the neighboring ``type A'' parameter regions also appear.

\Cref{fig:add.change.appa.vert.regions.A,fig:add.change.appa.vert.regions.B} show this transition again for the parameter regions $P^{20}_3$ and $P^{22}_4$.
The transition might be similar to the previous \Cref{sec:add.change.disb,sec:add.change.appa.hor} with a codimension-2 point that moves up or down, or it could happen at once with the left boundary of $P^{22}_4$ and the right boundary of $P^{20}_3$ being aligned perfectly for some parameter values of $a_L$ and $b_L$ on the line given by \Cref{equ:add.change.paramline}.

\begin{figure}
	\centering
	\subfloat[With $a_L = 2.8, b_L = -0.1$]{
		\includegraphics[width=.4 \textwidth]{62_MinimalRepr_Adding/2D_Regions_2.8_add_vert/Manual/result.png}
		\label{fig:add.change.appa.vert.regions.A}
	}
	\subfloat[With $a_L = 2.65, b_L = -0.05$]{
		\includegraphics[width=.4 \textwidth]{62_MinimalRepr_Adding/2D_Regions_2.65_add_vert/Manual/result.png}
		\label{fig:add.change.appa.vert.regions.B}
	} \\
	\subfloat[At point $A$]{
		\includegraphics[width=.4 \textwidth]{62_MinimalRepr_Adding/Cob_2.8_add_vert_A/Manual/result.png}
		\label{fig:add.change.appa.vert.cobweb.A}
	}
	\subfloat[At point $B$]{
		\includegraphics[width=.4 \textwidth]{62_MinimalRepr_Adding/Cob_2.65_add_vert_B/Manual/result.png}
		\label{fig:add.change.appa.vert.cobweb.B}
	}
	\caption[2D boundary scans and cobweb diagrams showing the appearance of vertical period-adding-like structures in the archetypal model]{
		2D boundary scans and cobweb diagrams showing the appearance of vertical period-adding-like structures in the archetypal model.
		The parameter $g_R\left(\frac{1}{2}\right) = \frac{1}{2} + \frac{1}{40}$ is fixed for all diagrams.
		(a) and (b) show boundary scans of parameter regions associated with different symbolic sequences.
		The parameters $a_L$ and $b_L$ are fixed in each 2D boundary scan, with the values given in the sub-captions.
		The parameters $\alpha = g_R\left(\frac{1}{4}\right)$ and $\beta = c_L$ are varied.
		(c) and (d) show cobweb diagrams at the parameter values marked with points $A$ and $B$ in (a) and (b), respectively.
	}
\end{figure}

\Cref{fig:add.change.appa.vert.cobweb.A} shows the coexistence of the two coexisting ``type A'' cycles while the ``type A'' parameter regions still overlap.
\Cref{fig:add.change.appa.vert.cobweb.B} shows the coexistence of the two coexisting hybrid cycles in the newly created parameter region $\left[P^{20}_3 \mid P^{22}_4\right]$.

The \glspl{bcb} at the left and right boundary of the hybrid parameter region follow the same rules as the \glspl{bcb} ``type B'' parameter regions.
At the left boundary, the hybrid cycles collide with the borders $d_0$ and $d_1$ from the left at the same time, pictured in \Cref{fig:add.appa.vert.bif.left}.
And at the right boundary, they collide with the same borders from the right at the same time, pictured in \Cref{fig:add.appa.vert.bif.right}.
Note that at the left boundary, the \glspl{bcb} of the hybrid cycles are left of the \gls{bcb} of the ``type A'' cycle.
This causes all three cycles to coexist in the parameter region between the \glspl{bcb}.
At the right boundary on the other hand, the \glspl{bcb} of the hybrid cycles are also left of the \gls{bcb} of the ``type A'' cycle.
This causes a space between the hybrid parameter region $\left[P^{20}_3 \mid P^{22}_4\right]$ and the ``type A'' parameter region $P^{20}_3$ where a period-adding-like structure emerges.
The cycle of the first stage of this period-adding-like structure is visible in purple in \Cref{fig:add.appa.vert.bif.right}.

\begin{figure}
	\centering
	\subfloat[Left boundary]{
		\includegraphics[width=.45 \textwidth]{62_MinimalRepr_Adding/1D_Bif_2.65_add_vert_BL/Manual/result.png}
		\label{fig:add.appa.vert.bif.left}
	}
	\subfloat[Right boundary]{
		\includegraphics[width=.45 \textwidth]{62_MinimalRepr_Adding/1D_Bif_2.65_add_vert_BR/Manual/result.png}
		\label{fig:add.appa.vert.bif.right}
	}
	\caption[Bifurcation diagrams for the vertical hybrid parameter regions in the increasing archetypal model]{
		Bifurcation diagrams at the vertical boundaries of the vertical hybrid parameter region $\left[P^{22}_4 \mid P^{20}_4\right]$.
		The parameters $a_L = 2.65, b_L = -0.05, g_R\left(\frac{1}{2}\right) = \frac{1}{2} + \frac{1}{40},$ and $\beta = c_L = 0.124797$ are fixed.
		The parameter $\alpha = g_R\left(\frac{1}{4}\right)$ is varied.
		(a) shows the bifurcation diagram at the left boundary, while (b) shows the bifurcation diagram at the right boundary.
	}
	\label{fig:add.appa.vert.bif}
\end{figure}
