\subsection{The Halved Model}
\label{sec:add.add.halved}

The idea behind the halved model is that the archetypal model can be looked at differently, because it is a circle map.
Let $m$ be the archetypal model for this section.
We know the model $m$ maps an input $x$ to $f(x) \mod 1$, meaning that if the output $f(x)$ is greater or equal to 1 we subtract 1 from it until it is in the range $[0, 1)$.
Similarly, we add 1 to it if it is smaller than 0 until it is in the desired range.
Now instead of confining the model to the domain of $[0, 1)$, we think of it repeating infinitely in both directions.
This process is called lifting of circle maps and is described by \Citeauthor{devaney2021introduction} in his book~\cite{devaney2021introduction}.
We can achieve this by mapping $T^m: x \mapsto f(x - \lfloor x \rfloor)$.
This trick maps any input $x \in \mathbb{R}$ into the domain $[0, 1)$ of the archetypal model $m$ and causes $T^m$ to repeat infinitely.
$T^m$ is now a lift of the model $m$ in the domain of all real numbers $\mathbb{R}$.
\Cref{fig:add.halved.lift} illustrates this concept for the cycle in the parameter region $P^{14}_3$.
The blue square is the full model.
One can see, that the branch $f_\D$ is outside the blue square at its right edge.
This is because it was cut off and continued at the bottom of the square before, due to the $\mod 1$ operation.

\begin{figure}
	\centering
	\includegraphics[width=.7 \textwidth]{63_MinimalRepr_Adding_Halved/Cob_Vis_s/Manual/result.png}
	\caption{Illustration of the lifted archetypal model}
	\label{fig:add.halved.lift}
\end{figure}

In this model, there are no cycles that have multiple rotations.
Instead, the cycles that had multiple rotations in the full model, manifest as a sequence of different blocks of the full model.
Meaning for the example $P^{14}_3$, the same blocks of $\A^4\B^3\C^4\D^3$ are repeating infinitely.
But for an example with multiple rotations, such as $\A\B\C\D\A^2\B^2\C^2\D^2$, the blocks will not all be the same.
Instead, the blocks $\A\B\C\D$ and $\A^2\B^2\C^2\D^2$ will be alternating.

Now we will take advantage of the symmetry in the model function $f$ of the archetypal model.
Since $f(x + \frac{1}{2}) \equiv f(x) + \frac{1}{2} \mod 1$, we can split the lifted model $T^m$ into smaller blocks of size $\frac{1}{2}$.
The function of the infinite model repeats in these smaller blocks.
These blocks are marked red in \Cref{fig:add.halved.lift}.
The red blocks represent the halved model, it is the smallest repeating part of the lifted model $T^m$.
Basically we choose the smallest model, of which $T^m$ is a lift.
This happens to be exactly our model $m$ folded in half.
So the halved archetypal model maps $x \mapsto g(x) \mod \frac{1}{2}$, where $g(x)$ is the same as in the archetypal model defined in \Cref{sec:setup.quad.hybrid.definition}.

When we scan the same area in the halved archetypal model as we did in \Cref{fig:add.add.like.hor.2D} for the horizontal \gls{pal} structures, we get one big structure that looks like \gls{pa}.
This is shown in \Cref{fig:add.halved.hor.2D}.
The red arrow indicates the parameter range for the 1D period scan in \Cref{fig:add.halved.hor.1D}.
The 1D scan shows that the periods in this structure add up as we would expect in \gls{pa} structures.

\begin{figure}
	\centering
	\subfloat[2D period scan]{
		\includegraphics[width=.45 \textwidth]{63_MinimalRepr_Adding_Halved/2D_Period_add_zoom_hor/result.png}
		\label{fig:add.halved.hor.2D}
	}
	\subfloat[1D period scan]{
		\includegraphics[width=.45 \textwidth]{63_MinimalRepr_Adding_Halved/1D_Period_hor_low/result.png}
		\label{fig:add.halved.hor.1D}
	}
	\caption[2D and 1D period scans of a horizontal period-adding structure in the halved increasing archetypal model]{
		2D and 1D period scans of a horizontal \gls{pa} structure in the halved increasing archetypal model.
		The fixed parameters are $a_L = 1, b_L = 0.8,$ and $g_R\left(\frac{1}{2}\right) = \frac{1}{2} + \frac{1}{40}$.
		(a) shows the 2D period scan where the parameters $\alpha = g_R\left(\frac{1}{4}\right)$ and $\beta = c_L$ are varied.
		The small arrow indicates the parameter range for the 1D period scan in (b).
		Here, only $\beta$ is varied.
		The numbers at the top mark the periods at the corresponding value for $\beta$.
	}
	\label{fig:add.add.halved.hor}
\end{figure}

We also take a look at the symbolic sequences associated with the parameter regions of this structure to make sure that this is really \gls{pa}.
\Cref{fig:add:add.halved.hor.tree} shows the Farey-tree with the symbolic sequences associated with the parameter regions of this structure.
One can see that the symbolic sequence of a child node is the concatenation of the symbolic sequences of the parent nodes, as we would expect from \gls{pa} structures.
It turns out that the hybrid parameter region $\left[P^{14}_3 \mid P^{12}_3\right]$ was also part of the horizontal \gls{pal} structure described in \Cref{sec:add.add.like}.
And the \gls{pal} structures in the archetypal model are consequences of the \gls{pa} structures in the halved archetypal model.

\begin{figure}
	\centering
	\includegraphics[width=\textwidth]{FareyTrees/Minrep_Adding_larger_Halved_3/adding.png}
	\caption[Farey-tree with the symbolic sequences of a horizontal \glsentrylong{pa} structure]{
		Farey-tree with the symbolic sequences associated with the parameter regions of the horizontal \gls{pa} structure marked with a red arrow in \Cref{fig:add.halved.hor.2D} up to three levels.
		Nodes of parameter regions associated with two coexisting cycles are colored yellow.
	}
	\label{fig:add:add.halved.hor.tree}
\end{figure}

In the following section...
\todo{Translating}
\todo{Groundwork for rules}
