\section{Adjusting the Parameters}
\label{sec:add.parameters}

The fixed parameters in the archetypal are $a_L = 4, b_L = -\frac{1}{2},$ and $g_R\left(\frac{1}{2}\right) = \frac{1}{2} + \frac{1}{40}$.
Only the branches $f_\A$ and $f_\C$ are not monotonously increasing.
Of the three fixed parameters, $a_L$ and $b_L$ influence the shape of the branches $f_\A$ and $f_\C$.
These parameters are adjusted to make the branches $f_\A$ and $f_\C$ monotonously increasing.
The new parameter values are $a_L = 1$ and $b_L = \frac{1}{2}$.
The new shape of the function can be seen in \Cref{fig:add.arch.new}, now all branches are monotonously increasing.
We will refer to this model as the piecewise-increasing archetypal model.

\Cref{fig:add.arch.new.period} shows a 2D scan of the periods associated with parameter regions in the archetypal model with these new values for the fixed parameters stated above.
In this scan, we can see that the ``type B'' parameter regions disappeared and ``type A'' parameter regions of the same chain seem to start overlapping.
Also, in between the chains there are now small parameter regions with much higher periods.
These structures look like \glsentrylong{pa} structures.
And indeed, it is plausible for \gls{pa} structures to emerge in such a map.
\Citeauthor{simpson2018saw} demonstrated in his work \cite{simpson2018saw} that a \gls{pws} circle map with two linear and increasing branches can exhibit \gls{pa}.
\Cref{fig:add.saw} shows this map called skew sawtooth next to the archetypal model function with increasing branches.
The skew sawtooth map is continuous while the archetypal model function with increasing branches is not.
And the archetypal model function with increasing branches has quadratic branches while all branches in the skew sawtooth map are linear.
But they are somewhat similar as we can see in the comparison in \Cref{fig:add.saw.vs.arch}.

In the following sections these structures are explored.
They are referred to as \gls{pal} structures.
But first, we will take a closer look at how the bifurcations structures change when adjusting the parameters $a_L$ and $b_L$ to make the branches $f_\A$ and $f_\C$ increasing.

\begin{figure}
	\centering
	\includegraphics[width=.6 \textwidth]{../Figures/7/7.1/result.png}
	\caption[2D scan of the periods associated with parameter regions in the archetypal model with increasing branches]{
		2D scan of the periods associated with parameter regions in the archetypal model with increasing branches.
		The parameters $a_L = 1, b_L = \frac{1}{2},$ and $g_R\left(\frac{1}{2}\right) = \frac{1}{2} + \frac{1}{40}$ are fixed.
		The parameters $\alpha = -g_R\left(\frac{1}{4}\right)$ and $\beta = c_L$ are varied in the ranges $[-0.45, -0.3]$ and $[0.075, 0.12]$.
	}
	\label{fig:add.arch.new.period}
\end{figure}

\begin{figure}
	\centering
	\subfloat[Skew sawtooth map]{
		\includegraphics[width=.4 \textwidth]{../Figures/7/7.2a/result.png}
		\label{fig:add.saw}
	}
	\subfloat[Function shape]{
		\includegraphics[width=.4 \textwidth]{../Figures/7/7.2b/result.png}
		\label{fig:add.arch.new}
	}
	\caption[Comparison of the archetypal model function with increasing branches and the skew sawtooth map]{
		Comparison of the archetypal model function with increasing branches and the skew sawtooth map.
		(a) shows the archetypal model function with the parameters $a_L = 1, b_L = \frac{1}{2}, c_L = 0.168, g_R\left(\frac{1}{4}\right) = -0.4 ,$ and $g_R\left(\frac{1}{2}\right) = \frac{1}{2} + \frac{1}{40}$.
		(b) shows the skew sawtooth map which is defined in \cite{simpson2018saw} with the parameters $a_L = 0.5$ and $a_R = 1.5$.
		The parameters happen to have similar names to the parameters of the archetypal model.
	}
	\label{fig:add.saw.vs.arch}
\end{figure}
