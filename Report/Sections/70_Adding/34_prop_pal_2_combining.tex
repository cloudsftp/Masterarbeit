\subsubsection{Symbolic Sequences in the \Glsentrylong{pal} Structures}

So far, we have derived rules for the periods and multistability of cycles that are associated with parameter regions of \gls{pal} structures in the archetypal model based on the corresponding \gls{pa} structure in the halved archetypal model.
Let us now derive rules for the symbolic sequences that child nodes are associated with based on the symbolic sequences their parent nodes are associated with.
For this, we first need to introduce a new operator.
It is defined in \Cref{def:merge}.

\begin{definition}[Merging two 4-syllables]
	\label{def:merge}
	The operation $\left[\phi_i \mid \psi_j\right]$ merges two 4-syllables $\phi_i$ and $\psi_j$.
	Let $\phi_i = \A^a\B^b\C^c\D^d$ and $\psi_j = \A^e\B^f\C^g\D^h$.
	Then $\left[\phi_i \mid \psi_j\right] = \A^a\B^b\C^g\D^h$.
	It concatenates the first 2-syllable of $\phi_i$ with the second 2-syllable of $\psi_j$.
\end{definition}

\begin{theorem}[Symbolic Sequences in Child Nodes I]
	\label{theorem:child.symbolic.1}
	The child node of two nodes that are both associated with a singular cycle each with the symbolic sequences $\phi = \phi_1 \dots \phi_n$ and $\psi = \psi_1 \dots \psi_m$ is associated with two coexisting cycles with the following symbolic sequences.
	\begin{align}
		\pi^a = \phi_1 \dots \phi_{\frac{n-1}{2}} \left[\phi_{\frac{n+1}{2}} \mid \psi_{\frac{m+1}{2}}\right] \psi_{\frac{m+3}{2}} \dots \psi_m
	\end{align}
	and
	\begin{align}
		\pi^b = \phi_{\frac{n+3}{2}} \dots \phi_n \psi_1 \dots \psi_{\frac{m-1}{2}} \left[\psi_{\frac{m+1}{2}} \mid \phi_{\frac{n+1}{2}}\right]
	\end{align}
\end{theorem}

\clearpage

\begin{proof} \phantom{x} \\
	Let $\sigma = \sigma_1\sigma_2 \dots \sigma_i$ with odd $i$, $\varrho = \varrho_1\varrho_2 \dots \varrho_j$ with odd $j$, $T(\sigma) = \phi$, and $T(\varrho) = \psi$.
	The child of both nodes in the halved archetypal model is associated with the symbolic sequence $\sigma\varrho = \sigma_1 \dots \sigma_i \varrho_1 \dots \varrho_j$.
	This manifests as two coexisting cycles in the archetypal model with the following symbolic sequences.
	\begin{align*}
		\pi^a & = T(\sigma\varrho) = T(\sigma_1 \dots \sigma_i \varrho_1 \dots \varrho_j)                                                         \\
		      & = t(\sigma_1\sigma_2) \dots t(\sigma_i\varrho_1) \dots t(\varrho_{j-1}\varrho_j)                                                  \\
		      & = \phi_1 \dots \phi_{\frac{n-1}{2}} t(\sigma_i\varrho_1) \psi_{\frac{m+3}{2}} \dots \psi_j                                        \\
		      & = \phi_1 \dots \phi_{\frac{n-1}{2}} \left[\phi_{\frac{n+1}{2}} \mid \psi_{\frac{m+1}{2}}\right] \psi_{\frac{m+3}{2}} \dots \psi_j
	\end{align*}
	and
	\begin{align*}
		\pi^b & = T(s_2(\sigma\varrho)) = T(\sigma_2 \dots \sigma_i \varrho_1 \dots \varrho_j \sigma_1)                                           \\
		      & = t(\sigma_2\sigma_3) \dots t(\sigma_{i-1}\sigma_i) t(\varrho_1\varrho_2) \dots t(\varrho_j\sigma_1)                              \\
		      & = \phi_{\frac{n+3}{2}} \dots \phi_n \psi_1 \dots \psi_{\frac{m-1}{2}} t(\varrho_j\sigma_1)                                        \\
		      & = \phi_{\frac{n+3}{2}} \dots \phi_n \psi_1 \dots \psi_{\frac{m-1}{2}} \left[\psi_{\frac{m+1}{2}} \mid \phi_{\frac{n+1}{2}}\right]
	\end{align*}
	\hfill $\blacksquare$
\end{proof}

\begin{theorem}[Symbolic Sequences in Child Nodes II]
	\label{theorem:child.symbolic.2}
	The child node of a node that is associated with a singular cycle with the symbolic sequence $\phi = \phi_1\phi_2 \dots \phi_n$ and a node that is associated with two coexisting cycles with the symbolic sequences $\phi^a = \phi^a_1\phi^a_2 \dots \phi^a_m$ and $\phi^b_1\phi^b_2 \dots \phi^b_m$ is associated with one of the following symbolic sequences.
	\begin{enumerate}
		\item If $\phi$ is associated with the left parent.
		      \begin{align}
			      \pi =
			      \phi_1 \dots \phi_{\frac{n-1}{2}} \left[\phi_{\frac{n+1}{2}} \mid \psi^b_m\right]
			      \psi^b_1 \dots \psi^b_{m-1} \left[\psi^b_m \mid \phi_{\frac{n+1}{2}}\right]
			      \phi_{\frac{n+3}{2}} \dots \phi_n \psi^a
		      \end{align}
		\item If $\phi$ is associated with the right parent.
		      \begin{align}
			      \pi =
			      \psi^a \phi_1 \dots \phi_{\frac{n-1}{2}} \left[\phi_{\frac{n+1}{2}} \mid \psi^b_m\right]
			      \psi^b_1 \dots \psi^b_{m-1} \left[\psi^b_m \mid \phi_{\frac{n+1}{2}}\right]
			      \phi_{\frac{n+3}{2}} \dots \phi_n
		      \end{align}
	\end{enumerate}
	Both cases are shift-equivalent to each other, but we must distinguish them in order to guarantee the correctness of the symbolic sequences in subsequent child nodes.
\end{theorem}

\begin{proof} \phantom{x}
	\begin{enumerate}
		\item Let $\sigma = \sigma_1\sigma_2 \dots \sigma_i$ with odd $i$, $\varrho = \varrho_1\varrho_2 \dots \varrho_j$ with even $j$, $T(\sigma) = \phi, T(\varrho) = \psi^a$, and $T(s_2(\varrho)) = \psi^b$.
		      The child of both nodes in the halved archetypal model is associated with the symbolic sequence $\sigma\varrho = \sigma_1\sigma_2 \dots \sigma_i \varrho_1\varrho_2 \dots \varrho_j$.
		      This will manifest as the following symbolic sequence in the archetypal model.
		      \begin{align*}
			      \pi & = T(\sigma\varrho)  = T(\sigma_1 \dots \sigma_i \varrho_1 \dots \varrho_j)                                                                                \\
			          & =
			      t(\sigma_1\sigma_2) \dots t(\sigma_i \varrho_1) \dots t(\varrho_j \sigma_1) \dots t(\sigma_{i-1}\sigma_i) t(\varrho_1\varrho_2) \dots t(\varrho_{j-1}\varrho_j) \\
			          & = \phi_1 \dots \phi_{\frac{n-1}{2}} t(\sigma_i \varrho_1)
			      \psi^b_1 \dots \varrho^b_{m-1} t(\varrho_j \sigma_1)
			      \phi_{\frac{n+3}{2}} \dots \phi_n
			      \psi^a_1 \dots \psi^a_m                                                                                                                                         \\
			          & =
			      \phi_1 \dots \sigma_{\frac{n-1}{2}} \left[\sigma_{\frac{n+1}{2}} \mid \varrho^b_m\right]
			      \psi^b_1 \dots \psi^b_{m-1} \left[\varrho^b_m \mid \sigma_{\frac{n+1}{2}}\right]
			      \phi_{\frac{n+3}{2}} \dots \phi_n
			      \psi^a                                                                                                                                                          \\
		      \end{align*}
		\item Let $\sigma = \sigma_1\sigma_2 \dots \sigma_i$ with even $i$ and $\varrho = \varrho_1\varrho_2 \dots \varrho_j$ with odd $j$ and $T(\sigma) = \psi^a, T(s_2(\sigma)) = \psi^b$, and $T(\varrho) = \phi$.
		      The child of both nodes in the halved archetypal model is associated with the symbolic sequence $\sigma\varrho = \sigma_1l_2 \dots \sigma_i \varrho_1\varrho_2 \dots \varrho_j$.
		      This will manifest as the following symbolic sequence in the archetypal model.
		      \begin{align*}
			      \pi & = T(\sigma\varrho) = T(\sigma_1 \dots \sigma_i \varrho_1 \dots \varrho_j)                                                                                             \\
			          & =
			      t(\sigma_1\sigma_2) \dots t(\sigma_{i-1}\sigma_i) t(\varrho_1\varrho_2) \dots t(\varrho_j \sigma_1) \dots t(\sigma_i\varrho_1) \dots t(\varrho_j\sigma_1) \dots t(\sigma_i) \\
			          & =
			      \psi^a_1 \dots \psi^a_m
			      \phi_1 \dots \phi_{\frac{n-1}{2}} t(\sigma_i \varrho_1)
			      \psi^b_1 \dots \psi^b_{m-1} t(\varrho_j \sigma_1)
			      \phi_{\frac{n+3}{2}} \dots \phi_n                                                                                                                                           \\
			          & =
			      \psi^a
			      \phi_1 \dots \phi_{\frac{n-1}{2}} \left[\phi_{\frac{n+1}{2}} \mid \psi^b_m\right]
			      \psi^b_1 \dots \psi^b_{m-1} \left[\psi^b_m \mid \phi_{\frac{n+1}{2}}\right]
			      \phi_{\frac{n+3}{2}} \dots \phi_n                                                                                                                                           \\
		      \end{align*}
	\end{enumerate}
	\hfill $\blacksquare$
\end{proof}

As mentioned before, the next case does not appear in the \gls{pal} structures we investigate.
But we will include it here for completeness.

\begin{theorem}[Symbolic Sequences in Child Nodes III]
	\label{theorem:child.symbolic.3}
	The child node of two nodes that are both associated with two coexisting cycles each with the symbolic sequences $\phi^a = \phi^a_1\phi^a_2\dots\phi^a_n, \phi^b = \phi^b_1\phi^b_2\dots\phi^b_n, \psi^a = \psi^a_1\psi^a_2\dots\psi^a_m,$ and $\psi^b = \psi^b_1\psi^b_2\dots\psi^b_m$ is associated with two coexisting cycles with the following symbolic sequences.
	\begin{align}
		\pi^a = \phi^a\psi^a
	\end{align}
	and
	\begin{align}
		\pi^b = \phi^b_1 \dots \phi^b_{n-1} \left[\phi^b_n \mid \psi^b_m\right] \psi^b_1 \dots \psi^b_{m-1} \left[\psi^b_m \mid \phi^b_n\right]
	\end{align}
\end{theorem}

\begin{proof} \phantom{x} \\
	Let $\sigma = \sigma_1 \dots \sigma_i$ with even $i$, $\varrho = \varrho_1 \dots \varrho_j$ with even $j$, $T(\sigma) = \phi^a, T(s_2(\sigma)) = \phi^b, T(\varrho) = \phi^a$, and $T(s_2(\varrho)) = \phi^b$.
	The child of both nodes in the halved archetypal model is associated with the symbolic sequence $\sigma\varrho$.
	This manifests as two coexisting cycles with the following symbolic sequences in the archetypal model.
	\begin{align*}
		\pi^a & = T(\sigma\varrho) = T(\sigma_1 \dots \sigma_i \varrho_1 \dots \varrho_j)                                 \\
		      & = t(\sigma_1\sigma_2) \dots t(\sigma_{i-i}\sigma_i) t(\varrho_1\varrho_2) \dots t(\varrho_{j-1}\varrho_j) \\
		      & = \phi^a_1 \dots \phi^a_n \psi^a_1 \dots \psi^a_m = \phi^a\psi^a
	\end{align*}
	and
	\begin{align*}
		\pi^b & = T(s_2(\sigma\varrho)) = T(\sigma_2 \dots \sigma_i \varrho_1 \dots \varrho_j \sigma_1)                                           \\
		      & = t(\sigma_2\sigma_3) \dots t(\sigma_i\varrho_1) \dots t(\varrho_j\sigma_1)                                                       \\
		      & = \phi^b_1 \dots \phi^b_{n-1} t(\sigma_i\varrho_1) \phi^b_1 \dots \phi^b_{m-1} t(\varrho_j\sigma_1)                               \\
		      & = \phi^b_1 \dots \phi^b_{n-1} \left[\phi^b_n \mid \phi^b_m\right] \phi^b_1 \dots \phi^b_{m-1} \left[\phi^b_m \mid \phi^b_n\right]
	\end{align*}
	\hfill $\blacksquare$
\end{proof}
