\subsection{Disappearance of ``Type B'' Parameter Regions}
\label{sec.change.disb}

For \Cref{fig:add.change.regions.1,fig:add.change.regions.2}, the ``type B'' parameter region $Q^{20}_3$ is complete.
In \Cref{fig:add.change.regions.4}, it is gone completely, instead the two ``type A'' parameter regions $P^{20}_3$ and $P^{20}_4$ now overlap.
\todo{In regions figure 3: labels $P^{20}_2$ wrong, should be $P^{20}_4$}

In between those two stages, we can see how the ``type B'' parameter region $Q^{20}_3$ disappears.
\Cref{fig:add.change.regions.3} shows the ``type A'' parameter regions $P^{20}_3$ and $P^{20}_4$ overlapping.
The point, where their boundaries cross is in the middle of where the ``type B'' parameter region was in \Cref{fig:add.change.regions.2}.
This point is also where the ``type B'' parameter region now ends.

\todo{Labels in figure wrong!}
\todo{In cobweb diagrams: enhance cycles at borders}
\begin{figure}
	\centering
	\subfloat[Regions]{
		\includegraphics[width=.3 \textwidth]{62_MinimalRepr_Adding/2D_Regions_2.675/Manual/result.png}
		\label{fig:add.change.disb.regions}
	}
	\subfloat[Cobweb at point $A$]{
		\includegraphics[width=.3 \textwidth]{62_MinimalRepr_Adding/Cob_2.675_A/Manual/result.png}
		\label{fig:add.change.disb.cob.A}
	}
	\subfloat[Cobweb at point $B$]{
		\includegraphics[width=.3 \textwidth]{62_MinimalRepr_Adding/Cob_2.675_B/Manual/result.png}
		\label{fig:add.change.disb.cob.B}
	}
	\caption{Disappearance of the ``type B'' parameter region}
\end{figure}

\Cref{fig:add.change.disb.regions} shows the same thing as \Cref{fig:add.change.regions.3} again but at parameter value closer to the diappearence of the ``type B'' parameter region.
We can see that the point where the boundaries of the ``type A'' parameter regions cross moved right and the ``type B'' parameter region got smaller.
This point is called a codimension-2 point because at this point, two bifurcations happen at the same time to the same cycle.

We know from \Cref{sec:arch.bif.sum} that the \gls{bcb} at the upper boundary of the ``type A'' parameter region $P^{20}_3$ is $\BCB_{d_1, d_3}^{\underline{\A}^7\B^3\underline{\C}^7\D^3}$.
And the \gls{bcb} at the lower boundary of the ``type A'' parameter region $P^{20}_4$ is $\BCB_{d_1, d_3}^{\A^6\underline{\B}^4\C^6\underline{\D}^4}$.
At the codimension-2 point, both these \glspl{bcb} happen at the same time and both cycles vanish.
We can see in \Cref{fig:add.change.disb.cob.A} that the ``type A'' cycles are very close to the borders $d_1$ and $d_3$, respectively.

We also know that the \glspl{bcb} at the top of the ``type B'' parameter region $Q^{20}_3$ are $\BCB_{d_1}^{\underline{\A}^7\B^3\C^6\D^4}$ and $\BCB_{d_3}^{\A^6\B^4\underline{\C}^7\D^3}$.
And the \glspl{bcb} at the lower boundary are $\BCB_{d_3}^{\A^7\B^3\C^6\underline{\D}^4}$ and $\BCB_{d_1}^{\A^7\underline{\B}^3\C^6\D^4}$.
At the codimension-2 point, both \glspl{bcb} $\BCB_{d_1}^{\underline{\A}^7\B^3\C^6\D^4}$ and $\BCB_{d_3}^{\A^7\B^3\C^6\underline{\D}^4}$ happen to the cycle $\Cycle{\A^7\B^3\C^6\D^4}$ at the same time and it vanishes.
Because of the symmetry, the \glspl{bcb} $\BCB_{d_3}^{\A^6\B^4\underline{\C}^7\D^3}$ and $\BCB_{d_1}^{\A^7\underline{\B}^3\C^6\D^4}$ happen to the cycle $\Cycle{\A^6\B^4\C^7\D^3}$ at the same time and it vanishes also.

This codimension-2 point moves right with higher values for $b_L$ along our line.
As soon as the codimension-2 point crosses the right boundary of the ``type B'' parameter region, the ``type B'' parameter region ceases to exist.
Instead, now the two ``type A'' parameter regions overlap without the codimension-2 point.
This new overlapping regions is bounded by simple ``type A'' boundary \glspl{bcb} as they are discussed in \Cref{sec:arch.bif.sum}.

\todo{Confirm bcbs with cobweb diagrams}

\todo{Order of left most cycle and other cycle => type B or type A. idk}
