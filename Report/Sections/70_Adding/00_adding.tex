\chapter{Period-adding in the Archetypal Model}
\label{chap:add}

It was shown that there are period-adding structures in piecewise linear continuous increasing circle maps by \Citeauthor{simpson2018saw}.
He calls these types of functions skew sawtooth maps~\cite{simpson2018saw}.
The archetypal model function is not continuous, has four branches instead of two, and only linear on two of four branches.
But it is also a circle map, \gls{pws}, and mostly increasing.
\Cref{fig:add.saw.vs.arch} shows both maps for comparison.
In this chapter, we modify the fixed parameters of the archetypal model to produce period-adding structures.
The idea is to make the archetypal model function as similar as possible to a skew sawtooth map.

\begin{figure}
	\centering
	\subfloat[Skew sawtooth map]{
		\includegraphics[width=.45 \textwidth]{81_Sawtooth/Sketch/result.png}
		\label{fig:add.saw}
	}
	\subfloat[Archetypal model function]{
		\includegraphics[width=.45 \textwidth]{60_MinimalRepr/Sketch/result.png}
		\label{fig:add.arch}
	}
	\caption[Comparison of a skew sawtooth map and the archetypal model function]{
		Comparison of a skew sawtooth map and the archetypal model function.
		(a) shows the skew sawtooth map with the parameters $a_L = 0.5$ and $a_R = 1.5$.
		The function is taken from \Citetitle{simpson2018saw}~\cite{simpson2018saw}.
		(b) shows the archetypal model function with the parameters $a_L = 4, b_L = -\frac{1}{2}, c_L = 0.168, g_R\left(\frac{1}{4}\right) = -0.4 ,$ and $g_R\left(\frac{1}{2}\right) = \frac{1}{2} + \frac{1}{40}$.
	}
	\label{fig:add.saw.vs.arch}
\end{figure}

\section{Adjusting the Parameters}
\label{sec:add.parameters}

Making the archetypal model continuous is in direct conflict with varying the parameters $\alpha = g_R\left(\frac{1}{4}\right)$ and $\beta = c_L$ individually.
But we can make the archetypal model always increasing.

The fixed parameters in the archetypal are $a_L = 4, b_L = -\frac{1}{2},$ and $g_R\left(\frac{1}{2}\right) = \frac{1}{2} + \frac{1}{40}$.
We can see in \Cref{fig:add.arch} that only the branches $f_\A$ and $f_\C$ are not monotonously increasing.
Of the three fixed parameters, $a_L$ and $b_L$ influence the shape of the branches $f_\A$ and $f_\C$.
We adjust these parameters to make the branches monotonously increasing.
The new parameter values are $a_L = 1$ and $b_L = \frac{1}{2}$.
The new shape of the function can be seen in \Cref{fig:add.arch.new}, now all branches are monotonously increasing.

\begin{figure}
	\centering
	\subfloat[Function shape]{
		\includegraphics[width=.45 \textwidth]{../Figures/7/7.2a/result.png}
		\label{fig:add.arch.new}
	}
	\subfloat[2D scan of periods]{
		\includegraphics[width=.45 \textwidth]{../Figures/7/7.2b/result.png}
		\label{fig:add.arch.new.period}
	}
	\caption[Model function shape and 2D period scan of the increasing archetypal model]{
		Model function shape and 2D period scan of the increasing archetypal model.
		(a) shows the archetypal model function with the parameters $a_L$ and $b_L$ adjusted such that the function is increasing.
		The parameters are $a_L = 1, b_L = \frac{1}{2}, c_L = 0.168, g_R\left(\frac{1}{4}\right) = -0.4 ,$ and $g_R\left(\frac{1}{2}\right) = \frac{1}{2} + \frac{1}{40}$.
		(b) shows a 2D scan of the periods with the parameters $a_L, b_L,$ and $g_R\left(\frac{1}{2}\right)$ fixed at the same values as (a).
		The parameters $\alpha = g_R\left(\frac{1}{4}\right)$ and $\beta = c_L$ are varied.
	}
\end{figure}

\Cref{fig:add.arch.new.period} shows a 2D scan of the periods in the archetypal model with the new parameters.
We will refer to this model as the increasing archetypal model.
In this scan, we can see that the ``type B'' parameter regions disappeared and ``type A'' parameter regions of the same chain seem to start overlapping.
Also, in between the chains there are now small parameter regions with much higher periods.
This looks like period-adding, and we will explore it in the following sections.
But first, we will take a closer look at how the bifurcations structures change when adjusting the parameters $a_L$ and $b_L$ to make the branches $f_\A$ and $f_\C$ increasing.

\todo{From here period-adding-like!}

\section{Changes to the Bifurcation Structure}
\label{sec:add.change}

\begin{enumerate}
	\item \begin{itemize}
		      \item ``type A'' parameter regions stop overlapping with the ``type A'' period regions above them (lower period, same number of points on branches $f_\B$ and $f_\D$)
		      \item Point where boundaries cross. On the left: no overlap, period-adding. On the right: overlap
		      \item This point moves right until no overlap exists anymore
	      \end{itemize}
	\item \begin{itemize}
		      \item ``type A'' parameter regions start overlapping with the ``type A'' period regions above right them (same period)
		      \item Point where boundaries cross. On the left: overlap. On the right: no overlap, ``type B'' parameter region
		      \item This point moves right, until no ``type B'' parameter region exists anymore
	      \end{itemize}
	\item \begin{itemize}
		      \item ``type A'' parameter regions stop overlapping with the ``type A'' parameter regions right to them (higher period, one more point on branches $f_\B$ and $f_\D$)
		      \item This seems to happen in an instant
	      \end{itemize}
\end{enumerate}

Timeline:
On the parameter line outlined above, the processes happen as follows.
First, the process (i) starts.
While it is happening, the process (ii) starts and finishes, before (i) finishes.
Lastly the process (iii) starts and finishes directly.
After all that, process (i) is the last to complete.

\subsection{Disappearance of ``Type B'' Parameter Regions}
\label{sec:add.change.disb}

For \Cref{fig:add.change.regions.1,fig:add.change.regions.2}, the ``type B'' parameter region $Q^{20}_3$ is complete.
In \Cref{fig:add.change.regions.4}, it is gone completely, instead the two ``type A'' parameter regions $P^{20}_3$ and $P^{20}_4$ now overlap.

\begin{figure}
	\centering
	\subfloat[Regions]{
		\includegraphics[width=.3 \textwidth]{../Figures/7/7.4a/result.png}
		\label{fig:add.change.disb.regions}
	}
	\subfloat[Cobweb at point $A$]{
		\includegraphics[width=.3 \textwidth]{../Figures/7/7.4b/result.png}
		\label{fig:add.change.disb.cob.A}
	}
	\subfloat[Cobweb at point $B$]{
		\includegraphics[width=.3 \textwidth]{../Figures/7/7.4c/result.png}
		\label{fig:add.change.disb.cob.B}
	}
	\caption[Short]{
		Disappearance of the ``type B'' parameter region
		\todo{Update caption}
	}
\end{figure}

In between those two stages, we can see how the ``type B'' parameter region $Q^{20}_3$ disappears.
\hl{Somewhere along the parameter line given by} \Cref{equ:add.change.paramline} between \Cref{fig:add.change.regions.2} and \Cref{fig:add.change.regions.3}, \hl{the lower left corner of the parameter region $P^{20}_4$ crosses the upper boundary of the parameter region $P^{20}_3$}.
\hl{This causes the ``type A'' parameter regions $P^{20}_3$ and $P^{20}_4$ to overlap in} \Cref{fig:add.change.regions.3}.
\hl{
	The point, where both boundaries cross is not a codimension-2 point, since the bifurcation at the lower boundary of the overlapping parameter region
}
\hl{We know from} \Cref{sec:arch.bif.sum} \hl{that the border collision bifurcation at the upper boundary of $P^{20}_3$ is $\BCB_{d_1, d_3}^{\A^6\underline{\B}^4\C^6\underline{\D}^4}$ and the border collision bifurcation at the lower boundary of $P^{20}_4$ is $\BCB_{d_1, d_3}^{\underline{\A}^7\B^3\underline{\C}^7\D^3}$}.
\hl{
	Therefore, this point is \textbf{not} a codimension-2 point, since the bifurcations happen to different cycle.
	Nonetheless, this is the right corner of the overlapping parameter region of $P^{20}_3 \cap P^{20}_4$.
}

\hl{
	At similar parameter values, where the lower left corner of $P^{20}_4$ crosses the upper boundary of $P^{20}_3$, the upper left corner and the lower left corner of the ``type B'' parameter region $Q^{20}_3$ collide causing the lower and upper boundaries of this parameter region to cross also.
}
\hl{We know from} \Cref{sec:arch.bif.sum} \hl{that the border collision bifurcations at the upper boundary of $Q^{20}_3$ are $\BCB_{d_1}^{\underline{\A}^7\B^3\C^6\D^4}$ and $\BCB_{d_3}^{\A^6\B^4\underline{\C}^7\D^3}$ and the border collision bifurcations at the lower boundary of $Q^{20}_3$ are $\BCB_{d_3}^{\A^7\B^3\C^6\underline{\D}^4}$ and $\BCB_{d_1}^{\A^6\underline{\B}^4\C^7\D^3}$}.
\hl{
	This is a codimension-2 point, since each of the coexisting cycles undergoes two different bifurcations at this point.
}
\hl{Also, this codimension-2 point is different from the codimension-2 points listed in} \Cref{sec:arch.bif.sum}, \hl{since those points always involve all four borders}.
\hl{
	Here, only the borders associated with vertical boundaries, namely $d_1$ and $d_3$, are involved in all four bifurcations at this point.
}

\hl{Along the parameter line given by} \Cref{equ:add.change.paramline} \hl{for increasing values of $b_L$, the lower boundary of $P^{20}_4$ and the upper boundary of $Q^{20}_3$ move down while the upper boundary of $P^{20}_3$ and the lower boundary of $Q^{20}_3$ moves up}.
\hl{
	This leads to both the right corner of the overlapping region $P^{20}_3 \cap P^{20}_4$ and the codimension-2 point which is the left corner of the ``type B'' parameter region $Q^{20}_3$ to move right.
}
\hl{We can observe this movement in} \Cref{fig:add.change.disb.regions}.
\hl{
	Here, those two corner points are near the right boundaries of $Q^{20}_3$ and $P^{20}_3$.
	As soon as the codimension-2 point of the boundaries ``type B'' parameter region crosses the right boundary of the ``type B'' parameter region, the ``type B'' parameter region vanishes.
	And as soon as the right corner point of the overlapping parameter region $P^{20}_3 \cap P^{20}_4$ collides with the upper right corner of $P^{20}_3$, the upper boundary of $P^{20}_3$ stops crossing the lower boundary $P^{20}_4$ and the overlapping parameter region $P^{20}_3 \cap P^{20}_4$ has four boundaries instead of three.
}

\Cref{fig:add.change.disb.cob.A} \hl{shows a cobweb diagram of the coexisting ``type A'' cycles in the emerging overlapping parameter region $P^{20}_3 \cap P^{20}_4$}.
\hl{
	We can see that the cycles are very close to colliding $d_1$ and $d_3$.
}
\hl{The same is true for the coexisting ``type B'' twin cycles in the cobweb diagram in} \Cref{fig:add.change.disb.cob.B}.

\subsection{Appearance of Period-adding structures}
\label{sec:add.change.appa}

In this section we will explore the appearance of the period-adding structures in between the chains of the same period.
This happens at the horizontal boundaries between ``type A'' parameter regions of different chains, as well as at the vertical boundaries.
We will first take care of the horizontal period-adding structures and then move on to the vertical period-adding structures.

\subsubsection{Horizontal Period-adding Structures}
\label{sec:add.change.appa.hor}

In \Cref{fig:add.change.regions.1}, the ``type A'' parameter regions $P^{20}_3$ and $P^{18}_3$, as well as $P^{22}_4$ and $P^{20}_4$ overlap.
This changes in \Cref{fig:add.change.regions.2}.
Here only the ``type A'' parameter regions $P^{20}_3$ and $P^{18}_3$ overlap, the parameter regions $P^{22}_4$ and $P^{20}_4$ stopped overlapping.
Instead, in the space between the two ``type A'' parameter region there are now two asymmetric coexisting twin cycles $\Cycle{\A^8\B^3\C^8\D^2}$ and $\Cycle{\A^8\B^2\C^8\D^3}$.
Those cycles are \textbf{not} ``type B'' cycles, because they only differ in the number of points on the branches $f_\B$ and $f_\D$.
Instead, we will call them hybrid cycles and ``type B'' cycles are a special case of hybrid cycles.
The notation $\left[P^{22}_4 \mid P^{20}_4\right]$ used in the diagrams was introduced in \Cref{sec:add.change} and is formally defined later in \Cref{sec:add.add.halved}.
Later in \Cref{fig:add.change.regions.4}, the ``type A'' parameter regions $P^{20}_3$ and $P^{18}_3$ also stop overlapping.
In between, there are also hybrid cycles, $\Cycle{\A^7\B^3\C^6\D^3}$ and $\Cycle{\A^6\B^3\C^7\D^3}$.
This parameter region is therefore labeled $\left[P^{20}_3 \mid P^{18}_3\right]$.

In between \Cref{fig:add.change.regions.2,fig:add.change.regions.3}, the ``type A'' parameter regions $P^{22}_4$ and $P^{20}_4$ don't stop overlapping completely.
Instead, they only stop overlapping on the left side of their shared boundaries and the rest is not pictured in these diagrams.
\Cref{fig:add.change.appa.hor.regions} shows better what happens to this overlapping region between \Cref{fig:add.change.regions.1,fig:add.change.regions.4}.
We also have a codimension-2 point that moves right as was the case in \Cref{sec:add.change.disb}.

\todo{Regions: labels wrong}
\todo{Cobwebs: enhance borders, replace (c), wrong pic}
\begin{figure}
	\centering
	\subfloat[Regions]{
		\includegraphics[width=.3 \textwidth]{62_MinimalRepr_Adding/2D_Regions_2.8_add_hor/Manual/result.png}
		\label{fig:add.change.appa.hor.regions}
	}
	\subfloat[At point $A$]{
		\includegraphics[width=.3 \textwidth]{62_MinimalRepr_Adding/Cob_2.8_add_hor_A/Manual/result.png}
		\label{fig:add.change.appa.hor.cob.A}
	}
	\subfloat[At point $B$]{
		\includegraphics[width=.3 \textwidth]{62_MinimalRepr_Adding/Cob_2.8_add_hor_A/Manual/result.png}
		\label{fig:add.change.appa.hor.cob.B}
	}
	\caption{Appearance of the horizontal period-adding cascade}
\end{figure}

We know from \Cref{sec:arch.bif.sum} that the \gls{bcb} at the upper boundary of the ``type A'' parameter region $P^{22}_4$ is $\BCB_{d_1, d_3}^{\underline{\A}^7\B^4\underline{\C}^7\D^4}$.
And the \gls{bcb} at the lower boundary of the ``type A'' parameter region $P^{20}_4$ is $\BCB_{d_1, d_3}^{\A^6\underline{\B}^4\C^6\underline{\D}^4}$.
Both these \glspl{bcb} are at the upper and lower boundaries of the overlapping region $P^{22}_4 \cup P^{20}_4$.
At the codimension-2 point, both these \glspl{bcb} happen at the same time and both cycles vanish.
We can see in \Cref{fig:add.change.appa.hor.cob.A} that the ``type A'' cycles are very close to the borders $d_1$ and $d_3$, respectively.

This codimension-2 point moves right with higher values for $b_L$ along our line.
As soon as the codimension-2 point crosses the right boundary of either the ``type A'' parameter region $P^{22}_4$ or $P^{20}_4$, the overlapping parameter region $P^{22}_4 \cup P^{20}_4$ ceases to exist.
Instead, there is space between the two ``type A'' parameter regions where there are now two hybrid cycles and period-adding between the hybrid cycles and either ``type A'' parameter region.

\todo{Labels for bifurcations missing underline}
\begin{figure}
	\centering
	\includegraphics[width=.7 \textwidth]{62_MinimalRepr_Adding/1D_Bif_2.8_add_hor_AU/Manual/result.png}
	\caption{Bifurcation diagram at the upper boundary of $\left[P^{22}_4 \mid P^{20}_4\right]$}
	\label{fig:add.change.appa.hor.bif}
\end{figure}

We assume that the \glspl{bcb} bounding the parameter regions with hybrid cycles follow the same rules as the \glspl{bcb} bounding the ``type B'' parameter regions.
\Cref{fig:add.change.appa.hor.bif} confirms this for the upper boundary.
So it is bounded at the top by the \glspl{bcb} $\BCB_{d_1}^{\underline{\A}^7\B^4\C^6\D^4}$ and $\BCB_{d_3}^{\A^6\B^4\underline{\C}^7\D^4}$.
And bounded at the bottom by the \glspl{bcb} $\BCB_{d_3}^{\A^7\B^4\C^6\underline{\D}^4}$ and $\BCB_{d_1}^{\underline{\A}^6\B^4\C^7\D^4}$.
At the codimension-2 point, both \glspl{bcb} $\BCB_{d_1}^{\underline{\A}^7\B^4\C^6\D^4}$ and $\BCB_{d_3}^{\A^7\B^4\C^6\underline{\D}^4}$ happen to the cycle $\Cycle{\A^7\B^4\C^6\D^4}$ at the same time and it vanishes.
Because of the symmetry, the \glspl{bcb} $\BCB_{d_3}^{\A^6\B^4\underline{\C}^7\D^4}$ and $\BCB_{d_1}^{\A^6\underline{\B}^4\C^7\D^4}$ happen to the cycle $\Cycle{\A^6\B^4\C^7\D^4}$ at the same time and it vanishes also.

\todo{also confirmed by cobweb with enhanced cycles at borders}

How this overlapping parameter regions disappears is similar to how the overlapping parameter region appears in \Cref{sec:add.change.disb}.
There, the codimension-2 point removes the ``type B'' parameter region and opens the overlapping parameter region.
Here, the codimension-2 point removes the overlapping parameter region and opens the parameter region with hybrid cycles, which behave very similar to ``type B'' cycles.

\subsubsection{Vertical Period-adding Structures}
\label{sec:add.change.appa.vert}

For the vertical period-adding structures, we could not find a codimension-2 point like for \Cref{sec:add.change.disb,sec:add.change.appa.hor}.
In \Crefrange{fig:add.change.regions.1}{fig:add.change.regions.2}, the parameter regions $P^{20}_3$ and $P^{22}_4$, as well as $P^{18}_3$ and $P^{20}_4$ overlap.
The appearance of the parameter region $\left[P^{20}_3 \mid P^{22}_4\right]$ in between $P^{20}_3$ and $P^{22}_4$ seems to happen at the same time as the appearance of the hybrid parameter region $\left[P^{18}_3 \mid P^{20}_4\right]$ in between $P^{18}_3$ and $P^{20}_4$.
At some parameter values on the line between \Cref{fig:add.change.regions.3,fig:add.change.regions.4}.
And with these hybrid parameter regions the vertical period-adding-like structures between them and the neighboring ``type A'' parameter regions.

\Cref{fig:add.change.appb.regions.A,fig:add.change.appb.regions.B} show this transition again for the parameter regions $P^{20}_3$ and $P^{22}_4$.
The transition might be similar to the previous \Cref{sec:add.change.disb,sec:add.change.appa.hor} with a codimension-2 point that moves up or down, or it could happen at once with the left boundary of $P^{22}_4$ and the right boundary of $P^{20}_3$ being aligned perfectly for some parameter values of $a_L$ and $b_L$ on the line given by \Cref{equ:add.change.paramline}.

\todo{Regions: labels wrong}
\todo{Cobwebs: replace (c) and enhance cycles at borders}
\begin{figure}
	\centering
	\subfloat[Regions scan before period-adding\\at $a_L = 2.8, b_L = -0.1$]{
		\includegraphics[width=.4 \textwidth]{62_MinimalRepr_Adding/2D_Regions_2.8_add_vert/Manual/result.png}
		\label{fig:add.change.appa.vert.regions.A}
	}
	\subfloat[Regions with period-adding\\at $a_L = 2.65, b_L = -0.05$]{
		\includegraphics[width=.4 \textwidth]{62_MinimalRepr_Adding/2D_Regions_2.65_add_vert/Manual/result.png}
		\label{fig:add.change.appa.vert.regions.B}
	} \\
	\subfloat[At point $A$]{
		\includegraphics[width=.4 \textwidth]{62_MinimalRepr_Adding/Cob_2.8_add_vert_A/Manual/result.png}
		\label{fig:add.change.appa.vert.cobweb.A}
	}
	\subfloat[At point $B$]{
		\includegraphics[width=.4 \textwidth]{62_MinimalRepr_Adding/Cob_2.65_add_vert_B/Manual/result.png}
		\label{fig:add.change.appa.vert.cobweb.B}
	}
	\caption{Appearance of the vertical period-adding cascade}
\end{figure}

\Cref{fig:add.change.appa.vert.cobweb.A} shows the coexistence of the two coexisting ``type A'' cycles while the ``type A'' parameter regions still overlap.
\Cref{fig:add.change.appa.vert.cobweb.B} shows the coexistence of the two coexisting hybrid cycles in the newly created parameter region $\left[P^{20}_3 \mid P^{22}_4\right]$.

\begin{figure}
	\centering
	\includegraphics[width=.7 \textwidth]{62_MinimalRepr_Adding/1D_Bif_2.65_add_vert_BR/Manual/result.png}
	\caption{Bifurcation diagram of the right boundary of $P_{10}^3 \oplus P_{11}^4$}
	\label{fig:add.appa.vert.bif}
\end{figure}

\todo{old:}

\todo{odd number of splits => no neg slope needed for asymmetry. odd number => needed (reorders cycles)}


\clearpage
\section{Investigation of the Period-adding-like Structures}
\label{sec:add.add}

\hl{
	Now we know, how the \gls{pal} structures develop.
	This section investigates the \gsl{pal} structures.
	First, there will be a description of the structures.
	Here we will see why they are called \glsentylong{pal} structures and not \glsentrylong{pa} structures.
}

\subsection{Description of the Structures in the Increasing Archetypal Model}
\label{sec:add.add.like}

The first step in the description of the \gls{pal} structures is plotting a 2D scan of the periods of such a structure.
Here, $b_L$ is changed to $0.8$, and $a_L$ and $g_R\left(\frac{1}{2}\right)$ are kept as described before, because the \gls{pal} structures are more pronounced at these parameter values.
\Cref{fig:add.add.like} shows a 2D period scan with these parameters at a corner of the space between chains.
Here, the parameter region $P^{12}_2$ is in the lower left corner and the parameter region $P^{12}_3$ of the same chain is in the upper right corner.
In the lower right corner is the parameter region $P^{14}_3$.

Next, this section takes a closer look at the horizontally \gls{pal} structures between the parameter regions $P^{14}_3$ and $P^{12}_3$.
After that, it examines the vertically oriented \gls{pal} structures between the parameter regions $P^{12}_2$ and $P^{14}_3$.
And lastly, it examines the complex looking structure in the corner between all three ``type A'' parameter regions.

\begin{figure}
	\centering
	\includegraphics[width=.7 \textwidth]{../Figures/7/7.11/result.png}
	\caption[2D scan of the periods associated with parameter regions in the archetypal model with increasing branches showing an overview of all three kinds of period-adding-like structures]{
		2D scan of the periods associated with parameter regions in the archetypal model with increasing branches showing an overview of all three kinds of \gls{pal} structures.
		The parameters $a_L = 1, b_L = 0.8,$ and $g_R\left(\frac{1}{2}\right) = \frac{1}{2} + \frac{1}{40}$ are fixed.
		The parameters $\alpha = -g_R\left(\frac{1}{4}\right)$ and $\beta = c_L$ are varied in the ranges $[-0.425, -0.39]$ and $[0.077, 0.088]$.
	}
	\label{fig:add.add.like}
\end{figure}

\subsubsection{\Glsentrylong{pal} Structures In-between Vertically Neighboring ``Type A'' Parameter Regions}

\Cref{fig:add.add.like.hor.2D} shows a 2D scan of the horizontally oriented \gls{pal} structures between the parameter regions $P^{14}_3$ and $P^{12}_3$.
We can see two \gls{pal} structures, one between the ``type A'' parameter region $P^{14}_3$ and the hybrid parameter region $\left[P^{14}_3 \mid P^{12}_3\right]$ and one in between the hybrid parameter region $\left[P^{14}_3 \mid P^{12}_3\right]$ and the ``type A'' parameter region $P^{12}_3$.
There is a red arrow in the 2D period scan that indicates the parameter range for the 1D period scan in \Cref{fig:add.add.like.hor.1D}.

Looking at the 1D period scan, one can see why the section is called \glsentrylong{pal}.
In the middle between the parameter regions associated with the periods $14$ and $13$, we would expect the most pronounced parameter region to be associated with the period $27$.
Instead, the most pronounced parameter region in-between those parameter regions is associated with the period $40$.
And the most pronounced parameter region between this parameter region and the parameter region associated with the period $14$, has period $27$ where we would expect period $54$.
We can see that the periods do not add in our case.

\begin{figure}
	\centering
	\subfloat[2D period scan]{
		\includegraphics[width=.45 \textwidth]{../Figures/7/7.12a/result.png}
		\label{fig:add.add.like.hor.2D}
	}
	\subfloat[1D period scan]{
		\includegraphics[width=.45 \textwidth]{../Figures/7/7.12b/result.png}
		\label{fig:add.add.like.hor.1D}
	}
	\caption[2D and 1D scans of the periods associated with parameter regions in the archetypal model with increasing branches showing horizontally oriented period-adding-like structures]{
		2D and 1D scans of the periods associated with parameter regions in the archetypal model with increasing branches showing horizontally oriented \gls{pal} structures.
		The parameters $a_L = 1, b_L = 0.8,$ and $g_R\left(\frac{1}{2}\right) = \frac{1}{2} + \frac{1}{40}$ are fixed.
		The parameters $\alpha = g_R\left(\frac{1}{4}\right)$ and $\beta = c_L$ are different in each diagram.
		(a) shows the 2D scan where the parameters $\alpha$ and $\beta$ are varied in the ranges $[-0.4, -0.39]$ and $[0.08, 0.088]$
		and (b) shows the 1D scan where the parameter $\alpha = -0.395$ is fixed and $\beta$ is varied in the range $[0.08175, 0.08275]$ which is marked with a red arrow in (a).
	}
	\label{fig:add.add.like.hor}
\end{figure}

The structure is not just a skewed \gls{pa} structure, as one might think since the period $27$ is associated with another parameter region that is not the most pronounced between the parameter regions associated with the periods $14$ and $13$, respectively.
And also $27 + 13 = 40$ which is associated with the parameter region between the parameter regions associated with the periods $27$ and $13$, respectively.
We can see that when examining the symbolic sequences associated with the parameter regions in that structure.
\Cref{fig:add.add.like.hor.tree} shows the Farey-tree with the symbolic sequences of this structure.
The starting nodes are associated with the symbolic sequences associated with the parameter regions $P^{14}_3$ and $\left[P^{14}_3 \mid P^{12}_3\right]$, respectively.
The parameter region associated with the period $27$ is the lowest node in level $2$, which is associated with two coexisting cycles.
These two cycles, $\A^4\B^3\C^4\D^3\A^4\B^3\C^3\D^3$ and $\A^4\B^3\C^4\D^3\A^3\B^3\C^4\D^3$ could be the result of concatenating the symbolic sequences $A^4\B^3\C^4\D^3$ and $\A^4\B^3\C^3\D^3$, as well as $\A^4\B^3\B^4\D^3$ and $\A^3\B^3\C^4\D^3$ of the parameter regions associated with the periods $14$ and $13$, respectively.
But the cycle $\A^4\B^3\C^4\D^3\A^3\B^3\B^4\D^3\A^4\B^3\C^3\D^3$ of the parameter region associated with the period $40$ cannot be a result of concatenating any pair of symbolic sequences $\A^4\B^3\C^3\D^3$ and $\A^3\B^3\C^4\D^3$ of the parameter region associated with the period $13$ and $\A^4\B^3\C^4\D^3\A^4\B^3\C^3\D^3$ and $\A^4\B^3\C^4\D^3\A^3\B^3\C^4\D^3$ of the parameter region associated with the period $27$.
One can see that this is truly no \gls{pa} structure.

\begin{figure}
	\centering
	\includegraphics[width=.7 \textwidth]{../Figures/7/7.13/adding.png}
	\caption[Farey-tree showing the symbolic sequences associated with the parameter regions of a horizontally oriented \glsentrylong{pal} structure]{
		Farey-tree showing the symbolic sequences associated with the parameter regions of the lower horizontally oriented \gls{pal} structure marked with a red arrow in \Cref{fig:add.add.like.hor.2D} up to two levels.
		Nodes of parameter regions associated with two coexisting cycles are colored yellow and the periods associated with the parameter regions are in parentheses.
	}
	\label{fig:add.add.like.hor.tree}
\end{figure}

\subsubsection{Vertical Period-adding-like Structures}

Next, vertically oriented \gls{pal} structures are examined.
\Cref{fig:add.add.like.vert.2D} shows a 2D period scan of this structure.
Here, we can also see two \gls{pal} structures, one in-between the parameter regions $P^{12}_2$ and $\left[P^{12}_2 \mid P^{14}_3\right]$ and one in-between the parameter regions $\left[P^{12}_2 \mid P^{14}_3\right]$ and $P^{14}_3$.
The \gls{pal} structure in-between the parameter regions $P^{12}_2$ and $\left[P^{12}_2 \mid P^{14}_3\right]$ is chosen for closer investigation.
Again, a red arrow marks the parameter range for the 1D period scan of this \gls{pal} structure in \Cref{fig:add.add.like.vert.1D}.

\begin{figure}
	\centering
	\subfloat[2D period scan]{
		\includegraphics[width=.45 \textwidth]{../Figures/7/7.14a/result.png}
		\label{fig:add.add.like.vert.2D}
	}
	\subfloat[1D period scan]{
		\includegraphics[width=.45 \textwidth]{../Figures/7/7.14b/result.png}
		\label{fig:add.add.like.vert.1D}
	}
	\caption[2D and 1D scans of the periods associated with parameter regions in the archetypal model with increasing branches showing vertically oriented period-adding-like structures]{
		2D and 1D scans of the periods associated with parameter regions in the archetypal model with increasing branches showing vertically oriented \gls{pal} structures.
		The parameters $a_L = 1, b_L = 0.8,$ and $g_R\left(\frac{1}{2}\right) = \frac{1}{2} + \frac{1}{40}$ are fixed.
		The parameters $\alpha = g_R\left(\frac{1}{4}\right)$ and $\beta = c_L$ are different in each diagram.
		(a) shows the 2D scan where the parameters $\alpha$ and $\beta$ are varied in the ranges $[-0.425, -0.405]$ and $[0.077, 0.078]$
		and (b) shows the 1D scan where the parameter $\beta = 0.07775$ is fixed and $\alpha$ is varied in the range $[-0.422, -0.418]$ which is marked with a red arrow in (a).
	}
	\label{fig:add.add.like.vert}
\end{figure}

\begin{figure}
	\centering
	\includegraphics[width=.7 \textwidth]{../Figures/7/7.15+17/adding.png}
	\caption[Farey-tree showing the symbolic sequences associated with the parameter regions of a vertically oriented \glsentrylong{pal} structure]{
		Farey-tree showing the symbolic sequences associated with the parameter regions of the left vertically oriented \gls{pal} structure marked with a red arrow in \Cref{fig:add.add.like.vert.2D} up to two levels.
		Nodes of parameter regions associated with two coexisting cycles are colored yellow and the periods associated with the parameter regions are in parentheses.
	}
	\label{fig:add.add.like.vert.tree}
\end{figure}

As before, the periods do not add as we would expect from a \gls{pa} structure.
The most pronounced parameter region between the parameter regions associated with the periods $12$ and $13$ is associated with the period $38$.
And the most pronounced parameter region between this parameter region and the parameter region associated with the period $12$ is associated with the period $25$, which should have been the most pronounced parameter region between the parameter regions associated with the periods $12$ and $13$, respectively.

Again, a Frey-tree with the symbolic sequences associated with the parameter regions in this \gls{pal} structure is provided in \Cref{fig:add.add.like.vert.tree}.
One can see that the expected concatenation of the symbolic sequence does not work.
Nor does the concatenation of the symbolic sequence of the parameter region associated with the period $25$, which is the lowest node in level $2$, with the symbolic sequences of the parameter region associated with the period $13$, which is the upper starting node, to get the symbolic sequences of the period region of the parameter region associated with the period $38$, which is the only node in level $1$.
Therefore, this is no \gls{pa} structure either.

\subsubsection{Period-adding-like Structures in the Corners}

Finally, the structure in the corner which \Cref{fig:add.add.like.corn.2D} shows a 2D period scan of is examined.
Here, we see many \gls{pal} structures.
These are organized as follows.
There is one horizontally oriented \gls{pal} structure between the parameter regions $\left[P^{12}_2 \mid P^{14}_3\right]$ and $P^{12}_3$.
And one horizontally oriented \gls{pal} structure between the parameter region $P^{12}_3$ and every parameter region in the vertically oriented \gls{pal} structure between the parameter regions $P^{12}_2$ and $\left[P^{12}_2 \mid P^{14}_3\right]$.
Analogously, there is one vertically oriented \gls{pal} structure between the parameter regions $P^{12}_2$ and $\left[P^{14}_3 \mid P^{12}_3\right]$.
And one vertically oriented \gls{pal} structure between the parameter region $P^{12}_2$ and every parameter region in the horizontally oriented \gls{pal} structure between the parameter regions $P^{14}_3$ and $\left[P^{14}_3 \mid P^{12}_3\right]$.
Similarly, there is also one horizontally oriented \gls{pal} structure between the parameter region $\left[P^{14}_3 \mid P^{12}_3\right]$ and every parameter region in the vertically oriented \gls{pal} structure between the parameter regions $\left[P^{12}_2 \mid P^{14}_3\right]$ and $P^{14}_3$.
And one vertically oriented \gls{pal} structure between the parameter region $\left[P^{12}_2 \mid P^{14}_3\right]$ and every parameter region in the horizontally oriented \gls{pal} structure between the parameter regions $P^{14}_3$ and $\left[P^{14}_3 \mid P^{12}_3\right]$.
A very similar phenomenon was discovered by \Citeauthor{tramontana2012period} where there are many \gls{pa} structures between some parameter region and every parameter region of a \gls{pa} structure~\cite{tramontana2012period}.

The 1D period scan in \Cref{fig:add.add.like.corn.1D} shows a 1D period scan of one of the simpler \gls{pal} structures.
It is the \gls{pal} structure between the parameter regions $P^{12}_2$ and $\left[P^{14}_3 \mid P^{12}_3\right]$ marked with a red arrow in \Cref{fig:add.add.like.corn.2D}.
The diagram looks exactly like the 1D period scan for the vertical \gls{pal} structure in \Cref{fig:add.add.like.vert.1D}.
Here, the periods also do not add up as we expect them to in \gls{pa} structures.

\begin{figure}
	\centering
	\subfloat[2D period scan]{
		\includegraphics[width=.45 \textwidth]{../Figures/7/7.16a/result.png}
		\label{fig:add.add.like.corn.2D}
	}
	\subfloat[1D period scan]{
		\includegraphics[width=.45 \textwidth]{../Figures/7/7.16b/result.png}
		\label{fig:add.add.like.corn.1D}
	}
	\caption[2D and 1D scans of the periods associated with parameter regions in the archetypal model with increasing branches showing diamond-shaped period-adding-like structures]{
		2D and 1D scans of the periods associated with parameter regions in the archetypal model with increasing branches showing diamond-shaped \gls{pal} structures.
		The parameters $a_L = 1, b_L = 0.8,$ and $g_R\left(\frac{1}{2}\right) = \frac{1}{2} + \frac{1}{40}$ are fixed.
		The parameters $\alpha = -g_R\left(\frac{1}{4}\right)$ and $\beta = c_L$ are different in each diagram.
		(a) shows the 2D scan where the parameters $\alpha$ and $\beta$ are varied in the ranges $[-0.418, -0.398]$ and $[0.08, 0.085]$
		and (b) shows the 1D scan where the parameter $\beta = 0.0825$ is fixed and $\alpha$ is varied in the range $[-0.414, -0.404]$ which is marked with a red arrow in (a).
	}
	\label{fig:add.add.like.corn}
\end{figure}

The symbolic sequences are different from the vertical \gls{pal} structure in \Cref{fig:add.add.like.vert.2D}.
Still the same argument as before works that this structure is not a skewed \gls{pa} structure.
Since there are infinitely many \gls{pal} structures in this corner it is very hard to describe each one and construct rules for the periods and symbolic sequences in the structures.
Luckily there is an easier way to describe the \gls{pal} structures in the increasing archetypal model and construct the wanted rules.
This involves the halved archetypal model, which is introduced in the next section.

\begin{figure}
	\centering
	\includegraphics[width=.7 \textwidth]{../Figures/7/7.15+17/adding.png}
	\caption[Farey-tree showing the symbolic sequences associated with the parameter regions of a diamond-shaped \glsentrylong{pal} structure]{
		Farey-tree showing the symbolic sequences associated with the parameter regions of the diamond-shaped \gls{pal} structure marked with a red arrow in \Cref{fig:add.add.like.vert.2D} up to two levels.
		Nodes of parameter regions associated with two coexisting cycles are colored yellow and the periods associated with the parameter regions are in parentheses.
	}
	\label{fig:add.add.like.corn.vert.tree}
\end{figure}

\clearpage
\subsection{The Halved Model}
\label{sec:add.add.halved}

The idea behind the halved model is that the model we have can be looked at differently than it was introduced in.
We know the model $m$ maps an input $x$ to $f(x) \mod 1$, meaning that if the output $f(x)$ is greater or equal to 1, we subtract 1 from it until it is in the range $[0, 1)$.
Similarly, we add 1 to it if it is smaller than 0 until it is in the desired range.
Now instead of confining the model to the domain of $[0, 1)$, we think of it repeating infinitely in both directions.
This process is called lifting of circle maps and is described by \Citeauthor{devaney2021introduction} in his book~\cite{devaney2021introduction}.
We can achieve this by mapping $T^m: x \mapsto f(x - \lfloor x \rfloor)$.
This trick maps the input $x$ into the domain, on which our model function produces sensible results and causes it to repeat infinitely.
$T^m$ is now a lift of the model $m$ in the domain of all real numbers $\mathbb{R}$.
\Cref{fig:minrep.infinite.model.concept} illustrates this concept for the cycle $P_7^3$.
The blue square is the full model.
One can see, that the branch $f_\D$ is outside the blue square at its right edge.
This is because it was cut off and continued at the bottom of the square before, due to the $\mod 1$ operation.

\todo{This makes sense in the original problem domain}

In this model, there are no cycles that have multiple rotations.
Instead, the cycles that had multiple rotations in the full model, manifest as a sequence of different blocks of the full model.
Meaning for the example $P_7^3$, the same blocks of $\A^4\B^3\C^4\D^3$ are repeating infinitely.
But for an example with multiple rotations, such as $\A\B\C\D\A^2\B^2\C^2\D^2$, the blocks will not all be the same.
Instead, the blocks $\A\B\C\D$ and $\A^2\B^2\C^2\D^2$ will be alternating.

Now the symmetry of our function $f$ comes into play.
Since $f(x + 0.5) = f(x) + 0.5$ for $x \in [0, 0.5)$, we can split the infinite model into smaller blocks than the blue block of the full model.
The function of the infinite model repeats in blocks of size 0.5, these blocks are marked red in \Cref{fig:minrep.infinite.model.concept}.
These red blocks represent the halved model, it is the smallest repeating part of the infinite model $T^m$.
Basically we choose the smallest model, of which $T^m$ is a lift.
This happens to be exactly our model $m$ folded in half.
Se the halved model $h$ is defined on the interval $[0, \frac{1}{2})$ and maps $x \mapsto g(x) \mod \frac{1}{2}$, where $g(x)$ is the same as in our model $m$, defined in \Cref{sec:minrep.definition}.

To get the symbolic sequence of a cycle in the halved model, we look at the pattern in which different red blocks repeat along the infinite model.
For our example in the picture, there is only one red block that repeats infinitely, $\L^4\R^3$.
The next section will explain, how to translate cycles between the halved and full model.

\begin{figure}
	\centering
	\includegraphics[width=.7 \textwidth]{63_MinimalRepr_Adding_Halved/Cob_Vis_s/Manual/result.png}
	\caption{Illustration of the infinite model concept.}
	\label{fig:minrep.infinite.model.concept}
\end{figure}

\begin{figure}
	\centering
	\includegraphics[width=\textwidth]{FareyTrees/Minrep_Adding1_Halved/adding.png}
	\caption{t}
	\label{fig:tree.adding1.hor.halved}
\end{figure}

\subsection{Translating Symbolic Sequences}
\label{sec:add.halved.tanslating}

First, we define naive algorithms for translating symbolic sequences between the archetypal and the halved archetypal model based on the idea of the lifted model.
Then, we derive regularities for translated symbolic sequences in the archetypal model.
Finally, we refine the algorithms using the regularities of translated symbolic sequences.

In the following, cycles of the archetypal model get the symbols $\phi, \psi,$ and $\pi$.
Cycles of the halved archetypal model get the symbols $\sigma, \rho,$ and $\tau$.
If there are two coexisting twin cycles, they are written as $\phi^a$ and $\phi^b$.
Also, the symbols for the symbolic sequences in the halved archetypal model are $\L$ and $\R$ instead of $\A, \B, \C,$ and $\D$ to avoid confusion.

\subsubsection{Naive Algorithms}

We start by formulating a naive algorithm for translating symbolic sequences from the archetypal model to the halved archetypal model.
This is the easier direction.
From this algorithm we don't learn much about the nature of the \gls{pal} structures in the archetypal model.
The algorithm for translating symbolic sequences in the other direction, from the halved archetypal model to the archetypal model, will be more important for that.

To translate a symbolic sequence of the archetypal model we start by writing it down.
For example, let $\phi = \A^4\B^3\C^4\D^3$.
Then we replace the symbols $\A$ and $\C$ by $\L$ and the symbols $\B$ and $\D$ by $\R$.
Now we have $\L^4\R^3\L^4\R^3$.
Finally, we have to check for redundancy in the resulting cycle.
In our example, the cycle $\L^4\R^3$ repeats twice in $\L^4\R^3\L^4\R^3$, so we just keep $\L^4\R^3$.

The opposite direction is trickier.
We start by writing down the symbolic sequence in the halved archetypal model.
For example $\sigma = \L^4\R^3\L4\R^3\L^3\R^3$.
Now we need to build pairs of rotations since each blue block fits exactly two red blocks.
If there is one rotation left over at the end, we wrap around or equivalently write down the original sequence again.
We repeat this until all rotations we have written down are paired up.

\begin{lemma}[How Often to Write Down the Symbolic Sequence]
	\label{lemma:writing.down}
	In order to translate a cycle from the halved archetypal model $\sigma$ with an even number of rotations $n$, we only need to write the original cycle down once.
	In order to translate a cycle from the halved archetypal model $\sigma$ with an odd number of rotations $n$, we need to write the original cycle down exactly twice.
\end{lemma}

\begin{proof} \phantom{x}
	\begin{enumerate}
		\item Let $n = 2i$. Then, we can build $i$ pairs of rotations and fit all $2i$ rotations of the original model.
		\item Let $n = 2i + 1$. We start by building $i$ pairs of rotations, fitting $2i$ rotations.
		      This will leave the last rotation unpaired, so we write down the sequence of $2i + 1$ rotations again.
		      Now we can pair up the last rotation of the first sequence we wrote down with the first rotation of the sequence we just wrote down.
		      $2i$ rotations remain, which we can fit into $i$ pairs.
	\end{enumerate}
	\hfill $\blacksquare$
\end{proof}

Notice that our example symbolic sequence has 3 rotations.
This means we have to write down the original sequence twice
$\sigma^2 = \left(\L^4\R^3\L^4\R^3\L^3\R^3\right)^2 = \L^4\R^3\L^4\R^3\L^3\R^3\L^4\R^3\L^4\R^3\L^3\R^3$.

Then we pair up the rotations, this corresponds to drawing blue boxes around the red boxes in the lifted model.
In our example, we get the pairs $\left(\L^4\R^3\L^4\R^3\right)\left(\L^3\R^3\L^4\R^3\right)\left(\L^4\R^3\L^3\R^3\right)$.
The pairs then have to be translated using the function $t$ defined below in \Cref{def:t}.
The resulting symbolic sequence is $T(h) = \A^4\B^3\C^4\D^3\A^3\B^3\C^4\D^3\A^4\B^3\C^3\D^3$.
The formal definition of $T$ is below in \Cref{def:T}.
But we first need to define the notion of syllables which is needed for the \Cref{def:t,def:T}.

\begin{definition}[Syllables]
	A syllable of a symbolic sequence is a subsequence of maximal length consisting of only one symbol.
	A 2-syllable is a pair of syllables that are next to each other.
	And a 4-syllable is a pair of 2-syllables that are next to each other.
\end{definition}

So for example, the symbolic sequence $\L^4\R^3\L^3\R^3$ has 4 syllables.
Its syllables are $\L^4$, $\L^3$, and two times $\R^3$.
And a 2-syllable of the cycle is $\L^4\R^3$.
In the halved archetypal model, a 2-syllable corresponds to one rotation.
The 2-syllables of a symbolic sequence $\sigma$ of a cycle in the halved archetypal model are written as $\sigma_i$.
In the archetypal model, a 4-syllable corresponds to one rotation.
The 4-syllables of a symbolic sequence $\phi$ of a cycle in the archetypal model are written as $\phi_i$.
These terms are used interchangeably in the rest of this chapter.

\begin{definition}[Translating 4-syllables from the Halved to the Full Archetypal Model]
	\label{def:t}
	The function $t$ maps two rotations, or a 4-syllable that starts with the symbol $\L$ of a symbolic sequence in the halved archetypal model to a single rotation in the archetypal model.
	It is defined in the following way.
	\begin{align}
		t:\: \L^a\R^b\L^c\R^d \mapsto \A^a\B^b\C^c\D^d
	\end{align}
\end{definition}

\begin{definition}[Translating Symbolic Sequences from the Halved to the Full Archetypal Model]
	\label{def:T}
	The function $T$ translates a symbolic sequence $\sigma = \sigma_1\sigma_2 \dots \sigma_n$ in the halved archetypal model to the archetypal model.
	Where $\sigma_i$ are the 2-syllables of $\sigma$.
	From \Cref{lemma:writing.down} we know how often we need to write down $\sigma$, and therefore also which 4-syllables to translate with $t$.
	\begin{align}
		T:\: \sigma \mapsto \begin{cases}
			                    t(\sigma_1\sigma_2) \dots t(\sigma_{n-1}\sigma_n)                           & \text{ if $n$ is even} \\
			                    t(\sigma_1\sigma_2) \dots t(\sigma_n\sigma_1) \dots t(\sigma_{n-1}\sigma_n) & \text{ else}
		                    \end{cases}
	\end{align}
\end{definition}

Now we have a way to translate full symbolic sequences.
But this is not enough, since a cycle in the halved archetypal model might manifest as multiple coexisting cycles in the archetypal model.
Let $\varrho = \L^4\R^3\L^3\R^3$ be another cycle in the halved archetypal model.
Since this cycle has an even number of rotations, the 2-syllables $\varrho_1 = \L^4\R^3$ and $\varrho_2 = \L^3\R^3$, we only need to write it down once.
And the translation of this cycle is $T(\varrho) = T(\varrho_1\varrho_2) = T\left(\L^4\R^3\L^3\R^3\right) = \A^4\B^3\C^3\D^3$.
But when we translate the cycle $\varrho' = \varrho_2\varrho_1$, which is indistinguishable from the cycle $\varrho$, we get $T(\varrho') = T(\varrho_2\varrho_1) = T\left(\L^3\R^3\L^4\R^3\right) = \A^3\B^3\C^4\D^3$.
This is a different cycle from $T(\varrho)$.
So we need to first obtain all indistinguishable cycles of the original cycle in the halved archetypal model, then translate each one, and finally check for indistinguishable results in order to not have any duplicate cycles in the archetypal model.
The concepts needed for this are defined below in \Cref{def:shifting,def:shift.equiv}.

\begin{definition}[Shifting Symbolic Sequences]
	\label{def:shifting}
	The function $s_2$ shifts a symbolic sequence $\sigma$ in the halved archetypal model by a single rotation, or equivalently by a 2-syllable.
	Let $\sigma = \sigma_1\sigma_2 \dots \sigma_n$.
	Then $s_2$ is defined in the following way:
	\begin{align}
		s_2:\: & \sigma_1\sigma_2 \dots \sigma_n \mapsto \sigma_2 \dots \sigma_n\sigma_1
	\end{align}
	In the archetypal model, there is a similar function, $s_4$ that shifts a symbolic sequence $\phi$ in the archetypal model by a single rotation.
	Let $\phi = \phi_1\phi_2 \dots \phi_n$.
	Then $s_4$ is defined in the following way:
	\begin{align}
		s_4:\: & \phi_1\phi_2 \dots \phi_n \mapsto \phi_2 \dots \phi_n\phi_1
	\end{align}
\end{definition}

\begin{definition}[Shift-equivalence]
	\label{def:shift.equiv}
	Two symbolic sequences $\sigma$ and $\varrho$ in the halved archetypal model are shift-equivalent $\sigma \equiv_2 \varrho$,
	if they both have the same number of rotations $n$
	and there is a number $0 \leq i < n$, such that $\sigma = s_2^i(\varrho)$,
	where $s_2^i(\varrho)$ is defined as $s_2^{i-1}(s_2(\varrho))$ and $s_2^1(\varrho) = s_2(\varrho)$.

	Two symbolic sequences $\phi$ and $\psi$ in the archetypal model are shift-equivalent $\phi \equiv_4 \psi$,
	if they both have the same number of rotations $n$
	and there is a number $0 \leq i < n$, such that $\phi = s_4^i(\psi)$,
	where $s_4^i(\psi)$ is defined as $s_4^{i-1}(s_4(\psi))$ and $s_4^1(\psi) = s_4(\psi)$.
\end{definition}

Note that the symbols for shift-equivalence are different for the halved archetypal model and the archetypal model.
We use $\equiv_2$ for the shift-equivalence in the halved archetypal model to express that the symbolic sequences are equivalent by shifting them by a multiple of two syllables with $s_2^i$.
Similarly, we use $\equiv_4$ for the shift-equivalence in the archetypal model to express that the symbolic sequences are equivalent by shifting them by a multiple of four syllables with $s_4^i$.

Since the halved archetypal model has two branches, any cycles that have shift-equivalent ($\equiv_2$) symbolic sequences are indistinguishable in the halved archetypal model.
Similarly, since the archetypal model has four branches, any cycles that have shift-equivalent ($\equiv_4$) symbolic sequences are indistinguishable in the archetypal model.
We obtain all symbolic sequences of cycles that are indistinguishable from the original cycle by shifting the symbolic sequence of the original cycle with $s_2$.
If the symbolic sequence $\tau$ of a cycle in the halved archetypal model has $n$ 2-syllables, the cycle has $n$ indistinguishable cycles with the symbolic sequences $\left\{s_2^i(\tau) \:|\: 0 \leq i < n\right\}$, since $s_2^n(\tau) = \tau$.
Returning to our initial example with the symbolic sequence $\sigma = \L^4\R^3\L^4\R^3\L^3\R^3$, there are three indistinguishable cycles with the symbolic sequences $\sigma = \L^4\R^3\L^4\R^3\L^3\R^3$, $s_2(\sigma) = \L^4\R^3\L^3\R^3\L^4\R^3$, and $s_2^2(\sigma) = \L^3\R^3\L^4\R^3\L^4\R^3$.

Now we translate each symbolic sequence of the indistinguishable cycles.
We already know that the translation of $\sigma$ is $T(\sigma) = \A^4\B^3\C^4\D^3\A^3\B^3\C^4\D^3\A^4\B^3\C^3\D^3$ from before.
The translation of $s_2(\sigma)$ is %$T\left(s_2(\sigma)\right) = T\left(\L^4\R^3\L^3\R^3\L^4\R^3\right) = \A^4\B^3\C^3\D^3\A^4\B^3\C^4\D^3\A^3\B^3\C^4\D^3$.
\begin{align*}
	T\left(s_2(\sigma)\right) & = T\left(\L^4\R^3\L^3\R^3\L^4\R^3\right)                                                       \\
	                          & = t\left(\L^4\R^3\L^3\R^3\right) t\left(\L^4\R^3\L^4\R^3\right) t\left(\L^3\R^3\L^4\R^3\right) \\
	                          & = \A^4\B^3\C^3\D^3\A^4\B^3\C^4\D^3\A^3\B^3\C^4\D^3
\end{align*}
And finally, the translation of $s_2^2(\sigma)$ is %$T\left(s_2(\sigma)\right) = T\left(\L^4\R^3\L^3\R^3\L^4\R^3\right) = \A^3\B^3\C^4\D^3\A^4\B^3\C^3\D^3\A^4\B^3\C^4\D^3$.
\begin{align*}
	T\left(s_2(\sigma)\right) & = T\left(\L^3\R^3\L^4\R^3\L^4\R^3\right)                                                       \\
	                          & = t\left(\L^3\R^3\L^4\R^3\right) t\left(\L^4\R^3\L^3\R^3\right) t\left(\L^4\R^3\L^4\R^3\right) \\
	                          & = \A^3\B^3\C^4\D^3\A^4\B^3\C^3\D^3\A^4\B^3\C^4\D^3
\end{align*}

Looking closely, we can see that $T(\sigma) = s_4^2\left(T\left(s_2(\sigma)\right)\right) = s_4\left(T\left(s_2^2(\sigma)\right)\right)$.
So the symbolic sequences of the cycles are shift-equivalent $T(\sigma) \equiv_4 T\left(s_2(\sigma)\right) \equiv_4 T\left(s_2^2(\sigma)\right)$ and therefore the cycles are indistinguishable.
This means that the cycle $\sigma$ manifests as a single cycle in the archetypal model.
The final result of the translation process is a set and, it is defined as the function $F$ which is formally defined in \Cref{def:F}.

\begin{definition}[Translating Cycles from the Halved to the Archetypal Model]
	\label{def:F}
	The function $F$ translates a cycle in the halved archetypal model with the symbolic sequence $\sigma$ that has $n$ 2-syllables to the archetypal model.
	The result is a set of the symbolic sequences of all the cycles that the original cycle manifests as in the archetypal model.
	\begin{align}
		F:\: \sigma \mapsto \left\{
		\prescript{}{i}{\phi} = T\left(s_2^i(\sigma)\right) \:|\:
		0 \leq i < n \land
		\not\exists 0 \leq j < i:\: \prescript{}{i}{\phi} \equiv_4 \prescript{}{j}{\phi}
		\right\}
	\end{align}
\end{definition}

Note that here, the left index is used for $\prescript{}{i}{\phi}$, since the right index is reserved for the 2-syllables of symbolic sequences in the halved archetypal model and 4-syllables of symbolic sequences in the archetypal model.

\subsubsection{Properties of Translated Symbolic Sequences in the Full Model}

With this naive algorithm, we can start to investigate rules for the \gls{pal} structures in the archetypal model.

\begin{lemma}[Shif-equivalence of Translated Symbolic Sequences in the Archetypal Model]
	\label{lemma:equivalence.translations}
	The translations of the two cycles $\sigma$ and $\rho = s_2^{2i}(\sigma)$ in the halved archetypal model are shift-equivalent $T(\sigma) \equiv_4 T(\rho)$ in the archetypal model for all integers $i$.
\end{lemma}

\begin{proof} \phantom{x} \\
	Let $\sigma = \sigma_1\sigma_2 \dots \sigma_n$, therefore $\rho = \sigma_{2i+1} \dots \sigma_n\sigma_1 \dots \sigma_{2i}$.
	The translations are $T(\sigma) = t(\sigma_1\sigma_2)t(\sigma_3\sigma_4) \dots t(\sigma_{n-1}\sigma_n)$
	and $T(\rho) = t(\sigma_{2i+1}\sigma_{2i+2}) \dots t(\sigma_{n-1}\sigma_n)t(\sigma_1\sigma_2) \dots t(\sigma_{2i-i}\sigma_{2i})$.
	We can see that $T(\rho) = s_4^i(T(\sigma))$ and therefore $T(\sigma) \equiv_4 T(\rho)$.

	\hfill $\blacksquare$
\end{proof}

\begin{theorem}[Coexistence of Translated Symbolic Sequences in the Archetypal Model]
	\label{theorem:coexistence.even}
	The manifestations of a cycle $\sigma$ in the halved archetypal model is either $F(\sigma) = \{T(\sigma), T(s_2(\sigma))\}$ or $F(\sigma) = \{T(\sigma)\}$.
	Specifically, \begin{enumerate}
		\item $F(\sigma) = \{T(\sigma), T(s_2(\sigma))\}$ if the number of rotations of the sequence $\sigma$ is even.
		\item $F(\sigma) = \{T(\sigma)\}$ if the number of rotations of the sequence $\sigma$ is odd.
	\end{enumerate}
\end{theorem}

\begin{proof} \phantom{x} \\
	Let $\sigma = \sigma_1\sigma_2 \dots \sigma_n$ be a symbolic sequence in the halved archetypal model with $n$ rotations.
	We know from \Cref{lemma:equivalence.translations} that the only possible candidates for $F(\sigma)$ are $T(\sigma)$ and $T(s_2(\sigma))$.
	These are the first two possibilities we check in the algorithm and the translations of all other shifts of the original cycle, $T(s_2^i(\sigma))$ with $2 \leq i < n$, are shift-equivalent to either $T(\sigma)$ or $T(s_2(\sigma))$.
	This follows directly from \Cref{lemma:equivalence.translations}.
	So, in the following, we only check for the shift-equivalence of these two candidates.
	\begin{enumerate}
		\item Let $n = 2i$.
		      \begin{align*}
			      T(h)                   & = t(\sigma_1\sigma_2) t(\sigma_3\sigma_4) \dots t(\sigma_{n-1}\sigma_n) \\
			      \nequiv_4 \: T(s_2(h)) & = t(\sigma_2\sigma_3) t(\sigma_4\sigma_5) \dots t(\sigma_n\sigma_1)
		      \end{align*}
		      The two candidates are not shift-equivalent because the pair $t(\sigma_1\sigma_2)$ in $T(\sigma)$ is not included in the other candidate $T(s_2(\sigma))$.
		      The same is true for any other pair, and therefore $F(\sigma) = \{T(\sigma), T(s_2(\sigma))\}$.
		\item Let $n = 2i + 1$.
		      \begin{align*}
			      T(h)                  & = t(\sigma_1\sigma_2) t(\sigma_3\sigma_4) \dots t(\sigma_n\sigma_1) t(\sigma_2\sigma_3) \dots t(\sigma_{n-1}\sigma_n) \\
			      \equiv_4 \: T(s_2(h)) & = t(\sigma_2\sigma_3) \dots t(\sigma_{n-1}\sigma_n) t(\sigma_1\sigma_2) t(\sigma_3\sigma_4) \dots t(\sigma_n\sigma_1)
		      \end{align*}
		      The two candidates are shift-equivalent.
		      By shifting the second candidate $T(s_2(\sigma))$ $2i$ times, we get the first candidate $T(\sigma)$.
		      Therefore, the second candidate is discarded and $F(\sigma) = \{T(\sigma)\}$.
	\end{enumerate}
	\hfill $\blacksquare$
\end{proof}

\subsubsection{Revised Algorithms}

With all these properties and functions we now can formulate a more compact algorithm, \Cref{alg:halved.to.full}, for translating symbolic sequences from the halved archetypal model to the archetypal model.
This revised algorithm is used in the following to construct the rules for the \gls{pal} structures in the archetypal model.

\begin{algorithm}
	\caption{Algorithm for the Translation of Symbolic Sequences from the Halved Archetypal Model to the Archetypal Model}
	\label{alg:halved.to.full}
	\begin{algorithmic}
		\Require $\sigma = \sigma_1\sigma_2 \dots \sigma_n$ with $n > 0$
		\If{$n$ is even}
		\State \Return $\{t(\sigma_1\sigma_2) t(\sigma_3\sigma_4) \dots t(\sigma_{n-1}\sigma_n), t(\sigma_2\sigma_3) t(\sigma_4\sigma_5) \dots t(\sigma_n\sigma_1)\}$
		\Else
		\State \Return $\{t(\sigma_1\sigma_2) \dots t(\sigma_{n}\sigma_1) \dots t(\sigma_{n-1}\sigma_n)\}$
		\EndIf
	\end{algorithmic}
\end{algorithm}

\clearpage

\Cref{alg:full.to.halved} shows the inverse algorithm for translating symbolic sequences from the archetypal model to the halved archetypal model for completeness.
It uses the inverse $t^{-1}$ of the function $t$.

\begin{definition}[Translating 4-syllables from Archetypal to Halved Archetypal]
	The function $t^{-1}$ maps a 4-syllable of a symbolic sequence in the archetypal model to the halved archetypal model.
	It is defined in the following way:
	\begin{align}
		t^{-1}: \quad \A^a\B^b\C^c\D^d \mapsto \L^a\R^b\L^c\R^d
	\end{align}
\end{definition}

\begin{algorithm}
	\caption{Algorithm for the Translation of Symbolic Sequences from the Archetypal Model to the Halved Archetypal Model}
	\label{alg:full.to.halved}
	\begin{algorithmic}
		\Require $\phi = \phi_1\phi_2 \dots \phi_n$ with $n > 0$
		\State $d \gets t^{-1}(\phi_1)t^{-1}(\phi_2) \dots t^{-1}(\phi_n) = \sigma_1\sigma_2 \dots \sigma_m$
		\Comment $m = 2n$ is even
		\State $\tau \gets \sigma_1\sigma_2 \dots \sigma_{\frac{m}{2}}$
		\If{$\sigma = \tau^2$}
		\State \Return $\tau$
		\Else
		\State \Return $\sigma$
		\EndIf
	\end{algorithmic}
\end{algorithm}

\subsection{Properties of the \Glsentrylong{pal} Structures in the Archetypal Model}
\label{sec:add.rules.pal}

With the revised algorithms we can now explain the \gls{pal} structures in the archetypal model.
We start by explaining why some cycles have a much higher period than expected in the \gls{pal} structures.
After that, we construct rules for the symbolic sequences in the \gls{pal} structures in the archetypal model.
And finally, we construct rules for the rotation-like numbers in the \gls{pal} structures in the archetypal model.

\subsubsection{Periods in the \Glsentrylong{pal} Structures}

First, we state that the function $t$ that translates 4-syllables from the halved archetypal model to the archetypal model preserves the period of the 4-syllable.
The function was defined in \Cref{sec:add.halved.tanslating}.

\begin{lemma}[$t$ Preserves Period]
	\label{lemma:t.preserves.period}
	The function $t$ preserves the period.
	\begin{align}
		|\sigma_1\sigma_2| = |t(\sigma_1\sigma_2)|
	\end{align}
\end{lemma}

\begin{proof}
	Let $\sigma_1\sigma_2 = \L^a\R^b\L^c\R^d$.
	\begin{align*}
		|\sigma_1\sigma_2| =  |\L^a\R^b\L^c\R^d|
		= a + b + c + d
		= |\A^a\B^b\C^c\D^d|
		= |t(\L^a\R^b\L^c\R^d)|
		= |t(\sigma_1\sigma_2)|
	\end{align*}
	\hfill $\blacksquare$
\end{proof}

\begin{theorem}[Period of Cycles in the Full Model]
	\begin{enumerate}
		\item If a cycle in the halved model manifests as two coexisting cycles in the full model, the period of either cycle is the same as the period of the cycle in the halved model.
		      \begin{align*}
			      |T(\sigma)| = |T(s_2(\sigma))| = |\sigma|
		      \end{align*}
		\item If a cycle in the halved model manifests as a single cycle in the full model, the period of this cycle is double the period of the cycle in the halved model.
		      \begin{align*}
			      |T(\sigma)| = 2 |\sigma|
		      \end{align*}
	\end{enumerate}
\end{theorem}

\clearpage

\begin{proof} \phantom{x}
	\begin{enumerate}
		\item From \Cref{theorem:coexistence.even} we know that if the cycle $\sigma$ in the halved model manifests as two coexisting cycles in the full model, $\sigma$ has an even number of rotations $n$.
		      And its translation is $T(\sigma) = t(\sigma_1\sigma_2) t(\sigma_3\sigma_4) \dots t(\sigma_{n-1}\sigma_n)$.
		      Combining this with the fact, that $t$ preserves the period of its input as described in \Cref{lemma:t.preserves.period}, we can calculate the period of $T(\sigma)$ in the following way.
		      \begin{align*}
			      |T(\sigma)| & = |t(\sigma_1\sigma_2) t(\sigma_3\sigma_4) \dots t(\sigma_{n-1}\sigma_n)|           \\
			                  & = |t(\sigma_1\sigma_2)| + |t(\sigma_3\sigma_4)| + \dots + |t(\sigma_{n-1}\sigma_n)| \\
			                  & = |\sigma_1\sigma_2| + |\sigma_3\sigma_4| + \dots + |\sigma_{n-1}\sigma_n|          \\
			                  & = |\sigma_1\sigma_2 \dots \sigma_n| = |\sigma|
		      \end{align*}
		      So the period of the cycle $T(\sigma)$ in the archetypal model is the same as the period of the cycle $\sigma$ in the halved archetypal model.
		      The same calculation can be done for $T(s(\sigma))$ and is omitted here.
		\item Similarly, we know that if the cycle $\sigma$ in the halved model manifests as a single cycle in the full model, $\sigma$ has an odd number of rotations $n$.
		      And its translation is $T(\sigma) = t(\sigma_1\sigma_2) \dots t(\sigma_n\sigma_1) \dots t(\sigma_{n-1}\sigma_n)$.
		      Its period can be calculated in the following way.
		      \begin{align*}
			      |T(\sigma)| & = |t(\sigma_1\sigma_2) \dots t(\sigma_n\sigma_1) \dots t(\sigma_{n-1}\sigma_n)|                      \\
			                  & = |t(\sigma_1\sigma_2)| + \dots + |t(\sigma_n\sigma_1)| + \dots + |t(\sigma_{n-1}\sigma_n)|          \\
			                  & = |\sigma_1\sigma_2| + \dots + |\sigma_n\sigma_1| + \dots + |\sigma_{n-1}\sigma_n|                   \\
			                  & = |\sigma_1\sigma_2 \dots \sigma_n\sigma_1 \dots \sigma_{n-1}\sigma_n| = |\sigma\sigma| = 2 |\sigma|
		      \end{align*}
		      So the period of the cycle $T(\sigma)$ in the full model is twice the period of the cycle $\sigma$ in the halved model.
	\end{enumerate}
	\hfill $\blacksquare$
\end{proof}

\todo{\Cref{lemma:t.preserves.period} should be Lemma 7.5}

With this property we can also explain the regularities for coexistence of two cycles in the Farey-trees of \gls{pal} structures.
\todo{Here Farey-tree of full horizontal pal structure}

The third case in \Cref{theorem:child.coexistence} can't be seen in the farey tree but it follows from the proof of the first two cases.
\todo{Last case not possible in our adding structures. proof!}

\begin{theorem}[Multiplicity of Cycles Associated With Child Nodes Based on the Multiplicity of Cycles Associated With its Parent Nodes]
	\label{theorem:child.coexistence}
	\begin{enumerate}
		\item The child of a node with a single cycle and a node with two coexisting cycles has a single cycle.
		\item The child of two nodes with a single cycle has two coexisting cycles.
		\item The child of two nodes with two coexisting cycles, has two coexisting cycles.
	\end{enumerate}
\end{theorem}

\begin{proof} \phantom{x}
	\begin{enumerate}
		\item A node with a single cycle in the full model is the manifestation of a cycle with an odd number of rotations in the halved model.
		      A node with two coexisting cycles in the full model is the manifestation of a cycle with an even number of rotations in the halved model.
		      Their child is the manifestation of the two cycles in the halved model glued together.
		      This glued-together cycle has an odd number of rotations and therefore manifests as a single cycle in the full model.
		\item Analogously, two cycles with an odd number of rotations glued together have an even number of rotations.
		      Therefore, this glued-together cycle manifests as two coexisting cycles in the full model.
		\item Analogously, two cycles with an even number of rotations glued together have an even number of rotations.
		      Therefore this glued-together cycle manifests as two coexisting cycles in the full model.
	\end{enumerate}
\end{proof}


This explains the pattern we observed in the periods of the \gls{pal} structures in \Cref{sec:add.add.like}.
We take another look at the 1D period scan of the horizontal \gls{pal} structure between the parameter regions $P^{14}_3$ and $\left[P^{14}_3  \mid P^{12}_3\right]$ in \Cref{fig:add.add.like.hor.1D}.
We get the information, whether a parameter region in the structure was associated with two coexisting cycles from the Farey-tree in \Cref{fig:add.add.like.hor.tree}.
The parameter region associated with the period $14$ is not associated with coexisting cycles, therefore the corresponding cycle in the halved archetypal model has an odd number of revolutions and its period is twice as high as the period associated with the same parameter region in the halved archetypal model, which is $7$.
The parameter region associated with the period $13$ is associated with coexisting cycles, therefore the corresponding cycle in the halved archetypal model has an even number of revolutions its period is the same as the period associated with the same parameter region in the halved archetypal mode.
The parameter region in between those two parameter regions is associated with the period $7 + 13 = 20$ in the halved archetypal model.
Also, its cycle has an odd number of revolutions, since it is the concatenation of one cycle with an even and one cycle with an odd number of revolutions.
We know from \Cref{theorem:period.pal} that this parameter region is associated with the period $40$ in the archetypal model.
This is true as we can observer this in the 1D period scan in \Cref{fig:add.add.hor.1D}.
Also, we know from \Cref{theorem:child.coexistence} that this parameter region is associated with two coexisting cycles.
This is true as we can see in the Farey-tree \todo{reference tree}

\todo{1D period scan again?}

\subsubsection{Rules for Combining Symbolic Sequences}

Furthermore, we can formulate rules for the cycles in the child node of two nodes in the period-adding structure in the full model.

\begin{definition}[Merging two 4-syllables]
	The operation $\left[\phi_i \mid \psi_j\right]$ merges the two 4-symmables $\phi_i$ and $\psi_j$.
	Let $\phi_i = \A^a\B^b\C^c\D^d$ and $\psi_j = \A^e\B^f\C^g\D^h$.
	Then $\left[\phi_i \mid \psi_j\right] = \A^a\B^b\C^g\D^h$.
	It concatenates the first 2-syllable of $\phi_i$ with the second 2-syllable of $\psi_j$.
\end{definition}

\begin{theorem}[The Cycles of a Child Node of a Node With a Singular Cycle and a Node With 2 Coexisting Cycles]
	The child node of a node with a singular cycle $\phi = \phi_1\phi_2 \dots \phi_n$ and a node with two coexisting cycles $\phi^a = \phi^a_1\phi^a_2 \dots \phi^a_m$ and $\phi^b_1\phi^b_2 \dots \phi^b_m$ will have one of the following cycles.
	\begin{enumerate}
		\item If $\phi$ is associated with the left parent.
		      \begin{align*}
			      \phi_1 \dots \phi_{\frac{n-1}{2}} \left[\phi_{\frac{n+1}{2}} \mid \phi^b_m\right]
			      \phi^b_1 \dots \phi^b_{m-1} \left[\phi^b_m \mid \phi_{\frac{n+1}{2}}\right]
			      \phi_{\frac{n+3}{2}} \dots \phi_n \phi^a
		      \end{align*}
		\item If $\phi$ is associated with the right parent.
		      \begin{align*}
			      \phi^a \phi_1 \dots \phi_{\frac{n-1}{2}} \left[\phi_{\frac{n+1}{2}} \mid \phi^b_m\right]
			      \phi^b_1 \dots \phi^b_{m-1} \left[\phi^b_m \mid \phi_{\frac{n+1}{2}}\right]
			      \phi_{\frac{n+3}{2}} \dots \phi_n
		      \end{align*}
		      Which is shift-equivalent to the first case.
		      But this distinction must be made to guarantee the correctness of cycles in subsequent child nodes.
	\end{enumerate}
\end{theorem}

\begin{proof} \phantom{x}
	\begin{enumerate}
		\item Let $\sigma = \sigma_1\sigma_2 \dots \sigma_i$ with odd $i$ and $\rho = \rho_1\rho_2 \dots \rho_j$ with even $j$ and $T(\sigma) = \phi, T(\rho) = \psi^a$, and $T(s_2(\rho)) = \psi^b$.
		      The child of both nodes in the halved model will have the cycle $\sigma\rho = \sigma_1\sigma_2 \dots \sigma_i \rho_1\rho_2 \dots \rho_j$.
		      This will manifest as the following cycle in the full model.
		      \begin{align*}
			      T(\sigma\rho) & = T(\sigma_1 \dots \sigma_i \rho_1 \dots \rho_j)                                                                              \\
			                    & =
			      t(\sigma_1\sigma_2) \dots t(\sigma_i \rho_1) \dots t(\rho_j \sigma_1) \dots t(\sigma_{i-1}\sigma_i) t(\rho_1\rho_2) \dots t(\rho_{j-1}\rho_j) \\
			                    & = \phi_1 \dots \phi_{\frac{n-1}{2}} t(\sigma_i \rho_1)
			      \psi^b_1 \dots \rho^b_{m-1} t(\rho_j \sigma_1)
			      \phi_{\frac{n+3}{2}} \dots \phi_n
			      \psi^a_1 \dots \psi^a_m                                                                                                                       \\
			                    & =
			      \phi_1 \dots \sigma_{\frac{n-1}{2}} \left[\sigma_{\frac{n+1}{2}} \mid \rho^b_m\right]
			      \psi^b_1 \dots \psi^b_{m-1} \left[\rho^b_m \mid \sigma_{\frac{n+1}{2}}\right]
			      \phi_{\frac{n+3}{2}} \dots \phi_n
			      \psi^a                                                                                                                                        \\
		      \end{align*}
		\item Let $\sigma = \sigma_1\sigma_2 \dots \sigma_i$ with even $i$ and $\rho = \rho_1\rho_2 \dots \rho_j$ with odd $j$ and $T(\sigma) = \psi^a, T(s_2(\sigma)) = \psi^b$, and $T(\rho) = \phi$.
		      The child of both nodes in the halved model will have the cycle $\sigma\rho = \sigma_1l_2 \dots \sigma_i \rho_1\rho_2 \dots \rho_j$.
		      This will manifest as the following cycle in the full model.
		      \begin{align*}
			      T(\sigma\rho) & = T(\sigma_1 \dots \sigma_i \rho_1 \dots \rho_j)                                                                                             \\
			                    & =
			      t(\sigma_1\sigma_2) \dots t(\sigma_{i-1}\sigma_i) t(\rho_1\rho_2) \dots t(\rho_j \sigma_1) \dots t(\sigma_i\rho_1) \dots t(\rho_j\sigma_1) \dots t(\sigma_i) \\
			                    & =
			      \psi^a_1 \dots \psi^a_m
			      \phi_1 \dots \phi_{\frac{n-1}{2}} t(\sigma_i \rho_1)
			      \psi^b_1 \dots \psi^b_{m-1} t(\rho_j \sigma_1)
			      \phi_{\frac{n+3}{2}} \dots \phi_n                                                                                                                            \\
			                    & =
			      \psi^a
			      \phi_1 \dots \phi_{\frac{n-1}{2}} \left[\phi_{\frac{n+1}{2}} \mid \psi^b_m\right]
			      \psi^b_1 \dots \psi^b_{m-1} \left[\psi^b_m \mid \phi_{\frac{n+1}{2}}\right]
			      \phi_{\frac{n+3}{2}} \dots \phi_n                                                                                                                            \\
		      \end{align*}
	\end{enumerate}
\end{proof}

\begin{theorem}
	The child node of two nodes with a singular cycle, $\phi = \phi_1 \dots \phi_n$ and $\psi = \psi_1 \dots \psi_m$ respectively, has the following two cycles.
	\begin{align*}
		\phi_1 \dots \phi_{\frac{n-1}{2}} \left[\phi_{\frac{n+1}{2}} \mid \psi_{\frac{m+1}{2}}\right] \psi_{\frac{m+3}{2}} \dots \psi_m
	\end{align*}
	and
	\begin{align*}
		\phi_{\frac{n+3}{2}} \dots \phi_n \psi_1 \dots \psi_{\frac{m-1}{2}} \left[\psi_{\frac{m+1}{2}} \mid \phi_{\frac{n+1}{2}}\right]
	\end{align*}
\end{theorem}

\begin{proof}
	Let $\sigma = \sigma_1\sigma_2 \dots \sigma_i$ with odd $i$ and $\rho = \rho_1\rho_2 \dots \rho_j$ with odd $j$ and $T(\sigma) = \phi$ and $T(\rho) = \psi$.
	The child of both nodes in the halved model will have the cycle $\sigma\rho = \sigma_1 \dots \sigma_i \rho_1 \dots \rho_j$.
	This will manifest as the following two cycles in the full model.
	\begin{align*}
		T(\sigma\rho) & = T(\sigma_1 \dots \sigma_i \rho_1 \dots \rho_j)                                                                                  \\
		              & = t(\sigma_1\sigma_2) \dots t(\sigma_i\rho_1) \dots t(\rho_{j-1}\rho_j)                                                           \\
		              & = \phi_1 \dots \phi_{\frac{n-1}{2}} t(\sigma_i\rho_1) \psi_{\frac{m+3}{2}} \dots \psi_j                                           \\
		              & = \phi_1 \dots \phi_{\frac{n-1}{2}} \left[\phi_{\frac{n+1}{2}} \mid \psi_{\frac{m+1}{2}}\right] \psi_{\frac{m+3}{2}} \dots \psi_j
	\end{align*}
	and
	\begin{align*}
		T(s(\sigma\rho)) & = T(\sigma_2 \dots \sigma_i \rho_1 \dots \rho_j \sigma_1)                                                                         \\
		                 & = t(\sigma_2\sigma_3) \dots t(\sigma_{i-1}\sigma_i) t(\rho_1\rho_2) \dots t(\rho_j\sigma_1)                                       \\
		                 & = \phi_{\frac{n+3}{2}} \dots \phi_n \psi_1 \dots \psi_{\frac{m-1}{2}} t(\rho_j\sigma_1)                                           \\
		                 & = \phi_{\frac{n+3}{2}} \dots \phi_n \psi_1 \dots \psi_{\frac{m-1}{2}} \left[\psi_{\frac{m+1}{2}} \mid \phi_{\frac{n+1}{2}}\right]
	\end{align*}
\end{proof}

\todo{last case not possible in our adding structures. proof!}

As mentioned before, the next case does not appear in the fare tree in \Cref{fig:tree.adding1.hor.full}.
But we will include it here for completeness.

\begin{theorem}
	The child node of two nodes with two coexisting cycles each, $\{\phi^a, \phi^b\}$ and $\{\phi^a, \phi^b\}$ respectively, has the following two cycles.
	\begin{align*}
		\phi^a\phi^a \qquad \text{and} \qquad
		\phi^b_1 \dots \phi^b_{n-1} \left[\phi^b_n \mid \phi^b_m\right] \phi^b_1 \dots \phi^b_{m-1} \left[\phi^b_m \mid \phi^b_n\right]
	\end{align*}
\end{theorem}

\begin{proof}
	Let $\sigma = \sigma_1 \dots \sigma_i$ with even $i$ and $\rho = \rho_1 \dots \rho_j$ with even $j$ and $T(\sigma) = \phi^a, T(s_2(\sigma)) = \phi^b, T(\rho) = \phi^a$, and $T(s_2(\rho)) = \phi^b$.
	The child of both nodes in the halved model will have the cycle $\sigma\rho$.
	This will manifest as the following two cycles in the full model.
	\begin{align*}
		T(\sigma\rho) & = T(\sigma_1 \dots \sigma_i \rho_1 \dots \rho_j) = t(\sigma_1\sigma_2) \dots t(\sigma_{i-i}\sigma_i) t(\rho_1\rho_2) \dots t(\rho_{j-1}\rho_j) \\
		              & = \phi^a_1 \dots \phi^a_n \phi^a_1 \dots \phi^a_m = \phi^a\phi^a
	\end{align*}
	and
	\begin{align*}
		T(s(\sigma\rho)) & = T(\sigma_2 \dots \sigma_i \rho_1 \dots \rho_j \sigma_1) = t(\sigma_2\sigma_3) \dots t(\sigma_i\rho_1) \dots t(\rho_j\sigma_1)   \\
		                 & = \phi^b_1 \dots \phi^b_{n-1} t(\sigma_i\rho_1) \phi^b_1 \dots \phi^b_{m-1} t(\rho_j\sigma_1)                                     \\
		                 & = \phi^b_1 \dots \phi^b_{n-1} \left[\phi^b_n \mid \phi^b_m\right] \phi^b_1 \dots \phi^b_{m-1} \left[\phi^b_m \mid \phi^b_n\right]
	\end{align*}
\end{proof}


