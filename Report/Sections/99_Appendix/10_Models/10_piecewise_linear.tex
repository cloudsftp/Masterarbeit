\subsection{Piecewise Linear Model}

The first model in this chapter and also the first constructed model is the piecewise-linear model.
It has 4 linear branches and is defined as the map $x_{n+1} = f(x_n) \mod 1$ where the following set of equations defines $f$.

\begin{align}
	f(x) & = \begin{cases}
		         g(x)                                        & \text{ if } x < \frac{1}{2} \\
		         g\left(x - \frac{1}{2}\right) + \frac{1}{2} & \text{ else}
	         \end{cases} \label{equ:app.model.lin.f} \\
	g(x) = \begin{cases}
		       g_L(x) = \alpha \cdot x + \beta            & \text{ if } x < \frac{1}{4} \\
		       g_R(x) = \alpha \cdot x - \frac{\alpha}{4} & \text{ else}
	       \end{cases} \label{equ:app.model.lin.g}
\end{align}

One can see that this model definition is a little different from the model definitions in the main part of the thesis.
For example, \Cref{equ:app.model.lin.g} also enforces the symmetry that is found in the original model in this model explicitly.
And \Cref{equ:app.model.lin.g} then breaks each half of the model function into two smaller parts.
One difference is that $\alpha$ influences the slope of all four branches and also influences the offset of the function $g_R$.
$g_R$ governs the branches $f_\B$ and $f_\D$ and its offset causes the branch $f_\B$ to start at $0$ and the branch $f_\D$ to start at $\frac{1}{2}$.

\begin{figure}
	\centering
	\includegraphics[width=0.7 \textwidth]{../Figures/A/A.1/result.png}
	\caption[2D scan of periods associated with parameter regions in the piecewise-linear model]{
		2D scan of periods associated with parameter regions in the piecewise-linear model.
		The red arrow marks the parameter range used for the 1D scan in \Cref{fig:app.model.lin.1D}.
	}
	\label{fig:app.model.lin.2D}
\end{figure}

\begin{figure}
	\centering
	\includegraphics[width=0.7 \textwidth]{../Figures/A/A.2/result.png}
	\caption{1D scan of periods of the piecewise-linear model}
	\label{fig:app.model.lin.1D}
\end{figure}

\todo{old:}
\Cref{fig:pcw.lin.2d} shows a bifurcation diagram of the model described above.
The parameter $\beta$ is varied on the interval $[0, 2 \pi]$, because the model will behave the same for $[0, 2 \pi] + k \cdot 2 \pi$.
This is because the result of the function will be taken modulo $2 \pi$ in \Cref{equ:pcw.lin.f}.
The parameter $\alpha$ is varied on the interval $[0, 1]$ because for $\alpha < 0$ nothing especially interesting happens and for $\alpha > 1$ the model shows no periodic behavior.

The two red lines mark the locations of the following one-dimensional scans keeping the parameter $\alpha$ fixed at $0.5$.
\Cref{fig:pcw.lin.1D} shows the scan along the lower red line.
There one can see a period-adding structure.
\Cref{fig:pcw.lin.1DPlusPi} shows the scan along the other red line.
The whole thing is not a period-adding structure, but there is a period-adding structure on the left and on the right of the line in the middle.
Both these structures don't look like period-adding structures normally do.
The interval in the middle of each structure is not the biggest interval with a constant period, but rather some interval that's further to the middle of both structures.

\Cref{fig:pcw.lin.CobwebA-C} shows a collection of cobwebs diagrams of the three points $P_A$ to $P_C$ marked in \Cref{fig:pcw.lin.1D}.
Note that in this period-doubling structure, all cycles exist in pairs.
At point $P_A$, there are two cycles of period 2 with the symbolic sequences $\A\B$ and $\C\D$ respectively.
When we make the parameter $\beta$ smaller, the cycles move toward the left of the arms.
\Cref{fig:pcw.lin.CobwebA} shows the cycles at the edge of the arms, shortly after that, the cycles collide with the border of the arms.

In the middle interval of \Cref{fig:pcw.lin.1D} with constant period 3, the cycles of $P_A$ are added with the fixed points of the left side.
This Farey-like-adding of cycles is common in period-adding structures~\cite{avrutin2019continuous}.
The fixed points are not shown here but are in the arms $\A$ and $\C$ respectively.
The resulting cycles are $\A^2\B$ and $\C^2\D$ and are shown in \Cref{fig:pcw.lin.CobwebB,fig:pcw.lin.CobwebC}.
These cobweb diagrams also make the movement of the cycles with decreasing parameter $\beta$, mentioned above, clear.
At $P_B$, $\beta$ is bigger and the cycles are at the right edge of the arms.
At $P_C$, the cycles are again at the left edge of the arms.

\begin{figure}
	\centering
	\begin{subfigure}{0.3\textwidth}
		\centering
		\includegraphics[width=\textwidth]{10_Linear_mod2pi/Cobweb_A/result.png}
		\caption{$P_A$}
		\label{fig:pcw.lin.CobwebA}
	\end{subfigure}
	\begin{subfigure}{0.3\textwidth}
		\centering
		\includegraphics[width=\textwidth]{10_Linear_mod2pi/Cobweb_B/result.png}
		\caption{$P_B$}
		\label{fig:pcw.lin.CobwebB}
	\end{subfigure}
	\begin{subfigure}{0.3\textwidth}
		\centering
		\includegraphics[width=\textwidth]{10_Linear_mod2pi/Cobweb_C/result.png}
		\caption{$P_C$}
		\label{fig:pcw.lin.CobwebC}
	\end{subfigure}
	\caption{Cobwebs for first 1D Scan}
	\label{fig:pcw.lin.CobwebA-C}
\end{figure}

\Cref{fig:pcw.lin.CobwebD-F} shows three cobwebs for different points of the second 1D scan in \Cref{fig:pcw.lin.1DPlusPi}.
You can see that the cycles here do not follow Farey-like-adding like in the other 1D scan.
This is of course because the three cobwebs do not belong to one period adding structure.
The cycles are nonetheless interesting.
The two-cycle at $P_D$, shown in \Cref{fig:pcw.lin.CobwebD}, has the symbolic sequence $\A\C$.
It is on the right edge of the arms at this point and after colliding and skipping the period adding structure, two coexisting cycles appear, shown in \Cref{fig:pcw.lin.CobwebE}.
They have symbolic sequences $\A\D\C$ and $\A\C\B$ respectively.
It is safe to assume that in the period adding structure between the two points, $P_D$ and $P_E$, the cycles follow Farrey-adding.

\Cref{fig:pcw.lin.CobwebF} shows the 4-cycle at the point $P_F$.
It has the symbolic sequence $\A\D\C\B$ and is at the left edge of the arms.
Lowering the parameter $\beta$ will cause the cycle to collide with the borders and this in turn will lead to the period adding structure we see in the 1D scan in \Cref{fig:pcw.lin.1DPlusPi}.
This period-adding structure will also follow the Farey-like-adding of this cycle and the 3-cycles at point $P_E$, shown in \Cref{fig:pcw.lin.CobwebE}.

\begin{figure}
	\centering
	\begin{subfigure}{0.3\textwidth}
		\centering
		\includegraphics[width=\textwidth]{10_Linear_mod2pi/Cobweb_D/result.png}
		\caption{$P_D$}
		\label{fig:pcw.lin.CobwebD}
	\end{subfigure}
	\begin{subfigure}{0.3\textwidth}
		\centering
		\includegraphics[width=\textwidth]{10_Linear_mod2pi/Cobweb_E/result.png}
		\caption{$P_E$}
		\label{fig:pcw.lin.CobwebE}
	\end{subfigure}
	\begin{subfigure}{0.3\textwidth}
		\centering
		\includegraphics[width=\textwidth]{10_Linear_mod2pi/Cobweb_F/result.png}
		\caption{$P_F$}
		\label{fig:pcw.lin.CobwebF}
	\end{subfigure}
	\caption{Cobwebs for second 1D Scan}
	\label{fig:pcw.lin.CobwebD-F}
\end{figure}
