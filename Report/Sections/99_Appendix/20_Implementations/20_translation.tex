\section{Translating Symbolic Sequences}

This section covers the implementation of the naive algorithm for translating symbolic sequences from the halved archetypal model into the archetypal model.
It is implemented in the programming language Rust.
The program can be obtained via \texttt{cargo} at the crate~\cite{TranslatingSymbolicCrate}.
\Cref{lst:app.impl.translating.cycles} shows the implementation of \texttt{HalvedCycle} and \texttt{FullCycle}.
Both structs use the type \texttt{Sequence} for representing the symbolic sequence in memory with different symbols each, \texttt{SYMBOLS\_FULL} and \texttt{SYMBOLS\_HALVED}.
From line 52, the trait \texttt{RotationEquivalence} is defined and implemented for the type \texttt{Sequence}.
It is used to implement equality via the \texttt{PartialEq} trait for both structs \texttt{HalvedCycle} and \texttt{FullCycle}.
As far as the program is concerned, two cycles that are rotation invariant are thus equal.
The implementations for displaying the cycles to the user are omitted.

\lstinputlisting[
	language=Rust,
	label=lst:app.impl.translating.cycles,
	caption=Implementation of Cycles for the Naive Translating Algorithm,
]
{Sections/99_Appendix/20_Implementations/20_0_cycles.rs}

\clearpage
\Cref{lst:app.impl.translating} shows the implementation of the naive algorithm for translating symbolic sequences.
It uses the definitions of \texttt{HalvedCycle} and \texttt{FullCycle} from \Cref{lst:app.impl.translating}.
The algorithm is implemented as the \texttt{TryFrom} trait.
This implementation returns a \texttt{Vec} of \texttt{FullCycle}s, since the cycle in the halved archetypal model may manifest as multiple cycles in the archetypal model.
It takes the symbolic sequence from the \texttt{HalvedCycle} and duplicates it such that the original symbolic sequence is now three times in the symbolic sequence in the variable \texttt{seq}.
Then it translates the symbolic sequences starting at different 2-syllables.
This translation step is wrapped in the function \texttt{translate\_to\_full\_from\_index}.
That function is basically the implementation of the function $T$ defined in \Cref{sec:add.halved.tanslating}.
Then the translated symbolic sequences are used to create \texttt{FullCycle}s.
Finally, the implementation filters rotation equivalent \texttt{FullCycle}s.
This is done with the \texttt{for}-loop that starts on line 20.
The detection of rotation equivalent cycles is easy, since the program considers rotation equivalent \texttt{FullCycle}s as equal.


\lstinputlisting[
	language=Rust,
	label=lst:app.impl.translating,
	caption=Implementation of the Naive Translating Algorithm,
]
{Sections/99_Appendix/20_Implementations/20_1_translating.rs}
