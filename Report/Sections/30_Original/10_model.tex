\section{Model Definition}
\label{sec:og.def}

The original model was defined by \Citeauthor{akyuz2022} in their master thesis as the map $\theta \mapsto F(\theta) \mod 2 \pi$.
Where $F$ is given by the following collection of equations~\cite{akyuz2022}.
\begin{align}
    F(\theta) = \begin{cases}
                    F_1(\theta) & \text{if } q \cdot \cos(\theta) > 0 \\
                    F_2(\theta) & \text{if } q \cdot \cos(\theta) < 0
                \end{cases}
\end{align}
Where $F_1$ is given by the following equation.
\begin{align}
    F_1(\theta) & = \begin{cases}
                        \theta + z_{L_+} + z_1 & \text{if } z_{L_+} < z_{L_0} \\
                        \theta + z_{L_0} + z_2 & \text{if } z_{L_+} > z_{L_0}
                    \end{cases}
\end{align}
And $F_2$ is similarly given by the following equation.
\begin{align}
    F_2(\theta) & = \begin{cases}
                        \theta + z_{R_+} + z_3 & \text{if } z_{R_+} < z_{R_0} \\
                        \theta + z_{R_0} + z_4 & \text{if } z_{R_+} > z_{R_0}
                    \end{cases}
\end{align}
$z_{L_+}, z_{L_0}, z_{R_+},$ and $z_{R_0}$ must satisfy the following equations.
\begin{subequations}
    \begin{align}
        (q \cdot \cos(\theta) + \mu \cdot \chi) \cdot e^{\lambda \cdot z_{L_+}}
         & = q \cdot \cos(\theta + z_{L_+} + z_1) + \mu \cdot \chi \\
        (q \cdot \cos(\theta) + \mu \cdot \chi) \cdot e^{\lambda \cdot z_{L_0}}
         & = q \cdot \cos(\theta + z_{L_0} + z_1) - \mu \cdot \chi \\
        (q \cdot \cos(\theta) + \mu \cdot \chi) \cdot e^{\lambda \cdot z_{R_+}}
         & = q \cdot \cos(\theta + z_{R_+} + z_1) + \mu \cdot \chi \\
        (q \cdot \cos(\theta) + \mu \cdot \chi) \cdot e^{\lambda \cdot z_{R_0}}
         & = q \cdot \cos(\theta + z_{R_0} + z_1) - \mu \cdot \chi
    \end{align}
\end{subequations}
While $z_1, z_2, z_3,$ and $z_4$ must satisfy these equations.
\begin{subequations}
    \begin{align}
        (q \cdot \cos(\theta + z_{L_+}) + \chi + 1) \cdot e^{\lambda \cdot z_1} - 1
         & = q \cdot  \cos(\theta + z_{L_+} + z_1) + \mu \cdot \chi \\
        (q \cdot \cos(\theta + z_{L_0}) + \chi + 1) \cdot e^{\lambda \cdot z_2} + 1
         & = q \cdot  \cos(\theta + z_{L_0} + z_2) - \mu \cdot \chi \\
        (q \cdot \cos(\theta + z_{R_+}) + \chi + 1) \cdot e^{\lambda \cdot z_3} - 1
         & = q \cdot  \cos(\theta + z_{L_+} + z_3) + \mu \cdot \chi \\
        (q \cdot \cos(\theta + z_{R_0}) + \chi + 1) \cdot e^{\lambda \cdot z_4} + 1
         & = q \cdot  \cos(\theta + z_{R_0} + z_4) - \mu \cdot \chi
    \end{align}
\end{subequations}

\todo{validate formulas, 2.5b and 2.5c missing $+ z_i$ in $\cos$}

The values for $\chi, \lambda, \mu,$ and $q$ come from the parameters of the model.
The parameters are $\chi_0, E_0, \beta, f, L, R, V_m,$ and $\mu$.
$\mu$ is directly applied in the equations above, while $\chi$ and $q$ consist of multiple parameters.
The values of $\chi$ and $q$ are given by the following equations.
\begin{align}
    \chi    & = \dfrac{R \cdot \chi_0}{\beta \cdot E_0} \\
    \lambda & = \dfrac{-R}{L \cdot 2 \cdot \pi \cdot f} \\
    q       & = \dfrac{R \cdot V_m}{\beta \cdot E_0}
\end{align}
\todo{stimmt das so?}
