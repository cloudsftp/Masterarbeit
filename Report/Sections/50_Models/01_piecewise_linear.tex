\section{Piecewise Linear Model}

The first model, we examined, is a simple piecewise linear function with four discontinuities.
Its discontinuities are at $\frac{\pi}{2}, \pi,$ and $\frac{3 \pi}{2}$.
\Cref{equ:pcw.lin.sympi} causes the discontinuity at $\pi$ and also the symmetry $f(x + \pi) \equiv f(x) + \pi \mod 2 \pi$.

\begin{align}
    f(x) & = g(x) \mod 2 \pi \\
    g(x) & = \begin{cases}
        h(x) & \text{ if } r(x) < \pi \\
        h(x) + \pi & \text{ else}
    \end{cases} \label{equ:pcw.lin.sympi}
\end{align}

Each arm then is governed by \Cref{equ:pcw.lin.discpihalves}.
It causes the discontinuities at $\frac{\pi}{2}$ and $\frac{3 \pi}{2}$ and also shows the linear nature of the function.
Both parameters $\alpha$ and $\beta$ act here.
$\alpha$ is the slope of the arms and $\beta$ is the offset of the first and third arms.
\todo{refer to later cobwebs}

\begin{align}
    h(x) & = \begin{cases}
        \alpha \cdot t(x) + \beta & \text{ if } s(x) < \frac{\pi}{2} \\
        \alpha \cdot t(x) & \text{ else}
    \end{cases} \label{equ:pcw.lin.discpihalves}
\end{align}

\Crefrange{equ:pcw.lin.r}{equ:pcw.lin.t} are used to make the above more readable.
They give the modulus of $x$ and some multiple of $\frac{\pi}{2}$.

\begin{align}
    r(x) & = x \mod 2 \pi \label{equ:pcw.lin.r} \\
    s(x) & = x \mod \pi \\
    t(x) & = x \mod \frac{\pi}{2} \label{equ:pcw.lin.t}
\end{align}
