\subsection{Behavior}
\label{sec:setup.arch.behavior}

\Cref{fig:setup.arch.period} shows the 2D scans of the periods associated wit parameter regions in the archetypal model.
As before, \Cref{fig:setup.arch.period.halved} shows the halved model to indicate ``type B'' parameter regions.
The structure is not very different from the previous constructed model in \Cref{sec:setup.quad.hyper.2}.
There are still chains of parameter regions associated with the same period next to each other with the period increasing by two for each chain.
And the types of the parameter regions in each chain alternate between ``type A'' and ``type B''.
Now the ``type B'' parameter regions are even more prominent in the chains for larger values of $\beta$.
This was not the case in the previous model with four quadratic branches.

\Cref{fig:setup.arch.cobwebs} shows cobweb diagrams at the parameter values that are marked with points in \Cref{fig:setup.arch.period}, as also done in previous sections of this chapter.
Here, the model also behaves like the original model with the symbolic sequence of the cycle at the point $A$ being $\A^6\B^3\C^6\D^3$, at the point $C$ being $\A^5\B^4\C^5\D^4$, and in between both parameter regions associated with each cycle, two coexisting cycles with the symbolic sequences $\A^6\B^3\C^5\D^4$ and $\A^5\B^4\C^6\D^3$ at the point $C$.
The next chapter covers the behavior of the archetypal model in-depth.

\begin{figure}
	\centering
	\subfloat[Model]{
		\includegraphics[width=.48 \textwidth]{../Figures/5/5.14a/result.png}
		\label{fig:setup.arch.period.full}
	}
	\subfloat[Halved Model]{
		\includegraphics[width=.48 \textwidth]{../Figures/5/5.14b/result.png}
		\label{fig:setup.arch.period.halved}
	}
	\caption[2D scans of the periods of the archetypal model]{
		2D scan of the periods of the archetypal model with composite parameters $g_R\left(\frac{1}{4}\right)$ and $g_R\left(\frac{1}{2}\right)$.
		The parameters $a_L = 4, b_L = -\frac{1}{2},$ and $g_R\left(\frac{1}{2}\right) = \frac{1}{2} + \frac{1}{40}$ are fixed.
		The parameters $\alpha = -g_R\left(\frac{1}{4}\right)$ and $\beta = c_L$ are varied in the ranges $[-0.45, -0.275]$ and $[0.15, 0.1875]$, respectively.
		The points $A, B,$ and $C$ mark the parameter values used for the cobweb diagrams in \Cref{fig:setup.arch.cobwebs}.
		(a) shows the scan for the model as defined above, while (b) shows the scan for the halved model where we can see ``type B'' parameter regions as they have higher periods than the ``type A'' parameter regions of the same chain.
		\todo{Add periods in figures}
	}
	\label{fig:setup.arch.period}
\end{figure}

\begin{figure}
	\centering
	\subfloat[$A$]{
		\includegraphics[width=.3 \textwidth]{../Figures/5/5.15a/result.png}
		\label{fig:setup.arch.cobweb.A}
	}
	\subfloat[$B$]{
		\includegraphics[width=.3 \textwidth]{../Figures/5/5.15b/result.png}
		\label{fig:setup.arch.cobweb.B}
	}
	\subfloat[$C$]{
		\includegraphics[width=.3 \textwidth]{../Figures/5/5.15c/result.png}
		\label{fig:setup.arch.cobweb.C}
	}
	\caption[Cobwebs of the archetypal model]{
		Cobweb diagrams at three parameter values of $\alpha = -g_R\left(\frac{1}{4}\right)$ and $\beta = c_L$ in the archetypal model.
		The other parameters are fixed as $a_L = 4, b_L = -\frac{1}{2},$ and $g_R\left(\frac{1}{2}\right) = \frac{1}{2} + \frac{1}{40}$.
		The parameter values are marked in \Cref{fig:setup.arch.period}.
		(a) shows the cycle $\Cycle{\A^6\B^3\C^6\D^3}$ the at point $A$where $\alpha = -0.4$ and $\beta = 0.16$,
		(b) shows the two coexisting cycles $\Cycle{\A^6\B^3\C^5\D^4}$ (green) and $\Cycle{\A^5\B^4\C^6\D^3}$ (red) at the point $B$where $\alpha = -0.378$ and $\beta = 0.1612$,
		and (c) shows the cycle $\Cycle{\A^5\B^4\C^5\D^4}$ at the point $C$where $\alpha = -0.36$ and $\beta = 0.164$.
	}
	\label{fig:setup.arch.cobwebs}
\end{figure}
