\subsection{Compound Parameter Definitions}
\label{sec:setup.quad.hyper.params}

The \hl{compound} parameters are $g_R\left(\frac{1}{4}\right)$ for the value at the left border of branches $f_\B$ and $f_\D$, $g_R\left(\frac{1}{2}\right)$ for the value at the right border of the branches, and finally $\left. \frac{d}{dx} g_R(x) \right|_{x = \frac{1}{2}}$ for the slope of the branches at the right border.
\hl{The compound parameter} $\left. \frac{d}{dx} g_R(x) \right|_{x = \frac{1}{2}} = 1.2$ \hl{is fixed}.
This way, the steepest slope is 1.2 which is just above 1.
Therefore, most of the function is still contractive.
\hl{Also, the compound parameter} $g_R\left(\frac{1}{2}\right) = \frac{1}{2} + \epsilon$ \hl{is fixed} with $\epsilon = 0.025$ to have the value at the right border of the branches $f_\B$ and $f_\D$ just above the bisector $y = x$.
The \hl{only compound parameter that is not fixed yet is} $g_R\left(\frac{1}{4}\right)$, the value \hl{of the model function} at the left border of the branches $f_\B$ and $f_\D$.
This compound parameter is varied.

\begin{subequations}
	\begin{align}
		g_R\left(\frac{1}{4}\right)                                     & = a_R \cdot \left(\frac{1}{4}\right)^2 + b_R \cdot \left(\frac{1}{4}\right) + c_R = \dfrac{a_R}{16} + \dfrac{b_R}{4} + c_R \label{equ:setup.quad.hyper.A} \\
		g_R\left(\frac{1}{2}\right)                                     & = a_R \cdot \left(\frac{1}{2}\right)^2 + b_R \cdot \left(\frac{1}{2}\right) + c_R = \dfrac{a_R}{4} + \dfrac{b_R}{2} + c_R \label{equ:setup.quad.hyper.B}  \\
		\left. \frac{d}{dx} g_R\left(x\right) \right|_{x = \frac{1}{2}} & = 2 \cdot a_R \cdot \left(\frac{1}{2}\right) + b_R \label{equ:setup.quad.hyper.C}
	\end{align}
\end{subequations}

\Crefrange{equ:setup.quad.hyper.A}{equ:setup.quad.hyper.A} are the values of the compound parameters $g_R\left(\frac{1}{4}\right), g_R\left(\frac{1}{2}\right),$ and $\left. \frac{d}{dx} g_R\left(x\right) \right|_{x = \frac{1}{2}}$.
This is a system of equations \hl{that need to be solved} for the parameters $a_R, b_R,$ and $c_R$.
To compute the parameters, \hl{one can write} the system \hl{of equations} as a matrix and invert it.
The matrix and its inverse are in \Cref{equ:setup.quad.hyper.matrix}.

\begin{align}
	\begin{pmatrix}
		\frac{1}{16} & \frac{1}{4} & 1 \\
		\frac{1}{4}  & \frac{1}{2} & 1 \\
		1            & 1           & 0
	\end{pmatrix}^{-1} & =
	\begin{pmatrix}
		16  & -16 & 4           \\
		-16 & 16  & -3          \\
		4   & -3  & \frac{1}{2}
	\end{pmatrix}
	\label{equ:setup.quad.hyper.matrix}
\end{align}

Hence, the equations for \hl{the parameters} $a_R, b_R,$ and $c_R$ in dependence of \hl{the compound parameters} $g_R\left(\frac{1}{4}\right), g_R\left(\frac{1}{2}\right),$ and $\left. \frac{d}{dx} g_R\left(x\right) \right|_{x = \frac{1}{2}}$ are \Crefrange{equ:setup.quad.hyper.aR}{equ:setup.quad.hyper.cR}.

\begin{align}
	a_R & = 16 \cdot g_R\left(\frac{1}{4}\right) - 16 \cdot g_R\left(\frac{1}{2}\right) + 4 \cdot \left. \frac{d}{dx} g_R\left(x\right) \right|_{x = \frac{1}{2}}     \label{equ:setup.quad.hyper.aR}     \\
	b_R & = -16 \cdot g_R\left(\frac{1}{4}\right) + 16 \cdot g_R\left(\frac{1}{2}\right) - 3 \cdot \left. \frac{d}{dx} g_R\left(x\right) \right|_{x = \frac{1}{2}} \label{equ:setup.quad.hyper.bR}        \\
	c_R & = 4 \cdot g_R\left(\frac{1}{4}\right) - 3 \cdot g_R\left(\frac{1}{2}\right) + \frac{1}{2} \cdot \left. \frac{d}{dx} g_R\left(x\right) \right|_{x = \frac{1}{2}} \label{equ:setup.quad.hyper.cR}
\end{align}
