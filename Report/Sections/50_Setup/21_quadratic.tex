\subsection{Piecewise Quadratic Model}
\label{sec:setup.quad}

We start with a model that is piecewise quadratic.
In the original model, the branches $f_\B$ and $f_\D$ are shaped more like cubic functions, but to keep the number of parameters low at the beginning, we model them as quadratic here.
The model has 4 branches and the same symmetry as the original model function.
We define the domain here as $[0, 1]$ instead of $[0, 2\pi]$.

The model is defined as the map $x_{n+1} = f(x_n) \mod 1$.
Where $f$ is given by the following collection of equations.
\begin{align}
	f(x) & = \begin{cases}
		         g(x)                             & \text{if } x < \frac{1}{2} \\
		         g(x - \frac{1}{2}) + \frac{1}{2} & \text{else}
	         \end{cases} \label{equ:quad.full.f}           \\
	g(x) & = \begin{cases}
		         g_L(x) = a_L \cdot x^2 + b_L \cdot x + c_L & \text{if } x < \frac{1}{4} \\
		         g_R(x) = a_R \cdot x^2 + b_R \cdot x + c_R & \text{else}
	         \end{cases} \label{equ:quad.full.g}
\end{align}

\Cref{equ:quad.full.f} explicitly states the discontinuity at $0$ and $3$.
It also explicitly states the symmetry of the model.
Each half of the model is then governed by \Cref{equ:quad.full.g}.
Here all the 6 parameters $a_L, a_R, b_L, b_R, c_L,$ and $c_R$ act.
