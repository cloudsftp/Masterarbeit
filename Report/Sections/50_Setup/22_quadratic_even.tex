\subsection{Centered Parabola-shaped Branches}
\label{sec:setup.quad.even}

This section examines the piecewise-quadratic model with the centered parabola-shaped branches, see the functions in \Cref{fig:setup.quad.even.cobwebs}.
To center the parabola-shaped branches, the parameter values $a_L = a_R = 6$, $b_L = -\frac{3}{2}$, and $b_R = -\frac{9}{2}$ are chosen and only the parameters $c_L$ and $c_R$ are varied.
Both varied parameters are in the ranges $[0.25, 0.6]$.

This emulates the effect that $\chi_0$ has on the branches $F_\A$ and $F_\C$.
Increasing $c_L$ increases the values of the branches $f_\A$ and $f_\C$.
The effects of $E_0$ on branches $F_\B$ and $F_\D$ are lowering the values of the function on the left sides of the branches, moving the local minima of the branches to the left, and reducing the value of the function at the minima.
Decreasing $c_R$ does not have the same effects on the branches $f_\B$ and $f_\D$ but rather lowers the values of the function for the whole branches.
\Cref{fig:setup.quad.even.period.full} shows 2D scans of the periods associated with parameter regions in this model.

\begin{figure}
	\centering
	\subfloat[Full]{
		\includegraphics[width=.48 \textwidth]{../Figures/5/5.5a/result.png}
		\label{fig:setup.quad.even.period.full}
	}
	\subfloat[Zoomed]{
		\includegraphics[width=.48 \textwidth]{../Figures/5/5.5b/result.png}
		\label{fig:setup.quad.even.period.zoomed}
	}
	\caption[2D scans of the periods associated with parameter regions in the piecewise quadratic model with centered parabola-shaped branches]{
		2D scans of the periods associated with parameter regions in the piecewise quadratic model with centered parabola-shaped branches
		The parameters $a_L = a_R = 6$, $b_L = -\frac{3}{2}$, and $b_R = -\frac{9}{2}$ are fixed.
		The parameters $c_L$ and $c_R$ are varied in different ranges.
		(a) shows the full structure with the parameters being varied in the range $[0.25, 0.6]$ each,
		(b) shows the ranges marked with a red rectangle in (a), $[0.29, 0.31]$ and $[0.4, 0.45]$.
		The marked points in (b) are the parameter values that are used for the cobweb diagrams in \Cref{fig:setup.quad.even.cobwebs}
	}
\end{figure}

\begin{figure}
	\centering
	\subfloat[$A$]{
		\includegraphics[width=.3 \textwidth]{../Figures/5/5.6a/result.png}
		\label{fig:setup.quad.even.cobweb.A}
	}
	\subfloat[$B$]{
		\includegraphics[width=.3 \textwidth]{../Figures/5/5.6b/result.png}
		\label{fig:setup.quad.even.cobweb.B}
	}
	\subfloat[$C$]{
		\includegraphics[width=.3 \textwidth]{../Figures/5/5.6c/result.png}
		\label{fig:setup.quad.even.cobweb.C}
	}
	\caption[Cobweb diagrams at three parameter values in the piecewise-quadratic model with centered parabola-shaped branches]{
		Cobweb diagrams at three parameter values in the piecewise-quadratic model with centered parabola-shaped branches.
		The parameters $a_L = a_R = 6$, $b_L = -\frac{3}{2}$, and $b_R = -\frac{9}{2}$ are fixed.
		The parameters $c_L$ and $c_R$ are different in every diagram and their values are marked with points in \Cref{fig:setup.quad.even.period.zoomed}.
		(a) shows the cycle $\Cycle{\A^3\B\C^3\D}$ at the point $A$ where $c_L = 0.2925$ and $c_R = 0.41$,
		(b) shows the two coexisting cycles $\Cycle{\A^3\B\C^3\D}$ shown in green and $\Cycle{\A^2\B\C^2\D}$ shown in red at the point $B$ where $c_L = 0.2975$ and $c_R = 0.425$,
		and (c) shows the cycle $\Cycle{\A^2\B\C^2\D}$ at the point $C$ where $c_L = 0.3025$ and $c_R = 0.44$.
	}
	\label{fig:setup.quad.even.cobwebs}
\end{figure}

A phenomenon like in the original model can not be found here.
But something very similar happens at the border of these wing-shaped parameter regions.
\Cref{fig:setup.quad.even.cobwebs} shows the cobweb diagrams at the points marked in \Cref{fig:setup.quad.even.period.zoomed}.
At point $A$, there is one stable cycle with period $8$.
This cycle is depicted in \Cref{fig:setup.quad.even.cobweb.A} and its symbolic sequence is $\A^3B\C^3\D$.
Point $B$ is in a parameter region, where 2 stable cycles coexist.
One can not see this in the 2D scans in \Cref{fig:setup.quad.even.period.full}, since it only ever picks up on one cycle.
\Cref{fig:setup.quad.even.cobweb.B} shows the coexisting cycles at this border.
The symbolic sequences of the two coexisting cycles are $\A^3B\C^3\D$ and $\A^2\B\C^2\D$.
In contrast to the original model, the cycle that existed before in \Cref{fig:setup.quad.even.cobweb.A} with the symbolic sequence $\A^3B\C^3\D$ still exists alongside the new cycle with the symbolic sequence $\A^2\B\C^2\D$.
At point $C$, there is again only one stable cycle.
It has the period $6$ and the symbolic sequence $\A^2\B\C^2\D$.
Therefore, this is the cycle that coexisted with the cycle with the symbolic sequence $\A^3B\C^3\D$ at point $C$.

This is different from the dynamics in the original model in two ways.
First, the cycles before and after the parameter region of coexistence have different periods.
And second, the cycles existing outside the parameter region of coexistence still exist inside the parameter region of coexistence.
In the original model, the cycles existing outside the parameter region of coexistence would disappear at the boundaries and new cycles would emerge inside this parameter region.
Here, one simply observes two overlapping parameter regions which is something different from the original model.
