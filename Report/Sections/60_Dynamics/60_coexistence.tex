\section{Coexistence Scenarios}
\label{sec:arch.coex}

The previous section investigated the bifurcations at the boundaries of ``type A'' and ``type B'' parameter regions.
There, one can see that there is a space between the border collision bifurcations of the ``type B'' cycles and the border collision bifurcation of the ``type A'' cycle where all three cycles coexist at each boundary of the ``type B'' parameter region.
\Cref{fig:arch.coex.regions.F} illustrates this overlap again, the overlapping regions are marked with the points $J, R, S,$ and $T$.
In \Cref{fig:arch.coex.regions.E}, one can see that ``type A'' parameter regions can also overlap with each other.
Here, the overlapping regions are marked with $M, N, O,$ and $P$.
One previously not considered case is that there can also be an overlap of two ``type A'' parameter regions and one ``type B'' parameter region.
This can be seen in \Cref{fig:arch.coex.regions.F} and is marked with points $V, W, X,$ and $Y$.

\begin{figure}
	\centering
	\subfloat[Parameter region $E_{16}$]{
		\includegraphics[width=.4 \textwidth]{../Figures/6/6.8a/result.png}
		\label{fig:arch.coex.regions.E}
	}
	\subfloat[Parameter region $F_{16}$]{
		\includegraphics[width=.4 \textwidth]{../Figures/6/6.8b/result.png}
		\label{fig:arch.coex.regions.F}
	}
	\caption[Magnified 2D scans of the boundaries of parameter regions with different symbolic sequences in the archetypal model]{
		Magnified 2D scans of the boundaries of parameter regions with different symbolic sequences in the archetypal model.
		The parameters $a_L = 4, b_L = -\frac{1}{2},$ and $g_R\left(\frac{1}{2}\right) = \frac{1}{2} + \frac{1}{40}$ are fixed.
		The parameters $\alpha = -g_R\left(\frac{1}{4}\right)$ and $\beta = c_L$ are varied in different ranges.
		(a) shows the ``type A'' parameter region marked with the point $E_{16}$ in \Cref{fig:arch.dyn.period}.
		The parameters are varied in the ranges $[-0.435, -0.37]$ and $[0.16, 0.169]$.
		At the point $L$ there is only one ``type A'' cycle, while the points $M, N, O,$ and $P$ mark locations with two coexisting ``type A'' cycles.
		(b) shows the ``type B'' parameter region marked with the point $F_{16}$ in \Cref{fig:arch.dyn.period}.
		The parameters are varied in the ranges $[-0.385, -0.365]$ and $[0.166, 0.169]$.
		At the point $Q$ there are two coexisting ``type B'' twin cycles, while the points $R, S, T,$ and $U$ mark locations with two coexisting ``type B'' twin cycles and one ``type A'' cycle.
		And the points $V, W, X,$ and $Y$ mark locations with two coexisting ``type B'' twin cycles and two ``type A'' cycles.
	}
	\label{fig:arch.coex.regions}
\end{figure}

Each coexistence scenario is covered in the following.
The first considered scenario is the simplest one where only one ``type A'' cycle exists on its own.


\subsection{Only One ``Type A'' Cycle}
\label{sec:arch.coex.A}

As mentioned above, the most simple case of coexistence in this model is the existence of a stable ``type A'' cycle on its own.
This is the case at the point $L$ in \Cref{fig:arch.coex.regions.E}.
Here, only the stable cycle $\Cycle{\A^5\B^3\C^5\D^3}$ exists.

For this case there is no extra cobweb diagram in this section, since there are already many cobweb diagrams of single ``type A'' cycles in the previous section, \Cref{sec:arch.dynamics}.
For example in \Cref{fig:arch.dyn.cobweb.A,fig:arch.dyn.cobweb.C,fig:arch.dyn.cobweb.E,fig:arch.dyn.cobweb.G,fig:arch.dyn.cobweb.E14,fig:arch.dyn.cobweb.E18}.
And the basin of attraction is the whole state space.

\subsection{Two ``Type A'' Cycles}
\label{sec:arch.coex.AA}

As mentioned at the beginning of \Cref{sec:arch.coex}, ``type A'' parameter regions can overlap.
This causes the coexistence of two ``type A'' cycles.
It can happen in four different ways.
Assuming the stable cycle of the parameter region in the middle is $\Cycle{\A^a\B^b\C^a\D^b}$, it can overlap with parameter regions, where either one of the following cycles is stable
\begin{enumerate*}
	\item $\Cycle{\A^{a-1}\B^b\C^{a-1}\D^b}$,
	\item $\Cycle{\A^a\B^{b+1}\C^a\D^{b+1}}$,
	\item $\Cycle{\A^{a+1}\B^b\C^{a+1}\D^b}$, and
	\item $\Cycle{\A^a\B^{b-1}\C^a\D^{b-1}}$.
\end{enumerate*}
For the specific case pictured in \Cref{fig:arch.coex.regions.E}, this results in the following coexistence scenarios.
\begin{enumerate}
	\item $\A^5\B^3\C^5\D^3$ and $\A^4\B^3\C^4\D^3$, marked with $M$ in \Cref{fig:arch.coex.regions.E},
	\item $\A^5\B^3\C^5\D^3$ and $\A^5\B^4\C^5\D^4$, marked with $N$ in \Cref{fig:arch.coex.regions.E},
	\item $\A^5\B^3\C^5\D^3$ and $\A^6\B^3\C^6\D^3$, marked with $O$ in \Cref{fig:arch.coex.regions.E}, and
	\item $\A^5\B^3\C^5\D^3$ and $\A^5\B^2\C^5\D^2$, marked with $P$ in \Cref{fig:arch.coex.regions.E}.
\end{enumerate}
\Cref{fig:arch.coex.cobweb.M} shows the cobweb diagram for the point $M$ in \Cref{fig:arch.coex.regions.E}.
Here, we can see the two cycles of the two different ``type A'' parameter regions.
The cycle $\Cycle{\A^5\B^3\C^5\D^3}$ is shown in blue and the cycle $\Cycle{\A^4\B^3\C^4\D^3}$ is shown in red, these colors will stay the same for other cobweb diagrams in this section.
The cobweb diagram also shows the basins of attraction of both cycles.
The basin of attraction of the cycle $\Cycle{\A^5\B^3\C^5\D^3}$ is shown in blue and the basin of attraction of the cycle $\Cycle{\A^4\B^3\C^4\D^3}$ is shown in red.

\begin{figure}
	\centering
	\subfloat[$M$]{
		\includegraphics[width=.45 \textwidth]{../Figures/6/6.9a/result.png}
		\label{fig:arch.coex.cobweb.M}
	}
	\subfloat[$Q$]{
		\includegraphics[width=.45 \textwidth]{../Figures/6/6.9b/result.png}
		\label{fig:arch.coex.cobweb.Q}
	}
	\caption[Cobweb diagrams at two parameter values in the archetypal model showing coexistence of two stable cycles and their basins of attraction]{
		Cobweb diagrams at two parameter values in the archetypal model showing coexistence of two stable cycles and their basins of attraction.
		The parameters $a_L = 4, b_L = -\frac{1}{2},$ and $g_R\left(\frac{1}{2}\right) = \frac{1}{2} + \frac{1}{40}$ are fixed.
		The parameters $\alpha = -g_R\left(\frac{1}{4}\right)$ and $\beta = c_L$ are differed in every diagram and are marked with points in \Cref{fig:arch.coex.regions}.
		(a) shows the cycles at the point $M$ where $\alpha = -0.4$ and $\beta = 0.168$.
		Here, two ``type A'' cycles coexist.
		(b) shows the cycles at the point $Q$ where $\alpha = -0.375$ and $\beta = 0.1678$.
		Here, two ``type B'' twin cycles coexist.
	}
\end{figure}

\subsection{Only one Pair of ``Type B'' Cycles}

Another very simple case is when a ``type B'' parameter region does not overlap with any other region.
This causes the coexistence of two ``type B'' twin cycles.
In this case, there are two coexisting stable cycles as discussed before in \Cref{sec:state.og.dynamics} and \Cref{sec:arch.dynamics}.
Here, the cycles are asymmetrical.
If one of the cycles is $\Cycle{\A^a\B^b\C^c\D^d}$ where $c = a - 1$ and $d = b + 1$, the other cycle is $\Cycle{\A^c\B^d\C^a\D^b}$.
In the concrete case marked with the point $Q$ in \Cref{fig:arch.coex.regions.F}, these cycles are $\Cycle{\A^5\B^3\C^4\D^4}$ and $\Cycle{\A^4\B^4\C^5\D^3}$.

\Cref{fig:arch.coex.cobweb.Q} shows the cobweb diagram at this point.
The cycle $\Cycle{\A^5\B^3\C^4\D^4}$ is shown in green and its basin of attraction also.
Its twin cycle $\Cycle{\A^4\B^4\C^5\D^3}$ is shown in brown and its basin of attraction is shown in yellow for better visibility.
Again, for the rest of this section, the colors will stay the same when we encounter these cycles in cobweb diagrams.

\subsection{One Pair of ``Type B'' Cycles and One ``Type A'' Cycle}
\label{sec:arch.coex.AB}

We can see in \Cref{fig:arch.coex.regions.F} that this ``type B'' parameter region can overlap with ``type A'' parameter regions.
This causes the coexistence of three cycles, two ``type B'' cycles and one ``type A'' cycle.
It can also happen in four different ways, as was the case with ``type A'' parameter regions overlapping with one another described in \Cref{sec:arch.coex.AA}.
Assuming the stable cycles of the parameter region in the middle are $\Cycle{\A^a\B^b\C^c\D^d}$ and $\Cycle{\A^c\B^d\C^a\D^b}$ with $c = a - 1$ and $d = b + 1$, it can overlap with parameter regions where either one of the following cycles is stable
\begin{enumerate*}
	\item $\Cycle{\A^c\B^d\C^c\D^d}$,
	\item $\Cycle{\A^a\B^d\C^a\D^d}$,
	\item $\Cycle{\A^a\B^b\C^a\D^b}$, and
	\item $\Cycle{\A^c\B^b\C^c\D^b}$.
\end{enumerate*}
For the concrete case pictured in \Cref{fig:arch.coex.regions.F}, this results in the following parameter regions
\begin{enumerate}
	\item $\A^5\B^3\C^4\D^4, \A^4\B^4\C^5\D^3,$ and $\A^4\B^4\C^4\D^4$, marked with $R$ in \Cref{fig:arch.coex.regions.F},
	\item $\A^5\B^3\C^4\D^4, \A^4\B^4\C^5\D^3,$ and $\A^5\B^4\C^5\D^4$, marked with $S$ in \Cref{fig:arch.coex.regions.F},
	\item $\A^5\B^3\C^4\D^4, \A^4\B^4\C^5\D^3,$ and $\A^5\B^3\C^5\D^3$, marked with $T$ in \Cref{fig:arch.coex.regions.F}, and
	\item $\A^5\B^3\C^4\D^4, \A^4\B^4\C^5\D^3,$ and $\A^4\B^3\C^4\D^3$, marked with $U$ in \Cref{fig:arch.coex.regions.F}.
\end{enumerate}

\Cref{fig:arch.coex.cobweb.U} shows the cobweb diagram at the point $U$ in \Cref{fig:arch.coex.regions.F}.
This point is chosen for the cobweb diagram, since here the parameter regions $\P_{\A^4\B^3\C^4\D^3}$ and $\P_{\A^5\B^3\C^4\D^4, \A^4\B^4\C^5\D^3}$ overlap and the cycles that exist at this point were already in the previous cobweb diagrams.
The colors for each cycle, as well as the color of their basins of attraction, are the same as in previous cobweb diagrams showing these cycles, \Cref{fig:arch.coex.cobweb.M,fig:arch.coex.cobweb.Q}.

\begin{figure}
	\centering
	\subfloat[$U$]{
		\includegraphics[width=.45 \textwidth]{../Figures/6/6.10a/result.png}
		\label{fig:arch.coex.cobweb.U}
	}
	\subfloat[$X$]{
		\includegraphics[width=.45 \textwidth]{../Figures/6/6.10b/result.png}
		\label{fig:arch.coex.cobweb.X}
	}
	\caption[Cobweb diagrams at two parameter values in the archetypal model showing coexistence of three and four stable cycles and their basins of attraction]{
		Cobweb diagrams at two parameter values in the archetypal model showing coexistence of three and four stable cycles and their basins of attraction.
		The parameters $a_L = 4, b_L = -\frac{1}{2},$ and $g_R\left(\frac{1}{2}\right) = \frac{1}{2} + \frac{1}{40}$ are fixed.
		The parameters $\alpha = -g_R\left(\frac{1}{4}\right)$ and $\beta = c_L$ are differed in every diagram and are marked with points in \Cref{fig:arch.coex.regions}.
		(a) shows the cycles at the point $U$ where $\alpha = -0.3797$ and $\beta = 0.168$.
		Here, two ``type B'' twin cycles coexist with one ``type A'' cycle.
		(b) shows the cycles at the point $X$ where $\alpha = -0.3805$ and $\beta = 0.1672$.
		Here, two ``type B'' twin cycles coexist with two ``type A'' cycles.
	}
\end{figure}

\subsection{One Pair of ``Type B'' Cycles And Two ``Type A'' Cycles}

When looking closer at \Cref{fig:arch.coex.regions.F}, we can see that the parameter regions described in the previous \Cref{sec:arch.coex.AB} can also overlap with one another.
There, one ``type B'' parameter region overlaps with two different ``type A'' parameter regions.
This results in parameter regions where there coexist two ``type B'' cycles and two ``type A'' cycles.
It can also happen in four cases, as with previous coexistence scenarios in \Cref{sec:arch.coex.AA,sec:arch.coex.AB}.
Assuming the ``type B'' cycles are $\Cycle{\A^a\B^b\C^c\D^b}$ and $\Cycle{\A^c\B^d\C^a\D^b}$ with $c = a - 1$ and $d = b + 1$, the cycles they will coexist with are the following pairs of the cycles discussed in \Cref{sec:arch.coex.AB}
\begin{enumerate*}
	\item $\Cycle{\A^c\B^d\C^c\D^d}$ and $\Cycle{\A^a\B^d\C^a\D^d}$,
	\item $\Cycle{\A^a\B^d\C^a\D^d}$ and $\Cycle{\A^a\B^b\C^a\D^b}$,
	\item $\Cycle{\A^a\B^b\C^a\D^b}$ and $\Cycle{\A^c\B^b\C^c\D^b}$, and
	\item $\Cycle{\A^c\B^b\C^c\D^b}$ and $\Cycle{\A^c\B^d\C^c\D^d}$.
\end{enumerate*}
For the concrete case pictured in \Cref{fig:arch.coex.regions.F}, this results in the following coexistence scenarios.
\begin{enumerate}
	\item $\A^5\B^3\C^4\D^4, \A^4\B^4\C^5\D^3, \A^4\B^4\C^4\D^4,$ and $\A^5\B^4\C^5\D^4$, marked with $V$ in \Cref{fig:arch.coex.regions.F},
	\item $\A^5\B^3\C^4\D^4, \A^4\B^4\C^5\D^3, \A^5\B^4\C^5\D^4,$ and $\A^5\B^3\C^5\D^3$, marked with $W$ in \Cref{fig:arch.coex.regions.F},
	\item $\A^5\B^3\C^4\D^4, \A^4\B^4\C^5\D^3, \A^5\B^3\C^5\D^3,$ and $\A^4\B^3\C^4\D^3$, marked with $X$ in \Cref{fig:arch.coex.regions.F}, and
	\item $\A^5\B^3\C^4\D^4, \A^4\B^4\C^5\D^3, \A^4\B^3\C^4\D^3,$ and $\A^4\B^4\C^4\D^4$, marked with $Y$ in \Cref{fig:arch.coex.regions.F}.
\end{enumerate}

\Cref{fig:arch.coex.cobweb.X} shows the cobweb diagram for the point $X$ in \Cref{fig:arch.coex.regions.F}.
Again, this point is chosen so that the coexisting cycles were already pictured in previous cobweb diagrams in this section.
The colors for each cycle as well as the color of their basins of attraction are the same as in previous cobweb diagrams.
If we compare this cobweb diagram to the cobweb diagram in \Cref{fig:arch.coex.cobweb.U} of point $U$ in \Cref{fig:arch.coex.regions.F}, we can see that the cycles $\Cycle{\A^4\B^3\C^4\D^3}$ shown in red, $\Cycle{\A^5\B^3\C^4\D^4}$ shown in green, and $\Cycle{\A^4\B^4\C^5\D^3}$ shown in brown exist at almost the same point.
The same is true for their basins of attraction.
But there is a new cycle, $\Cycle{A^5\B^3\C^5\D^3}$ shown in blue that emerged between the basins of attraction of the two ``type B'' cycles, $\Cycle{\A^5\B^3\C^4\D^4}$ shown in green and $\Cycle{\A^4\B^4\C^5\D^3}$ shown in brown.

In this cobweb diagram, as well as in the previous ones, we can see that the borders of the function $d_0$, $d_1$, $d_2$, and $d_3$ as well as their preimages seperate basins of attraction from each other.
In each diagram, the neighborhoods of the borders $d_1$ and $d_2$ are magnified.
From these two borders, we can also reason about the two other borders $d_0$ and $d_3$ thanks to the symmetry of the model.
The basins of attraction of the cycles of ``type A'' parameter regions, such as $\Cycle{\A^5\B^3\C^5\D^3}$ shown in blue and $\Cycle{\A^4\B^3\C^4\D^3}$ shown in red, are the same on each half of the model.
For the cycles of ``type B'' parameter regions, such as $\Cycle{\A^5\B^3\C^4\D^4}$ shown in green and $\Cycle{\A^4\B^4\C^5\D^3}$ shown in brown, the cycles and their basins of attraction swap places.
So at border $d_3$, the basin of attraction to the left will be of the cycle $\A^5\B^3\C^5\D^3$ shown in blue still, but the basin of attraction to the right will be of $\Cycle{\A^5\B^3\C^4\D^4}$ shown in green instead of $\Cycle{\A^4\B^4\C^5\D^3}$ shown in brown.

\subsection{Possible Period Combinations in Coexistence Scenarios}

The previous sections cover all possible coexistence scenarios in the archetypal model.
We can see that only parameter regions of the same chain of parameter regions associated with the same period or parameter regions of neighboring chains can overlap.
Therefore, coexisting cycles differ in their periods by two at most.

\subsection{Coexistence Scenarios in the Original Model}

\Citeauthor{akyuz2022} covered all these coexistence scenarios in \cite{akyuz2022} with one exception.
The coexistence of four cycles was not described in his preliminary investigation of the original model.
This section confirms that this scenario also exists in the original model.
For this, \Cref{fig:arch.coex.og.regions} shows a 2D scan of the boundaries of parameter regions zoomed in the lower left corner of the ``type B'' parameter region that is associated with the twin cycles $\Cycle{\A^3\B^2\C^2\D^3}$ and $\Cycle{\A^2\B^3\C^3\D^2}$.
Since it is magnified a lot, it has more artifacts that are not actual boundaries of parameter regions than previous scans of boundaries.
Therefore, the important boundaries are marked with arrows.
The lower boundary of the ``type B'' parameter regions is marked with $\BCB^{\A^3\B^2\C^2\underline{\D}^3}_{d_3}, \BCB^{\A^3\underline{\B}^2\C^2\D^3}_{d_1}$ and its left boundary is marked with $\BCB^{\A^3\B^2\C^2\underline{\D}^3}_{d_0}, \BCB^{\A^3\underline{\B}^2\C^2\D^3}_{d_2}$.
The parameter region below the ``type B'' parameter region is the ``type A'' parameter region that is associated with the cycle $\Cycle{\A^3\B^2\C^3\D^2}$.
Its upper boundary is marked with $\BCB^{\underline{\A}^3\B^2\underline{\C}^3\D^2}_{d_1, d_3}$.
And finally, the parameter region to the left of the ``type B'' parameter region is the ``type A'' parameter that is region associated with the cycle $\A^2\B^2\C^2\D^2$.
Its right boundary is marked with $\BCB^{\underline{\A}^2\B^2\underline{\C}^2\D^2}_{d_0, d_2}$.
All other lines in the diagram are numerical artifacts that do not correspond to boundaries of parameter regions.

The point $X$ marks parameter values that are in the overlapping area of all three parameter regions mentioned above.
\Cref{fig:arch.coex.og.regions} shows the cobweb diagram at these parameter values.
Here all four cycles coexist, two cycles from each ``type A'' parameter region and two cycles from the ``type B'' parameter region.
This is hard to see, since the two cycles $\Cycle{\A^3\B^2\C^3\D^2}$ shown in blue and $\Cycle{\A^2\B^3\C^3\D^2}$ shown in brown are almost on top of each other in the blowup plot.
But one can see that they are different cycles, as they are visibly separated right of the border $d_3$.

%in next document for better layout
