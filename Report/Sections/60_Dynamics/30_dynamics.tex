\section{Model Dynamics}
\label{sec:arch.dynamics}

As already hinted at in \Cref{sec:setup.arch.behavior}, the dynamics of this archetypal model are similar to the dynamics of the original model.
This section examines the behavior more thoroughly.
\Cref{fig:arch.dyn.period} displays 2D scans showing the periods of the stable cycles associated with parameter regions in the archetypal model with fixed parameters $a_L = 4, b_L = -\frac{1}{2},$ and $g_R\left(\frac{1}{2}\right) = \frac{1}{2} + \frac{1}{40}$.
The parameters $\alpha = -g_R\left(\frac{1}{4}\right)$ and $\beta = c_R$ are varied in the ranges $[-0.55, -0.275]$ and $[0.15, 0.1875]$, respectively.
As before in \Cref{fig:setup.quad.hyper.2.period,fig:setup.arch.period}, the first scan, \Cref{fig:arch.dyn.period.full}, shows the periods associated with parameter regions in the archetypal model and the second scan, \Cref{fig:arch.dyn.period.halved}, shows the periods associated with parameter regions in the halved archetypal model.
It was preliminarily defined in \Cref{sec:setup.quad.hyper} and \Cref{sec:add.add.halved} defines it formally.

\begin{figure}
	\centering
	\subfloat[Model]{
		\includegraphics[width=.48 \textwidth]{../Figures/6/6.1a/result.png}
		\label{fig:arch.dyn.period.full}
	}
	\subfloat[Halved Model]{
		\includegraphics[width=.48 \textwidth]{../Figures/6/6.1b/result.png}
		\label{fig:arch.dyn.period.halved}
	}
	\caption[2D scans of the periods of the archetypal model]{
		2D scans of the periods of the archetypal model.
		The parameters $a_L = 4, b_L = -\frac{1}{2},$ and $g_R\left(\frac{1}{2}\right) = \frac{1}{2} + \frac{1}{40}$ are fixed.
		The parameters $\alpha = -g_R\left(\frac{1}{4}\right)$ and $\beta = c_L$ are varied in the ranges $[-0.55, -0.275]$ and $[0.15, 0.1875]$, respectively.
		The points $A_{16}, B_{16}, C_{16}, D_{16}, E_{16},$ and $F_{16}$ mark the parameter values used for the cobweb diagrams in \Cref{fig:arch.dyn.cobwebs.1,fig:arch.dyn.cobwebs.2}.
		(a) shows the scan for the archetypal model as defined above, while (b) shows the scan of the halved archetypal model where we can see ``type B'' parameter regions.
	}
	\label{fig:arch.dyn.period}
\end{figure}

\begin{figure}
	\centering
	\subfloat[$A_{16}$]{
		\includegraphics[width=.3 \textwidth]{../Figures/6/6.2a/result.png}
		\label{fig:arch.dyn.cobweb.A}
	}
	\subfloat[$B_{16}$]{
		\includegraphics[width=.3 \textwidth]{../Figures/6/6.2b/result.png}
		\label{fig:arch.dyn.cobweb.B}
	}
	\subfloat[$C_{16}$]{
		\includegraphics[width=.3 \textwidth]{../Figures/6/6.2c/result.png}
		\label{fig:arch.dyn.cobweb.C}
	}
	\caption[Cobweb diagrams of the archetypal model]{
		Cobweb diagrams at three parameter values of $\alpha = -g_R\left(\frac{1}{4}\right)$ and $\beta = c_L$ in the archetypal model.
		The other parameters are fixed as $a_L = 4, b_L = -\frac{1}{2},$ and $g_R\left(\frac{1}{2}\right) = \frac{1}{2} + \frac{1}{40}$.
		The parameter values are marked with the points $A_{16}, B_{16},$ and $C_{16}$ in \Cref{fig:arch.dyn.period}.
		(a) shows the cycle $\Cycle{\A^7\B^1\C^7\D^1}$ at the point $A_{16}$ where $\alpha = -0.52$ and $\beta = 0.156$,
		(b) shows the two coexisting cycles $\Cycle{\A^7\B^1\C^6\D^2}$ (green) and $\Cycle{\A^6\B^2\C^7\D^1}$ (red) at the point $B_{16}$ where $\alpha = -0.498$ and $\beta = 0.1575$,
		and (c) shows the cycle $\Cycle{\A^6\B^2\C^6\D^2}$ at the point $C_{16}$ where $\alpha = -0.47$ and $\beta = 0.159$.
	}
	\label{fig:arch.dyn.cobwebs.1}
\end{figure}

As in \cite{akyuz2022}, this section examines the chain of parameter regions associated with stable cycles of period $16$.
The parameter regions are marked with the points $A_{16}$ to $G_{16}$ in \Cref{fig:arch.dyn.period}.
\Cref{fig:arch.dyn.cobwebs.1} shows cobweb diagrams for the points $A_{16}, B_{16},$ and $C_{16}$.
The parameter region containing $A_{16}$ is denoted $\P_{\A^7\B\C^7\D}$ since it only associated with the stable cycle $\Cycle{\A^7\B\C^7\D}$.
\Cref{fig:state.og.dynamics.cobweb.A} shows the cobweb diagram of this cycle.
The parameter region, therefore, is a ``type A'' parameter region associated with only one stable cycle of period 16.
The next parameter region contains the point $B_{16}$ and is associated with two stable cycles $\Cycle{\A^7\B\C^6\D^2}$ and $\Cycle{\A^6\B^2\C^7\D}$.
\Cref{fig:state.og.dynamics.cobweb.B} shows the cobweb diagram of these two cycles.
The parameter region containing these cycles is denoted $\P_{\A^7\B\C^6\D^2, \A^6\B^2\C^7\D}$.
By the same logic, the parameter region marked with point $C_{16}$ is denoted $\P_{\A^6\B^2\C^6\D^2}$.
\Cref{fig:arch.dyn.cobweb.C} shows the cobweb diagram of that cycle.

\begin{figure}
	\centering
	\subfloat[$D_{16}$]{
		\includegraphics[width=.3 \textwidth]{../Figures/6/6.3a/result.png}
		\label{fig:arch.dyn.cobweb.D}
	}
	\subfloat[$E_{16}$]{
		\includegraphics[width=.3 \textwidth]{../Figures/6/6.3b/result.png}
		\label{fig:arch.dyn.cobweb.E}
	}
	\subfloat[$F_{16}$]{
		\includegraphics[width=.3 \textwidth]{../Figures/6/6.3c/result.png}
		\label{fig:arch.dyn.cobweb.F}
	}
	\caption[Cobweb diagrams of the archetypal model]{
		Cobweb diagrams at three parameter values of $\alpha = -g_R\left(\frac{1}{4}\right)$ and $\beta = c_L$ in the archetypal model.
		The other parameters are fixed as $a_L = 4, b_L = -\frac{1}{2},$ and $g_R\left(\frac{1}{2}\right) = \frac{1}{2} + \frac{1}{40}$.
		The parameter values are marked with the points $D_{16}, E_{16},$ and $F_{16}$ in \Cref{fig:arch.dyn.period}.
		(a) shows the cycle $\Cycle{\A^6\B^2\C^6\D^2}$ at the point $D_{16}$ where $\alpha = -0.428$ and $\beta = 0.1615$,
		(b) shows the two coexisting cycles $\Cycle{\A^6\B^2\C^5\D^3}$ (green) and $\Cycle{\A^5\B^3\C^6\D^2}$ (red) at the point $E_{16}$ where $\alpha = -0.4$ and $\beta = 0.165$,
		and (c) shows the cycle $\Cycle{\A^5\B^3\C^5\D^3}$ at the point $C_{16}$ where $\alpha = -0.367$ and $\beta = 0.173$.
	}
	\label{fig:arch.dyn.cobwebs.2}
\end{figure}

Thus far these parameter regions follow the rules laid out in \cite{akyuz2022} and listed again in \Cref{sec:state.og.dynamics}.
The first parameter region has the stable cycle $\Cycle{\A^{\frac{P}{2} - 1}\B\C^{\frac{P}{2} - 1}\D}$ where $P = 16$ is the period.
It is a ``type A'' parameter region and the next parameter region is of ``type B'', as the rules require.
This parameter region has the stable cycles $\Cycle{\A^{\frac{P}{2} - 1}\B\C^{\frac{P}{2} - 2}\D^2}$ and $\Cycle{\A^{\frac{P}{2} - 2}\B^2\C^{\frac{P}{2} - 1}\D}$.
This also agrees with the rules.
The next parameter region is of ``type A'' again and has the stable cycle $\Cycle{A^{\frac{P}{2} - 1}\B^2\C^{\frac{P}{2} - 2}\D^2}$ as expected by the rules.
This chain continues to abide by the rules as can be seen in the cobweb diagrams of points $D_{16}, E_{16},$ and $F_{16}$ in \Cref{fig:arch.dyn.cobwebs.2}.
The corresponding parameter regions are $\P_{\A^6\B^2\C^5\D^3, \A^5\B^3\C^6\D^2}, \P_{\A^5\B^3\C^5\D^3},$ and $\P_{\A^5\B^3\C^4\D^4, \A^4\B^4\C^5\D^3}$ respectively.

\begin{figure}
	\centering
	\subfloat[$G_{16}$]{
		\includegraphics[width=.3 \textwidth]{../Figures/6/6.4a/result.png}
		\label{fig:arch.dyn.cobweb.G}
	}
	\subfloat[$E_{14}$]{
		\includegraphics[width=.3 \textwidth]{../Figures/6/6.4b/result.png}
		\label{fig:arch.dyn.cobweb.E14}
	}
	\subfloat[$E_{18}$]{
		\includegraphics[width=.3 \textwidth]{../Figures/6/6.4c/result.png}
		\label{fig:arch.dyn.cobweb.E18}
	}
	\caption[Cobweb diagrams of the archetypal model]{
		Cobweb diagrams at three parameter values of $\alpha = -g_R\left(\frac{1}{4}\right)$ and $\beta = c_L$ in the archetypal model.
		The other parameters are fixed as $a_L = 4, b_L = -\frac{1}{2},$ and $g_R\left(\frac{1}{2}\right) = \frac{1}{2} + \frac{1}{40}$.
		The parameter values are marked with the points $G_{16}, E_{14},$ and $E_{18}$ in \Cref{fig:arch.dyn.period}.
		(a) shows the cycle $\Cycle{\A^5\B^3\C^5\D^3}$ at the point $G_{16}$ where $\alpha = -0.36$ and $\beta = 0.173$,
		(b) shows the cycle $\Cycle{\A^4\B^3\C^4\D^3}$ at the point $E_{14}$ where $\alpha = -0.4$ and $\beta = 0.173$,
		and (c) shows the cycle $\Cycle{\A^6\B^3\C^6\D^3}$ at the point $E_{18}$ where $\alpha = -0.4$ and $\beta = 0.16$.
	}
	\label{fig:arch.dyn.cobwebs.3}
\end{figure}

The stable cycle in the parameter region marked with point $E_{14}$ has period $14$.
This agrees with the rule that the neighboring chains differ in their periods by 2.
Its symbolic sequence is $\A^4\B^3\C^4\D^3$ and its cobweb diagram can be seen in \Cref{fig:arch.dyn.cobweb.E14}.
Similarly, the cycle associated with the parameter regions marked with $E_{16}$ has period $16$.
Its symbolic sequence is $\A^6\B^3\C^6\D^3$ and its cobweb diagram can be seen in \Cref{fig:arch.dyn.cobweb.E}.
So the cycles in the ``type A'' regions directly above another ``type A'' parameter region with stable cycle $\Cycle{\A^a\B^b\C^a\D^b}$ has the stable cycle $\Cycle{\A^{a-1}\B^b\C^{a-1}\D^b}$.

The regularities of symbolic sequences associated with ``type A'' parameter regions that are neighboring horizontally is explored next.
The parameter region with point $C_{16}$ is directly left of the parameter region with point $E_{18}$.
It has the stable cycle $\Cycle{\A^6\B^2\C^6\D^2}$, while the parameter region with point $E_{18}$ has the stable cycle $\Cycle{\A16\B^3\C^6\D^3}$.
Similarly, the stable cycle in the parameter region marked with point $G_{16}$ which is directly right of the parameter region marked with point $E_{14}$ that has the stable cycle $\Cycle{\A^4\B^4\C^4\D^4}$, has the stable cycle $\Cycle{\A^4\B^3\C^4\D^3}.$
Therefore, the cycles in the ``type A'' regions directly left of another ``type A'' parameter region with stable cycle $\Cycle{\A^a\B^b\C^a\D^b}$ has the stable cycle $\Cycle{\A^a\B^{b-1}\C^a\D^{b-1}}$.
\Cref{fig:arch.dyn.cobwebs.3} shows the cobweb diagrams of the archetypal model at the points $G_{16}, E_{14},$ and $E_{18}$.

These regularities of horizontally and vertically neighboring ``type A'' parameter regions make sense.
From \Cref{sec:setup.arch.parameterfx} we know that the effect of increasing $\alpha$ is that the values of the model function get smaller on the left sides of the branches $f_\B$ and $f_\D$.
This leads to a narrower canal between the branches $f_\B$ and $f_\D$ and the bisector $x = y$.
Cycles thus have more points on these branches.
This explains why the cycles associated with the ``type A'' parameter region to the right of another ``type A'' parameter region have one point more on each of the branches $f_\B$ and $f_\D$.
Similarly, we know from \Cref{sec:setup.arch.parameterfx} that the effect of decreasing $\beta$ is that the values of the model function get smaller on the whole branches $f_\A$ and $f_\C$.
This leads to a narrower canal between the branches $f_\A$ and $f_\C$ and the bisector $x = y$.
This explains why the cycles associated with the ``type A'' parameter region below another ``type A'' parameter region have one point more on each of the branches $f_\A$ and $f_\C$.
