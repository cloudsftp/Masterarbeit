\section{Bifurcations}
\label{sec:arch.bif}

This section explores the bifurcations that happen at the borders of ``type A'' and ``type B'' parameter regions, respectively.
\Cref{fig:arch.dyn.regions.full} shows the borders of the parameter regions in full.
\Cref{fig:arch.dyn.regions.zoomed} is a zoomed-in version, that pictures the parameter region that contain the point $F_{16}$ of \Cref{fig:arch.dyn.period}.
It is a ``type B'' parameter region with the stable cycles $\Cycle{\A^5\B^3\C^4\D^4}$ and $\Cycle{\A^4\B^4\C^5\D^3}$.
Every one of its boundaries has a ``type A'' parameter region on the other side.
Therefore, we only need to describe the 4 boundaries of this ``type B'' parameter region in depth to cover all the boundaries of both ``type A'' and ``type B'' parameter regions.

\begin{figure}
	\centering
	\subfloat[Full]{
		\includegraphics[width=.4\textwidth]{60_MinimalRepr/2D_Regions_Whole/result-halved.png}
		\label{fig:arch.dyn.regions.full}
	}
	\subfloat[Zoomed-in]{
		\includegraphics[width=.4\textwidth]{60_MinimalRepr/2D_Regions_F_Boundaries/result.png}
		\label{fig:arch.dyn.regions.zoomed}
	}
	\caption[2D scans of the boundaries of parameter regions with different symbolic sequences in the archetypal model]{
		2D scans of the boundaries of parameter regions with different symbolic sequences in the archetypal model.
		The parameters $a_L = 4, b_L = -\frac{1}{2},$ and $g_R\left(\frac{1}{4}\right) = 0.525$ are fixed.
		In (a), the parameters $\alpha = -g_R\left(\frac{1}{4}\right)$ and $\beta = c_L$ are varied in the ranges $[-0.55, -0.275]$ and $[0.15, 0.1875]$, respectively.
		(b) is a zoomed-in version with the same parameters being varied in the ranges $[-0.385, -0.365]$ and $[0.166, 0.169]$, respectively.
		It focuses on the ``type B'' parameter region marked with point $F_{16}$ in \Cref{fig:arch.dyn.period}.
		Its boundaries are marked with $F_{16}^\uparrow, F_{16}^\downarrow, F_{16}^\leftarrow,$ and $F_{16}^\rightarrow$.
	}
	\label{fig:arch.dyn.regions}
\end{figure}

\subsection{The Boundary $F_{16}^\uparrow$}
\label{sec:arch.bif.U}

\todo{There was a copy paste error in a caption: $g_R\left(\frac{1}{4}\right) = 0.525$ should be $\frac{1}{2}$. Check captions!!!}
\todo{In bifurcation figures: adjust labels, underline symbols}
\begin{figure}
	\centering
	\includegraphics[width=.7 \textwidth]{60_MinimalRepr/1D_Bif_LFU16/Manual/result.png}
	\caption[1D bifurcation diagram at the boundary $F_{16}^\uparrow$ in the archetypal model]{
		1D bifurcation diagram at the boundary $F_{16}^\uparrow$ in the archetypal model.
		The parameters $a_L = 4, b_L = -\frac{1}{2}, g_R\left(\frac{1}{2}\right) = 0.525,$ and $\alpha = g_R\left(\frac{1}{4}\right) = -0.375$ are fixed.
		The parameter $\beta = c_L$ is varied in the range marked with an arrow in \Cref{fig:arch.dyn.regions.zoomed}.
		On the left, the whole state space is pictured while the right side enhances the area of the state space around the borders involved in the pictured \glspl{bcb}.
	}
	\label{fig:arch.bif.F.up}
\end{figure}

\Cref{fig:arch.bif.F.up} shows the bifurcation diagram of the first considered boundary, $F_{16}^\uparrow$.
To better differentiate between the two coexisting ``type B'' cycles of the parameter region marked with point $F_{16}$, they are plotted in different colors.
The cycle $\Cycle{\A^5\B^3\C^4\D^4}$ is green and its twin cycle $\Cycle{\A^4\B^4\C^5\D^3}$ is red.
One can see that the cycle $\Cycle{\A^5\B^3\C^4\D^4}$ (green) collides with the border $d_1$ when it vanishes.
To be more precise the point $x_4^{\A^5\B^3\C^4\D^4}$, which is the 5th point of the cycle $\Cycle{\A^5\B^3\C^4\D^4}$, collides with the border $d_1$.
This is a \gls{bcb}, and it is denoted as $\BCB_{d_1}^{\underline{\A}^5\B^3\C^4\D^4}$.

The lower index of $\BCB$ indicates the border of the model function that is involved in the bifurcation.
The upper index of $\BCB$ indicates two things.
First, the object that collides with the border of the model function.
In our case this is the cycle $\Cycle{\A^5\B^3\C^4\D^4}$.
Second, the underlined symbol indicates the branch of the model function, the colliding point of the cycle belongs to.
Together with the information which border is involved in the \gls{bcb}, one can determine which point of the cycle collided with the border.
For example, we know that a point of the cycle on branch $f_\A$ collides with the border $d_1$, which is the right border of the branch $f_\A$.
Since there are $5$ points on branch $f_\A$, we can derive that the point $x_4^{\A^5\B^3\C^4\D^4}$ is involved in the \gls{bcb}.

A similar thing that happens to cycle $\Cycle{\A^5\B^3\C^4\D^4}$ (green) happens to its twin cycle $\Cycle{\A^4\B^4\C^5\D^3}$ but shifted by $\frac{1}{2}$ in the state space because of the symmetry in the model.
Here, the point $x_{12}^{A^4\B^4\C^5\D^3}$ collides with the border $d_3$ and the bifurcation is denoted as $\BCB_{d_3}^{A^4\B^4\underline{\C}^5\D^3}$.
In both cases, the cycles collide from the left side of the border.

The ``type A'' parameter region above is $\P_{\A^4\B^4\C^4\D^4}$.
The cycle $\Cycle{\A^4\B^4\C^4\D^4}$ (blue), which is stable in that parameter region, collides with the same borders the ``type B'' cycles collide with, $d_1$ and $d_3$.
But here, two points of the same cycle collide with two different borders at the same parameter values.
Point $x_{4}^{A^4\B^4\C^4\D^4}$ collides with the border $d_1$ while point $x_{12}^{A^4\B^4\C^4\D^4}$ collides with $d_3$.
Both collisions happen from the right side of the borders.
So one point of the cycle on the branch $f_{\B}$ collides with $d_1$ and one point on the branch $f_{\D}$ collides with $d_3$.
This is unusual for border collision bifurcations but is explained by the symmetry of both the cycle and the model function.
The bifurcation is denoted as $\BCB_{d_1, d_3}^{\A^4\underline{\B}^4\C^4\underline{\D}^4}$.

\subsection{The Boundary $F_{16}^\downarrow$}
\label{sec:arch.bif.D}

\begin{figure}
	\centering
	\includegraphics[width=.7 \textwidth]{60_MinimalRepr/1D_Bif_LFD16/Manual/result.png}
	\caption[1D bifurcation diagram at the boundary $F_{16}^\downarrow$ in the archetypal model]{
		1D bifurcation diagram at the boundary $F_{16}^\downarrow$ in the archetypal model.
		The parameters $a_L = 4, b_L = -\frac{1}{2}, g_R\left(\frac{1}{2}\right) = 0.525,$ and $\alpha = g_R\left(\frac{1}{4}\right) = -0.3775$ are fixed.
		The parameter $\beta = c_L$ is varied in the range marked with the arrow $F_{16}^\downarrow$ in \Cref{fig:arch.dyn.regions.zoomed}.
		On the left, the whole state space is pictured while the right side enhances the area of the state space around the borders involved in the pictured \glspl{bcb}.
	}
	\label{fig:arch.bif.F.down}
\end{figure}

At the lower boundary $F_{16}^\downarrow$, the two cycles $\Cycle{\A^5\B^3\C^4\D^4}$ and $\Cycle{\A^4\B^4\C^5\D^3}$ also collide with the borders $d_1$ and $d_3$, this time from the right side of the borders.
But while the cycle $\Cycle{\A^5\B^5\C^4\D^4}$ (green) collides with the border $d_1$ at the upper boundary, here it collides with the border $d_3$.
To be more precise, the point $x_{12}^{\A^5\B^3\C^4\D^4}$ collides with the border $d_3$.
Meaning one point on the branch $f_{\D}$ collides with the border $d_3$.
This \gls{bcb} is written as $\BCB_{d_3}^{\A^5\B^3\C^4\underline{\D}^4}$.
Similarly, the point $x_{4}^{\A^4\B^4\C^5\D^3}$ of the cycle $\Cycle{\A^4\B^4\C^5\D^3}$ (red) now collides with the border $d_1$ from the right side of the border.
Meaning that one point of branch $f_{\B}$ collides with the border $d_1$.
This \gls{bcb} is written as $\BCB_{d_1}^{\A^4\underline{\B}^4\C^5\D^3}$.

The ``type A'' parameter region below the ``type B'' parameter region is $\P_{\A^5\B^3\C^5\D^3}$.
The cycle $\P_{\A^5\B^3\C^5\D^3}$ (blue) collides with the same borders as the ``type B'' cycles, just like before at the upper boundary $F_{16}^\uparrow$.
Again, two points of this cycle collide with two different borders, $d_1$ and $d_2$, at the same parameter values.
But here they collide from the left side.
The point colliding with $d_1$ is $x_{4}^{A^5\B^3\C^5\D^3}$ and the point colliding with $d_3$ is $x_{12}^{A^5\B^3\C^5\D^3}$.
So one point on the branch $f_{\A}$ collides with the border $d_1$ and one point on the branch $f_{\C}$ collides with the border $d_3$.
This bifurcation is written as $\BCB_{d_1, d_3}^{\underline{\A}^5\B^3\underline{\C}^5\D^3}$.

\subsection{The Boundary $F_{16}^\leftarrow$}
\label{sec:arch.bif.L}

\begin{figure}
	\centering
	\includegraphics[width=.7 \textwidth]{60_MinimalRepr/1D_Bif_LFL16/Manual/result.png}
	\caption[1D bifurcation diagram at the boundary $F_{16}^\leftarrow$ in the archetypal model]{
		1D bifurcation diagram at the boundary $F_{16}^\leftarrow$ in the archetypal model.
		The parameters $a_L = 4, b_L = -\frac{1}{2}, g_R\left(\frac{1}{2}\right) = 0.525,$ and $\beta = c_L = 0.1675$ are fixed.
		The parameter $\beta = g_R\left(\frac{1}{4}\right)$ is varied in the range marked with the arrow $F_{16}^\leftarrow$ in \Cref{fig:arch.dyn.regions.zoomed}.
		On the left, the whole state space is pictured while the right side enhances the area of the state space around the borders involved in the pictured \glspl{bcb}.
	}
	\label{fig:arch.bif.F.left}
\end{figure}

Now we will take a look at the horizontal boundaries of this ``type B'' parameter region.
At the left boundary $F_{16}^\leftarrow$, the two cycles $\Cycle{\A^5\B^3\C^4\D^4}$ and $\Cycle{\A^4\B^4\C^5\D^3}$ collide with the borders $d_1$ and $d_2$ from the right.
These are different borders than the borders involved in the \glspl{bcb} at the vertical boundaries $F_{16}^\uparrow$ and $F_{16}^\downarrow$.
The point $x_{7}^{\A^4\B^4\B^5\D^3}$, which is on branch $f_{\B}$, collides with $d_2$ while the point $x_{15}^{\A^5\B^3\C^4\D^4}$, which is on branch $f_{\D}$, collides with the border $d_0$.
These bifurcations are written $\BCB_{d_0}^{\A^5\underline{\B}^3\C^4\D^4}$ and $\BCB_{d_2}^{\A^4\B^4\C^5\underline{\D}^3}$ respectively.

The parameter region left to the ``type B'' parameter region is $\P_{\A^4\B^3\C^4\D^3}$.
As before with the vertical boundaries $F_{16}^\uparrow$ and $F_{16}^\downarrow$, the cycle of the neighboring ``type A'' parameter region collides with the same borders as the ``type B'' cycles but from the opposite direction.
The point $x_{0}^{\A^4\B^3\C^4\D^3}$, which is on branch $f_{\A}$, collides with the border $d_0$ while the point $x_{7}^{\A^4\B^3\C^4\D^3}$, which is on branch $f_{\C}$, collides with the border $d_2$.
This bifurcation is denoted as $\BCB_{d_0, d_2}^{\A^4\B^3\C^4\D^3}$.

\subsection{The Boundary $F_{16}^\rightarrow$}
\label{sec:arch.bif.R}

\begin{figure}
	\centering
	\includegraphics[width=.7 \textwidth]{60_MinimalRepr/1D_Bif_LFR16/Manual/result.png}
	\label{fig:arch.bif.F.right}
	\caption[1D bifurcation diagram at the boundary $F_{16}^\rightarrow$ in the archetypal model]{
		1D bifurcation diagram at the boundary $F_{16}^\rightarrow$ in the archetypal model.
		The parameters $a_L = 4, b_L = -\frac{1}{2}, g_R\left(\frac{1}{2}\right) = 0.525,$ and $\beta = c_L = 0.1675$ are fixed.
		The parameter $\beta = g_R\left(\frac{1}{4}\right)$ is varied in the range marked with the arrow $F_{16}^\rightarrow$ in \Cref{fig:arch.dyn.regions.zoomed}.
		On the left, the whole state space is pictured while the right side enhances the area of the state space around the borders involved in the pictured \glspl{bcb}.
	}
\end{figure}

At the right boundary $F_{16}^\rightarrow$, the two cycles $\Cycle{\A^5\B^3\C^4\D^4}$ (green) and $\Cycle{\A^4\B^4\C^5\D^3}$ (red) collide with the borders $d_0$ and $d_2$ from left of the borders.
The first point of cycle $\Cycle{A^4B^4\C^5\D^3}$ (green) $x_{0}^{\A^4\B^4\C^5\D^3}$ collides with the border $d_0$, while the point $x_{8}^{\A^5\B^3\C^4\D^4}$ of its twin cycle $\Cycle{\A^5\B^3\C^4\D^4}$ (green) collides with the border $d_2$.
This means that one point of the cycle $\Cycle{\A^4\B^4\C^5\D^3}$ (green) on the branch $f_{\A}$ collides with the border $d_0$ and one point of the cycle $\Cycle{\A^5\B^3\C^4\D^4}$ (red) on the branch $f_{\C}$ collides with the border $d_2$.
The bifurcations are written as $\BCB_{d_2}^{\A^5\B^3\underline{\C}^4\D^4}$ and $\BCB_{d_0}^{\underline{\A}^4\B^4\C^5\D^3}$, respectively.

The ``type A'' parameter region right of this parameter region is $\P_{\A^5\B^4\C^5\D^4}$.
Here again collides the ``type A'' cycle with the same borders as the ``type B'' cycles but from the opposite direction.
In this case, two of the points of the cycle $\Cycle{\A^5\B^4\C^5\D^4}$ (blue) collide with the borders $d_0$ and $d_1$ at the same parameter value from left of the borders.
To be more precise, the point $x_{17}^{\A^5\B^4\C^5\D^4}$, which is on the branch $f_{\D}$, collides with the border $d_0$ while the point $x_{8}^{\A^5\B^4\C^5\D^4}$, which is on the branch $f_{\B}$, collides with the border $d_2$.
This bifurcation is written as $\BCB_{d_0, d_2}^{\A^5\underline{\B}^4\C^5\underline{\D}^4}$.

\subsection{Summary of Rules for Bifurcations}

The bifurcations are spread out in the previous sections.
And the bifurcations of the ``type A'' parameter regions are out-of-order.
Thus, we will generalize and summarize the rules for the bifurcations at the boundaries of either type of parameter region.

\subsubsection{``Type A'' Parameter Regions}

Let the stable cycle in the ``type A'' parameter region be $\Cycle{\A^a\B^b\C^a\D^b}$.

\begin{enumerate}
	\item At the upper boundary there will be the bifurcation $\BCB_{d_1, d_3}^{\underline{\A}^a\B^b\underline{\C}^a\D^b}$.
	\item At the lower boundary there will be the bifurcation $\BCB_{d_1, d_3}^{\A^a\underline{\B}^b\C^a\underline{\D}^b}$.
	\item At the left boundary there will be the bifurcation $\BCB_{d_0, d_2}^{\A^a\underline{\B}^b\C^a\underline{\D}^b}$.
	\item At the right boundary there will be the bifurcation $\BCB_{d_0, d_2}^{\underline{\A}^a\B^b\underline{\C}^a\D^b}$.
\end{enumerate}

\subsubsection{``Type B'' Parameter Regions}

Let the stable cycles in the ``type B'' parameter region be $\Cycle{\A^a\B^b\C^c\D^d}$ and $\Cycle{\A^c\B^d\C^a\D^b}$.
Where $c = a - 1$ and $d = b + 1$.

\begin{enumerate}
	\item At the upper boundary there will be the bifurcations $\BCB_{d_1}^{\underline{\A}^a\B^b\C^c\D^d}$ and $\BCB_{d_3}^{\A^c\B^d\underline{\C}^a\D^b}$.
	\item At the lower boundary there will be the bifurcations $\BCB_{d_3}^{\A^a\B^b\C^c\underline{\D}^d}$ and $\BCB_{d_1}^{\A^c\underline{\B}^d\C^a\D^b}$.
	\item At the left boundary there will be the bifurcations $\BCB_{d_0}^{\A^a\underline{\B}^b\C^c\D^d}$ and $\BCB_{d_2}^{\A^c\B^d\C^a\underline{\D}^b}$.
	\item At the right boundary there will be the bifurcations $\BCB_{d_2}^{\A^a\B^b\underline{\C}^c\D^d}$ and $\BCB_{d_0}^{\underline{\A}^c\B^d\C^a\D^b}$.
\end{enumerate}

\subsubsection{Validation}

These rules agree with the rules for bifurcations, laid  out by \Citeauthor{akyuz2022}~\cite{akyuz2022}.

\todo{Pattern of bifurcations: same symbols underlined, but in type B spread across 2 bifurcations}
\todo{Plus depending on direction, model half of bifurcation swaps between twin cycles. There is an explanation using the halved model but we do not have it here}
