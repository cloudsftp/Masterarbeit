\section{Model Definition}

The final model, which shows the same bifurcation phenomenon as the original model, is fairly simple.
Now there are no implicit equations and the model can be defined by the following explicit equations.
The model maps $x \mapsto f(x) \mod 1$ where ... define $f$.

\begin{align}
    f(x) & = \begin{cases}
                 g(x)       & \text{ if } x < 0.5 \\
                 g(x) + 0.5 & \text{ else}
             \end{cases} \\
    g(x) & = \begin{cases}
                 l(x) & \text{ if } x < 0.25 \\
                 r(x) & \text{ else}
             \end{cases}
\end{align}

For the definitions of $l$ and $r$, we center the variable to better understand the effects of the coefficients.
Branches $\A$ and $\C$ exist on the intervals $[0, 0.25)$ and $[0.5, 0.75)$ respectively.
Therefore, we center the variable for the polynomials of these branches by moving it $0.125$ to the left.
\Cref{equ:final.def.tl} achieves this.
Branches $\B$ and $\D$ exist on the intervals $[0.25, 0.5)$ and $[0.75, 1)$ respectively.
The offset here is $0.375$ by the same logic and \Cref{equ:final.def.tr} achieves this.

\begin{align}
    s(x)   & = x \mod 0.5                                   \\
    t_L(x) & = s(x) - \dfrac{1}{8} \label{equ:final.def.tl} \\
    t_R(x) & = s(x) - \dfrac{3}{8} \label{equ:final.def.tr}
\end{align}

The actual polynomials defining $l$ and $r$ are then defined by the following equations.

\begin{align}
    l(x) & = a_L \cdot t_L(x)^2 + b_L \cdot t_L(x) + c_L \\
    r(x) & = b_R \cdot t_R(x) + c_R
\end{align}

\subsection{The Parameters}

To get this model to show the desired behavior, we only change the coefficient $c_L$ directly via the parameter $p_y$.
The coefficients $a_L = 4$ and $b_L = 0.5$ are fixed.
The coefficients $b_R$ and $c_R$ are changed indirectly by the two parameters $A$ and $B$.
The parameter $A$
