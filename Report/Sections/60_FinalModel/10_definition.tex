\section{Model Definition}

The final model, which shows the same bifurcation phenomenon as the original model, is fairly simple.
Now there are no implicit equations and the model can be defined by the following explicit equations.
This section defines this model in two parts.
First, it defines all the explicit equations of the model which has some internal parameters.
Then it discusses the definitions of the external parameters to achieve the desired behavior.
Furthermore, this section then will discuss the effects of the parameters on the model function and compare it to the parameter effects in the original model.

\subsection{The Model Equations}

The model maps $x \mapsto f(x) \mod 1$ where \Crefrange{equ:final.def.f}{equ:final.def.r} define $f$.
\begin{align}
    f(x) & = \begin{cases}
                 g(x)       & \text{ if } x < 0.5 \\
                 g(x) + 0.5 & \text{ else}
             \end{cases}
    \label{equ:final.def.f}
    \\
    g(x) & = \begin{cases}
                 l(x) & \text{ if } x < 0.25 \\
                 r(x) & \text{ else}
             \end{cases}
\end{align}

For the definitions of $l$ and $r$, we center the variable to better understand the effects of the coefficients.
Branches $\A$ and $\C$ exist on the intervals $[0, 0.25)$ and $[0.5, 0.75)$ respectively.
Therefore, we center the variable for the polynomials of these branches by moving it $0.125$ to the left.
\Cref{equ:final.def.tl} achieves this.
Branches $\B$ and $\D$ exist on the intervals $[0.25, 0.5)$ and $[0.75, 1)$ respectively.
The offset here is $0.375$ by the same logic and \Cref{equ:final.def.tr} achieves this.
\begin{align}
    s(x)   & = x \mod 0.5                                   \\
    t_L(x) & = s(x) - \dfrac{1}{8} \label{equ:final.def.tl} \\
    t_R(x) & = s(x) - \dfrac{3}{8} \label{equ:final.def.tr}
\end{align}

The actual polynomials defining $l$ and $r$ are then given by the following equations.
\begin{align}
    l(x) & = a_L \cdot t_L(x)^2 + b_L \cdot t_L(x) + c_L \\
    r(x) & = b_R \cdot t_R(x) + c_R
    \label{equ:final.def.r}
\end{align}

\subsection{The Model Parameters}

To get this model to show the desired behavior, we only change the coefficient $c_L$ directly via the parameter $p_y$ ($c_L = p_y$).
The coefficients $a_L = 4$ and $b_L = 0.5$ are fixed.
The coefficients $b_R$ and $c_R$ are changed indirectly by the two parameters $A$ and $B$, which define characteristics of the function $r$.
$A$ is the value of the function $r$ at the left border of branch $\B$, so $r(0.25) = A$.
And $B$ is the value of the function $r$ at the right border of branch $\B$, so $r(0.5) = B$.
These constraints will yield the following set of equations.
\begin{subequations}
    \begin{align}
        r(0.25) & = - b_R \cdot \frac{1}{8} + c_R = A
        \label{equ:final.def.param.constr.A}
        \\
        r(0.5)  & = b_R \cdot \frac{1}{8} + c_R = B
        \label{equ:final.def.param.constr.B}
    \end{align}
\end{subequations}

Solving for $b_R$ and $c_R$ will give us explicit definitions of these two internal parameters depending on both $A$ and $B$.
Adding both \Cref{equ:final.def.param.constr.A} and \Cref{equ:final.def.param.constr.B} will yield the \Cref{equ:final.def.param.cR} for $c_R$.
\begin{align}
    2 \cdot c_R = A + B \implies c_R = \dfrac{A + B}{2}
    \label{equ:final.def.param.cR}
\end{align}

Similarily, subtracting \Cref{equ:final.def.param.constr.A} from \Cref{equ:final.def.param.constr.B} will yield the \Cref{equ:final.def.param.bR} for $b_R$.
\begin{align}
    \dfrac{b_R}{4} = B - A \implies b_R = 4 \cdot (B - A)
    \label{equ:final.def.param.bR}
\end{align}

For our purposes, we also fix the parameter $B = 0.525$, so that the right border of the branches $\B$ and $\D$ are just above the bisector.
We only change the parameter $A$ via the parameter $p_x$, where we negate the value of $p_x$ to get a similar 2D-scan to the original function.
So $A = -p_x$.

\Cref{tab:final.def.parameters.overview} lists the values of all parameters of the model.
The first part focuses on the parameters of the function $l$, while the second part lists the parameters needed for the function $r$.

\begin{table}
    \centering
    \begin{tabular}{|c|c|}
        \hline
        Model Parameter & Value                       \\ \hline \hline
        $a_L$           & $4$                         \\ \hline
        $b_L$           & $0.5$                       \\ \hline
        $c_L$           & $p_y$                       \\ \hline \hline
        $b_R$           & $4 \cdot (B - A)$           \\ \hline
        $c_R$           & $\frac{1}{2} \cdot (A + B)$ \\ \hline
        $A$             & $-p_x$                      \\ \hline
        $B$             & $0.525$                     \\ \hline
    \end{tabular}
    \caption{Overview of Parameter Values of Final Model}
    \label{tab:final.def.parameters.overview}
\end{table}
