\subsection{Model Dynamics}
\label{sec:state.og.dynamics}

The aforementioned interesting dynamics of the original model occur for the fixed parameters $\beta = 1, f = 150, L = 4.2 \cdot 10^{-3}, R = 2, V_m = 5,$ and $\mu = 0.5$.
The parameters $E_0$ and $\chi_0$ are varied in the ranges $[14, 28]$ and $[0.1, 0.65]$, respectively.
Scanning this parameter plane for the period of stable cycles results in \Cref{fig:state.og.dynamics.period}.

\begin{figure}
	\centering
	\includegraphics[width=0.6\textwidth]{../Figures/2/2.3/result.png}
	\caption[2D scan of the periods in the original model]{
		2D scan of the periods in the original model.
		The parameters $\beta = 1, f = 150, L = 4.2 \cdot 10^{-3}, R = 2, V_m = 5,$ and $\mu = 0.5$ are fixed.
		The parameters $E_0$ and $\chi_0$ are varied in the ranges $[14, 28]$ and $[0.1, 0.65]$, respectively.
		Each color represents a different period, where lighter colors correspond to higher periods.
		The numbers in the picture are the period of the corresponding chain of parameter regions.
		The three points $A, B,$ and $C$ mark the parameter values for the cobweb diagrams in \Cref{fig:state.og.dynamics.cobwebs}.
	}
	\label{fig:state.og.dynamics.period}
\end{figure}

The colors in the 2D scan \Cref{fig:state.og.dynamics.period} indicate the period of the stable cycle in those regions.
Brighter colors correspond to higher periods.
Points $A, B$ and $C$ are in the parameter region, which has stable cycles with the period 12.
The parameter regions differ in another way, the cycles in these regions are associated with different symbolic sequences.
As mentioned before, it describes on which branches of the model function the points of the cycle exist.
Some cobweb diagrams illustrate the difference between the parameter regions.

\begin{align}
	F(\theta + \pi) \equiv F(\theta) + \pi \mod 2\pi \label{equ:state.og.sym}
\end{align}

\Citeauthor{akyuz2022} pointed out a symmetry in the original model.
\Cref{equ:state.og.sym} describes this symmetry~\cite{akyuz2022}.
This means that the shapes of the branches $F_\A$ and $F_\C$ are identical.
The branch $F_\C$ is exactly $\pi$ to the right of $F_\A$ and its values are $\pi$ larger.
The same is true for the branches $F_\B$ and $F_\D$.
It follows that if $x$ is a part of a cycle in the original model, the point $x + \pi$ belongs to a cycle as well.
Therefore, only the two following cases are possible:

\begin{enumerate}[label=(\Alph*)]
	\item The points $x$ and $x + \pi$ belong to the same cycle.
	      This cycle is therefore symmetric, and such cycles are referred to as ``type A'' cycles in this thesis.
	      Such cycles must have an even period because there are as many points on the intervals $I_\A$ and $I_\B$ as there are on the intervals $I_\C$ and $I_\D$.
	\item The points $x$ and $x + \pi$ belong to different cycles.
	      Then there are at least 2 coexisting cycles with the same period.
	      This is not obvious and therefore proven below.
	      Such cycles \hl {are referred to as} ``type B'' cycles in this thesis.
\end{enumerate}

\begin{proof}[At Least Two Coexisting ``Type B'' Cycles] \phantom{x} \\
	Let $\O_k = \left\{x_i \:\mid\: 0 \leq i < k\right\}$ be a $k$-cycle of the original model where $x_i + \pi \not\in O_k$ for some $0 \leq i < k$.
	It follows that $x_i + \pi \not\in \O_k$ for any $x_i \in \O_k$.

	Then there is a second cycle $\O'_k = \left\{x_i + \pi \:\mid\: x_i \in \O_k\right\}$.
	This cycle $\O'_k$ is completely disjunct from the first cycle $\O_k$ per definition.
	Therefore, there are at least two coexisting cycles $\O_k \neq \O'_k$ with the same period. \hfill	$\blacksquare$
\end{proof}

\Cref{fig:state.og.dynamics.cobwebs} shows the cobweb diagrams at points $A$ and $C$.
Both parameter regions have only one stable cycle of period 12.
The stable cycle at point $A$ has the symbolic sequence $\A^3\B^3\C^3\D^3$ and the cycle at point $C$ has the symbolic sequence $\A^2\B^4\C^2\D^4$.
This thesis follows the convention that the branch with the smallest positive boundaries is called branch $\A$.
And the next branch is called branch $\B$ and so on.

\Cref{fig:state.og.dynamics.cobweb.B} shows the cobweb diagram at point $B$.
By looking closely, one can see that there are 2 coexisting cycles in this cobweb diagram.
One cycle has the symbolic sequence $\A^3\B^3\C^2\D^4$, while the other one has the symbolic sequence $\A^2\B^2\C^3\D^3$.
Both cycles are asymmetric.
And they are similar to each other in the way that $\Cycle{\A^3\B^3\C^2\D^4}$ behaves on the branches $\A$ and $\B$ like $\Cycle{\A^2\B^2\C^3\D^3}$ on the branches $\C$ and $\D$ and vice versa.
One can think of the cycles being equivalent by shifting them by $\pi$ in either direction.
Due to the symmetry of the model, an asymmetric stable cycle necessarily must exist alongside another asymmetric stable cycle that is itself, but shifted by $\pi$.
These cycles also behave similarly to both the cycles at points $A$ and $C$.
The cycle $\Cycle{\A^3\B^3\C^2\D^4}$ behaves like the cycle $\Cycle{\A^3\B^3\C^3\D^3}$ at point $A$ on its left half, while it behaves like the cycle $\Cycle{\A^2\B^4\C^2\D^4}$ at point $C$ on its right half.
The same is true for the cycle $\Cycle{\A^2\B^4\C^3\D^3}$ but reversed since it is the other cycle shifted by $\pi$.
The previously described parameter regions with only one symmetrical stable cycle are called ``type A'' parameter regions in this thesis.
And the just described parameter regions with 2 asymmetrical coexisting cycles are called ``type B'' parameter regions in this thesis.

\begin{figure}
	\centering
	\subfloat[$A$]{
		\includegraphics[width=.3 \textwidth]{../Figures/2/2.4a/result.png}
		\label{fig:state.og.dynamics.cobweb.A}
	}
	\subfloat[$B$]{
		\includegraphics[width=.3 \textwidth]{../Figures/2/2.4b/result.png}
		\label{fig:state.og.dynamics.cobweb.B}
	}
	\subfloat[$C$]{
		\includegraphics[width=.3 \textwidth]{../Figures/2/2.4b/result.png}
		\label{fig:state.og.dynamics.cobweb.C}
	}
	\caption[Cobweb diagrams of the original model]{
		Cobweb diagrams at three values of the parameters $E_0$ and $\chi_0$ in the original model with fixed parameters $\beta = 1, f = 150, L = 4.2 \cdot 10^{-3}, R = 2, V_m = 5,$ and $\mu = 0.5$.
		The parameter values of $E_0$ and $\chi_0$ are marked in \Cref{fig:state.og.dynamics.period}.
		(a) shows the cycle $\Cycle{\A^3\B^3\C^3\D^3}$ at point $A$,
		(b) shows the two coexisting cycles $\Cycle{\A^3\B^3\C^2\D^4}$ (green) and $\Cycle{\A^2\B^4\C^3\D^3}$ (red) at point $B$,
		and (c) shows the cycle $\Cycle{\A^2\B^4\C^2\D^4}$ at point $C$.
		\todo{Adjust caption: precise parameter values}
	}
	\label{fig:state.og.dynamics.cobwebs}
\end{figure}

This behavior is peculiar.
To summarize, we have chains of parameter regions associated with the same period.
The type of the parameter regions alternates between ``type A'' and ``type B''.
When the stable cycle in one ``type A'' parameter region is $\Cycle{\A^a\B^b\C^a\D^b}$, the stable cycle in the next ``type A'' parameter region is $\Cycle{\A^{c}\B^{d}\C^{c}\D^{d}}$ with $c = a - 1$ and $d = b + 1$.
The ``type B'' parameter region in between two ``type A'' parameter regions of a chain with cycles $\Cycle{\A^a\B^b\C^c\D^d}$ and $\Cycle{\A^{c}\B^{d}\C^{a}\D^{b}}$, has the two cycles $\Cycle{\A^x\B^y\C^{x-1}\D^{y+1}}$ and $\Cycle{\A^{x-1}\B^{y+1}\C^x\D^y}$.
Also, there is not only one chain, but many chains next to each other where the period increases by 2 from one chain to the next.
