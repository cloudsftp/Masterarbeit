\section{Investigated Dynamical System}
\label{sec:state.og}

As mentioned before, this thesis considers a complex dynamical system.
It is referred to as the original model from now on.
This section gives a definition of the model and some properties.
Furthermore, it showcases an unusual bifurcation structure this model exhibits.
This bifurcation structure is the core motivation for this thesis.

\subsection{Model Origin}
\label{sec:state.og.orig}

The original model describes the behavior of a DC to AC power converter with hysteresis control.
In continuous-time, the converter is described by a differential equation.
The differential equation indicates the rate of change of the output of the power converter based on the current output and the current time.
Here, the current time is given as a phase since the converter tries to output alternating current that follows a sine wave.
The converter switches to achieve its goal.
This switching causes discontinuities in the model function.

From this time-continuous model, one can derive a time-discrete model.
This model maps a phase at which the converter switches to the phase at which the converter will switch the next time.
It is also \gls{pws} and discontinuous like the time-continuous model from which it is derived.
The next subsection gives an in-depth definition of this map.

\subsection{Model Definition}
\label{sec:state.og.def}

The model was defined by Zhusubaliyev and was preliminarily investigated by \Citeauthor{akyuz2022} in his thesis ``\Citetitle{akyuz2022}''~\cite{akyuz2022}.
As mentioned before, we skip the time-continuous model and only focus on the time-discrete one.
It is defined as the map $\theta_{n-1} = F(\theta_n) \mod 2 \pi$ where $F$ is defined by the following equations.
\begin{align}
	F(\theta) = \begin{cases}
		            F_1(\theta) & \text{if } q \cdot \cos(\theta) > 0 \\
		            F_2(\theta) & \text{if } q \cdot \cos(\theta) < 0
	            \end{cases}
\end{align}
where $F_1$ and $F_2$ are given by
\begin{subequations}
	\begin{align}
		F_1(\theta) & = \begin{cases}
			                \theta + z_{L_+} + z_1 & \text{if } z_{L_+} < z_{L_0} \\
			                \theta + z_{L_0} + z_2 & \text{if } z_{L_+} > z_{L_0}
		                \end{cases} \\
		F_2(\theta) & = \begin{cases}
			                \theta + z_{R_+} + z_3 & \text{if } z_{R_+} < z_{R_0} \\
			                \theta + z_{R_0} + z_4 & \text{if } z_{R_+} > z_{R_0}
		                \end{cases}
	\end{align}
\end{subequations}
The values for the parameters $z_{L_+}, z_{L_0}, z_{R_+}, z_{R_0}, z_1, z_2, z_3$, and $z_4$ are the smallest positive solutions to the implicit equations \Crefrange{equ:setup.og.def.impl.1.A}{equ:setup.og.def.impl.2.D}.
\begin{subequations}
	\begin{align}
		(q \cdot \cos(\theta) + \mu \cdot \chi) \cdot e^{\lambda \cdot z_{L_+}}
		 & = q \cdot \cos(\theta + z_{L_+}) + \chi \label{equ:setup.og.def.impl.1.A}                  \\
		(q \cdot \cos(\theta) + \mu \cdot \chi) \cdot e^{\lambda \cdot z_{L_0}}
		 & = q \cdot \cos(\theta + z_{L_0}) - \chi                                                    \\
		(q \cdot \cos(\theta) - \mu \cdot \chi) \cdot e^{\lambda \cdot z_{R_+}}
		 & = q \cdot \cos(\theta + z_{R_+}) - \chi                                                    \\
		(q \cdot \cos(\theta) - \mu \cdot \chi) \cdot e^{\lambda \cdot z_{R_0}}
		 & = q \cdot \cos(\theta + z_{R_0}) + \chi \label{equ:setup.og.def.impl.1.D}
		\\
		(q \cdot \cos(\theta + z_{L_+}) + \chi + 1) \cdot e^{\lambda \cdot z_1} - 1
		 & = q \cdot  \cos(\theta + z_{L_+} + z_1) + \mu \cdot \chi \label{equ:setup.og.def.impl.2.A} \\
		(q \cdot \cos(\theta + z_{L_0} + z_2) - \chi - 1) \cdot e^{\lambda \cdot z_2} + 1
		 & = q \cdot  \cos(\theta + z_{L_0} + z_2) - \mu \cdot \chi                                   \\
		(q \cdot \cos(\theta + z_{R_+}) + \chi + 1) \cdot e^{\lambda \cdot z_3} - 1
		 & = q \cdot  \cos(\theta + z_{L_+} + z_1) + \mu \cdot \chi                                   \\
		(q \cdot \cos(\theta + z_{R_0} + z_4) - \chi - 1) \cdot e^{\lambda \cdot z_4} + 1
		 & = q \cdot  \cos(\theta + z_{R_0} + z_2) - \mu \cdot \chi \label{equ:setup.og.def.impl.2.D}
	\end{align}
\end{subequations}

The values for $\chi, \lambda, \mu,$ and $q$ come from the parameters of the model.
The parameters are $\chi_0, E_0, \beta, f, L, R, V_m,$ and $\mu$.
$\mu$ is directly applied in the equations above, while $\chi, \lambda,$ and $q$ are computed from multiple parameters.
The values of $\chi, \lambda,$ and $q$ are given by the following \Crefrange{equ:setup.og.def.param.chi}{equ:setup.og.def.param.q}.
\begin{subequations}
	\begin{align}
		\chi    & = \dfrac{R \cdot \chi_0}{\beta \cdot E_0} \label{equ:setup.og.def.param.chi} \\
		\lambda & = \dfrac{-R}{L \cdot 2 \cdot \pi \cdot f}                                    \\
		q       & = \dfrac{R \cdot V_m}{\beta \cdot E_0} \label{equ:setup.og.def.param.q}
	\end{align}
\end{subequations}
