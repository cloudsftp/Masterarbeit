\section{The Investigated Dynamical System}
\label{sec:state.og}

\hl{(Title changed)}

As mentioned before, we will take a closer look at a complex \hl{dynamical system} in this thesis.
From now on we will refer to it as the original model.
\hl{
	This section gives a definition of the model and some properties.
	Furthermore, it showcases the unusual bifurcation structures, this model exhibits.
	Those are the core motivation for this thesis.
}

\todo{Here I switch from system to model. Is it ok?}

\subsection{Model Origin}
\label{sec:state.og.orig}

The \hl{original} model describes the behavior of a DC to AC power converter with hysteresis control.
This power converter switches and the model is therefore \hl{\gls{pws}}.
In the time-continuous case, the converter is described using a differential equation.
The differential equation indicates the rate of change of the output of the power converter based on the current output and the current time.
Where the current time is given as a phase since the converter tries to output alternating current that follows a sine wave.
Every time the output of the converter hits an upper or lower limit, it switches.
This causes the discontinuities.

From this time-continuous model, \hl{one} can derive a \hl{time-discrete model}.
This model maps \hl{a} phase at which the converter \hl{switches} to the phase at which the converter will switch the next time.
The next subsection will give an in-depth definition of this map.

\subsection{Model Definition}
\label{sec:state.og.def}

The model was defined by Zhusubaliyev and \Citeauthor{akyuz2022} did an analysis of that model in his thesis ``\Citetitle{akyuz2022}''.
\hl{As mentioned before,} we ignore the \hl{time-continuous} case and only focus on the time-discrete model.
It is defined as the map $\theta \mapsto F(\theta) \mod 2 \pi$ where $F$ is defined by the following equations~\cite{akyuz2022}.
\begin{align}
	F(\theta) = \begin{cases}
		            F_1(\theta) & \text{if } q \cdot \cos(\theta) > 0 \\
		            F_2(\theta) & \text{if } q \cdot \cos(\theta) < 0
	            \end{cases}
\end{align}
Where $F_1$ is given by the following equation.
\begin{align}
	F_1(\theta) & = \begin{cases}
		                \theta + z_{L_+} + z_1 & \text{if } z_{L_+} < z_{L_0} \\
		                \theta + z_{L_0} + z_2 & \text{if } z_{L_+} > z_{L_0}
	                \end{cases}
\end{align}
And $F_2$ is similarly given by the following equation.
\begin{align}
	F_2(\theta) & = \begin{cases}
		                \theta + z_{R_+} + z_3 & \text{if } z_{R_+} < z_{R_0} \\
		                \theta + z_{R_0} + z_4 & \text{if } z_{R_+} > z_{R_0}
	                \end{cases}
\end{align}
\hl{
	The values for the parameters $z_{L_+}, z_{L_0}, z_{R_+},$ and $z_{R_0}$ are the smallest positive solutions to the implicit equations \Crefrange{equ:setup.og.def.impl.1.A}{equ:setup.og.def.impl.2.D}.
	Note, that these parameters are part of the following block of equations, as well as the block of equations after that.
}
\begin{subequations}
	\begin{align}
		(q \cdot \cos(\theta) + \mu \cdot \chi) \cdot e^{\lambda \cdot z_{L_+}}
		 & = q \cdot \cos(\theta + z_{L_+}) + \chi \label{equ:setup.og.def.impl.1.A} \\
		(q \cdot \cos(\theta) + \mu \cdot \chi) \cdot e^{\lambda \cdot z_{L_0}}
		 & = q \cdot \cos(\theta + z_{L_0}) - \chi                                   \\
		(q \cdot \cos(\theta) - \mu \cdot \chi) \cdot e^{\lambda \cdot z_{R_+}}
		 & = q \cdot \cos(\theta + z_{R_+}) - \chi                                   \\
		(q \cdot \cos(\theta) - \mu \cdot \chi) \cdot e^{\lambda \cdot z_{R_0}}
		 & = q \cdot \cos(\theta + z_{R_0}) + \chi \label{equ:setup.og.def.impl.1.D}
	\end{align}
\end{subequations}
\hl{The values for the parameters $z_1, z_2, z_3$, and $z_4$ are the smallest positive solutions to the \Crefrange{equ:setup.og.def.impl.2.A}{equ:setup.og.def.impl.2.D}}.
\begin{subequations}
	\begin{align}
		(q \cdot \cos(\theta + z_{L_+}) + \chi + 1) \cdot e^{\lambda \cdot z_1} - 1
		 & = q \cdot  \cos(\theta + z_{L_+} + z_1) + \mu \cdot \chi \label{equ:setup.og.def.impl.2.A} \\
		(q \cdot \cos(\theta + z_{L_0} + z_2) - \chi - 1) \cdot e^{\lambda \cdot z_2} + 1
		 & = q \cdot  \cos(\theta + z_{L_0} + z_2) - \mu \cdot \chi                                   \\
		(q \cdot \cos(\theta + z_{R_+}) + \chi + 1) \cdot e^{\lambda \cdot z_3} - 1
		 & = q \cdot  \cos(\theta + z_{L_+} + z_1) + \mu \cdot \chi                                   \\
		(q \cdot \cos(\theta + z_{R_0} + z_4) - \chi - 1) \cdot e^{\lambda \cdot z_4} + 1
		 & = q \cdot  \cos(\theta + z_{R_0} + z_2) - \mu \cdot \chi \label{equ:setup.og.def.impl.2.D}
	\end{align}
\end{subequations}

The values for $\chi, \lambda, \mu,$ and $q$ come from the parameters of the model.
The parameters are $\chi_0, E_0, \beta, f, L, R, V_m,$ and $\mu$.
$\mu$ is directly applied in the equations above, while $\chi, \lambda,$ and $q$ \hl{are computed from multiple parameters}.
The values of $\chi, \lambda,$ and $q$ are given by the \hl{following \Crefrange{equ:setup.og.def.param.chi}{equ:setup.og.def.param.q}}.
\begin{subequations}
	\begin{align}
		\chi    & = \dfrac{R \cdot \chi_0}{\beta \cdot E_0} \label{equ:setup.og.def.param.chi} \\
		\lambda & = \dfrac{-R}{L \cdot 2 \cdot \pi \cdot f}                                    \\
		q       & = \dfrac{R \cdot V_m}{\beta \cdot E_0} \label{equ:setup.og.def.param.q}
	\end{align}
\end{subequations}
