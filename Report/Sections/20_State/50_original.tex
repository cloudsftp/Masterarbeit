\section{The Investigated Model}
\label{sec:state.og}

As mentioned before, we will take a closer look at a complex model in this thesis.
From now on we will refer to it as the original model.
In this section, we will give a definition and some properties of the model.
Furthermore, we will showcase the unusual bifurcation structures, this model exhibits.
Which are the core motivation for this thesis.

\subsection{Model Origin}
\label{sec:state.og.orig}

The model describes the behavior of a DC to AC power converter with hysteresis control.
This power converter switches and the model is therefore discontinuous.
In the time continuous case, the converter is described using a differential equation.
The differential equation indicates the rate of change of the output of the power converter based on the current output and the current time.
Where the current time is given as a phase since the converter tries to output alternating current that follows a sine wave.
Every time the output of the converter hits an upper or lower limit, it switches.
This causes the discontinuities.

From this time continuous model, we can derive a model in continuous time.
This model maps the phase at which the converter switched to the phase at which the converter will switch the next time.
The next subsection will give an in-depth definition of this map.

\subsection{Model Definition}
\label{sec:state.og.def}

The model was defined by Zhusubaliyev and \Citeauthor{akyuz2022} did an analysis of that model in his thesis ``\Citetitle{akyuz2022}''.
We ignore the continuous time case and only focus on the time discrete model.
It is defined as the map $\theta \mapsto F(\theta) \mod 2 \pi$ where $F$ is defined by the following equations~\cite{akyuz2022}.
\begin{align}
	F(\theta) = \begin{cases}
		            F_1(\theta) & \text{if } q \cdot \cos(\theta) > 0 \\
		            F_2(\theta) & \text{if } q \cdot \cos(\theta) < 0
	            \end{cases}
\end{align}
Where $F_1$ is given by the following equation.
\begin{align}
	F_1(\theta) & = \begin{cases}
		                \theta + z_{L_+} + z_1 & \text{if } z_{L_+} < z_{L_0} \\
		                \theta + z_{L_0} + z_2 & \text{if } z_{L_+} > z_{L_0}
	                \end{cases}
\end{align}
And $F_2$ is similarly given by the following equation.
\begin{align}
	F_2(\theta) & = \begin{cases}
		                \theta + z_{R_+} + z_3 & \text{if } z_{R_+} < z_{R_0} \\
		                \theta + z_{R_0} + z_4 & \text{if } z_{R_+} > z_{R_0}
	                \end{cases}
\end{align}
$z_{L_+}, z_{L_0}, z_{R_+},$ and $z_{R_0}$ must satisfy the following equations.
\begin{subequations}
	\begin{align}
		(q \cdot \cos(\theta) + \mu \cdot \chi) \cdot e^{\lambda \cdot z_{L_+}}
		 & = q \cdot \cos(\theta + z_{L_+}) + \chi \\
		(q \cdot \cos(\theta) + \mu \cdot \chi) \cdot e^{\lambda \cdot z_{L_0}}
		 & = q \cdot \cos(\theta + z_{L_0}) - \chi \\
		(q \cdot \cos(\theta) - \mu \cdot \chi) \cdot e^{\lambda \cdot z_{R_+}}
		 & = q \cdot \cos(\theta + z_{R_+}) - \chi \\
		(q \cdot \cos(\theta) - \mu \cdot \chi) \cdot e^{\lambda \cdot z_{R_0}}
		 & = q \cdot \cos(\theta + z_{R_0}) + \chi
	\end{align}
\end{subequations}
They also have to satisfy the following equations.
So do the parameters $z_1, z_2, z_3$, and $z_4$, which were not included in the previous set of implicit equations.
\begin{subequations}
	\begin{align}
		(q \cdot \cos(\theta + z_{L_+}) + \chi + 1) \cdot e^{\lambda \cdot z_1} - 1
		 & = q \cdot  \cos(\theta + z_{L_+} + z_1) + \mu \cdot \chi \\
		(q \cdot \cos(\theta + z_{L_0} + z_2) - \chi - 1) \cdot e^{\lambda \cdot z_2} + 1
		 & = q \cdot  \cos(\theta + z_{L_0} + z_2) - \mu \cdot \chi \\
		(q \cdot \cos(\theta + z_{R_+}) + \chi + 1) \cdot e^{\lambda \cdot z_3} - 1
		 & = q \cdot  \cos(\theta + z_{L_+} + z_1) + \mu \cdot \chi \\
		(q \cdot \cos(\theta + z_{R_0} + z_4) - \chi - 1) \cdot e^{\lambda \cdot z_4} + 1
		 & = q \cdot  \cos(\theta + z_{R_0} + z_2) - \mu \cdot \chi
	\end{align}
\end{subequations}

The values of $z_{L_+}, z_{L_0}, z_{R_+}, z_{R_0}, z_1, z_2, z_3,$ and $z_4$ are the smallest positive solutions that satisfy the implicit equations above.

\todo{2.7 c + d? $z_3, z_4$?}

The values for $\chi, \lambda, \mu,$ and $q$ come from the parameters of the model.
The parameters are $\chi_0, E_0, \beta, f, L, R, V_m,$ and $\mu$.
$\mu$ is directly applied in the equations above, while $\chi$ and $q$ consist of multiple parameters.
The values of $\chi$ and $q$ are given by the following equations.
\begin{subequations}
	\begin{align}
		\chi    & = \dfrac{R \cdot \chi_0}{\beta \cdot E_0} \\
		\lambda & = \dfrac{-R}{L \cdot 2 \cdot \pi \cdot f} \\
		q       & = \dfrac{R \cdot V_m}{\beta \cdot E_0}
	\end{align}
\end{subequations}
