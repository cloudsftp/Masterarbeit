\section{Non-linear Dynamical Systems}

Dynamical systems can either be time-continuous or time-discrete.
In this thesis we only consider time-discrete dynamical systems, also called maps.
The time evolution of the state of such a system results from an iterative application of the model function governing the dynamics of the system, i.e. $x_{n+1} = f(x_n)$.
Depending on the properties of the model function $f$, one can distinguish between smooth, \gls{pws} continuous, and \gls{pws} discontinuous dynamical systems.

We are most interested in invariant sets.
In the following, we define fixed points and cycles.

\begin{definition}[Fixed Point]
	Let $x_{n+1} = f(x_n)$ be a time-discrete 1D non-linear dynamical system where the model function is $f: \mathbb{R} \to \mathbb{R}$.
	Then $x \in X$ is a fixed point, if $f(x) = x$.
\end{definition}

\begin{definition}[Iterate Function]
	Let $f: \mathbb{R} \to \mathbb{R}$ be the model function of a 1D non-linear dynamical system.
	Then $f^k$ is the $k$-th iterate of this function, defined recursively by $f^k(x) = f\left(f^{k-1}(x)\right)$ and $f^1(x) = f(x)$.
	This means, we apply the function $k$ times.
\end{definition}

\begin{definition}[Cycle]
	Let $f: \mathbb{R} \to \mathbb{R}$ be the model function of a 1D non-linear dynamical system.
	Then the sequence of $k$ points $\O_k = \{x_i \:\mid\: 0 \leq i < k\}$ is a $k$-cycle if it satisfies the following conditions.
	\begin{align}
		\forall 1 \leq i < k: \quad & f(x_i) = x_{i+1} \quad \land \quad f(x_{k-1}) = x_0
	\end{align}
	If $x_0 \neq x_i$ for $1 \leq i < k$, then $k$ is the prime period of $\O_k$.
\end{definition}

\hl{
	Each point $x_i$ of a $k$-cycle satisfies $f^k(x_i) = x_i$.
}

\begin{definition}[Stability]
	Let $f: \mathbb{R} \to \mathbb{R}$ be the model function of a 1D non-linear dynamical system again.
	A fixed point $x^*$ is stable, if $|f'(x^*)| < 1$.
	Similarly, a $k$-cycle $\O_k = \{x_i \:\mid\: 0 \leq i < k\}$ is stable, if the following condition is true for all $0 \leq i < k$.
	\begin{align}
		\left| \left. \frac{d}{dx}f^k(x) \:\right|_{x = x_i}\right| & < 1
	\end{align}
\end{definition}

\hl{
	Note that the derivative of $f^k$ evaluated at any point $x_i$ of the $k$-cycle $\O_k$ takes the same value.
	That value being the product of the derivative of the function $f$ evaluated at all points $x_i$.
}
\begin{align}
	\left. \frac{d}{dx}f^k(x) \:\right|_{x = x_i} & = \prod_{j=0}^{k-1} \left( \left. \frac{d}{dx} f(x) \:\right|_{x = x_j} \right)
\end{align}

\begin{definition}[Bifurcation]
	A bifurcation is a qualitative change of the state space topology of a model under an infinitesimal small parameter variation.
\end{definition}
Meaning for example that solutions to the model function or its iterate (fixed points and cycles) disappear or appear under a very small change to parameters.
The only bifurcations that will be important in this thesis are introduced in the next section.
That section will cover \gls{pws} dynamical systems.
