\section{Non-linear Dynamical Systems}

Dynamical systems can either be time-continuous or time-discrete.
In this thesis, only time-discrete dynamical systems, also called maps, are considered.
The time evolution of the state of such a system results from an iterative application of the model function governing the dynamics of the system, i.e. $x_{n+1} = f(x_n)$.
Depending on the properties of the model function $f$, one can distinguish between smooth, \gls{pws} continuous, and \gls{pws} discontinuous dynamical systems.

This thesis primarily investigates invariant sets.
Next, two types of invariant sets are defined, fixed points and cycles.

\begin{definition}[Fixed Point]
	Let $x_{n+1} = f(x_n)$ be a time-discrete non-linear dynamical system where the model function is $f: X \to X$.
	Then $x \in X$ is a fixed point, if $f(x) = x$.
\end{definition}

\begin{definition}[Iterate Function]
	Let $f: X \to X$ be the model function of a time-discrete non-linear dynamical system.
	Then $f^k$ is the $k$-th iterate of this function, defined recursively by $f^k(x) = f\left(f^{k-1}(x)\right)$ and $f^1(x) = f(x)$.
	This means, the function is applied $k$ times.
\end{definition}

\begin{definition}[Cycle]
	Let $f: X \to X$ be the model function of a time-discrete non-linear dynamical system.
	Then the sequence of $k$ points $\O_k = \{x_i \:\mid\: 0 \leq i < k\}$ is a $k$-cycle if it satisfies the following conditions.
	\begin{align}
		\forall 1 \leq i < k: \quad & f(x_i) = x_{i+1} \quad \land \quad f(x_{k-1}) = x_0
	\end{align}
	If $x_0 \neq x_i$ for $1 \leq i < k$, then $k$ is the prime period of $\O_k$.
\end{definition}

Each point $x_i$ of a $k$-cycle satisfies $f^k(x_i) = x_i$.

\begin{definition}[Stability]
	Let $f: X \to X$ be the model function of a time-discrete non-linear dynamical system where $X$ is one-dimensional.
	A fixed point $x^*$ is stable, if $|f'(x^*)| < 1$.
	Similarly, a $k$-cycle $\O_k = \{x_i \:\mid\: 0 \leq i < k\}$ is stable, if the following condition is true for all $0 \leq i < k$.
	\begin{align}
		\left| \left. \frac{d}{dx}f^k(x) \:\right|_{x = x_i}\right| & < 1
	\end{align}
\end{definition}

Note that the derivative of $f^k$ evaluated at any point $x_i$ of the $k$-cycle $\O_k$ takes the same value.
That value being the product of the derivative of the function $f$ evaluated at all points $x_i$.
\begin{align}
	\left. \frac{d}{dx}f^k(x) \:\right|_{x = x_i} & = \prod_{j=0}^{k-1} \left( \left. \frac{d}{dx} f(x) \:\right|_{x = x_j} \right)
\end{align}

This thesis only covers stable cycles.
Stable cycles have the property that different points of the state space that are not necessarily part of the cycle will eventually end up in the cycle.
The set of those points are called the basin of attraction of the cycle.

\begin{definition}[Basin of Attraction of a Cycle]
	Let $f: X \to X$ be the model function of a non-linear dynamical system.
	Let $\O_k$ be a stable cycle in that system.
	Then the basin of attraction of this cycle is
	\begin{align}
		\left\{x \in X \:\mid\: \exists n > 0: \:f^n(x) \in \O_k\right\}
	\end{align}
\end{definition}

\begin{definition}[Bifurcation]
	A bifurcation is a qualitative change of the state space topology of a model under an infinitesimal small parameter variation.
\end{definition}
Meaning for example that solutions to the model function or its iterate (fixed points and cycles) disappear or appear under a very small change to parameters.
The only bifurcations that are important in this thesis are introduced in the next section.
That section covers \gls{pws} dynamical systems.
