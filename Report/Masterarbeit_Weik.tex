% !TeX spellcheck = en-US
% !TeX encoding = utf8
% !TeX program = pdflatex
% !BIB program = biber
% -*- coding:utf-8 mod:LaTeX -*-


% vv  scroll down to line 200 for content  vv


\let\ifdeutsch\iffalse
\let\ifenglisch\iftrue
\input{pre-documentclass}
\documentclass[
  % fontsize=11pt is the standard
  a4paper,  % Standard format - only KOMAScript uses paper=a4 - https://tex.stackexchange.com/a/61044/9075
  twoside,  % we are optimizing for both screen and two-side printing. So the page numbers will jump, but the content is configured to stay in the middle (by using the geometry package)
  bibliography=totoc,
  %               idxtotoc,   %Index ins Inhaltsverzeichnis
  %               liststotoc, %List of X ins Inhaltsverzeichnis, mit liststotocnumbered werden die Abbildungsverzeichnisse nummeriert
  headsepline,
  cleardoublepage=empty,
  parskip=half,
  %               draft    % um zu sehen, wo noch nachgebessert werden muss - wichtig, da Bindungskorrektur mit drin
  draft=false
]{scrbook}
\input{template-config.tex}
\usepackage[inline,shortlabels]{enumitem}

\usepackage[normalem]{ulem}
\newcommand\hl{\bgroup\markoverwith
  {\textcolor{green}{\rule[-.5ex]{.1pt}{2.5ex}}}\ULon}

\usepackage{mathtools}

\setlist[enumerate]{label=(\roman*)}

\graphicspath{{../Simulation/Models/}{Figures/}}

\renewcommand{\P}{\mathcal{P}}

\newcommand{\A}{\mathcal{A}}
\newcommand{\B}{\mathcal{B}}
\newcommand{\C}{\mathcal{C}}
\newcommand{\D}{\mathcal{D}}

\newcommand{\X}{\mathcal{X}}

\renewcommand{\L}{\mathcal{L}}
\newcommand{\R}{\mathcal{R}}

\DeclareMathOperator{\AL}{L}
\DeclareMathOperator{\AR}{R}
\DeclareMathOperator{\AW}{W}
\DeclareMathOperator{\AMi}{Mi}
\DeclareMathOperator{\AB}{B}

\newcommand{\BCB}{\xi}

\newcommand{\BranchInterval}{I}
\newcommand{\Branch}{f}

\renewcommand{\O}{\mathcal{O}}
\newcommand{\Cycle}[1]{\O_{#1}}

% EN: For theorems, replacement for amsthm
\usepackage[amsmath,hyperref]{ntheorem}

\theorempreskipamount 2ex plus1ex minus0.5ex
\theorempostskipamount 2ex plus1ex minus0.5ex
\theoremstyle{break}
\newtheorem{theorem}{Theorem}[chapter]
\newtheorem{lemma}[theorem]{Lemma}
\newtheorem{definition}{Definition}[chapter]

\theoremstyle{nonumberplain}
\newtheorem{proof}{Proof}

\usepackage{listing-rust}


\makeatletter
\def\ext@algorithm{lol}% algorithm captions will be written to the .lol file
% share the list making commands and redefine the heading
\AtBeginDocument{%
  \let\l@algorithm\l@lstlisting
  \let\c@algorithm\c@lstlisting
  \let\thealgorithm\thelstlisting
  \renewcommand{\lstlistlistingname}{Algorithms and program code}%
}
\makeatother



\usepackage[
  title={Investigation of Multistability-affected Period Adding Structure},
  author={Fabian Weik},
  type=master,
  institute=ipvs, % systems theory: istac,
  course={Informatik},
  examiner={Prof. Dr. Holger Schwarz},
  supervisor={Prof. Dr. Viktor Avrutin},
  startdate={March 1, 2023},
  enddate={September 1, 2023}
]{scientific-thesis-cover}

\makeindex

\bibliography{bibliography.bib}

\newif\ifshowfigures  % At the time of definition it is false!
\showfigurestrue % yes, we show figures
% Invert
\newif\ifdropfigures
\dropfigurestrue
\ifshowfigures
  \dropfiguresfalse
\fi
\ifdropfigures
  \renewcommand{\includegraphics}[2][]{%
  }% Ok, `\includegraphics does nothing any longer
\fi

\newacronym{pws}{PWS}{piecewise smooth}


\begin{document}

%tex4ht-Konvertierung verschönern
\iftex4ht
  % tell tex4ht to create picures also for formulas starting with '$'
  % WARNING: a tex4ht run now takes forever!
  \Configure{$}{\PicMath}{\EndPicMath}{}
  %$ % <- syntax highlighting fix for emacs
  \Css{body {text-align:justify;}}

  %conversion of .pdf to .png
  \Configure{graphics*}
  {pdf}
  {\Needs{"convert \csname Gin@base\endcsname.pdf
      \csname Gin@base\endcsname.png"}%
    \Picture[pict]{\csname Gin@base\endcsname.png}%
  }
\fi

%\VerbatimFootnotes %verbatim text in Fußnoten erlauben. Geht normalerweise nicht.

\input{template-commands.tex}
\pagenumbering{arabic}
\Titelblatt

%Eigener Seitenstil fuer die Kurzfassung und das Inhaltsverzeichnis
\deftriplepagestyle{preamble}{}{}{}{}{}{\pagemark}
%Doku zu deftriplepagestyle: scrguide.pdf
\pagestyle{preamble}
\renewcommand*{\chapterpagestyle}{preamble}

%Kurzfassung / abstract
%auch im Stil vom Inhaltsverzeichnis
\ifdeutsch
  \section*{Kurzfassung}
\else
  \section*{Abstract}
\fi

The theory of non-linear dynamical systems is very effective at explaining physical systems and phenomena.
Unfortunately, the theory is developed mainly for smooth systems.
And the models of some systems, such as power converters, are inherently piecewise-smooth.
This thesis is concerned with such a model that is piecewise-smooth, discontinuous, and symmetric.
The definition of this model is very complex, and it exhibits a unique period-incrementing structure that is affected by multistability.
This thesis identifies the characteristics of the model that lead to this unique bifurcation structure by constructing a simplified model that exhibits the same bifurcation structure, the archetypal model.
It follows a description of the dynamics of the archetypal model and an explanation of the unique bifurcation structure using the description of the dynamics.
Also, this thesis demonstrates that the proposed archetypal model can exhibit structures related to \glsentrylong{pa}.
These structures behave unexpectedly.
And this behavior is explained by taking advantage of the symmetry in the archetypal model.
In the same way, rules for these period-adding structures are constructed.


\cleardoublepage


% BEGIN: Verzeichnisse

\iftex4ht
\else
  \microtypesetup{protrusion=false}
\fi

%%%
% Literaturverzeichnis ins TOC mit aufnehmen, aber nur wenn nichts anderes mehr hilft!
% \addcontentsline{toc}{chapter}{Literaturverzeichnis}
%
% oder zB
%\addcontentsline{toc}{section}{Abkürzungsverzeichnis}
%
%%%

%Produce table of contents
%
%In case you have trouble with headings reaching into the page numbers, enable the following three lines.
%Hint by http://golatex.de/inhaltsverzeichnis-schreibt-ueber-rand-t3106.html
%
%\makeatletter
%\renewcommand{\@pnumwidth}{2em}
%\makeatother
%
\tableofcontents

% Bei einem ungünstigen Seitenumbruch im Inhaltsverzeichnis, kann dieser mit
% \addtocontents{toc}{\protect\newpage}
% an der passenden Stelle im Fließtext erzwungen werden.

\listoffigures
%\listoftables

%Wird nur bei Verwendung von der lstlisting-Umgebung mit dem "caption"-Parameter benoetigt
%\lstlistoflistings
%ansonsten:
\ifdeutsch
  \listof{Listing}{Verzeichnis der Listings}
\else
  %  \listof{Listing}{List of Listings}
\fi

%\listofalgorithms %Ist nur für Algorithmen, die mittels \begin{algorithm} umschlossen werden, nötig

% Abkürzungsverzeichnis
\printnoidxglossaries

\iftex4ht
\else
  %Optischen Randausgleich und Grauwertkorrektur wieder aktivieren
  \microtypesetup{protrusion=true}
\fi

% END: Verzeichnisse


% Headline and footline
\renewcommand*{\chapterpagestyle}{scrplain}
\pagestyle{scrheadings}
\pagestyle{scrheadings}
\ihead[]{}
\chead[]{}
\ohead[]{\headmark}
\cfoot[]{}
\ofoot[\usekomafont{pagenumber}\thepage]{\usekomafont{pagenumber}\thepage}
\ifoot[]{}


%% vv  scroll down for content  vv %%































%%%%%%%%%%%%%%%%%%%%%%%%%%%%%%%%%%%%%%%%%%%%%%%%%%%%%%%%%%%%%%%%%%%%%%%%%%%%%%
%
% Main content starts here
%
%%%%%%%%%%%%%%%%%%%%%%%%%%%%%%%%%%%%%%%%%%%%%%%%%%%%%%%%%%%%%%%%%%%%%%%%%%%%%%

\todo{Intro, fundamentals, releated work, etc.}

\chapter{Examined Models}

\section{Piecewise-linear Model}

The first model in this chapter and also the first constructed model is the piecewise-linear model.
It has 4 linear branches and is defined as the map $x_{n+1} = f(x_n) \mod 1$ where the following set of equations defines $f$.

\begin{align}
	f(x) & = \begin{cases}
		         g(x)                                        & \text{ if } x < \frac{1}{2} \\
		         g\left(x - \frac{1}{2}\right) + \frac{1}{2} & \text{ else}
	         \end{cases} \label{equ:app.model.lin.f} \\
	g(x) & = \begin{cases}
		         g_L(x) = \alpha \cdot x + \beta            & \text{ if } x < \frac{1}{4} \\
		         g_R(x) = \alpha \cdot x - \frac{\alpha}{4} & \text{ else}
	         \end{cases} \label{equ:app.model.lin.g}
\end{align}

One can see that this model definition is a little different from the model definitions in the main part of the thesis.
For example, \Cref{equ:app.model.lin.g} also enforces the symmetry that is found in the original model in this model explicitly.
And \Cref{equ:app.model.lin.g} then breaks each half of the model function into two smaller parts.
One difference is that $\alpha$ influences the slope of all four branches and also influences the offset of the function $g_R$.
$g_R$ governs the branches $f_\B$ and $f_\D$ and its offset causes the branch $f_\B$ to start at $0$ and the branch $f_\D$ to start at $\frac{1}{2}$.

\begin{figure}
	\centering
	\subfloat[2D scan]{
		\includegraphics[width=0.7 \textwidth]{../Figures/A/A.1a/result.png}
		\label{fig:app.model.lin.2D}
	} \\
	\subfloat[1D scan]{
		\includegraphics[width=0.5 \textwidth]{../Figures/A/A.1b/result.png}
		\label{fig:app.model.lin.1D}
	}
	\caption[2D and 1D scans of the periods associated with parameter regions in the piecewise-linear model]{
		2D and 1D scans of the periods associated with parameter regions in the piecewise-linear model
		The parameters $\alpha$ and $\beta$ are different in every diagram.
		(a) shows the 2D scan with the parameters $\alpha$ and $\beta$ both being varied in the range $[0, 1]$,
		(b) shows the 1D scan with the parameter $\alpha = \frac{1}{2}$ fixed and $\beta$ varied in the range $[0.1, 0.3]$,
	}
\end{figure}

\Cref{fig:app.model.lin.2D} shows a 2D scan of the periods associated with parameter regions in this model.
The structures look a lot like \gls{pa} structures.
Scanning the periods in 1D, results in \Cref{fig:app.model.lin.1D}.
This indeed shows a pattern that is typical for \gls{pa} structures.


\section{Piecewise Quadratic Model}

In this section, we will examine the dynamics of a piecewise quadratic model.
Starting with a model with 4 branches and symmetry, like in \Cref{sec:og.full}.
After that, we will reduce the model to just two branches using symmetry.

\subsection{Full Model}

The full model is the map $x \mapsto f(x) \mod 6$.
Where $f$ is given by the following collection of equations.
\begin{align}
    f(x) & = \begin{cases}
        g(x) & \text{if } r(x) < 3 \\
        g(x) + 3 & \text{else}
    \end{cases} \label{equ:quad.full.f} \\
    g(x) & = \begin{cases}
        a_L \cdot s_L(x)^2 + b_L \cdot x + c_L & \text{if } s(x) < \frac{3}{2} \\
        a_R \cdot s_R(x)^2 + b_R \cdot x + c_R & \text{else}
    \end{cases} \label{equ:quad.full.g}
\end{align}

\Cref{equ:quad.full.f} causes the disontinuity in the middle at $x = 3$.
It also makes sure, the symmetry $f(x + 3) \equiv f(x) + 3 \mod 6$ is true.
Each half of the model is then governed by \Cref{equ:quad.full.g}.
Here all the 6 parameters $a_L, a_R, b_L, b_R, c_L,$ and $c_R$ act.

\Crefrange{equ:quad.full.s}{equ:quad.full.sr} provide adjusted values of x for either choosing between branches in both halves or substituting in the quadratic formulas of each branch.
\begin{subequations}
\begin{align}
    s(x) & = x \mod 3 \label{equ:quad.full.s} \\
    s_L(x) & = s(x) - \frac{3}{4} \\
    s_R(x) & = s(x) - \frac{9}{4} \label{equ:quad.full.sr}
\end{align}
\end{subequations}

\subsection{Variation of Single Parameters}

We start by examining the behavior of the quadratic model under variations of single parameters like $a_L, a_R, b_L, b_R, c_L,$ and $c_R$.

\subsubsection{Fixing $a_L = a_R = 1, b_L = b_R = 0$}

\Cref{fig:quadratic.full.2d.full} shows a 2D-scan of the periods of the stable cycles.
The parameters $a_L = a_R = 1$ and $b_L = b_R = 0$ are fixed and the parameters $c_L$ and $c_R$ are varied.
Both are varied within the range $[0, 6]$ because beyond that the diagram just repeats infinitely.
The structure in the middle left is enhanced and depicted in \Cref{fig:quadratic.full.2d.z1}.

\begin{figure}
    \centering
    \begin{subfigure}{0.4\textwidth}
        \centering
        \includegraphics[width=\textwidth]{21_Quadratic_mod6/2D_Period_Full/result.png}
        \caption{Full}
        \label{fig:quadratic.full.2d.full}
    \end{subfigure}
    \begin{subfigure}{0.4\textwidth}
        \centering
        \includegraphics[width=\textwidth]{21_Quadratic_mod6/2D_Period_Zoomed1/result.png}
        \caption{Zoomed}
        \label{fig:quadratic.full.2d.z1}
    \end{subfigure}
    \caption{2D Scan of Full Quadratic Model}
\end{figure}

A phenomenon like in the original model could not be found here.
But something very similar happens on the border of these wings.
\Cref{fig:quad.full.Cobwebs} shows the cobwebs along the line marked in \Cref{fig:quadratic.full.2d.z1}.
Before the border, there is one stable cycle with period 8.
This cycle is depicted in \Cref{fig:quad.full.CobwebA} and its symbolic sequence is $\A^3B\C^3\D$.
At the border, there is an area where two cycles coexist.
You cannot see this in the 2D scans above, since it only ever picks up on one cycle.
\Cref{fig:quad.full.CobwebB} shows the coexisting cycles at this border.
In contrast to the original model, the cycle that existed before in \Cref{fig:quad.full.CobwebA}, still exists alongside the new cycle with period 6.
The symbolic sequence of the new cycle is $\A^2\B\C^2\D$.

This is different from the dynamics in the original model in two ways.
First, the cycles before and after the area of coexistence have different periods.
And second, the cycles existing outside the area of coexistence still exist inside the area of coexistence.
In the original model, the cycles existing outside the area of coexistence would disappear at the border and new cycles would emerge inside this area.

\begin{figure}
    \centering
    \begin{subfigure}{0.3\textwidth}
        \centering
        \includegraphics[width=\textwidth]{21_Quadratic_mod6/Cobweb_A/result.png}
        \caption{Before border}
        \label{fig:quad.full.CobwebA}
    \end{subfigure}
    \begin{subfigure}{0.3\textwidth}
        \centering
        \includegraphics[width=\textwidth]{21_Quadratic_mod6/Cobweb_B/result.png}
        \caption{At border}
        \label{fig:quad.full.CobwebB}
    \end{subfigure}
    \begin{subfigure}{0.3\textwidth}
        \centering
        \includegraphics[width=\textwidth]{21_Quadratic_mod6/Cobweb_C/result.png}
        \caption{After border}
        \label{fig:quad.full.CobwebC}
    \end{subfigure}
    \caption{Cobwebs along marked line}
    \label{fig:quad.full.Cobwebs}
\end{figure}

\subsubsection{Fixing $a_L = a_R = b_L = b_R = 1$}

If you compare the functions in the cobwebs \Crefrange{fig:quad.full.CobwebA}{fig:quad.full.CobwebB} to the functions of the original model in \Crefrange{fig:yunus.2pi.CobwebA2}{fig:yunus.2pi.CobwebD2} you can see some differences.
One is, that the right end of the branches $\B$ and $\D$ is higher than the left part.
To get our model to look a little more like the original model, we now skew the branches by fixing also $b_L = b_R = 1$

The 2D scan for the periods when varying $c_L$ and $c_R$ now looks different.
\Cref{fig:quadratic.full.skew.2d.full} shows the full scan, while \Cref{fig:quadratic.full.skew.2d.z1} shows an enhanced version of it, depicting the artefact in the middle of the left half of the full scan.

\begin{figure}
    \centering
    \begin{subfigure}{0.4\textwidth}
        \centering
        \includegraphics[width=\textwidth]{21_Quadratic_mod6/Skew/2D_Period_SFull/result.png}
        \caption{Full}
        \label{fig:quadratic.full.skew.2d.full}
    \end{subfigure}
    \begin{subfigure}{0.4\textwidth}
        \centering
        \includegraphics[width=\textwidth]{21_Quadratic_mod6/Skew/2D_Period_SZoomed1/result.png}
        \caption{Zoomed}
        \label{fig:quadratic.full.skew.2d.z1}
    \end{subfigure}
    \caption{2D Scan of Full Skewed Quadratic Model}
\end{figure}

\todo{replace with new cobwebs}


\begin{figure}
    \centering
    \begin{subfigure}{0.3\textwidth}
        \centering
        \includegraphics[width=\textwidth]{21_Quadratic_mod6/Cobweb_A/result.png}
        \caption{Before border}
        \label{fig:quad.full.skew.CobwebA}
    \end{subfigure}
    \begin{subfigure}{0.3\textwidth}
        \centering
        \includegraphics[width=\textwidth]{21_Quadratic_mod6/Cobweb_B/result.png}
        \caption{At border}
        \label{fig:quad.full.skew.CobwebB}
    \end{subfigure}
    \begin{subfigure}{0.3\textwidth}
        \centering
        \includegraphics[width=\textwidth]{21_Quadratic_mod6/Cobweb_C/result.png}
        \caption{After border}
        \label{fig:quad.full.skew.CobwebC}
    \end{subfigure}
    \caption{Cobwebs along marked line}
    \label{fig:quad.full.skew.Cobwebs}
\end{figure}

\subsubsection{Fixing $a_L = a_R = b_L = 1, c_R = 2.3$}

To better mimic the behavior of the original model function, we now fix the parameters $a_L, a_R, b_L,$ and $c_R$.
Fixing $a_L$ and $b_L$ and varying $c_L$ in the range $[0.8, 1.4]$ will cause the branches $\A$ and $\C$ to move upwards, just as we observed in the original model in \Cref{sec:og.param.effects}.
To get the left part of branches $\B$ and $\D$ to move downwards while also moving the local minima of those branches to the lower left, we vary $b_R$ in the range $[0, 2]$.
\Cref{fig:quadratic.full.cLbR.2d.full} shows a 2D scan of the periods in this model.


\begin{figure}
    \centering
    \includegraphics[width=0.6\textwidth]{21_Quadratic_mod6/TestingDifferentParameters/2D_Period_cLbR_Lower/result.png}
    \caption{2D Scan of Quadratic Model Imitating the Original Model}
    \label{fig:quadratic.full.cLbR.2d.full}
\end{figure}

\Cref{fig:quad.full.cLbR.Cobwebs} shows cobwebs along the red line in \Cref{fig:quadratic.full.cLbR.2d.full}.
You can see, that the 14-cycle that existed at the beginning in \Cref{fig:quad.full.cLbR.CobwebA} with symbolic sequence $\A^3\B^4\C^3\D^4$ still exists at the end in \Cref{fig:quad.full.cLbR.CobwebC}.
In \Cref{fig:quad.full.cLbR.CobwebB} you can also see another 14-cycle coexisting.
It has the symbolic sequence $\A^2\B^5\C^2\D^5$.
This is different from the phenomenon, we are searching for.
The reason for this coexistence is, that the arm of the wing crosses this area of period 14.
You can see the arm above the right end of the red line passing through the arm of the area we are examining.

\begin{figure}
    \centering
    \begin{subfigure}{0.3\textwidth}
        \centering
        \includegraphics[width=\textwidth]{21_Quadratic_mod6/TestingDifferentParameters/Cobweb_cLbR/result_A.png}
        \caption{Before border}
        \label{fig:quad.full.cLbR.CobwebA}
    \end{subfigure}
    \begin{subfigure}{0.3\textwidth}
        \centering
        \includegraphics[width=\textwidth]{21_Quadratic_mod6/TestingDifferentParameters/Cobweb_cLbR/result_B.png}
        \caption{At border}
        \label{fig:quad.full.cLbR.CobwebB}
    \end{subfigure}
    \begin{subfigure}{0.3\textwidth}
        \centering
        \includegraphics[width=\textwidth]{21_Quadratic_mod6/TestingDifferentParameters/Cobweb_cLbR/result_C.png}
        \caption{After border}
        \label{fig:quad.full.cLbR.CobwebC}
    \end{subfigure}
    \caption{Cobwebs along marked line}
    \label{fig:quad.full.cLbR.Cobwebs}
\end{figure}

\subsection{Combining Parameter}

Thus far, we have not seen type B regions, that are not caused by overlapping type A regions.
We now want to change multiple parameters at the same time to imitate the function of the original model better.
For this, we introduce new parameters, $p_x$ and $p_y$, and define the actual model parameters dependent on those two.

\subsubsection{Defining $a_R = 1 + p_x$, $b_R = 2 \cdot px$, and $c_L = p_y$}

\todo{better imitation but nothing was found}

\subsubsection{Defining $a_R = 1 + 2 \cdot p_x$, $b_R = px$, and $c_L = p_y$}

This definition of $a_R$ and $b_R$ is similar to before, but now $p_x$ has double the effect on $a_R$ and half the effect on $b_R$.
It was created by accident since it does not imitate the original model as nicely as before.
\Cref{fig:quadratic.full.2aR1bR_cL.2d.full} shows the 2D scan of the different periods.
Regions we will have a closer look at, are marked with red rectangles.

\begin{figure}
    \centering
    \includegraphics[width=0.6\textwidth]{21_002_Quadratic_2aR1bR_cL/2D_Period_Selected/result.png}
    \caption{2D Scan of Periods of Quadratic Model with ...}
    \label{fig:quadratic.full.2aR1bR_cL.2d.full}
\end{figure}

The first enhanced region, shown in \Cref{fig:quadratic.full.2aR1bR_cL.2d.1}, has two areas with stable cycles of period 6 that overlap.
\Cref{fig:quadratic.regions.2aR1bR_cL.2d.1} shows the borders of the two regions.
It was created by halving the model and scanning for the borders of regions of different periods.
We will see that the bottom area is a type B area, and therefore the period in the halved model is double the period in the full model.

\begin{figure}
    \centering
    \begin{subfigure}{0.4\textwidth}
        \centering
        \includegraphics[width=\textwidth]{21_002_Quadratic_2aR1bR_cL/P6/2D_Period_P6/result.png}
        \caption{Periods}
        \label{fig:quadratic.full.2aR1bR_cL.2d.1}
    \end{subfigure}
    \begin{subfigure}{0.4\textwidth}
        \centering
        \includegraphics[width=\textwidth]{21_002_Quadratic_2aR1bR_cL/P6/2D_Regions_P6/result.png}
        \caption{Period Regions}
        \label{fig:quadratic.regions.2aR1bR_cL.2d.1}
    \end{subfigure}
    \caption{2D Scans of First Marked Region}
\end{figure}

\Cref{fig:quad.full.2aR1bR_cL.Cobwebs} shows cobweb diagrams at the three points marked in \Cref{fig:quadratic.full.2aR1bR_cL.2d.1,fig:quadratic.regions.2aR1bR_cL.2d.1}.
At point $A$, we have two stable coexisting cycles of period 6 with symbolic sequences $\A\B\C^3D$ and $\A^3\B\C\D$.
You can see them in \Cref{fig:quad.full.2aR1bR_cL.CobwebA}.
This region is therefore a type B region since we have two coexisting cycles that are symmetric by rotationp.
At point $C$, we only have one stable cycle of period 6 with symbolic sequence $\A^2\B\C^3\D$.
\Cref{fig:quad.full.2aR1bR_cL.CobwebC} shows this cycle.
The upper region, therefore, is a type A region.
Both these regions overlap like in the original model.
\Cref{fig:quad.full.2aR1bR_cL.CobwebB} shows the stable cycles at point $C$ in the overlapping area.
\todo{similarity to overlap in og model}

\begin{figure}
    \centering
    \begin{subfigure}{0.3\textwidth}
        \centering
        \includegraphics[width=\textwidth]{21_002_Quadratic_2aR1bR_cL/P6/Cobweb_P6_A/result.png}
        \caption{At Point A}
        \label{fig:quad.full.2aR1bR_cL.CobwebA}
    \end{subfigure}
    \begin{subfigure}{0.3\textwidth}
        \centering
        \includegraphics[width=\textwidth]{21_002_Quadratic_2aR1bR_cL/P6/Cobweb_P6_B/result.png}
        \caption{At Point B}
        \label{fig:quad.full.2aR1bR_cL.CobwebB}
    \end{subfigure}
    \begin{subfigure}{0.3\textwidth}
        \centering
        \includegraphics[width=\textwidth]{21_002_Quadratic_2aR1bR_cL/P6/Cobweb_P6_C/result.png}
        \caption{At Point C}
        \label{fig:quad.full.2aR1bR_cL.CobwebC}
    \end{subfigure}
    \caption{Cobwebs at Different Points}
    \label{fig:quad.full.2aR1bR_cL.Cobwebs}
\end{figure}



\newpage


\printbibliography

All links were last followed on
\todo{add date}

\appendix
\only<beamer>{\titleframe}

\begin{frame}{Overview}
	\tableofcontents
\end{frame}


\thanksframe



\pagestyle{empty}
\renewcommand*{\chapterpagestyle}{empty}
\Versicherung
\end{document}
