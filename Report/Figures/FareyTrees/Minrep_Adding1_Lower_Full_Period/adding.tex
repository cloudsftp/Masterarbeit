\documentclass{article}
\usepackage{xcolor}
\newcommand{\red}{\color{red}}
\usepackage{tikz}
\usetikzlibrary{arrows,calc}
\usepackage[width=128mm]{geometry}
\pagestyle{empty}

\renewcommand{\L}{\scriptstyle\mathcal{L}}
\newcommand{\R}{\scriptstyle\mathcal{R}}

\newcommand{\A}{\scriptstyle\mathcal{A}}
\newcommand{\B}{\scriptstyle\mathcal{B}}
\newcommand{\C}{\scriptstyle\mathcal{C}}
\newcommand{\D}{\scriptstyle\mathcal{D}}

% key executed at each level to set the with of the nodes
\pgfkeysdef{/tikz/widthlevel}%
 {\setlength{\NodeWidth}{\BaseWidth}%
  \addtolength{\NodeWidth}{#1\WidthIncrement}%
  \addtolength{\NodeWidth}{\WidthIncrement}}

% set node width manually per key
\pgfkeysdef{/tikz/setwidth}%
 {\setlength{\NodeWidth}{#1}}

% ========================================
% =============  PARAMETERS  =============
% ========================================

\newcommand{\Scale}{1.0}                                     % scaling factor (including text scaling)
\newlength{\TreeDiameter}\setlength{\TreeDiameter}{100mm}    % the 'vertical' diameter of the tree
\newlength{\LevelDistance}\setlength{\LevelDistance}{2.5em}    % the distance from level to level
\newlength{\BaseWidth}\setlength{\BaseWidth}{1ex}            % the node width at level 0 (actually 'level -1')
\newlength{\WidthIncrement}\setlength{\WidthIncrement}{1em}  % the node width increment from level to level
\newlength{\NodeHeight}\setlength{\NodeHeight}{1.5em}        % the node height
\newlength{\NodeWidth}                                       % the node width register

\pgfkeysdef{/tikz/settextwidth}%
 {\settowidth{\NodeWidth}{\tiny#1}}

\settoheight{\NodeHeight}{\tiny$(\R\L^2)^3$}

\begin{document}

\begin{tikzpicture}
	[transform shape,scale=\Scale,
		grow'=right,
		every node/.style={text width=\NodeWidth,
				draw,rectangle,rounded corners=1pt,
				minimum height=\NodeHeight+4pt,
				inner sep=2pt},
		lbl/.style={text width=,draw=none,inner sep=},
		level/.style={widthlevel=#1, % automatically set node width
				sibling distance=\TreeDiameter/(2^#1),
				level distance=\LevelDistance+#1\WidthIncrement},
		level 1/.style={settextwidth=$10$, level distance=2em},    %,level distance=1cm
		level 2/.style={settextwidth=$10$, level distance=3em},    %,level distance=1cm
		level 3/.style={settextwidth=$100$, level distance=4em},    %,level distance=1cm
		edge from parent/.style={}] % don't draw tree edges


	% ========================================
	% ===============  NODES  ================
	% ========================================

	\path[widthlevel=-1,setwidth=0.7em] node[name=r] at (0,\NodeHeight) {\tiny$13$}
	node[name=l] at (0,-\TreeDiameter-\NodeHeight) {\tiny$14$};

	\path[setwidth=0.7em] node[name=lr] at (\LevelDistance,-.5\TreeDiameter) {\tiny$40$}
	child{ node{\tiny$66$}
			child{ node{\tiny$92$}
					child{ node{\tiny$118$} }
					child{ node{\tiny$79$} }
				}
			child{ node{\tiny$53$}
					child{ node{\tiny$172$} }
					child{ node{\tiny$146$} }
				}
		}
	%---------%
	child{ node{\tiny$27$}
			child{ node{\tiny$94$}
					child{ node{\tiny$87$} }
					child{ node{\tiny$148$} }
				}
			child{ node{\tiny$68$}
					child{ node{\tiny$122$} }
					child{ node{\tiny$41$} }
				}
		};

	% ========================================
	% ===============  EDGES  ================
	% ========================================

	\path[>=latex,->] % set arrowhead
	(r) edge (lr.north west)
	(r) edge (lr-1.north west)
	(r) edge (lr-1-1.north west)
	(r) edge (lr-1-1-1.north west)
	(l) edge (lr.south west)
	(l) edge (lr-2.south west)
	(l) edge (lr-2-2.south west)
	(l) edge (lr-2-2-2.south west)
	;

	\path[>=latex,->] % set arrowhead
	% level 1
	(lr)         edge (lr-1.south west)
	(lr)         edge (lr-1-2.south west)
	(lr)         edge (lr-1-2-2.south west)
	(lr)         edge (lr-2.north west)
	(lr)         edge (lr-2-1.north west)
	(lr)         edge (lr-2-1-1.north west)

	% level 2
	(lr-1)       edge (lr-1-1.south west)
	(lr-1)       edge (lr-1-1-2.south west)
	(lr-1)       edge (lr-1-2.north west)
	(lr-1)       edge (lr-1-2-1.north west)

	(lr-2)       edge (lr-2-2.north west)
	(lr-2)       edge (lr-2-2-1.north west)
	(lr-2)       edge (lr-2-1.south west)
	(lr-2)       edge (lr-2-1-2.south west)

	% level 3

	(lr-1-1)       edge (lr-1-1-1.south west)
	(lr-1-1)       edge (lr-1-1-2.north west)

	(lr-1-2)       edge (lr-1-2-1.south west)
	(lr-1-2)       edge (lr-1-2-2.north west)

	(lr-2-2)       edge (lr-2-2-2.north west)
	(lr-2-2)       edge (lr-2-2-1.south west)

	(lr-2-1)       edge (lr-2-1-2.north west)
	(lr-2-1)       edge (lr-2-1-1.south west)

	;

	\node[text width=1ex,draw=black!1,very thin] at (0,  3.2\NodeHeight) {\scriptsize 0};
	\node[text width=1ex,draw=black!1,very thin] at (0.8,3.2\NodeHeight) {\scriptsize 1};
	\node[text width=1ex,draw=black!1,very thin] at (1.6,3.2\NodeHeight) {\scriptsize 2};
	\node[text width=1ex,draw=black!1,very thin] at (2.6,3.2\NodeHeight) {\scriptsize 3};
	\node[text width=1ex,draw=black!1,very thin] at (4.0,3.2\NodeHeight) {\scriptsize 4};

	% horizontal line beneath column labels
	\draw[thick,->] (-0.5,2.3\NodeHeight) -- (4.5,2.3\NodeHeight);

\end{tikzpicture}

\end{document}
